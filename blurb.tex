% Blurb on back cover, gets included in lulu cover as well
% as regular version, so be careful with formatting

  {
  \parindent=0pt
  \parskip=\baselineskip
  {\OPTbacktitlefont
  \textit{引言摘录:}}
  \OPTbackfont

  \emph{同伦类型论 (Homotopy Type Theory)} 是数学的一个新分支,它以令人惊讶的方式结合了多个不同领域的元素。它基于最近发现的 \emph{同伦论 (Homotopy Theory)} 和 \emph{类型论 (Type Theory)} 之间的联系。
  它涉及的主题范围广泛,从球面同伦群、类型检查算法到弱 $\infty$-群 (Weak $\infty$-Groupoids) 的定义。

  同伦类型论为数学基础带来了新的思想。
  一方面,有 Voevodsky 的微妙而美丽的 \emph{同值性公理 (Univalence Axiom)}。
  同值性公理特别意味着同构结构可以被视为相同,这一原则虽然与传统基础的“官方”教义不符,却一直被数学家们在实际工作中愉快地使用。
  另一方面,我们有 \emph{高阶归纳类型 (Higher Inductive Types)},它们为同伦论中的一些基本空间和构造提供了直接的逻辑描述:球面、圆柱体、截断、局部化等。
  这些思想在经典的集合论基础中无法直接捕捉,但在同伦类型论中结合后,它们形成了一种全新的“同伦类型逻辑 (Logic of Homotopy Types)”。

  这表明了一种数学基础的新概念,具有内在的同伦内容,一种数学对象的“不变量”概念——以及便捷的计算机实现,可以作为数学家工作的实用辅助工具。
  这就是 \emph{同值性基础 (Univalent Foundations)} 计划。

  本书旨在首次系统性地阐述同值性基础的基本内容,并收集这种新型推理风格的实例——但不要求读者掌握或学习任何形式逻辑,也不需要使用任何计算机证明助手。
  我们相信,同值性基础最终将成为集合论的可行替代方案,成为大多数数学家进行非形式化数学时的“隐式基础”。

  \bigskip

  \begin{center}
  {\Large
  \textit{在 HomotopyTypeTheory.org 获取本书的免费副本。}}
  \end{center}
}
