\chapter{等价性 (Equivalences)}
\label{cha:equivalences}

我们现在详细研究在\cref{sec:basics-equivalences}中简要介绍的\emph{类型的等价性 (equivalence of types)}概念。具体来说,我们将给出几种不同的方式来定义具有前述特性的一种类型 $\isequiv(f)$。回想一下,我们希望 $\isequiv(f)$ 具有以下性质,在此处我们再次陈述这些性质:
\begin{enumerate}
  \item $\qinv(f) \to \isequiv (f)$。\label{item:beb1}
  \item $\isequiv (f) \to \qinv(f)$。\label{item:beb2}
  \item $\isequiv(f)$ 是一个纯命题 (mere proposition)。\label{item:beb3}
\end{enumerate}
这里 $\qinv(f)$ 表示 $f$ 的准逆 (quasi-inverse) 类型:
\begin{equation*}
  \sm{g:B\to A} \big((f \circ g \htpy \idfunc[B]) \times (g\circ f \htpy \idfunc[A])\big)。
\end{equation*}
根据函数外延性 (function extensionality),这意味着 $\qinv(f)$ 等价于类型
\begin{equation*}
  \sm{g:B\to A} \big((f \circ g = \idfunc[B]) \times (g\circ f = \idfunc[A])\big)。
\end{equation*}
我们将定义三种具有性质~\ref{item:beb1}--\ref{item:beb3} 的类型,分别称为:
\begin{itemize}
  \item 半伴随等价 (half adjoint equivalences),
  \item 双可逆映射 (bi-invertible maps),
  \index{function!bi-invertible}
  和
  \item 可缩函数 (contractible functions)。
\end{itemize}
我们还将证明这些类型是等价的。这些名称特意显得有些繁琐,因为在我们知道它们都是等价的并具有性质~\ref{item:beb1}--\ref{item:beb3}之后,我们会简单地使用“等价性 (equivalence)”这个词,而不需要指定我们选择了哪种特定定义。但为了本章中的比较目的,我们需要为每个定义提供不同的名称。

在我们研究等价性的不同概念之前,我们先稍微解释一下为什么需要一个不同于准可逆性的概念。

\section{准逆 (Quasi-inverses)}
\label{sec:quasi-inverses}

\index{quasi-inverse|(}%
我们已经提到 $\qinv(f)$ 是不令人满意的,因为它不是一个纯命题,而我们希望一个给定的函数最多以一种方式“成为等价的”。然而,我们还没有证明 $\qinv(f)$ 不是一个纯命题。在本节中,我们展示一个具体的反例。

\begin{lem}\label{lem:qinv-autohtpy}
如果 $f:A\to B$ 是使得 $\qinv (f)$ 可居住的 (inhabited),那么
\[\eqv{\qinv(f)}{\Parens{\prd{x:A}(x=x)}}。\]
\end{lem}
\begin{proof}
  根据假设,$f$ 是一个等价性,也就是说我们有 $e:\isequiv(f)$,因此 $(f,e):\eqv A B$。根据单值性 (univalence),$\idtoeqv:(A=B) \to (\eqv A B)$ 是一个等价性,因此我们可以假设 $(f,e)$ 的形式为 $\idtoeqv(p)$,其中 $p:A=B$。然后根据路径归纳 (path induction),我们可以假设 $p$ 是 $\refl{A}$,在这种情况下,$f$ 是 $\idfunc[A]$。因此,我们简化为证明 $\eqv{\qinv(\idfunc[A])}{(\prd{x:A}(x=x))}$。现在,根据定义,我们有
  \[ \qinv(\idfunc[A]) \jdeq
  \sm{g:A\to A} \big((g \htpy \idfunc[A]) \times (g \htpy \idfunc[A])\big)。
  \]
  根据函数外延性,这等价于
  \[ \sm{g:A\to A} \big((g = \idfunc[A]) \times (g = \idfunc[A])\big)。
  \]
  根据 \cref{ex:sigma-assoc},这等价于
  \[ \sm{h:\sm{g:A\to A} (g = \idfunc[A])} (\proj1(h) = \idfunc[A])。
  \]
  然而,根据 \cref{thm:contr-paths},$\sm{g:A\to A} (g = \idfunc[A])$ 是以 $(\idfunc[A],\refl{\idfunc[A]})$ 为中心的可缩 (contractible) 类型;因此根据 \cref{thm:omit-contr},这个类型等价于 $\idfunc[A] = \idfunc[A]$。根据函数外延性,$\idfunc[A] = \idfunc[A]$ 等价于 $\prd{x:A} x=x$。
\end{proof}

\noindent
我们注意到 \cref{ex:qinv-autohtpy-no-univalence} 要求在避免单值性的情况下证明上述引理。

因此,我们需要一些 $A$,它允许 $\prd{x:A}(x=x)$ 的非平凡 (nontrivial) 元素。将 $A$ 视为一个更高阶的群胚 (higher groupoid),$\prd{x:A}(x=x)$ 的一个居留元素 (inhabitant) 是从 $A$ 的恒等函子 (identity functor) 到其自身的自然变换 (natural transformation)。这样的变换被称为一个范畴的\define{中心 (center of a category)},\index{center!of a category}%
\index{category!center of}%
因为自然性公理要求它们与所有态射 (morphisms) 交换。传统上,如果 $A$ 只是一个被视为单对象群胚 (one-object groupoid) 的群体,那么这将恰好得出通常群论意义上的中心。这为以下内容提供了一些动机。

\begin{lem}\label{lem:autohtpy}
假设我们有一个类型 $A$ 和 $a:A$ 以及 $q:a=a$,使得
\begin{enumerate}
  \item 类型 $a=a$ 是一个集合 (set)。\label{item:autohtpy1}
  \item 对于所有 $x:A$,我们有 $\brck{a=x}$。\label{item:autohtpy2}
  \item 对于所有 $p:a=a$,我们有 $p\ct q = q \ct p$。\label{item:autohtpy3}
\end{enumerate}
那么存在 $f:\prd{x:A} (x=x)$,且有 $f(a)=q$。
\end{lem}
\begin{proof}
  令 $g:\prd{x:A} \brck{a=x}$ 为~\ref{item:autohtpy2}给出的值。首先我们
  观察到每个类型 $\id[A]xy$ 是一个集合。因为作为一个集合是一个纯命题,我们可以应用命题截断 (propositional truncation) 的归纳原则,假设 $g(x)=\bproj
  p$ 且 $g(y)=\bproj{p'}$,其中 $p:a=x$ 和 $p':a=y$。在这种情况下,与
  $p$ 和 $\opp{p'}$ 组成的映射等价于 $\eqv{(x=y)}{(a=a)}$。但由于~\ref{item:autohtpy1} 中 $(a=a)$ 是一个集合,因此 $(x=y)$ 也是一个集合。

  现在,我们希望通过为每个 $x$ 分配路径 $\opp{g(x)}
  \ct q \ct g(x)$ 来定义 $f$,但这不奏效,因为 $g(x)$ 不属于 $a=x$,而是 $\brck{a=x}$,并且类型 $(x=x)$ 可能不是一个纯命题,因此我们不能使用命题截断的归纳法。相反,我们可以应用 \cref{sec:unique-choice} 中提到的技巧:我们唯一地刻画我们希望构造的对象。让我们定义,对于每个 $x:A$,类型
  \[ B(x) \defeq \sm{r:x=x} \prd{s:a=x} (r = \opp s \ct q\ct s)。\]
  我们声称每个 $B(x)$ 是一个纯命题。
  由于这个声明本身是一个纯命题,我们可以再次应用命题截断的归纳法,并假设 $g(x) = \bproj p$,其中 $p:a=x$。
  现在假设给定 $(r,h)$ 和 $(r',h')$ 在 $B(x)$ 中;然后我们有
  \[ h(p) \ct \opp{h'(p)} : r = r'。\]
  剩下的是展示,当沿着这个等式传递时,$h$ 与 $h'$ 相同,这通过在恒等类型和函数类型中传输 (\cref{sec:compute-paths,sec:compute-pi}),减少为展示
  \[ h(s) = h(p) \ct \opp{h'(p)} \ct h'(s) \]
  对于任何 $s:a=x$。
  但这两侧都是 $(x=x)$ 的元素之间的等式,因此它遵循我们之前的观察,即 $(x=x)$ 是一个集合。

  因此,每个 $B(x)$ 是一个纯命题;我们声称 $\prd{x:A} B(x)$。
  给定 $x:A$,我们现在可以调用命题截断的归纳法,假设 $g(x) = \bproj p$,其中 $p:a=x$。
  我们定义 $r \defeq \opp p \ct q \ct p$;为了填充 $B(x)$,剩下的是展示对于任何 $s:a=x$ 我们有
  $r = \opp s \ct q \ct s$。
  操作路径,这减少为展示 $q\ct (p\ct \opp s) = (p\ct \opp s) \ct q$。
  但这只是~\ref{item:autohtpy3} 的一个实例。
\end{proof}

\begin{thm}\label{thm:qinv-notprop}
存在类型 $A$ 和 $B$ 以及一个函数 $f:A\to B$ 使得 $\qinv(f)$ 不是一个纯命题。
\end{thm}
\begin{proof}
  这足以展示一个类型 $X$,使得 $\prd{x:X} (x=x)$ 不是一个纯命题。
  定义 $X\defeq \sm{A:\type} \brck{\bool=A}$,如在 \cref{thm:no-higher-ac} 的证明中所述。
  展示一个 $f:\prd{x:X} (x=x)$,它不同于 $\lam{x} \refl{x}$ 即可。

  令 $a \defeq (\bool,\bproj{\refl{\bool}}) : X$,并令 $q:a=a$ 是对应于非恒等的等价性 $e:\eqv\bool\bool$ 的路径,其定义为 $e(\bfalse)\defeq\btrue$ 和 $e(\btrue)\defeq\bfalse$。
  我们希望应用 \cref{lem:autohtpy} 来构建一个 $f$。
  根据 $X$ 的定义,子集类型中的等式 (\cref{subsec:prop-subsets}) 和单值性,我们有 $\eqv{(a=a)}{(\eqv{\bool}{\bool})}$,这是一个集合,因此~\ref{item:autohtpy1} 成立。
  类似地,根据 $X$ 的定义和子集类型中的等式,我们有~\ref{item:autohtpy2}。
  最后,\cref{ex:eqvboolbool} 表明每个等价性 $\eqv\bool\bool$ 都等于 $\idfunc[\bool]$ 或 $e$,因此我们可以通过四种情况分析展示~\ref{item:autohtpy3}。

  因此,我们有 $f:\prd{x:X} (x=x)$,且有 $f(a) = q$。
  由于 $e$ 不等于 $\idfunc[\bool]$,$q$ 不等于 $\refl{a}$,因此 $f$ 不等于 $\lam{x} \refl{x}$。
  因此,$\prd{x:X} (x=x)$ 不是一个纯命题。
\end{proof}

更普遍地,\cref{lem:autohtpy} 表明任何“Eilenberg--Mac Lane 空间 (Eilenberg--Mac Lane space)” $K(G,1)$,其中 $G$ 是一个非平凡的阿贝尔 (abelian) 群体,将提供一个反例;参见 \cref{cha:homotopy}。我们使用的类型 $X$ 结果是等价于 $K(\mathbb{Z}_2,1)$。在 \cref{cha:hits} 中,我们将看到圆 $S^1 = K(\mathbb{Z},1)$ 是另一个易于描述的例子。

我们现在继续描述更好的等价性概念。

\index{quasi-inverse|)}%

%%%%%%%%%%%%%%%%%%%%%%%%%%%%%%%%%%%%%%
\section{半伴随等价 (Half adjoint equivalences)}
\label{sec:hae}
%%%%%%%%%%%%%%%%%%%%%%%%%%%%%%%%%%%%%%

\index{equivalence!half adjoint|(defstyle}%
\index{half adjoint equivalence|(defstyle}%
\index{adjoint!equivalence!of types, half|(defstyle}%

在 \cref{sec:quasi-inverses} 中,我们得出结论认为 $\qinv(f)$ 通过丢弃一个可缩类型等价于 $\prd{x:A} (x=x)$。粗略地说,类型 $\qinv(f)$ 包含三个数据 $g$、$\eta$ 和 $\epsilon$,其中两个 ($g$ 和 $\eta$) 可以一起在 $f$ 是等价性时被视为可缩的。问题在于去掉这些数据后还剩下一个 ($\epsilon$)。为了解决这个问题,想法是增加一个\emph{额外的}数据,这样 $\epsilon$ 和它一起构成一个可缩类型。

\begin{defn}\label{defn:ishae}
一个函数 $f:A\to B$ 是一个\define{半伴随等价 (half adjoint equivalence)},
如果存在 $g:B\to A$ 和同伦 $\eta: g \circ f \htpy \idfunc[A]$ 和 $\epsilon:f \circ g \htpy \idfunc[B]$,使得存在一个同伦
\[\tau : \prd{x:A} \map{f}{\eta x} = \epsilon(fx)。\]
\end{defn}

因此我们定义类型 $\ishae(f)$ 为
\begin{equation*}
  \sm{g:B\to A}{\eta: g \circ f \htpy \idfunc[A]}{\epsilon:f \circ g \htpy \idfunc[B]} \prd{x:A} \map{f}{\eta x} = \epsilon(fx)。
\end{equation*}
注意在上述定义中,$\eta$ 和 $\epsilon$ 的一致性条件 (coherence condition) 仅涉及 $f$。我们可以考虑一个涉及 $g$ 的类似一致性条件:
\[\upsilon : \prd{y:B} \map{g}{\epsilon y} = \eta(gy)\]
并得到一个类似的定义 $\ishae'(f)$。

幸运的是,事实证明这些条件中的每一个都意味着另一个:

\begin{lem}\label{lem:coh-equiv}
对于函数 $f : A \to B$ 和 $g:B\to A$ 以及同伦 $\eta: g \circ f \htpy \idfunc[A]$ 和 $\epsilon:f \circ g \htpy \idfunc[B]$,以下条件是逻辑等价的:
\begin{itemize}
  \item $\prd{x:A} \map{f}{\eta x} = \epsilon(fx)$
  \item $\prd{y:B} \map{g}{\epsilon y} = \eta(gy)$
\end{itemize}
\end{lem}
\begin{proof}
  证明一个方向就足够了;另一个方向通过替换 $A$、$f$ 和 $\eta$ 为 $B$、$g$ 和 $\epsilon$ 分别得到。
  设 $\tau : \prd{x:A}\;\map{f}{\eta x} = \epsilon(fx)$。
  固定 $y : B$。
  使用 $\epsilon$ 的自然性 (naturality) 并应用 $g$,我们得到以下路径的交换图:
  \[\uppercurveobject{{ }}\lowercurveobject{{ }}\twocellhead{{ }}
  \xymatrix@C=3pc{gfgfgy \ar@{=}^-{gfg(\epsilon y)}[r] \ar@{=}_{g(\epsilon (fgy))}[d] & gfgy \ar@{=}^{g(\epsilon y)}[d] \\ gfgy \ar@{=}_{g(\epsilon y)}[r] & gy
  }\]
  在图的左侧使用 $\tau(gy)$ 得到
  \[\uppercurveobject{{ }}\lowercurveobject{{ }}\twocellhead{{ }}
  \xymatrix@C=3pc{gfgfgy \ar@{=}^-{gfg(\epsilon y)}[r] \ar@{=}_{gf(\eta (gy))}[d] & gfgy \ar@{=}^{g(\epsilon y)}[d] \\ gfgy \ar@{=}_{g(\epsilon y)}[r] & gy
  }\]
  使用 $\eta$ 与 $g \circ f$ 的交换 (\cref{cor:hom-fg}),我们有
  \[\uppercurveobject{{ }}\lowercurveobject{{ }}\twocellhead{{ }}
  \xymatrix@C=3pc{gfgfgy \ar@{=}^-{gfg(\epsilon y)}[r] \ar@{=}_{\eta (gfgy)}[d] & gfgy \ar@{=}^{g(\epsilon y)}[d] \\ gfgy \ar@{=}_{g(\epsilon y)}[r] & gy
  }\]
  然而,根据 $\eta$ 的自然性,我们也有
  \[\uppercurveobject{{ }}\lowercurveobject{{ }}\twocellhead{{ }}
  \xymatrix@C=3pc{gfgfgy \ar@{=}^-{gfg(\epsilon y)}[r] \ar@{=}_{\eta (gfgy)}[d] & gfgy \ar@{=}^{\eta(gy)}[d] \\ gfgy \ar@{=}_{g(\epsilon y)}[r] & gy
  }\]
  因此,取消所有但右侧的同伦,我们得到 $g(\epsilon y) = \eta(g y)$ 如所期望的。
\end{proof}

然而,重要的是我们不要在 $\ishae (f)$ 的定义中包含\emph{两者} $\tau$ 和 $\upsilon$(因此称之为“\emph{半}伴随等价 (half adjoint equivalence)”)。如果我们这么做,那么在取消可缩类型后,我们仍然会剩下一个数据——除非我们添加另一个更高的一致性条件。一般来说,如果我们在一个奇数个一致性条件之后切断,我们预期会得到一个良好行为的类型。

当然,很明显 $\ishae(f) \to\qinv(f)$:只需忘记一致性数据。另一个方向是同伦理论 (homotopy theory) 和范畴论 (category theory) 中的标准论点的一种版本。

\begin{thm}\label{thm:equiv-iso-adj}
对于任何 $f:A\to B$,我们有 $\qinv(f)\to\ishae(f)$。
\end{thm}
\begin{proof}
  假设 $(g,\eta,\epsilon)$ 是 $f$ 的准逆 (quasi-inverse)。我们必须提供一个四元组 $(g',\eta',\epsilon',\tau)$ 来证明 $f$ 是一个半伴随等价 (half adjoint equivalence)。为了定义 $g'$ 和 $\eta'$,我们可以选择显而易见的选择,将 $g'
  定义为 $g$,将 $\eta'定义为 $\eta$。然而,在 $\epsilon'$ 的定义中,我们需要开始考虑 $\tau$ 的构造,因此不能直接按照常规选择 $\epsilon'$ 为 $\epsilon$。相反,我们选择
  \begin{equation*}
    \epsilon'(b) \defeq \opp{\epsilon(f(g(b)))}\ct (\ap{f}{\eta(g(b))}\ct \epsilon(b))。
  \end{equation*}
  现在我们需要找到
  \begin{equation*}
    \tau(a): \ap{f}{\eta(a)}=\opp{\epsilon(f(g(f(a))))}\ct (\ap{f}{\eta(g(f(a)))}\ct \epsilon(f(a)))。
  \end{equation*}
  首先注意到,根据 \cref{cor:hom-fg},我们有
%$\eta(g(f(a)))\ct\eta(a)=\ap{g}{\ap{f}{\eta(a)}}\ct\eta(a)$ 并因此它遵循 $\eta(g(f(a)))=\ap{g}{\ap{f}{\eta(a)}}$。因此,我们可以应用
  \cref{lem:htpy-natural} 计算
  \begin{align*}
    \ap{f}{\eta(g(f(a)))}\ct \epsilon(f(a))
    & = \ap{f}{\ap{g}{\ap{f}{\eta(a)}}}\ct \epsilon(f(a))\\
    & = \epsilon(f(g(f(a))))\ct \ap{f}{\eta(a)}
  \end{align*}
  从而得到所需的路径 $\tau(a)$。
\end{proof}

结合 \cref{lem:coh-equiv}(或对称化的证明),我们也有 $\qinv(f)\to\ishae'(f)$。

剩下的是展示 $\ishae(f)$ 是一个纯命题。
为此,我们需要知道等价性的纤维 (fiber) 是可缩的。

\begin{defn}\label{defn:homotopy-fiber}
函数 $f:A\to B$ 在点 $y:B$ 上的\define{纤维 (fiber)}
\indexdef{fiber}%
\indexsee{function!fiber of}{fiber}%
定义为
\[ \hfib f y \defeq \sm{x:A} (f(x) = y)。\]
\end{defn}

在同伦理论中,这被称为 $f$ 的\emph{同伦纤维 (homotopy fiber)}。
\cref{sec:computational} 中的路径引理得出以下关于纤维中路径的刻画:

\begin{lem}\label{lem:hfib}
对于任何 $f : A \to B$、$y : B$ 以及 $(x,p),(x',p') : \hfib{f}{y}$,我们有
\[ \big((x,p) = (x',p')\big) \eqvsym \Parens{\sm{\gamma : x = x'} f(\gamma) \ct p' = p} \qedhere\]
\end{lem}

\begin{thm}\label{thm:contr-hae}
如果 $f:A\to B$ 是一个半伴随等价,那么对于任何 $y:B$,纤维 $\hfib f y$ 是可缩的。
\end{thm}
\begin{proof}
  令 $(g,\eta,\epsilon,\tau) : \ishae(f)$,并固定 $y : B$。
  作为 $\hfib{f}{y}$ 的缩并中心 (center of contraction),我们选择 $(gy, \epsilon y)$。
  现在取任何 $(x,p) : \hfib{f}{y}$;我们要构造从 $(gy, \epsilon y)$ 到 $(x,p)$ 的路径。
  根据 \cref{lem:hfib},足以给出一个路径 $\gamma : \id{gy}{x}$,使得 $\ap f\gamma \ct p = \epsilon y$。
  我们取 $\gamma \defeq \opp{g(p)} \ct \eta x$。
  然后我们有
  \begin{align*}
    f(\gamma) \ct p & = \opp{fg(p)} \ct f (\eta x) \ct p \\
    & = \opp{fg(p)} \ct \epsilon(fx) \ct p \\
    & = \epsilon y
  \end{align*}
  其中第二个等式由 $\tau x$ 得到,第三个等式是 $\epsilon$ 的自然性。
\end{proof}

我们现在定义封装可缩数据对的类型。
以下类型将准逆 $g$ 与其中一个同伦结合起来。

\begin{defn}\label{defn:linv-rinv}
给定一个函数 $f:A\to B$,我们定义类型
\begin{align*}
  \linv(f) &\defeq \sm{g:B\to A} (g\circ f\htpy \idfunc[A])\\
  \rinv(f) &\defeq \sm{g:B\to A} (f\circ g\htpy \idfunc[B])
\end{align*}
分别为 $f$ 的\define{左逆 (left inverses)}
\indexdef{left!inverse}%
\indexdef{inverse!left}%
和\define{右逆 (right inverses)}。
如果 $\linv(f)$ 可居住,我们称 $f$ 是\define{左可逆 (left invertible)}的,如果 $\rinv(f)$ 可居住,我们称 $f$ 是\define{右可逆 (right invertible)}的。
\end{defn}

\begin{lem}\label{thm:equiv-compose-equiv}
如果 $f:A\to B$ 有一个准逆,那么以下映射也有准逆:
\begin{align*}
(f\circ \blank) &: (C\to A) \to (C\to B)\\
(\blank\circ f) &: (B\to C) \to (A\to C)。
\end{align*}
\end{lem}
\begin{proof}
  如果 $g$ 是 $f$ 的准逆,那么 $(g\circ \blank)$ 和 $(\blank\circ g)$ 分别是 $(f\circ \blank)$ 和 $(\blank\circ f)$ 的准逆。
\end{proof}

\begin{lem}\label{lem:inv-hprop}
如果 $f : A \to B$ 有一个准逆,那么类型 $\rinv(f)$ 和 $\linv(f)$ 是可缩的。
\end{lem}
\begin{proof}
  根据函数外延性,我们有
  \[\eqv{\linv(f)}{\sm{g:B\to A} (g\circ f = \idfunc[A])}。\]
  但这是 $\Sigma$ 的纤维 (fiber) (\cref{subsec:prop-subsets}),因此
  根据 \cref{thm:equiv-compose-equiv,thm:equiv-iso-adj,thm:contr-hae},它是可缩的。
  类似地,$\rinv(f)$ 等价于 $\Sigma$ 上的 $\idfunc[B]$ 的纤维 (fiber),因此也是可缩的。
\end{proof}

接下来,我们定义将另一个同伦与额外的一致性数据结合起来的类型。\index{coherence}%

\begin{defn}\label{defn:lcoh-rcoh}
对于 $f : A \to B$,一个左逆 $(g,\eta) : \linv(f)$ 和一个右逆 $(g,\epsilon) : \rinv(f)$,我们表示
\begin{align*}
  \lcoh{f}{g}{\eta} & \defeq \sm{\epsilon : f\circ g \htpy \idfunc[B]} \prd{y:B} g(\epsilon y) = \eta (gy), \\
  \rcoh{f}{g}{\epsilon} & \defeq \sm{\eta : g\circ f \htpy \idfunc[A]} \prd{x:A} f(\eta x) = \epsilon (fx)。
\end{align*}
\end{defn}

\begin{lem}\label{lem:coh-hfib}
对于任何 $f,g,\epsilon,\eta$,我们有
\begin{align*}
  \lcoh{f}{g}{\eta} & \eqvsym {\prd{y:B} \id[\hfib{g}{gy}]{(fgy,\eta(gy))}{(y,\refl{gy})}}, \\
  \rcoh{f}{g}{\epsilon} & \eqvsym {\prd{x:A} \id[\hfib{f}{fx}]{(gfx,\epsilon(fx))}{(x,\refl{fx})}}。
\end{align*}
\end{lem}
\begin{proof}
  使用 \cref{lem:hfib}。
\end{proof}

\begin{lem}\label{lem:coh-hprop}
如果 $f$ 是一个半伴随等价,那么对于任何 $(g,\epsilon) : \rinv(f)$,类型 $\rcoh{f}{g}{\epsilon}$ 是可缩的。
\end{lem}
\begin{proof}
  根据 \cref{lem:coh-hfib} 和从属函数类型保持可缩空间的事实,足以展示对于每个 $x:A$,类型 $\id[\hfib{f}{fx}]{(gfx,\epsilon(fx))}{(x,\refl{fx})}$ 是可缩的。
  但根据 \cref{thm:contr-hae},$\hfib{f}{fx}$ 是可缩的,并且任何可缩空间的路径空间本身就是可缩的。
\end{proof}

\begin{thm}\label{thm:hae-hprop}
对于任何 $f : A \to B$,类型 $\ishae(f)$ 是一个纯命题。
\end{thm}
\begin{proof}
  根据 \cref{ex:prop-inhabcontr},假设 $f$ 是一个半伴随等价,并展示 $\ishae(f)$ 是可缩的就足够了。
  现在根据 $\Sigma$ 的结合律 (\cref{ex:sigma-assoc}),类型 $\ishae(f)$ 等价于
  \[\sm{u : \rinv(f)} \rcoh{f}{\proj{1}(u)}{\proj{2}(u)}。\]
  但根据 \cref{lem:inv-hprop,lem:coh-hprop} 和 $\Sigma$ 保持可缩性的事实,后者类型也是可缩的。
\end{proof}

因此,我们已经证明 $\ishae(f)$ 具有作为类型 $\isequiv(f)$ 的所有三种特性。
在接下来的两节中,我们将讨论一些其他可能性。

\index{equivalence!half adjoint|)}%
\index{half adjoint equivalence|)}%
\index{adjoint!equivalence!of types, half|)}%

\section{双可逆映射 (Bi-invertible maps)}
\label{sec:biinv}

\index{function!bi-invertible|(defstyle}%
\index{bi-invertible function|(defstyle}%
\index{equivalence!as bi-invertible function|(defstyle}%

使用在 \cref{sec:hae} 引入的语言,我们可以重新表述在 \cref{sec:basics-equivalences} 提出的定义,如下所示。

\begin{defn}\label{defn:biinv}
我们说 $f:A\to B$ 是 \define{双可逆 (bi-invertible)} 映射,
如果它同时具有左逆元和右逆元:
\[ \biinv (f) \defeq \linv(f) \times \rinv(f). \]
\end{defn}

在 \cref{sec:basics-equivalences} 中,我们证明了 $\qinv(f)\to\biinv(f)$ 和 $\biinv(f)\to\qinv(f)$。
剩下的是以下内容。

\begin{thm}\label{thm:isprop-biinv}
对于任意 $f:A\to B$,类型 $\biinv(f)$ 是一个单纯命题 (mere proposition)。
\end{thm}
\begin{proof}
  我们假设 $f$ 是双可逆的并证明 $\biinv(f)$ 是收缩的 (contractible)。
  但由于 $\biinv(f)\to\qinv(f)$,根据 \cref{lem:inv-hprop},在这种情况下 $\linv(f)$ 和 $\rinv(f)$ 都是收缩的,而收缩类型的乘积也是收缩的。
\end{proof}

注意,这也符合在 \cref{sec:hae} 开头提出的建议:我们将 $g$ 和 $\eta$ 组合成一个收缩类型,并添加一个额外的数据,它与 $\epsilon$ 组合成一个收缩类型。
不同之处在于,我们没有添加一个 \emph{更高的} 数据(一个二维路径)来与 $\epsilon$ 组合,而是添加了一个 \emph{更低的} 数据(一个与左逆元分开的右逆元)。

\begin{cor}\label{thm:equiv-biinv-isequiv}
对于任意 $f:A\to B$,我们有 $\eqv{\biinv(f)}{\ishae(f)}$。
\end{cor}
\begin{proof}
  我们有 $\biinv(f) \to \qinv(f) \to \ishae(f)$ 和 $\ishae(f) \to \qinv(f) \to \biinv(f)$。
  由于 $\ishae(f)$ 和 $\biinv(f)$ 都是单纯命题,因此根据 \cref{lem:equiv-iff-hprop} 可以得出等价性。
\end{proof}

\index{function!bi-invertible|)}%
\index{bi-invertible function|)}%
\index{equivalence!as bi-invertible function|)}%

\section{收缩纤维 (Contractible fibers)}
\label{sec:contrf}

\index{function!contractible|(defstyle}%
\index{contractible!function|(defstyle}%
\index{equivalence!as contractible function|(defstyle}%

注意,我们关于 $\ishae(f)$ 和 $\biinv(f)$ 的证明中,使用了等价的纤维是收缩的这一事实。
实际上,这个性质本身就是等价的一个充分定义。

\begin{defn}[收缩映射 (Contractible maps)] \label{defn:equivalence}
映射 $f:A\to B$ 是 \define{收缩的 (contractible)},
如果对所有 $y:B$,纤维 $\hfib f y$ 是收缩的。
\end{defn}

因此,类型 $\iscontr(f)$ 被定义为
\begin{align}
  \iscontr(f) &\defeq \prd{y:B} \iscontr(\hfib f y)\label{eq:iscontrf}
  % \\
  % &\defeq \prd{y:B} \iscontr (\setof{x:A | f(x) = y}).
\end{align}
注意,在 \cref{sec:contractibility} 中,我们定义了 \emph{类型} 收缩的含义。
在这里,我们定义了 \emph{映射} 收缩的含义。
我们的术语遵循一般同伦理论的实践,即如果一个映射的所有(同伦)纤维都具有某个性质,则该映射具有该性质。
因此,当映射 $A\to\unit$ 是收缩的时,类型 $A$ 是收缩的。
从 \cref{cha:hlevels} 开始,我们也会称收缩映射和类型为 \emph{$(-2)$-截断的 ($(-2)$-truncated)}。

我们已经在 \cref{thm:contr-hae} 中展示了 $\ishae(f) \to \iscontr(f)$。
反之亦然:

\begin{thm}\label{thm:lequiv-contr-hae}
对于任意 $f:A\to B$ 我们有 ${\iscontr(f)} \to {\ishae(f)}$。
\end{thm}
\begin{proof}
  让 $P : \iscontr(f)$。我们通过将每个 $y : B$ 映射到 $y$ 处纤维的收缩中心来定义一个逆映射 $g : B \to A$:
  \[ g(y) \defeq \proj{1}(\proj{1}(Py))。\]
  因此,我们可以通过将 $y$ 映射到 $g(y)$ 确实属于 $y$ 处纤维的见证,来定义同伦 $\epsilon$:
  \[ \epsilon(y) \defeq \proj{2}(\proj{1}(P y))。\]
  剩下的是定义 $\eta$ 和 $\tau$。这当然意味着给出一个 $\rcoh{f}{g}{\epsilon}$ 的元素。根据 \cref{lem:coh-hfib},这相当于给出对于每个 $x:A$,从 $f$ 在 $fx$ 处的纤维中的 $(gfx,\epsilon(fx))$ 到 $(x,\refl{fx})$ 的路径。但这很简单:对于任意 $x : A$,类型 $\hfib{f}{fx}$ 根据假设是收缩的,因此这样的路径必须存在。我们可以显式地构造它为
  \[\opp{\big(\proj{2}(P(fx))(gfx,\epsilon(fx))\big)} \ct \big(\proj{2}(P(fx)) (x,\refl{fx})\big). \qedhere \]
\end{proof}

同样容易看出:

\begin{lem}\label{thm:contr-hprop}
对于任意 $f$,类型 $\iscontr(f)$ 是一个单纯命题。
\end{lem}
\begin{proof}
  根据 \cref{thm:isprop-iscontr},每个类型 $\iscontr (\hfib f y)$ 是一个单纯命题。
  因此,根据 \cref{thm:isprop-forall},\eqref{eq:iscontrf} 也是如此。
\end{proof}

\begin{thm}\label{thm:equiv-contr-hae}
对于任意 $f:A\to B$ 我们有 $\eqv{\iscontr(f)}{\ishae(f)}$。
\end{thm}
\begin{proof}
  我们已经建立了 ${\iscontr(f)} \Leftrightarrow {\ishae(f)}$ 的逻辑等价关系,并且它们都是单纯命题 (\cref{thm:contr-hprop,thm:hae-hprop})。
  因此,\cref{lem:equiv-iff-hprop} 适用。
\end{proof}

通常,我们通过给出一个准逆元来证明一个函数是等价的,但有时这个定义更方便。
例如,它暗示了当证明一个函数是等价的时,我们可以假设它的陪域是非空的。

\begin{cor}\label{thm:equiv-inhabcod}
如果 $f:A\to B$ 使得 $B\to \isequiv(f)$,那么 $f$ 是等价的。
\end{cor}
\begin{proof}
  为了证明 $f$ 是等价的,足以证明对于任意 $y:B$,$\hfib f y$ 是收缩的。
  但是如果 $e:B\to \isequiv(f)$,那么对于任意 $y$,我们有 $e(y):\isequiv(f)$,因此 $f$ 是等价的,并且因此 $\hfib f y$ 是收缩的,如所需。
\end{proof}

\index{function!contractible|)}%
\index{contractible!function|)}%
\index{equivalence!as contractible function|)}%

\section{关于等价定义的思考 (On the definition of equivalences)}
\label{sec:concluding-remarks}

\indexdef{equivalence}
我们已经证明所有三个等价的定义满足三个理想的性质,并且它们是成对等价的:
\[ \iscontr(f) \eqvsym \ishae(f) \eqvsym \biinv(f). \]
(还有更多可能的等价定义,但我们将在此处讨论这三个。
参见 \cref{ex:brck-qinv} 和本章中的习题,以了解更多。)
因此,我们可以选择其中任何一个作为 $\isequiv (f)$ 的“定义”。
为了明确起见,我们选择定义
\[ \isequiv(f) \defeq \ishae(f)。\]
\index{mathematics!formalized}%
这个选择对形式化很有利,因为 $\ishae(f)$ 包含了最直接有用的数据。
另一方面,对于其他目的,$\biinv(f)$ 通常更容易处理,因为它不包含二维路径,并且它的两个对称部分可以独立处理。
然而,对于本书的目的,具体选择几乎没有区别。

在本章的其余部分,我们研究等价的一些其他性质和特征。
\index{equivalence!properties of}%


\section{满射与嵌入 (Surjections and embeddings)}
\label{sec:mono-surj}

\index{set}
当 $A$ 和 $B$ 是集合 (sets) 并且 $f:A\to B$ 是等价的时,我们也称它为 \define{同构 (isomorphism)}
\indexdef{isomorphism!of sets}%
或 \define{双射 (bijection)}。
\indexdef{bijection}%
\indexsee{function!bijective}{bijection}%
(对于不是集合的类型,我们避免使用这些词,因为在同伦理论和高阶范畴论中,它们通常表示比同伦等价更严格的“相同”概念。)
在集合论中,当且仅当一个函数既是单射 (injective) 又是满射 (surjective) 时,它是一个双射。
在类型论中也是如此,如果我们适当地表述这些条件。
为清楚起见,在处理非集合的类型时,我们将使用 \emph{嵌入 (embeddings)} 一词来代替单射。

\begin{defn}\label{defn:surj-emb}
设 $f:A\to B$。
\begin{enumerate}
  \item 我们说 $f$ 是 \define{满射 (surjective)}
  \indexsee{surjective!function}{function, surjective}%
  \indexdef{function!surjective}%
  (或称为 \define{满射 (surjection)})
  \indexsee{surjection}{function, surjective}%
  如果对于每个 $b:B$,我们有 $\brck{\hfib f b}$。
  \item 我们说 $f$ 是一个 \define{嵌入 (embedding)}
  \indexdef{function!embedding}%
  \indexsee{embedding}{function, embedding}%
  如果对于每个 $x,y:A$,函数 $\apfunc f : (\id[A]xy) \to (\id[B]{f(x)}{f(y)})$ 是一个等价。
\end{enumerate}
\end{defn}

换句话说,如果 $f$ 的每个纤维只是居住的,或者等价地,如果对于所有 $b:B$,仅存在一个 $a:A$ 使得 $f(a)=b$,则 $f$ 是满射。
在传统逻辑符号中,如果 $\fall{b:B}\exis{a:A} (f(a)=b)$,则 $f$ 是满射。
这必须与更强的断言 $\prd{b:B}\sm{a:A} (f(a)=b)$ 区分开来;如果这成立,我们说 $f$ 是一个 \define{分裂满射 (split surjection)}。
\indexsee{split!surjection}{function, split surjective}%
\indexsee{surjection!split}{function, split surjective}%
\indexsee{surjective!function!split}{function, split surjective}%
\indexdef{function!split surjective}%
(因为这个后者的类型等价于 $\sm{g:B\to A}\prd{b:B} (f(g(b))=b)$,所以分裂满射与 \cref{sec:contractibility} 中定义的 \emph{再牵引 (retraction)} 是相同的。)
\index{retraction}%
\index{function!retraction}%

\cref{sec:axiom-choice} 中的选择公理 (axiom of choice) 恰好说明了每个 \emph{集合之间的} 满射都是分裂的。
然而,在单一性公理 (univalence axiom) 的存在下,简单地说,\emph{所有} 满射都不是分裂的。
在 \cref{thm:no-higher-ac} 中,我们构建了一个类型族 $Y:X\to \type$,使得 $\prd{x:X} \brck{Y(x)}$ 但 $\neg \prd{x:X} Y(x)$;
对于任何此类族,$Y$ 的第一次投影 $(\sm{x:X} Y(x)) \to X$ 是一个未分裂的满射。

如果 $A$ 和 $B$ 是集合,那么根据 \cref{lem:equiv-iff-hprop},当且仅当
\begin{equation}
  \prd{x,y:A} (\id[B]{f(x)}{f(y)}) \to (\id[A]xy).\label{eq:injective}
\end{equation}
时,$f$ 是嵌入的。
在这种情况下,我们说 $f$ 是 \define{单射 (injective)},
\indexsee{injective function}{function, injective}%
\indexdef{function!injective}%
或 \define{单射 (injection)}。
\indexsee{injection}{function, injective}%
对于非集合的类型,我们避免使用这些词,因为它们可能会被解释为~\eqref{eq:injective},这是对非集合类型不合适的概念。
同样地,任何集合之间的函数如果且仅如果它是一个适当意义上的 \emph{上同态 (epimorphism)},则它是满射,但这对更一般的类型来说也是不成立的,并且满射通常是更重要的概念。

\begin{thm}\label{thm:mono-surj-equiv}
一个函数 $f:A\to B$ 当且仅当它既是满射又是嵌入时是等价的。
\end{thm}
\begin{proof}
  如果 $f$ 是等价的,那么每个 $\hfib f b$ 都是收缩的,因此 $\brck{\hfib f b}$ 也是收缩的,因此 $f$ 是满射的。
  我们在 \cref{thm:paths-respects-equiv} 中证明了任何等价都是嵌入的。

  反之,假设 $f$ 是一个满射嵌入。
  设 $b:B$;我们证明 $\sm{x:A}(f(x)=b)$ 是收缩的。
  因为 $f$ 是满射的,所以仅存在一个 $a:A$ 使得 $f(a)=b$。
  因此,$f$ 在 $b$ 上的纤维是非空的;剩下的是证明它是一个单纯命题。
  为此,假设给定 $x,y:A$ 且 $p:f(x)=b$ 和 $q:f(y)=b$。
  然后由于 $\apfunc f$ 是一个等价的,存在 $r:x=y$ 使得 $\apfunc f (r) = p \ct \opp q$。
  然而,使用在 $\Sigma$-类型中的路径表征,后者等式重新排列为 $\trans{r}{p} = q$。
  因此,连同 $r$ 一起,它展示了在 $f$ 在 $b$ 上的纤维中的 $(x,p) = (y,q)$。
\end{proof}

\begin{cor}
  对于任意 $f:A\to B$ 我们有
  \[ \isequiv(f) \eqvsym (\mathsf{isEmbedding}(f) \times \mathsf{isSurjective}(f))。\]
\end{cor}
\begin{proof}
  作为一个满射和嵌入都是单纯命题;现在应用 \cref{lem:equiv-iff-hprop}。
\end{proof}

当然,这不能用作“等价”的定义,因为嵌入的定义涉及到等价的概念。
然而,这个表征仍然是有用的;参见 \cref{sec:whitehead}。
我们将在 \cref{cha:hlevels} 中对其进行推广。

% \section{Fiberwise equivalences}
\section{等价的闭包性质 (Closure properties of equivalences)}
\label{sec:equiv-closures}
\label{sec:fiberwise-equivalences}
\index{equivalence!properties of}%

我们已经在 \cref{thm:equiv-eqrel} 中看到,等价关系在复合下是封闭的。
此外,我们还有:

\begin{thm}[三选二性质 (The 2-out-of-3 property)]\label{thm:two-out-of-three}
\index{2-out-of-3 property}%
假设 $f:A\to B$ 和 $g:B\to C$。
如果 $f$、$g$ 和 $g\circ f$ 中的任意两个是等价的,那么第三个也是等价的。
\end{thm}
\begin{proof}
  如果 $g\circ f$ 和 $g$ 是等价的,那么 $\opp{(g\circ f)} \circ g$ 是 $f$ 的准逆 (quasi-inverse)。
  一方面,我们有 $\opp{(g\circ f)} \circ g \circ f \htpy \idfunc[A]$,而另一方面我们有
  \begin{align*}
    f \circ \opp{(g\circ f)} \circ g
    &\htpy \opp g \circ g \circ f \circ \opp{(g\circ f)} \circ g\\
    &\htpy \opp g \circ g\\
    &\htpy \idfunc[B].
  \end{align*}
  同样地,如果 $g\circ f$ 和 $f$ 是等价的,那么 $f\circ \opp{(g\circ f)}$ 是 $g$ 的准逆。
\end{proof}

这是同伦理论中关于等价的一个标准闭包条件。
同样众所周知的是,它们在以下意义下对于再牵引 (retracts) 也是封闭的。

\index{retract!of a function|(defstyle)}%

\begin{defn}\label{defn:retract}
若有一个图
\begin{equation*}
  \xymatrix{
      {A} \ar[r]^{s} \ar[d]_{g}
    &
      {X} \ar[r]^{r} \ar[d]_{f}
    &
      {A} \ar[d]^{g}
    \\
    {B} \ar[r]_{s'}
    &
      {Y} \ar[r]_{r'}
    &
      {B}
  }
\end{equation*}
则称函数 $g:A\to B$ 是函数 $f:X\to Y$ 的 \define{再牵引 (retract)},其中存在
\begin{enumerate}
  \item 一个同伦 $R:r\circ s \htpy \idfunc[A]$。
  \item 一个同伦 $R':r'\circ s' \htpy\idfunc[B]$。
  \item 一个同伦 $L:f\circ s\htpy s'\circ g$。
  \item 一个同伦 $K:g\circ r\htpy r'\circ f$。
  \item 对于每个 $a:A$,存在一个路径 $H(a)$ 见证下图的交换性
  \begin{equation*}
    \xymatrix@C=3pc{
        {g(r(s(a)))} \ar@{=}[r]^-{K(s(a))} \ar@{=}[d]_{\ap g{R(a)}}
      &
        {r'(f(s(a)))} \ar@{=}[d]^{\ap{r'}{L(a)}}
      \\
      {g(a)} \ar@{=}[r]_-{\opp{R'(g(a))}}
      &
        {r'(s'(g(a)))}
    }
  \end{equation*}
\end{enumerate}
\end{defn}

回顾一下,在 \cref{sec:contractibility} 中我们定义了类型如何成为另一类型的再牵引。
这是上述定义的一个特例,其中 $B$ 和 $Y$ 是 $\unit$。
相反,正如与可收缩性一样,映射的再牵引会诱导其纤维的再牵引。

\begin{lem}\label{lem:func_retract_to_fiber_retract}
如果函数 $g:A\to B$ 是函数 $f:X\to Y$ 的再牵引,那么对于每个 $b:B$,$\hfib{g}b$ 是 $\hfib{f}{s'(b)}$ 的再牵引,其中 $s':B\to Y$ 如 \cref{defn:retract} 所述。
\end{lem}

\begin{proof}
  假设 $g:A\to B$ 是 $f:X\to Y$ 的再牵引。那么对于任何 $b:B$,我们有函数
  \begin{align*}
    \varphi_b &:\hfiber{g}b\to\hfib{f}{s'(b)}, &
    \varphi_b(a,p) & \defeq \pairr{s(a),L(a)\ct s'(p)},\\
    \psi_b &:\hfib{f}{s'(b)}\to\hfib{g}b, &
    \psi_b(x,q) &\defeq \pairr{r(x),K(x)\ct r'(q)\ct R'(b)}.
  \end{align*}
  那么我们有 $\psi_b(\varphi_b({a,p}))\equiv\pairr{r(s(a)),K(s(a))\ct r'(L(a)\ct s'(p))\ct R'(b)}$。
  我们声称 $\psi_b$ 是一个具有截面的再牵引 $\varphi_b$ 对于所有 $b:B$,也就是说,对于所有 $(a,p):\hfib g b$,我们有 $\psi_b(\varphi_b({a,p}))= \pairr{a,p}$。
  换句话说,我们要证明
  \begin{equation*}
    \prd{b:B}{a:A}{p:g(a)=b} \psi_b(\varphi_b({a,p}))= \pairr{a,p}。
  \end{equation*}
  通过重新排列前两个 $\Pi$ 并应用 \cref{thm:omit-contr} 的一个版本,这等价于
  \begin{equation*}
    \prd{a:A}\psi_{g(a)}(\varphi_{g(a)}({a,\refl{g(a)}}))=\pairr{a,\refl{g(a)}}。
  \end{equation*}
  对于任何 $a$,根据 \cref{thm:path-sigma},这个对偶对等价于一对等式。由于 $R(a):r(s(a))= a$,所以我们只需要证明
  \begin{equation*}
    \trans{R(a)}{K(s(a))\ct r'(L(a))\ct R'(g(a))} = \refl{g(a)}。
  \end{equation*}
  但这种变换计算为 $\opp{g(R(a))}\ct K(s(a))\ct r'(L(a))\ct R'(g(a))$,所以所需的路径由 $H(a)$ 给出。
\end{proof}

\begin{thm}\label{thm:retract-equiv}
如果 $g$ 是等价映射 $f$ 的再牵引,那么 $g$ 也是等价映射。
\end{thm}
\begin{proof}
  根据 \cref{lem:func_retract_to_fiber_retract},$g$ 的每个纤维都是 $f$ 的纤维的再牵引。
  因此,根据 \cref{thm:retract-contr},如果后者都是可收缩的,那么前者也是。
\end{proof}

\index{retract!of a function|)}%

\index{fibration}%
\index{total!space}%
最后,我们展示了纤维等价性 (fiberwise equivalences) 可以通过全空间等价 (equivalences of total spaces) 来表征。
为了解释这个术语,回顾 \cref{sec:fibrations},其中类型族 $P:A\to\type$ 可以看作是 $A$ 上的一个纤维丛,其全空间为 $\sm{x:A} P(x)$,纤维丛的投影为 $\proj1:\sm{x:A} P(x) \to A$。
从这个角度来看,给定两个类型族 $P,Q:A\to\type$,我们可以称函数 $f:\prd{x:A} (P(x)\to Q(x))$ 为 \define{纤维映射 (fiberwise map)} 或 \define{纤维变换 (fiberwise transformation)}。
\indexsee{transformation!fiberwise}{fiberwise transformation}%
\indexsee{function!fiberwise}{fiberwise transformation}%
\index{fiberwise!transformation|(defstyle}%
\indexsee{fiberwise!map}{fiberwise transformation}%
\indexsee{map!fiberwise}{fiberwise transformation}
这样的映射在全空间上诱导一个函数:

\begin{defn}\label{defn:total-map}
给定类型族 $P,Q:A\to\type$ 和一个映射 $f:\prd{x:A} P(x)\to Q(x)$,我们定义
\begin{equation*}
  \total f  \defeq \lam{w}\pairr{\proj{1}w,f(\proj{1}w,\proj{2}w)} : \sm{x:A}P(x)\to\sm{x:A}Q(x)。
\end{equation*}
\end{defn}

\begin{thm}\label{fibwise-fiber-total-fiber-equiv}
假设 $f$ 是类型 $A$ 上族 $P$ 和 $Q$ 之间的一个纤维变换,并设 $x:A$ 和 $v:Q(x)$。那么我们有一个等价关系
\begin{equation*}
  \eqv{\hfib{\total{f}}{\pairr{x,v}}}{\hfib{f(x)}{v}}。
\end{equation*}
\end{thm}
\begin{proof}
  我们计算如下:
  \begin{align}
    \hfib{\total{f}}{\pairr{x,v}}
    & \jdeq \sm{w:\sm{x:A}P(x)}\pairr{\proj{1}w,f(\proj{1}w,\proj{2}w)}=\pairr{x,v}。
    \notag \\
    & \eqv{}{} \sm{a:A}{u:P(a)}\pairr{a,f(a,u)}=\pairr{x,v}。
    \tag{by~\cref{ex:sigma-assoc}} \\
    & \eqv{}{} \sm{a:A}{u:P(a)}{p:a=x}\trans{p}{f(a,u)}=v。
    \tag{by \cref{thm:path-sigma}} \\
    & \eqv{}{} \sm{a:A}{p:a=x}{u:P(a)}\trans{p}{f(a,u)}=v。
    \notag \\
    & \eqv{}{} \sm{u:P(x)}f(x,u)=v。
    \tag{$*$}\label{eq:uses-sum-over-paths} \\
    & \jdeq \hfib{f(x)}{v}。 \notag
  \end{align}
  等式~\eqref{eq:uses-sum-over-paths} 由 \cref{thm:omit-contr,thm:contr-paths,ex:sigma-assoc} 得出。
\end{proof}

我们称纤维变换 $f:\prd{x:A} P(x)\to Q(x)$ 为 \define{纤维等价 (fiberwise equivalence)}%
\indexdef{fiberwise!equivalence}%
\indexdef{equivalence!fiberwise}%
如果每个 $f(x):P(x) \to Q(x)$ 是一个等价。

\begin{thm}\label{thm:total-fiber-equiv}
假设 $f$ 是类型 $A$ 上族 $P$ 和 $Q$ 之间的一个纤维变换。
则 $f$ 是纤维等价当且仅当 $\total{f}$ 是等价映射。
\end{thm}

\begin{proof}
  设 $f$、$P$、$Q$ 和 $A$ 如定理所述。
  根据 \cref{fibwise-fiber-total-fiber-equiv},对所有 $x:A$ 和 $v:Q(x)$,$\hfib{\total{f}}{\pairr{x,v}}$ 可收缩当且仅当 $\hfib{f(x)}{v}$ 可收缩。
  因此,$\hfib{\total{f}}{w}$ 对所有 $w:\sm{x:A}Q(x)$ 可收缩当且仅当 $\hfib{f(x)}{v}$ 对所有 $x:A$ 和 $v:Q(x)$ 可收缩。
\end{proof}

\index{fiberwise!transformation|)}%

\section{对象分类器 (The object classifier)}
\label{sec:object-classification}

在类型论中,我们有一个基本的“类型族”的概念,即函数 $B:A\to\type$。
我们已经看到,这样的族在某种程度上类似于同伦理论中的纤维丛 (fibration),纤维丛为投影 $\proj1:\sm{a:A} B(a) \to A$。
同伦理论中的一个基本事实是每个映射都等价于一个纤维丛。
借助单值化 (univalence),我们可以在类型论中证明同样的事情。

\begin{lem}\label{thm:fiber-of-a-fibration}
对于任意类型族 $B:A\to\type$,$\proj1:\sm{x:A} B(x) \to A$ 在 $a:A$ 处的纤维等价于 $B(a)$:
\[ \eqv{\hfib{\proj1}{a}}{B(a)} \]
\end{lem}
\begin{proof}
  我们有
  \begin{align*}
    \hfib{\proj1}{a} &\defeq \sm{u:\sm{x:A} B(x)} \proj1(u)=a\\
    &\eqvsym \sm{x:A}{b:B(x)} (x=a)\\
    &\eqvsym \sm{x:A}{p:x=a} B(x)\\
    &\eqvsym B(a)
  \end{align*}
  使用恒等类型的左泛性质 (left universal property)。
\end{proof}

\begin{lem}\label{thm:total-space-of-the-fibers}
对于任意函数 $f:A\to B$,我们有 $\eqv{A}{\sm{b:B}\hfib{f}{b}}$。
\end{lem}
\begin{proof}
  我们有
  \begin{align*}
    \sm{b:B}\hfib{f}{b} &\defeq \sm{b:B}{a:A} (f(a)=b)\\
    &\eqvsym \sm{a:A}{b:B} (f(a)=b)\\
    &\eqvsym A
  \end{align*}
  利用 $\sm{b:B} (f(a)=b)$ 是可收缩的事实。
\end{proof}

\begin{thm}\label{thm:nobject-classifier-appetizer}
对于任意类型 $B$,存在一个等价
\begin{equation*}
  \chi:\Parens{\sm{A:\type} (A\to B)}\eqvsym (B\to\type)。
\end{equation*}
\end{thm}
\begin{proof}
  我们需要构造准逆
  \begin{align*}
    \chi & : \Parens{\sm{A:\type} (A\to B)}\to B\to\type\\
    \psi & : (B\to\type)\to\Parens{\sm{A:\type} (A\to B)}。
  \end{align*}
  我们定义 $\chi$ 为 $\chi((A,f),b)\defeq\hfiber{f}b$,定义 $\psi$ 为 $\psi(P)\defeq\Pairr{(\sm{b:B} P(b)),\proj1}$。
  现在我们需要验证 $\chi\circ\psi\htpy\idfunc{}$ 和 $\psi\circ\chi \htpy\idfunc{}$。
  \begin{enumerate}
    \item 设 $P:B\to\type$。
    通过 \cref{thm:fiber-of-a-fibration},
    $\hfiber{\proj1}{b}\eqvsym P(b)$ 对于任意 $b:B$,所以这立即得出
    $P\htpy\chi(\psi(P))$。
    \item 设 $f:A\to B$ 是一个函数。我们需要找到一个路径
    \begin{equation*}
      \Pairr{\tsm{b:B} \hfiber{f}b,\,\proj1}=\pairr{A,f}。
    \end{equation*}
    首先注意到,通过 \cref{thm:total-space-of-the-fibers},我们有
    $e:\sm{b:B} \hfiber{f}b\eqvsym A$,定义为 $e(b,a,p)\defeq a$ 和 $e^{-1}(a)
    \defeq(f(a),a,\refl{f(a)})$。
    根据 \cref{thm:path-sigma},剩下的部分是要证明 $\trans{(\ua(e))}{\proj1} = f$。
    但通过单值化的计算规则和~\eqref{eq:transport-arrow},我们有 $\trans{(\ua(e))}{\proj1} = \proj1\circ e^{-1}$,并且 $e^{-1}$ 的定义立即得出 $\proj1 \circ e^{-1} \jdeq f$。\qedhere
  \end{enumerate}
\end{proof}

\noindent
\indexdef{object!classifier}%
\indexdef{classifier!object}%
\index{.infinity1-topos@$(\infty,1)$-topos}%
特别地,这意味着我们在更高拓扑论 (higher topos theory) 的意义上有一个对象分类器 (object classifier)。
回顾 \cref{def:pointedtype} 中 $\pointed\type$ 表示带指点类型的类型 $\sm{A:\type} A$。

\begin{thm}\label{thm:object-classifier}
设 $f:A\to B$ 是一个函数。则图
\begin{equation*}
  \vcenter{\xymatrix{
    A\ar[r]^-{\vartheta_f} \ar[d]_{f} &
    \pointed{\type}\ar[d]^{\proj1}\\
    B\ar[r]_{\chi_f} &
    \type
  }}
\end{equation*}
是一个拉回方块 (pullback square)(见 \cref{ex:pullback})。
其中函数 $\vartheta_f$ 定义为
\begin{equation*}
  \lam{a} \pairr{\hfiber{f}{f(a)},\pairr{a,\refl{f(a)}}}。
\end{equation*}
\end{thm}
\begin{proof}
  注意我们有等价关系
  \begin{align*}
    A & \eqvsym \sm{b:B} \hfiber{f}b\\
    & \eqvsym \sm{b:B}{X:\type}{p:\hfiber{f}b= X} X\\
    & \eqvsym \sm{b:B}{X:\type}{x:X} \hfiber{f}b= X\\
    & \eqvsym \sm{b:B}{Y:\pointed{\type}} \hfiber{f}b = \proj1 Y\\
    & \jdeq B\times_{\type}\pointed{\type}
  \end{align*}
  这给我们一个复合等价 $e:A\eqvsym B\times_\type\pointed{\type}$。
  我们可以通过以下步骤逐步显示这个复合等价的操作:
  \begin{align*}
    a & \mapsto \pairr{f(a),\; \pairr{a,\refl{f(a)}}}\\
    & \mapsto \pairr{f(a), \; \hfiber{f}{f(a)}, \; \refl{\hfiber{f}{f(a)}}, \; \pairr{a,\refl{f(a)}}}\\
    & \mapsto \pairr{f(a), \; \hfiber{f}{f(a)}, \; \pairr{a,\refl{f(a)}}, \; \refl{\hfiber{f}{f(a)}}}。
  \end{align*}
  因此,我们得到同伦 $f\htpy\proj1\circ e$ 和 $\vartheta_f\htpy \proj2\circ e$。
\end{proof}


\section{一致性 (Univalence) 蕴含函数外延性 (Function Extensionality)}
\label{sec:univalence-implies-funext}

\index{函数外延性 (function extensionality)!来源于一致性的证明 (proof from univalence)}%
在本章的最后一节中,我们将证明一致性公理 (univalence axiom) 蕴含了函数外延性 (function extensionality)。因此,在本节中我们\emph{不}使用函数外延性公理。该证明分为两步。首先,我们在 \cref{uatowfe} 中展示了一致性公理 (univalence) 蕴含了函数外延性的弱形式,定义见下文 \cref{weakfunext}。然后,弱函数外延性原则 (weak function extensionality) 又蕴含了通常的函数外延性,而这一过程不依赖于一致性公理 (\cref{wfetofe})。

\index{一致性公理 (univalence axiom)}%
令 $\type$ 为一个宇宙;我们将在适当的位置明确指出何时假设其为一致的。

\begin{defn}\label{weakfunext}
\define{弱函数外延性原则 (weak function extensionality principle)}
\indexdef{函数外延性 (function extensionality)!弱 (weak)}%
断言存在一个函数
\begin{equation*}
  \Parens{\prd{x:A}\iscontr(P(x))} \to\iscontr\Parens{\prd{x:A}P(x)}
\end{equation*}
对于任意类型 $A$ 上的类型族 $P:A\to\type$。
\end{defn}

使用函数外延性可以轻松证明以下引理;关键在于它也可以在不假设函数外延性的情况下从一致性得到。

\begin{lem} \label{UA-eqv-hom-eqv}
假设 $\type$ 是一致的,对于任意的 $A,B,X:\type$ 和任意的 $e:\eqv{A}{B}$,存在一个等价性
\begin{equation*}
  \eqv{(X\to A)}{(X\to B)}
\end{equation*}
其底层映射由 $e$ 的底层函数通过后合成给出。
\end{lem}

\begin{proof}
  % 通过对 $\eqv{}{}$ 的归纳立即可得 (参见 \cref{thm:equiv-induction})。
  如 \cref{lem:qinv-autohtpy} 的证明中,我们可以假设 $e = \idtoeqv(p)$,其中 $p:A=B$。
  然后通过路径归纳法 (path induction),我们可以假设 $p$ 为 $\refl{A}$,因此 $e = \idfunc[A]$。
  但在这种情况下,后合成 $e$ 是恒等映射,因此是等价的。
\end{proof}

\begin{cor}\label{contrfamtotalpostcompequiv}
设 $P:A\to\type$ 是一族可收缩类型 (contractible types),即 $\narrowequation{\prd{x:A}\iscontr(P(x))}$。
那么投影 $\proj{1}:(\sm{x:A}P(x))\to A$ 是一个等价性。如果假设 $\type$ 是一致的,那么通过后合成 $\proj{1}$ 立即得到一个等价性
\begin{equation*}
  \alpha : \eqv{\Parens{A\to\sm{x:A}P(x)}}{(A\to A)}。
\end{equation*}
\end{cor}

\begin{proof}
  根据 \cref{thm:fiber-of-a-fibration},对于 $\proj{1}:\sm{x:A}P(X)\to A$ 和 $x:A$,我们有一个等价性
  \begin{equation*}
    \eqv{\hfiber{\proj{1}}{x}}{P(x)}。
  \end{equation*}
  因此,当每个 $P(x)$ 是可收缩时,$\proj{1}$ 是一个等价性。这个断言现在是 \cref{UA-eqv-hom-eqv} 的一个推论。
\end{proof}

特别地,上述等价性在 $\idfunc[A]$ 处的同伦纤维是可收缩的。因此,我们可以通过展示依赖函数类型 $\prd{x:A}P(x)$ 是 $\hfiber{\alpha}{\idfunc[A]}$ 的一个收缩来证明一致性蕴含了弱函数外延性。

\begin{thm}\label{uatowfe}
在一个一致的宇宙 $\type$ 中,假设 $P:A\to\type$ 是一族可收缩类型,并且令 $\alpha$ 为 \cref{contrfamtotalpostcompequiv} 中的函数。
那么 $\prd{x:A}P(x)$ 是 $\hfiber{\alpha}{\idfunc[A]}$ 的一个收缩。由此可见,$\prd{x:A}P(x)$ 是可收缩的。换句话说,一致性公理蕴含了弱函数外延性原则。
\end{thm}

\begin{proof}
  定义函数
  \begin{align*}
    \varphi &: (\tprd{x:A}P(x))\to\hfiber{\alpha}{\idfunc[A]},\\
    \varphi(f) &\defeq (\lam{x} (x,f(x)),\refl{\idfunc[A]}),
    \intertext{以及}
    \psi &: \hfiber{\alpha}{\idfunc[A]}\to \tprd{x:A}P(x),\\
    \psi(g,p) &\defeq \lam{x} \trans {\happly (p,x)}{\proj{2} (g(x))}。
  \end{align*}
  那么 $\psi(\varphi(f))=\lam{x} f(x)$,即 $f$,根据依赖函数类型的唯一性原理 (uniqueness principle)。
\end{proof}

现在我们展示弱函数外延性蕴含了通常的函数外延性。
回忆自~\eqref{eq:happly} 的函数 $\happly (f,g) : (f = g)\to(f\htpy g)$,其将函数的相等性转化为同伦。在接下来的证明中,未使用一致性公理。

\begin{thm}\label{wfetofe}
\index{函数外延性 (function extensionality)}%
弱函数外延性蕴含函数外延性 \cref{axiom:funext}。
\end{thm}

\begin{proof}
  我们想要证明
  \begin{equation*}
    \prd{A:\type}{P:A\to\type}{f,g:\prd{x:A}P(x)}\isequiv(\happly (f,g))。
  \end{equation*}
  由于纤维映射 (fiberwise map) 当且仅当它是纤维的等价性时会在整个空间上诱导出等价性 (\cref{thm:total-fiber-equiv}),我们只需展示以下类型的函数
  \begin{equation*}
    \Parens{\sm{g:\prd{x:A}P(x)}(f= g)} \to \sm{g:\prd{x:A}P(x)}(f\htpy g)
  \end{equation*}
  由 $\lam{g:\prd{x:A}P(x)} \happly (f,g)$ 诱导出的函数是一个等价性。
  由于左侧的类型是可收缩的 (\cref{thm:contr-paths}),我们只需证明右侧的类型:
  \begin{equation}\label{eq:uatofesp}
  \sm{g:\prd{x:A}P(x)}\prd{x:A}f(x)= g(x)
  \end{equation}
  是可收缩的。
  现在 \cref{thm:ttac} 表明这与以下类型等价:
  \begin{equation}\label{eq:uatofeps}
  \prd{x:A}\sm{u:P(x)}f(x)= u。
  \end{equation}
  \cref{thm:ttac} 的证明使用了函数外延性,但仅对其中一个复合函数使用。因此,在不假设函数外延性的情况下,我们可以得出~\eqref{eq:uatofesp} 是~\eqref{eq:uatofeps} 的一个收缩。并且~\eqref{eq:uatofeps} 是一个可收缩类型的乘积,由弱函数外延性原则可知,它是可收缩的;因此~\eqref{eq:uatofesp} 也是可收缩的。
\end{proof}

\sectionNotes

带有拟逆的连续映射空间具有错误的同伦类型,而不是“同伦等价空间”的事实在代数拓扑学中是众所周知的。在这种情况下,“同伦等价空间”$(\eqv AB)$通常仅定义为由同伦等价构成的函数空间 $(A\to B)$ 的子空间。在类型论中,这将最接近于 $\sm{f:A\to B} \brck{\qinv(f)}$;见 \cref{ex:brck-qinv}。

在同伦类型论 (Homotopy Type Theory) 中,第一个等价性定义是由 Voevodsky 提出的,即我们称为 $\iscontr(f)$ 的定义。其他定义的可能性后来被不同的人注意到。关于伴随等价性 (adjoint equivalences) 的基本定理,例如 \cref{lem:coh-equiv,thm:equiv-iso-adj} 是从高等范畴论 (higher category theory) 和同伦论 (homotopy theory) 中的标准事实改编而来的。使用双可逆性 (bi-invertibility) 作为等价性定义是由 Andr\'e Joyal 提出的。

在 \cref{sec:mono-surj,sec:equiv-closures} 中讨论的等价性性质在同伦论中是众所周知的。它们中的大多数首次在类型论中被 Voevodsky 证明。

每个函数等价于纤维化 (fibration) 是同伦论中的标准事实。对象分类器 (object classifier) 的概念
\index{对象分类器 (object classifier)}%
\index{分类器 (classifier)!对象 (object)}%
在 $(\infty,1)$-范畴 (category)
\index{.infinity1-category@$(\infty,1)$-范畴 (category)}%
理论中(\cref{thm:nobject-classifier-appetizer} 的范畴对应物)是 Rezk 提出的 (参见~\cite{Rezk05,lurie:higher-topoi})。

最后,一致性蕴含函数外延性 (\cref{sec:univalence-implies-funext}) 的事实归功于 Voevodsky。我们的证明是他证明的简化版。\cref{ex:funext-from-nondep} 也归功于 Voevodsky。

\sectionExercises

\begin{ex}\label{ex:two-sided-adjoint-equivalences}
考虑 $f:A\to B$ 的“两侧伴随等价性 (two-sided adjoint equivalence) 数据”类型,
\begin{narrowmultline*}
  \sm{g:B\to A}{\eta: g \circ f \htpy \idfunc[A]}{\epsilon:f \circ g \htpy \idfunc[B]}
  \narrowbreak
  \Parens{\prd{x:A} \map{f}{\eta x} = \epsilon(fx)} \times
  \Parens{\prd{y:B} \map{g}{\epsilon y} = \eta(gy) }。
\end{narrowmultline*}
根据 \cref{lem:coh-equiv},我们知道如果 $f$ 是等价性,那么该类型是可居住的。给出一个与 \cref{lem:qinv-autohtpy} 类似的该类型的表征。

你能给出一个例子来展示这个类型通常不是一个单纯命题 (mere proposition) 吗?
(在学习 \cref{cha:hits} 后,这将变得更容易。)
\end{ex}

\begin{ex}\label{ex:symmetric-equiv}
证明对于任意 $A,B:\UU$,以下类型等价于 $\eqv A B$。
\begin{equation*}
  \sm{R:A\to B\to \type}
  \Parens{\prd{a:A} \iscontr\Parens{\sm{b:B} R(a,b)}} \times
  \Parens{\prd{b:B} \iscontr\Parens{\sm{a:A} R(a,b)}}。
\end{equation*}
你能从中提取出一个满足 $\isequiv(f)$ 的三个要求的类型定义吗?
\end{ex}

\begin{ex} \label{ex:qinv-autohtpy-no-univalence}
在不使用一致性的情况下重新表述 \cref{lem:qinv-autohtpy} 的证明。
\end{ex}

\begin{ex}[不稳定八面体公理 (The unstable octahedral axiom)]\label{ex:unstable-octahedron}
\index{公理 (axiom)!不稳定八面体 (unstable octahedral)}%
\index{不稳定八面体公理 (octahedral axiom, unstable)}%
假设 $f:A\to B$ 和 $g:B\to C$ 以及 $b:B$。
\begin{enumerate}
  \item 证明存在一个自然映射 $\hfib{g\circ f}{g(b)} \to \hfib{g}{g(b)}$,其纤维在 $(b,\refl{g(b)})$ 处等价于 $\hfib f b$。
  \item 证明 $\eqv{\hfib{g\circ f}{c}}{\sm{w:\hfib{g}{c}} \hfib f {\proj1 w}}$。
\end{enumerate}
\end{ex}

\begin{ex}\label{ex:2-out-of-6}
\index{2-out-of-6 性质 (property)}%
证明等价性满足 \emph{2-out-of-6 性质}:给定 $f:A\to B$ 和 $g:B\to C$ 以及 $h:C\to D$,如果 $g\circ f$ 和 $h\circ g$ 是等价的,那么 $f$, $g$, $h$ 和 $h\circ g\circ f$ 也是等价的。
使用此结果来给出 \cref{thm:paths-respects-equiv} 的更高级别的证明。
\end{ex}

\begin{ex}\label{ex:qinv-univalence}
对于 $A,B:\UU$,定义
\[ \mathsf{idtoqinv}_{A,B} :(A=B) \to \sm{f:A\to B}\qinv(f) \]
通过路径归纳法 (path induction) 以显而易见的方式定义。
令 \textbf{\textsf{qinv}-一致性 (univalence)} 表示一致性公理的修改形式,该公理断言对于所有 $A,B:\UU$,函数 $\mathsf{idtoqinv}_{A,B}$ 具有一个拟逆。
\begin{enumerate}
  \item 证明在 \cref{sec:univalence-implies-funext} 中,可以用 \qinv-一致性 代替一致性来证明函数外延性。
  \item 证明在 \cref{thm:qinv-notprop} 中,可以用 \qinv-一致性 代替一致性。
  \item 证明 \qinv-一致性 是不一致的 (inconsistent)(即允许构造 $\emptyt$ 的一个居住者)。因此,在一致性声明中使用“良好”版本的 $\isequiv$ 是至关重要的。
\end{enumerate}
\end{ex}

\begin{ex}\label{ex:embedding-cancellable}
证明当且仅当以下两个条件成立时,函数 $f:A\to B$ 是一个嵌入 (embedding):
\begin{enumerate}
  \item $f$ 是 \emph{左可消的 (left cancellable)},即对于任意 $x,y:A$,如果 $f(x)=f(y)$,那么 $x=y$。\label{item:ex:ec1}
  \item 对于任意 $x:A$,映射 $\apfunc f: \Omega(A,x) \to \Omega(B,f(x))$ 是一个等价性。\label{item:ex:ec2}
\end{enumerate}
(特别地,如果 $A$ 是一个集合,那么 $f$ 是一个嵌入,如果且仅当它是左可消的,并且对于所有 $x:A$,$\Omega(B,f(x))$ 是可收缩的。)
给出例子证明 \ref{item:ex:ec1} 和 \ref{item:ex:ec2} 中的任一项都不能推出另一项。
\end{ex}

\begin{ex}\label{ex:cancellable-from-bool}
证明左可消函数 $\bool\to B$ 的类型(参见 \cref{ex:embedding-cancellable})等价于 $\sm{x,y:B}(x\neq y)$。
给出一个类似的显式表征 $\bool\to B$ 的嵌入类型。
\end{ex}

\begin{ex}\label{ex:funext-from-nondep}
\textbf{天真的非依赖函数外延性公理 (na\"{i}ve non-dependent function extensionality axiom)} 断言,对于 $A,B:\type$ 和 $f,g:A\to B$,存在一个函数 $(\prd{x:A} f(x)=g(x)) \to (f=g)$。
\indexdef{函数外延性 (function extensionality)!非依赖 (non-dependent)}%
修改 \cref{sec:univalence-implies-funext} 中的论证,证明该公理蕴含了完整的函数外延性公理 (\cref{axiom:funext})。
\end{ex}

% Local Variables:
% TeX-master: "hott-online"
% End:
