\chapter{等价性 (Equivalences)}
\label{cha:equivalences}

我们现在详细研究在\cref{sec:basics-equivalences}中简要介绍的\emph{类型的等价性 (equivalence of types)}概念。具体来说,我们将给出几种不同的方式来定义具有前述特性的一种类型 $\isequiv(f)$。回想一下,我们希望 $\isequiv(f)$ 具有以下性质,在此处我们再次陈述这些性质:
\begin{enumerate}
  \item $\qinv(f) \to \isequiv (f)$。\label{item:beb1}
  \item $\isequiv (f) \to \qinv(f)$。\label{item:beb2}
  \item $\isequiv(f)$ 是一个纯命题 (mere proposition)。\label{item:beb3}
\end{enumerate}
这里 $\qinv(f)$ 表示 $f$ 的准逆 (quasi-inverse) 类型:
\begin{equation*}
  \sm{g:B\to A} \big((f \circ g \htpy \idfunc[B]) \times (g\circ f \htpy \idfunc[A])\big)。
\end{equation*}
根据函数外延性 (function extensionality),这意味着 $\qinv(f)$ 等价于类型
\begin{equation*}
  \sm{g:B\to A} \big((f \circ g = \idfunc[B]) \times (g\circ f = \idfunc[A])\big)。
\end{equation*}
我们将定义三种具有性质~\ref{item:beb1}--\ref{item:beb3} 的类型,分别称为:
\begin{itemize}
  \item 半伴随等价 (half adjoint equivalences),
  \item 双可逆映射 (bi-invertible maps),
  \index{function!bi-invertible}
  和
  \item 可缩函数 (contractible functions)。
\end{itemize}
我们还将证明这些类型是等价的。这些名称特意显得有些繁琐,因为在我们知道它们都是等价的并具有性质~\ref{item:beb1}--\ref{item:beb3}之后,我们会简单地使用“等价性 (equivalence)”这个词,而不需要指定我们选择了哪种特定定义。但为了本章中的比较目的,我们需要为每个定义提供不同的名称。

在我们研究等价性的不同概念之前,我们先稍微解释一下为什么需要一个不同于准可逆性的概念。

\section{准逆 (Quasi-inverses)}
\label{sec:quasi-inverses}

\index{quasi-inverse|(}%
我们已经提到 $\qinv(f)$ 是不令人满意的,因为它不是一个纯命题,而我们希望一个给定的函数最多以一种方式“成为等价的”。然而,我们还没有证明 $\qinv(f)$ 不是一个纯命题。在本节中,我们展示一个具体的反例。

\begin{lem}\label{lem:qinv-autohtpy}
如果 $f:A\to B$ 是使得 $\qinv (f)$ 可居住的 (inhabited),那么
\[\eqv{\qinv(f)}{\Parens{\prd{x:A}(x=x)}}。\]
\end{lem}
\begin{proof}
  根据假设,$f$ 是一个等价性,也就是说我们有 $e:\isequiv(f)$,因此 $(f,e):\eqv A B$。根据单值性 (univalence),$\idtoeqv:(A=B) \to (\eqv A B)$ 是一个等价性,因此我们可以假设 $(f,e)$ 的形式为 $\idtoeqv(p)$,其中 $p:A=B$。然后根据路径归纳 (path induction),我们可以假设 $p$ 是 $\refl{A}$,在这种情况下,$f$ 是 $\idfunc[A]$。因此,我们简化为证明 $\eqv{\qinv(\idfunc[A])}{(\prd{x:A}(x=x))}$。现在,根据定义,我们有
  \[ \qinv(\idfunc[A]) \jdeq
  \sm{g:A\to A} \big((g \htpy \idfunc[A]) \times (g \htpy \idfunc[A])\big)。
  \]
  根据函数外延性,这等价于
  \[ \sm{g:A\to A} \big((g = \idfunc[A]) \times (g = \idfunc[A])\big)。
  \]
  根据 \cref{ex:sigma-assoc},这等价于
  \[ \sm{h:\sm{g:A\to A} (g = \idfunc[A])} (\proj1(h) = \idfunc[A])。
  \]
  然而,根据 \cref{thm:contr-paths},$\sm{g:A\to A} (g = \idfunc[A])$ 是以 $(\idfunc[A],\refl{\idfunc[A]})$ 为中心的可缩 (contractible) 类型;因此根据 \cref{thm:omit-contr},这个类型等价于 $\idfunc[A] = \idfunc[A]$。根据函数外延性,$\idfunc[A] = \idfunc[A]$ 等价于 $\prd{x:A} x=x$。
\end{proof}

\noindent
我们注意到 \cref{ex:qinv-autohtpy-no-univalence} 要求在避免单值性的情况下证明上述引理。

因此,我们需要一些 $A$,它允许 $\prd{x:A}(x=x)$ 的非平凡 (nontrivial) 元素。将 $A$ 视为一个更高阶的群胚 (higher groupoid),$\prd{x:A}(x=x)$ 的一个居留元素 (inhabitant) 是从 $A$ 的恒等函子 (identity functor) 到其自身的自然变换 (natural transformation)。这样的变换被称为一个范畴的\define{中心 (center of a category)},\index{center!of a category}%
\index{category!center of}%
因为自然性公理要求它们与所有态射 (morphisms) 交换。传统上,如果 $A$ 只是一个被视为单对象群胚 (one-object groupoid) 的群体,那么这将恰好得出通常群论意义上的中心。这为以下内容提供了一些动机。

\begin{lem}\label{lem:autohtpy}
假设我们有一个类型 $A$ 和 $a:A$ 以及 $q:a=a$,使得
\begin{enumerate}
  \item 类型 $a=a$ 是一个集合 (set)。\label{item:autohtpy1}
  \item 对于所有 $x:A$,我们有 $\brck{a=x}$。\label{item:autohtpy2}
  \item 对于所有 $p:a=a$,我们有 $p\ct q = q \ct p$。\label{item:autohtpy3}
\end{enumerate}
那么存在 $f:\prd{x:A} (x=x)$,且有 $f(a)=q$。
\end{lem}
\begin{proof}
  令 $g:\prd{x:A} \brck{a=x}$ 为~\ref{item:autohtpy2}给出的值。首先我们
  观察到每个类型 $\id[A]xy$ 是一个集合。因为作为一个集合是一个纯命题,我们可以应用命题截断 (propositional truncation) 的归纳原则,假设 $g(x)=\bproj
  p$ 且 $g(y)=\bproj{p'}$,其中 $p:a=x$ 和 $p':a=y$。在这种情况下,与
  $p$ 和 $\opp{p'}$ 组成的映射等价于 $\eqv{(x=y)}{(a=a)}$。但由于~\ref{item:autohtpy1} 中 $(a=a)$ 是一个集合,因此 $(x=y)$ 也是一个集合。

  现在,我们希望通过为每个 $x$ 分配路径 $\opp{g(x)}
  \ct q \ct g(x)$ 来定义 $f$,但这不奏效,因为 $g(x)$ 不属于 $a=x$,而是 $\brck{a=x}$,并且类型 $(x=x)$ 可能不是一个纯命题,因此我们不能使用命题截断的归纳法。相反,我们可以应用 \cref{sec:unique-choice} 中提到的技巧:我们唯一地刻画我们希望构造的对象。让我们定义,对于每个 $x:A$,类型
  \[ B(x) \defeq \sm{r:x=x} \prd{s:a=x} (r = \opp s \ct q\ct s)。\]
  我们声称每个 $B(x)$ 是一个纯命题。
  由于这个声明本身是一个纯命题,我们可以再次应用命题截断的归纳法,并假设 $g(x) = \bproj p$,其中 $p:a=x$。
  现在假设给定 $(r,h)$ 和 $(r',h')$ 在 $B(x)$ 中;然后我们有
  \[ h(p) \ct \opp{h'(p)} : r = r'。\]
  剩下的是展示,当沿着这个等式传递时,$h$ 与 $h'$ 相同,这通过在恒等类型和函数类型中传输 (\cref{sec:compute-paths,sec:compute-pi}),减少为展示
  \[ h(s) = h(p) \ct \opp{h'(p)} \ct h'(s) \]
  对于任何 $s:a=x$。
  但这两侧都是 $(x=x)$ 的元素之间的等式,因此它遵循我们之前的观察,即 $(x=x)$ 是一个集合。

  因此,每个 $B(x)$ 是一个纯命题;我们声称 $\prd{x:A} B(x)$。
  给定 $x:A$,我们现在可以调用命题截断的归纳法,假设 $g(x) = \bproj p$,其中 $p:a=x$。
  我们定义 $r \defeq \opp p \ct q \ct p$;为了填充 $B(x)$,剩下的是展示对于任何 $s:a=x$ 我们有
  $r = \opp s \ct q \ct s$。
  操作路径,这减少为展示 $q\ct (p\ct \opp s) = (p\ct \opp s) \ct q$。
  但这只是~\ref{item:autohtpy3} 的一个实例。
\end{proof}

\begin{thm}\label{thm:qinv-notprop}
存在类型 $A$ 和 $B$ 以及一个函数 $f:A\to B$ 使得 $\qinv(f)$ 不是一个纯命题。
\end{thm}
\begin{proof}
  这足以展示一个类型 $X$,使得 $\prd{x:X} (x=x)$ 不是一个纯命题。
  定义 $X\defeq \sm{A:\type} \brck{\bool=A}$,如在 \cref{thm:no-higher-ac} 的证明中所述。
  展示一个 $f:\prd{x:X} (x=x)$,它不同于 $\lam{x} \refl{x}$ 即可。

  令 $a \defeq (\bool,\bproj{\refl{\bool}}) : X$,并令 $q:a=a$ 是对应于非恒等的等价性 $e:\eqv\bool\bool$ 的路径,其定义为 $e(\bfalse)\defeq\btrue$ 和 $e(\btrue)\defeq\bfalse$。
  我们希望应用 \cref{lem:autohtpy} 来构建一个 $f$。
  根据 $X$ 的定义,子集类型中的等式 (\cref{subsec:prop-subsets}) 和单值性,我们有 $\eqv{(a=a)}{(\eqv{\bool}{\bool})}$,这是一个集合,因此~\ref{item:autohtpy1} 成立。
  类似地,根据 $X$ 的定义和子集类型中的等式,我们有~\ref{item:autohtpy2}。
  最后,\cref{ex:eqvboolbool} 表明每个等价性 $\eqv\bool\bool$ 都等于 $\idfunc[\bool]$ 或 $e$,因此我们可以通过四种情况分析展示~\ref{item:autohtpy3}。

  因此,我们有 $f:\prd{x:X} (x=x)$,且有 $f(a) = q$。
  由于 $e$ 不等于 $\idfunc[\bool]$,$q$ 不等于 $\refl{a}$,因此 $f$ 不等于 $\lam{x} \refl{x}$。
  因此,$\prd{x:X} (x=x)$ 不是一个纯命题。
\end{proof}

更普遍地,\cref{lem:autohtpy} 表明任何“Eilenberg--Mac Lane 空间 (Eilenberg--Mac Lane space)” $K(G,1)$,其中 $G$ 是一个非平凡的阿贝尔 (abelian) 群体,将提供一个反例;参见 \cref{cha:homotopy}。我们使用的类型 $X$ 结果是等价于 $K(\mathbb{Z}_2,1)$。在 \cref{cha:hits} 中,我们将看到圆 $S^1 = K(\mathbb{Z},1)$ 是另一个易于描述的例子。

我们现在继续描述更好的等价性概念。

\index{quasi-inverse|)}%

%%%%%%%%%%%%%%%%%%%%%%%%%%%%%%%%%%%%%%
\section{半伴随等价 (Half adjoint equivalences)}
\label{sec:hae}
%%%%%%%%%%%%%%%%%%%%%%%%%%%%%%%%%%%%%%

\index{equivalence!half adjoint|(defstyle}%
\index{half adjoint equivalence|(defstyle}%
\index{adjoint!equivalence!of types, half|(defstyle}%

在 \cref{sec:quasi-inverses} 中,我们得出结论认为 $\qinv(f)$ 通过丢弃一个可缩类型等价于 $\prd{x:A} (x=x)$。粗略地说,类型 $\qinv(f)$ 包含三个数据 $g$、$\eta$ 和 $\epsilon$,其中两个 ($g$ 和 $\eta$) 可以一起在 $f$ 是等价性时被视为可缩的。问题在于去掉这些数据后还剩下一个 ($\epsilon$)。为了解决这个问题,想法是增加一个\emph{额外的}数据,这样 $\epsilon$ 和它一起构成一个可缩类型。

\begin{defn}\label{defn:ishae}
一个函数 $f:A\to B$ 是一个\define{半伴随等价 (half adjoint equivalence)},
如果存在 $g:B\to A$ 和同伦 $\eta: g \circ f \htpy \idfunc[A]$ 和 $\epsilon:f \circ g \htpy \idfunc[B]$,使得存在一个同伦
\[\tau : \prd{x:A} \map{f}{\eta x} = \epsilon(fx)。\]
\end{defn}

因此我们定义类型 $\ishae(f)$ 为
\begin{equation*}
  \sm{g:B\to A}{\eta: g \circ f \htpy \idfunc[A]}{\epsilon:f \circ g \htpy \idfunc[B]} \prd{x:A} \map{f}{\eta x} = \epsilon(fx)。
\end{equation*}
注意在上述定义中,$\eta$ 和 $\epsilon$ 的一致性条件 (coherence condition) 仅涉及 $f$。我们可以考虑一个涉及 $g$ 的类似一致性条件:
\[\upsilon : \prd{y:B} \map{g}{\epsilon y} = \eta(gy)\]
并得到一个类似的定义 $\ishae'(f)$。

幸运的是,事实证明这些条件中的每一个都意味着另一个:

\begin{lem}\label{lem:coh-equiv}
对于函数 $f : A \to B$ 和 $g:B\to A$ 以及同伦 $\eta: g \circ f \htpy \idfunc[A]$ 和 $\epsilon:f \circ g \htpy \idfunc[B]$,以下条件是逻辑等价的:
\begin{itemize}
  \item $\prd{x:A} \map{f}{\eta x} = \epsilon(fx)$
  \item $\prd{y:B} \map{g}{\epsilon y} = \eta(gy)$
\end{itemize}
\end{lem}
\begin{proof}
  证明一个方向就足够了;另一个方向通过替换 $A$、$f$ 和 $\eta$ 为 $B$、$g$ 和 $\epsilon$ 分别得到。
  设 $\tau : \prd{x:A}\;\map{f}{\eta x} = \epsilon(fx)$。
  固定 $y : B$。
  使用 $\epsilon$ 的自然性 (naturality) 并应用 $g$,我们得到以下路径的交换图:
  \[\uppercurveobject{{ }}\lowercurveobject{{ }}\twocellhead{{ }}
  \xymatrix@C=3pc{gfgfgy \ar@{=}^-{gfg(\epsilon y)}[r] \ar@{=}_{g(\epsilon (fgy))}[d] & gfgy \ar@{=}^{g(\epsilon y)}[d] \\ gfgy \ar@{=}_{g(\epsilon y)}[r] & gy
  }\]
  在图的左侧使用 $\tau(gy)$ 得到
  \[\uppercurveobject{{ }}\lowercurveobject{{ }}\twocellhead{{ }}
  \xymatrix@C=3pc{gfgfgy \ar@{=}^-{gfg(\epsilon y)}[r] \ar@{=}_{gf(\eta (gy))}[d] & gfgy \ar@{=}^{g(\epsilon y)}[d] \\ gfgy \ar@{=}_{g(\epsilon y)}[r] & gy
  }\]
  使用 $\eta$ 与 $g \circ f$ 的交换 (\cref{cor:hom-fg}),我们有
  \[\uppercurveobject{{ }}\lowercurveobject{{ }}\twocellhead{{ }}
  \xymatrix@C=3pc{gfgfgy \ar@{=}^-{gfg(\epsilon y)}[r] \ar@{=}_{\eta (gfgy)}[d] & gfgy \ar@{=}^{g(\epsilon y)}[d] \\ gfgy \ar@{=}_{g(\epsilon y)}[r] & gy
  }\]
  然而,根据 $\eta$ 的自然性,我们也有
  \[\uppercurveobject{{ }}\lowercurveobject{{ }}\twocellhead{{ }}
  \xymatrix@C=3pc{gfgfgy \ar@{=}^-{gfg(\epsilon y)}[r] \ar@{=}_{\eta (gfgy)}[d] & gfgy \ar@{=}^{\eta(gy)}[d] \\ gfgy \ar@{=}_{g(\epsilon y)}[r] & gy
  }\]
  因此,取消所有但右侧的同伦,我们得到 $g(\epsilon y) = \eta(g y)$ 如所期望的。
\end{proof}

然而,重要的是我们不要在 $\ishae (f)$ 的定义中包含\emph{两者} $\tau$ 和 $\upsilon$(因此称之为“\emph{半}伴随等价 (half adjoint equivalence)”)。如果我们这么做,那么在取消可缩类型后,我们仍然会剩下一个数据——除非我们添加另一个更高的一致性条件。一般来说,如果我们在一个奇数个一致性条件之后切断,我们预期会得到一个良好行为的类型。

当然,很明显 $\ishae(f) \to\qinv(f)$:只需忘记一致性数据。另一个方向是同伦理论 (homotopy theory) 和范畴论 (category theory) 中的标准论点的一种版本。

\begin{thm}\label{thm:equiv-iso-adj}
对于任何 $f:A\to B$,我们有 $\qinv(f)\to\ishae(f)$。
\end{thm}
\begin{proof}
  假设 $(g,\eta,\epsilon)$ 是 $f$ 的准逆 (quasi-inverse)。我们必须提供一个四元组 $(g',\eta',\epsilon',\tau)$ 来证明 $f$ 是一个半伴随等价 (half adjoint equivalence)。为了定义 $g'$ 和 $\eta'$,我们可以选择显而易见的选择,将 $g'
  定义为 $g$,将 $\eta'定义为 $\eta$。然而,在 $\epsilon'$ 的定义中,我们需要开始考虑 $\tau$ 的构造,因此不能直接按照常规选择 $\epsilon'$ 为 $\epsilon$。相反,我们选择
  \begin{equation*}
    \epsilon'(b) \defeq \opp{\epsilon(f(g(b)))}\ct (\ap{f}{\eta(g(b))}\ct \epsilon(b))。
  \end{equation*}
  现在我们需要找到
  \begin{equation*}
    \tau(a): \ap{f}{\eta(a)}=\opp{\epsilon(f(g(f(a))))}\ct (\ap{f}{\eta(g(f(a)))}\ct \epsilon(f(a)))。
  \end{equation*}
  首先注意到,根据 \cref{cor:hom-fg},我们有
%$\eta(g(f(a)))\ct\eta(a)=\ap{g}{\ap{f}{\eta(a)}}\ct\eta(a)$ 并因此它遵循 $\eta(g(f(a)))=\ap{g}{\ap{f}{\eta(a)}}$。因此,我们可以应用
  \cref{lem:htpy-natural} 计算
  \begin{align*}
    \ap{f}{\eta(g(f(a)))}\ct \epsilon(f(a))
    & = \ap{f}{\ap{g}{\ap{f}{\eta(a)}}}\ct \epsilon(f(a))\\
    & = \epsilon(f(g(f(a))))\ct \ap{f}{\eta(a)}
  \end{align*}
  从而得到所需的路径 $\tau(a)$。
\end{proof}

结合 \cref{lem:coh-equiv}(或对称化的证明),我们也有 $\qinv(f)\to\ishae'(f)$。

剩下的是展示 $\ishae(f)$ 是一个纯命题。
为此,我们需要知道等价性的纤维 (fiber) 是可缩的。

\begin{defn}\label{defn:homotopy-fiber}
函数 $f:A\to B$ 在点 $y:B$ 上的\define{纤维 (fiber)}
\indexdef{fiber}%
\indexsee{function!fiber of}{fiber}%
定义为
\[ \hfib f y \defeq \sm{x:A} (f(x) = y)。\]
\end{defn}

在同伦理论中,这被称为 $f$ 的\emph{同伦纤维 (homotopy fiber)}。
\cref{sec:computational} 中的路径引理得出以下关于纤维中路径的刻画:

\begin{lem}\label{lem:hfib}
对于任何 $f : A \to B$、$y : B$ 以及 $(x,p),(x',p') : \hfib{f}{y}$,我们有
\[ \big((x,p) = (x',p')\big) \eqvsym \Parens{\sm{\gamma : x = x'} f(\gamma) \ct p' = p} \qedhere\]
\end{lem}

\begin{thm}\label{thm:contr-hae}
如果 $f:A\to B$ 是一个半伴随等价,那么对于任何 $y:B$,纤维 $\hfib f y$ 是可缩的。
\end{thm}
\begin{proof}
  令 $(g,\eta,\epsilon,\tau) : \ishae(f)$,并固定 $y : B$。
  作为 $\hfib{f}{y}$ 的缩并中心 (center of contraction),我们选择 $(gy, \epsilon y)$。
  现在取任何 $(x,p) : \hfib{f}{y}$;我们要构造从 $(gy, \epsilon y)$ 到 $(x,p)$ 的路径。
  根据 \cref{lem:hfib},足以给出一个路径 $\gamma : \id{gy}{x}$,使得 $\ap f\gamma \ct p = \epsilon y$。
  我们取 $\gamma \defeq \opp{g(p)} \ct \eta x$。
  然后我们有
  \begin{align*}
    f(\gamma) \ct p & = \opp{fg(p)} \ct f (\eta x) \ct p \\
    & = \opp{fg(p)} \ct \epsilon(fx) \ct p \\
    & = \epsilon y
  \end{align*}
  其中第二个等式由 $\tau x$ 得到,第三个等式是 $\epsilon$ 的自然性。
\end{proof}

我们现在定义封装可缩数据对的类型。
以下类型将准逆 $g$ 与其中一个同伦结合起来。

\begin{defn}\label{defn:linv-rinv}
给定一个函数 $f:A\to B$,我们定义类型
\begin{align*}
  \linv(f) &\defeq \sm{g:B\to A} (g\circ f\htpy \idfunc[A])\\
  \rinv(f) &\defeq \sm{g:B\to A} (f\circ g\htpy \idfunc[B])
\end{align*}
分别为 $f$ 的\define{左逆 (left inverses)}
\indexdef{left!inverse}%
\indexdef{inverse!left}%
和\define{右逆 (right inverses)}。
如果 $\linv(f)$ 可居住,我们称 $f$ 是\define{左可逆 (left invertible)}的,如果 $\rinv(f)$ 可居住,我们称 $f$ 是\define{右可逆 (right invertible)}的。
\end{defn}

\begin{lem}\label{thm:equiv-compose-equiv}
如果 $f:A\to B$ 有一个准逆,那么以下映射也有准逆:
\begin{align*}
(f\circ \blank) &: (C\to A) \to (C\to B)\\
(\blank\circ f) &: (B\to C) \to (A\to C)。
\end{align*}
\end{lem}
\begin{proof}
  如果 $g$ 是 $f$ 的准逆,那么 $(g\circ \blank)$ 和 $(\blank\circ g)$ 分别是 $(f\circ \blank)$ 和 $(\blank\circ f)$ 的准逆。
\end{proof}

\begin{lem}\label{lem:inv-hprop}
如果 $f : A \to B$ 有一个准逆,那么类型 $\rinv(f)$ 和 $\linv(f)$ 是可缩的。
\end{lem}
\begin{proof}
  根据函数外延性,我们有
  \[\eqv{\linv(f)}{\sm{g:B\to A} (g\circ f = \idfunc[A])}。\]
  但这是 $\Sigma$ 的纤维 (fiber) (\cref{subsec:prop-subsets}),因此
  根据 \cref{thm:equiv-compose-equiv,thm:equiv-iso-adj,thm:contr-hae},它是可缩的。
  类似地,$\rinv(f)$ 等价于 $\Sigma$ 上的 $\idfunc[B]$ 的纤维 (fiber),因此也是可缩的。
\end{proof}

接下来,我们定义将另一个同伦与额外的一致性数据结合起来的类型。\index{coherence}%

\begin{defn}\label{defn:lcoh-rcoh}
对于 $f : A \to B$,一个左逆 $(g,\eta) : \linv(f)$ 和一个右逆 $(g,\epsilon) : \rinv(f)$,我们表示
\begin{align*}
  \lcoh{f}{g}{\eta} & \defeq \sm{\epsilon : f\circ g \htpy \idfunc[B]} \prd{y:B} g(\epsilon y) = \eta (gy), \\
  \rcoh{f}{g}{\epsilon} & \defeq \sm{\eta : g\circ f \htpy \idfunc[A]} \prd{x:A} f(\eta x) = \epsilon (fx)。
\end{align*}
\end{defn}

\begin{lem}\label{lem:coh-hfib}
对于任何 $f,g,\epsilon,\eta$,我们有
\begin{align*}
  \lcoh{f}{g}{\eta} & \eqvsym {\prd{y:B} \id[\hfib{g}{gy}]{(fgy,\eta(gy))}{(y,\refl{gy})}}, \\
  \rcoh{f}{g}{\epsilon} & \eqvsym {\prd{x:A} \id[\hfib{f}{fx}]{(gfx,\epsilon(fx))}{(x,\refl{fx})}}。
\end{align*}
\end{lem}
\begin{proof}
  使用 \cref{lem:hfib}。
\end{proof}

\begin{lem}\label{lem:coh-hprop}
如果 $f$ 是一个半伴随等价,那么对于任何 $(g,\epsilon) : \rinv(f)$,类型 $\rcoh{f}{g}{\epsilon}$ 是可缩的。
\end{lem}
\begin{proof}
  根据 \cref{lem:coh-hfib} 和从属函数类型保持可缩空间的事实,足以展示对于每个 $x:A$,类型 $\id[\hfib{f}{fx}]{(gfx,\epsilon(fx))}{(x,\refl{fx})}$ 是可缩的。
  但根据 \cref{thm:contr-hae},$\hfib{f}{fx}$ 是可缩的,并且任何可缩空间的路径空间本身就是可缩的。
\end{proof}

\begin{thm}\label{thm:hae-hprop}
对于任何 $f : A \to B$,类型 $\ishae(f)$ 是一个纯命题。
\end{thm}
\begin{proof}
  根据 \cref{ex:prop-inhabcontr},假设 $f$ 是一个半伴随等价,并展示 $\ishae(f)$ 是可缩的就足够了。
  现在根据 $\Sigma$ 的结合律 (\cref{ex:sigma-assoc}),类型 $\ishae(f)$ 等价于
  \[\sm{u : \rinv(f)} \rcoh{f}{\proj{1}(u)}{\proj{2}(u)}。\]
  但根据 \cref{lem:inv-hprop,lem:coh-hprop} 和 $\Sigma$ 保持可缩性的事实,后者类型也是可缩的。
\end{proof}

因此,我们已经证明 $\ishae(f)$ 具有作为类型 $\isequiv(f)$ 的所有三种特性。
在接下来的两节中,我们将讨论一些其他可能性。

\index{equivalence!half adjoint|)}%
\index{half adjoint equivalence|)}%
\index{adjoint!equivalence!of types, half|)}%

\section{Bi-invertible maps}
\label{sec:biinv}

\index{function!bi-invertible|(defstyle}%
\index{bi-invertible function|(defstyle}%
\index{equivalence!as bi-invertible function|(defstyle}%

Using the language introduced in \cref{sec:hae}, we can restate the definition proposed in \cref{sec:basics-equivalences} as follows.

\begin{defn}\label{defn:biinv}
  We say $f:A\to B$ is \define{bi-invertible}
  if it has both a left inverse and a right inverse:
  \[ \biinv (f) \defeq \linv(f) \times \rinv(f). \]
\end{defn}

In \cref{sec:basics-equivalences} we proved that $\qinv(f)\to\biinv(f)$ and $\biinv(f)\to\qinv(f)$.
What remains is the following.

\begin{thm}\label{thm:isprop-biinv}
  For any $f:A\to B$, the type $\biinv(f)$ is a mere proposition.
\end{thm}
\begin{proof}
  We may suppose $f$ to be bi-invertible and show that $\biinv(f)$ is contractible.
  But since $\biinv(f)\to\qinv(f)$, by \cref{lem:inv-hprop} in this case both $\linv(f)$ and $\rinv(f)$ are contractible, and the product of contractible types is contractible.
\end{proof}

Note that this also fits the proposal made at the beginning of \cref{sec:hae}: we combine $g$ and $\eta$ into a contractible type and add an additional datum which combines with $\epsilon$ into a contractible type.
The difference is that instead of adding a \emph{higher} datum (a 2-dimensional path) to combine with $\epsilon$, we add a \emph{lower} one (a right inverse that is separate from the left inverse).

\begin{cor}\label{thm:equiv-biinv-isequiv}
  For any $f:A\to B$ we have $\eqv{\biinv(f)}{\ishae(f)}$.
\end{cor}
\begin{proof}
  We have $\biinv(f) \to \qinv(f) \to \ishae(f)$ and $\ishae(f) \to \qinv(f) \to \biinv(f)$.
  Since both $\ishae(f)$ and $\biinv(f)$ are mere propositions, the equivalence follows from \cref{lem:equiv-iff-hprop}.
\end{proof}

\index{function!bi-invertible|)}%
\index{bi-invertible function|)}%
\index{equivalence!as bi-invertible function|)}%

\section{Contractible fibers}
\label{sec:contrf}

\index{function!contractible|(defstyle}%
\index{contractible!function|(defstyle}%
\index{equivalence!as contractible function|(defstyle}%

Note that our proofs about $\ishae(f)$ and $\biinv(f)$ made essential use of the fact that the fibers of an equivalence are contractible.
In fact, it turns out that this property is itself a sufficient definition of equivalence.

\begin{defn}[Contractible maps] \label{defn:equivalence}
  A map $f:A\to B$ is \define{contractible}
  if for all $y:B$, the fiber $\hfib f y$ is contractible.
\end{defn}

Thus, the type $\iscontr(f)$ is defined to be
\begin{align}
  \iscontr(f) &\defeq \prd{y:B} \iscontr(\hfib f y)\label{eq:iscontrf}
  % \\
  % &\defeq \prd{y:B} \iscontr (\setof{x:A | f(x) = y}).
\end{align}
Note that in \cref{sec:contractibility} we defined what it means for a \emph{type} to be contractible.
Here we are defining what it means for a \emph{map} to be contractible.
Our terminology follows the general homotopy-theoretic practice of saying that a map has a certain property if all of its (homotopy) fibers have that property.
Thus, a type $A$ is contractible just when the map $A\to\unit$ is contractible.
From \cref{cha:hlevels} onwards we will also call contractible maps and types \emph{$(-2)$-truncated}.

We have already shown in \cref{thm:contr-hae} that $\ishae(f) \to \iscontr(f)$.
Conversely:

\begin{thm}\label{thm:lequiv-contr-hae}
For any $f:A\to B$ we have ${\iscontr(f)} \to {\ishae(f)}$.
\end{thm}
\begin{proof}
Let $P : \iscontr(f)$. We define an inverse mapping $g : B \to A$ by sending each $y : B$ to the center of contraction of the fiber at $y$:
\[ g(y) \defeq \proj{1}(\proj{1}(Py)). \]
We can thus define the homotopy $\epsilon$ by mapping $y$ to the witness that $g(y)$ indeed belongs to the fiber at $y$:
\[ \epsilon(y) \defeq \proj{2}(\proj{1}(P y)). \]
It remains to define $\eta$ and $\tau$. This of course amounts to giving an element of $\rcoh{f}{g}{\epsilon}$. By \cref{lem:coh-hfib}, this is the same as giving for each $x:A$ a path from $(gfx,\epsilon(fx))$ to $(x,\refl{fx})$ in the fiber of $f$ over $fx$. But this is easy: for any $x : A$, the type $\hfib{f}{fx}$
is contractible by assumption, hence such a path must exist. We can construct it explicitly as
\[\opp{\big(\proj{2}(P(fx))(gfx,\epsilon(fx))\big)} \ct \big(\proj{2}(P(fx)) (x,\refl{fx})\big). \qedhere \]
\end{proof}

It is also easy to see:

\begin{lem}\label{thm:contr-hprop}
  For any $f$, the type $\iscontr(f)$ is a mere proposition.
\end{lem}
\begin{proof}
  By \cref{thm:isprop-iscontr}, each type $\iscontr (\hfib f y)$ is a mere proposition.
  Thus, by \cref{thm:isprop-forall}, so is~\eqref{eq:iscontrf}.
\end{proof}

\begin{thm}\label{thm:equiv-contr-hae}
  For any $f:A\to B$ we have $\eqv{\iscontr(f)}{\ishae(f)}$.
\end{thm}
\begin{proof}
  We have already established a logical equivalence ${\iscontr(f)} \Leftrightarrow {\ishae(f)}$, and both are mere propositions (\cref{thm:contr-hprop,thm:hae-hprop}).
  Thus, \cref{lem:equiv-iff-hprop} applies.
\end{proof}

Usually, we prove that a function is an equivalence by exhibiting a quasi-inverse, but sometimes this definition is more convenient.
For instance, it implies that when proving a function to be an equivalence, we are free to assume that its codomain is inhabited.

\begin{cor}\label{thm:equiv-inhabcod}
  If $f:A\to B$ is such that $B\to \isequiv(f)$, then $f$ is an equivalence.
\end{cor}
\begin{proof}
  To show $f$ is an equivalence, it suffices to show that $\hfib f y$ is contractible for any $y:B$.
  But if $e:B\to \isequiv(f)$, then given any such $y$ we have $e(y):\isequiv(f)$, so that $f$ is an equivalence and hence $\hfib f y$ is contractible, as desired.
\end{proof}

\index{function!contractible|)}%
\index{contractible!function|)}%
\index{equivalence!as contractible function|)}%

\section{On the definition of equivalences}
\label{sec:concluding-remarks}

\indexdef{equivalence}
We have shown that all three definitions of equivalence satisfy the three desirable properties and are pairwise equivalent:
\[ \iscontr(f) \eqvsym \ishae(f) \eqvsym \biinv(f). \]
(There are yet more possible definitions of equivalence, but we will stop with these three.
See \cref{ex:brck-qinv} and the exercises in this chapter for some more.)
Thus, we may choose any one of them as ``the'' definition of $\isequiv (f)$.
For definiteness, we choose to define
\[ \isequiv(f) \defeq \ishae(f).\]
\index{mathematics!formalized}%
This choice is advantageous for formalization, since $\ishae(f)$ contains the most directly useful data.
On the other hand, for other purposes, $\biinv(f)$ is often easier to deal with, since it contains no 2-dimensional paths and its two symmetrical halves can be treated independently.
However, for purposes of this book, the specific choice will make little difference.

In the rest of this chapter, we study some other properties and characterizations of equivalences.
\index{equivalence!properties of}%


\section{Surjections and embeddings}
\label{sec:mono-surj}

\index{set}
When $A$ and $B$ are sets and $f:A\to B$ is an equivalence, we also call it as \define{isomorphism}
\indexdef{isomorphism!of sets}%
or a \define{bijection}.
\indexdef{bijection}%
\indexsee{function!bijective}{bijection}%
(We avoid these words for types that are not sets, since in homotopy theory and higher category theory they often denote a stricter notion of ``sameness'' than homotopy equivalence.)
In set theory, a function is a bijection just when it is both injective and surjective.
The same is true in type theory, if we formulate these conditions appropriately.
For clarity, when dealing with types that are not sets, we will speak of \emph{embeddings} instead of injections.

\begin{defn}\label{defn:surj-emb}
  Let $f:A\to B$.
  \begin{enumerate}
  \item We say $f$ is \define{surjective}
    \indexsee{surjective!function}{function, surjective}%
    \indexdef{function!surjective}%
    (or a \define{surjection})
    \indexsee{surjection}{function, surjective}%
    if for every $b:B$ we have $\brck{\hfib f b}$.
  \item We say $f$ is an \define{embedding}
    \indexdef{function!embedding}%
    \indexsee{embedding}{function, embedding}%
    if for every $x,y:A$ the function $\apfunc f : (\id[A]xy) \to (\id[B]{f(x)}{f(y)})$ is an equivalence.
  \end{enumerate}
\end{defn}

In other words, $f$ is surjective if every fiber of $f$ is merely inhabited, or equivalently if for all $b:B$ there merely exists an $a:A$ such that $f(a)=b$.
In traditional logical notation, $f$ is surjective if $\fall{b:B}\exis{a:A} (f(a)=b)$.
This must be distinguished from the stronger assertion that $\prd{b:B}\sm{a:A} (f(a)=b)$; if this holds we say that $f$ is a \define{split surjection}.
\indexsee{split!surjection}{function, split surjective}%
\indexsee{surjection!split}{function, split surjective}%
\indexsee{surjective!function!split}{function, split surjective}%
\indexdef{function!split surjective}%
(Since this latter type is equivalent to $\sm{g:B\to A}\prd{b:B} (f(g(b))=b)$, being a split surjection is the same as being a \emph{retraction} as defined in \cref{sec:contractibility}.)
\index{retraction}%
\index{function!retraction}%

The axiom of choice from \cref{sec:axiom-choice} says exactly that every surjection \emph{between sets} is split.
However, in the presence of the univalence axiom, it is simply false that \emph{all} surjections are split.
In \cref{thm:no-higher-ac} we constructed a type family $Y:X\to \type$ such that $\prd{x:X} \brck{Y(x)}$ but $\neg \prd{x:X} Y(x)$;
for any such family, the first projection $(\sm{x:X} Y(x)) \to X$ is a surjection that is not split.

If $A$ and $B$ are sets, then by \cref{lem:equiv-iff-hprop}, $f$ is an embedding just when
\begin{equation}
  \prd{x,y:A} (\id[B]{f(x)}{f(y)}) \to (\id[A]xy).\label{eq:injective}
\end{equation}
In this case we say that $f$ is \define{injective},
\indexsee{injective function}{function, injective}%
\indexdef{function!injective}%
or an \define{injection}.
\indexsee{injection}{function, injective}%
We avoid these word for types that are not sets, because they might be interpreted as~\eqref{eq:injective}, which is an ill-behaved notion for non-sets.
It is also true that any function between sets is surjective if and only if it is an \emph{epimorphism} in a suitable sense, but this also fails for more general types, and surjectivity is generally the more important notion.

\begin{thm}\label{thm:mono-surj-equiv}
  A function $f:A\to B$ is an equivalence if and only if it is both surjective and an embedding.
\end{thm}
\begin{proof}
  If $f$ is an equivalence, then each $\hfib f b$ is contractible, hence so is $\brck{\hfib f b}$, so $f$ is surjective.
  And we showed in \cref{thm:paths-respects-equiv} that any equivalence is an embedding.

  Conversely, suppose $f$ is a surjective embedding.
  Let $b:B$; we show that $\sm{x:A}(f(x)=b)$ is contractible.
  Since $f$ is surjective, there merely exists an $a:A$ such that $f(a)=b$.
  Thus, the fiber of $f$ over $b$ is inhabited; it remains to show it is a mere proposition.
  For this, suppose given $x,y:A$ with $p:f(x)=b$ and $q:f(y)=b$.
  Then since $\apfunc f$ is an equivalence, there exists $r:x=y$ with $\apfunc f (r) = p \ct \opp q$.
  However, using the characterization of paths in $\Sigma$-types, the latter equality rearranges to $\trans{r}{p} = q$.
  Thus, together with $r$ it exhibits $(x,p) = (y,q)$ in the fiber of $f$ over $b$.
\end{proof}

\begin{cor}
  For any $f:A\to B$ we have
  \[ \isequiv(f) \eqvsym (\mathsf{isEmbedding}(f) \times \mathsf{isSurjective}(f)).\]
\end{cor}
\begin{proof}
  Being a surjection and an embedding are both mere propositions; now apply \cref{lem:equiv-iff-hprop}.
\end{proof}

Of course, this cannot be used as a definition of ``equivalence'', since the definition of embeddings refers to equivalences.
However, this characterization can still be useful; see \cref{sec:whitehead}.
We will generalize it in \cref{cha:hlevels}.


% \section{Fiberwise equivalences}
\section{Closure properties of equivalences}
\label{sec:equiv-closures}
\label{sec:fiberwise-equivalences}
\index{equivalence!properties of}%


% We end this chapter by observing some important closure properties of equivalences.
We have already seen in \cref{thm:equiv-eqrel} that equivalences are closed under composition.
Furthermore, we have:

\begin{thm}[The 2-out-of-3 property]\label{thm:two-out-of-three}
  \index{2-out-of-3 property}%
  Suppose $f:A\to B$ and $g:B\to C$.
  If any two of $f$, $g$, and $g\circ f$ are equivalences, so is the third.
\end{thm}
\begin{proof}
  If $g\circ f$ and $g$ are equivalences, then $\opp{(g\circ f)} \circ g$ is a quasi-inverse to $f$.
  On the one hand, we have $\opp{(g\circ f)} \circ g \circ f \htpy \idfunc[A]$, while on the other we have
  \begin{align*}
    f \circ \opp{(g\circ f)} \circ g
    &\htpy \opp g \circ g \circ f \circ \opp{(g\circ f)} \circ g\\
    &\htpy \opp g \circ g\\
    &\htpy \idfunc[B].
  \end{align*}
  Similarly, if $g\circ f$ and $f$ are equivalences, then $f\circ \opp{(g\circ f)}$ is a quasi-inverse to $g$.
\end{proof}

This is a standard closure condition on equivalences from homotopy theory.
Also well-known is that they are closed under retracts, in the following sense.

\index{retract!of a function|(defstyle}%

\begin{defn}\label{defn:retract}
A function $g:A\to B$ is said to be a \define{retract}
of a function $f:X\to Y$ if there is a diagram
\begin{equation*}
  \xymatrix{
    {A} \ar[r]^{s} \ar[d]_{g}
    &
    {X} \ar[r]^{r} \ar[d]_{f}
    &
    {A} \ar[d]^{g}
    \\
    {B} \ar[r]_{s'}
    &
    {Y} \ar[r]_{r'}
    &
    {B}
  }
\end{equation*}
for which there are
\begin{enumerate}
\item a homotopy $R:r\circ s \htpy \idfunc[A]$.
\item a homotopy $R':r'\circ s' \htpy\idfunc[B]$.
\item a homotopy $L:f\circ s\htpy s'\circ g$.
\item a homotopy $K:g\circ r\htpy r'\circ f$.
\item for every $a:A$, a path $H(a)$ witnessing the commutativity of the square
\begin{equation*}
  \xymatrix@C=3pc{
    {g(r(s(a)))} \ar@{=}[r]^-{K(s(a))} \ar@{=}[d]_{\ap g{R(a)}}
    &
    {r'(f(s(a)))} \ar@{=}[d]^{\ap{r'}{L(a)}}
    \\
    {g(a)} \ar@{=}[r]_-{\opp{R'(g(a))}}
    &
    {r'(s'(g(a)))}
  }
\end{equation*}
\end{enumerate}
\end{defn}

Recall that in \cref{sec:contractibility} we defined what it means for a type to be a retract of another.
This is a special case of the above definition where $B$ and $Y$ are $\unit$.
Conversely, just as with contractibility, retractions of maps induce retractions of their fibers.

\begin{lem}\label{lem:func_retract_to_fiber_retract}
If a function $g:A\to B$ is a retract of a function $f:X\to Y$, then $\hfib{g}b$ is a retract of $\hfib{f}{s'(b)}$
for every $b:B$, where $s':B\to Y$ is as in \cref{defn:retract}.
\end{lem}

\begin{proof}
Suppose that $g:A\to B$ is a retract of $f:X\to Y$. Then for any $b:B$ we have the functions
\begin{align*}
\varphi_b &:\hfiber{g}b\to\hfib{f}{s'(b)}, &
\varphi_b(a,p) & \defeq \pairr{s(a),L(a)\ct s'(p)},\\
\psi_b &:\hfib{f}{s'(b)}\to\hfib{g}b, &
\psi_b(x,q) &\defeq \pairr{r(x),K(x)\ct r'(q)\ct R'(b)}.
\end{align*}
Then we have $\psi_b(\varphi_b({a,p}))\equiv\pairr{r(s(a)),K(s(a))\ct r'(L(a)\ct s'(p))\ct R'(b)}$.
We claim $\psi_b$ is a retraction with section $\varphi_b$ for all $b:B$, which is to say that for all $(a,p):\hfib g b$ we have $\psi_b(\varphi_b({a,p}))= \pairr{a,p}$.
In other words, we want to show
\begin{equation*}
\prd{b:B}{a:A}{p:g(a)=b} \psi_b(\varphi_b({a,p}))= \pairr{a,p}.
\end{equation*}
By reordering the first two $\Pi$s and applying a version of \cref{thm:omit-contr}, this is equivalent to
\begin{equation*}
\prd{a:A}\psi_{g(a)}(\varphi_{g(a)}({a,\refl{g(a)}}))=\pairr{a,\refl{g(a)}}.
\end{equation*}
For any $a$, by \cref{thm:path-sigma}, this equality of pairs is equivalent to a pair of equalities. The first components are equal by $R(a):r(s(a))= a$, so we need only show
\begin{equation*}
\trans{R(a)}{K(s(a))\ct r'(L(a))\ct R'(g(a))} = \refl{g(a)}.
\end{equation*}
But this transportation computes as $\opp{g(R(a))}\ct K(s(a))\ct r'(L(a))\ct R'(g(a))$, so the required path is given by $H(a)$.
\end{proof}

\begin{thm}\label{thm:retract-equiv}
  If $g$ is a retract of an equivalence $f$, then $g$ is also an equivalence.
\end{thm}
\begin{proof}
  By \cref{lem:func_retract_to_fiber_retract}, every fiber of $g$ is a retract of a fiber of $f$.
  Thus, by \cref{thm:retract-contr}, if the latter are all contractible, so are the former.
\end{proof}

\index{retract!of a function|)}%

\index{fibration}%
\index{total!space}%
Finally, we show that fiberwise equivalences can be characterized in terms of equivalences of total spaces.
To explain the terminology, recall from \cref{sec:fibrations} that a type family $P:A\to\type$ can be viewed as a fibration over $A$ with total space $\sm{x:A} P(x)$, the fibration being the projection $\proj1:\sm{x:A} P(x) \to A$.
From this point of view, given two type families $P,Q:A\to\type$, we may refer to a function $f:\prd{x:A} (P(x)\to Q(x))$ as a \define{fiberwise map} or a \define{fiberwise transformation}.
\indexsee{transformation!fiberwise}{fiberwise transformation}%
\indexsee{function!fiberwise}{fiberwise transformation}%
\index{fiberwise!transformation|(defstyle}%
\indexsee{fiberwise!map}{fiberwise transformation}%
\indexsee{map!fiberwise}{fiberwise transformation}
Such a map induces a function on total spaces:

\begin{defn}\label{defn:total-map}
  Given type families $P,Q:A\to\type$ and a map $f:\prd{x:A} P(x)\to Q(x)$, we define
  \begin{equation*}
    \total f  \defeq \lam{w}\pairr{\proj{1}w,f(\proj{1}w,\proj{2}w)} : \sm{x:A}P(x)\to\sm{x:A}Q(x).
  \end{equation*}
\end{defn}

\begin{thm}\label{fibwise-fiber-total-fiber-equiv}
Suppose that $f$ is a fiberwise transformation between families $P$ and
$Q$ over a type $A$ and let $x:A$ and $v:Q(x)$. Then we have an equivalence
\begin{equation*}
\eqv{\hfib{\total{f}}{\pairr{x,v}}}{\hfib{f(x)}{v}}.
\end{equation*}
\end{thm}
\begin{proof}
  We calculate:
\begin{align}
  \hfib{\total{f}}{\pairr{x,v}}
  & \jdeq \sm{w:\sm{x:A}P(x)}\pairr{\proj{1}w,f(\proj{1}w,\proj{2}w)}=\pairr{x,v}
  \notag \\
  & \eqv{}{} \sm{a:A}{u:P(a)}\pairr{a,f(a,u)}=\pairr{x,v}
  \tag{by~\cref{ex:sigma-assoc}} \\
  & \eqv{}{} \sm{a:A}{u:P(a)}{p:a=x}\trans{p}{f(a,u)}=v
  \tag{by \cref{thm:path-sigma}} \\
  & \eqv{}{} \sm{a:A}{p:a=x}{u:P(a)}\trans{p}{f(a,u)}=v
  \notag \\
  & \eqv{}{} \sm{u:P(x)}f(x,u)=v
  \tag{$*$}\label{eq:uses-sum-over-paths} \\
  & \jdeq \hfib{f(x)}{v}. \notag
\end{align}
The equivalence~\eqref{eq:uses-sum-over-paths} follows from \cref{thm:omit-contr,thm:contr-paths,ex:sigma-assoc}.
\end{proof}

We say that a fiberwise transformation $f:\prd{x:A} P(x)\to Q(x)$ is a \define{fiberwise equivalence}%
\indexdef{fiberwise!equivalence}%
\indexdef{equivalence!fiberwise}
if each $f(x):P(x) \to Q(x)$ is an equivalence.

\begin{thm}\label{thm:total-fiber-equiv}
Suppose that $f$ is a fiberwise transformation between families
$P$ and $Q$ over a type $A$.
Then $f$ is a fiberwise equivalence if and only if $\total{f}$ is an equivalence.
\end{thm}

\begin{proof}
Let $f$, $P$, $Q$ and $A$ be as in the statement of the theorem.
By \cref{fibwise-fiber-total-fiber-equiv} it follows for all
$x:A$ and $v:Q(x)$ that
$\hfib{\total{f}}{\pairr{x,v}}$ is contractible if and only if
$\hfib{f(x)}{v}$ is contractible.
Thus, $\hfib{\total{f}}{w}$ is contractible for all $w:\sm{x:A}Q(x)$ if and only if $\hfib{f(x)}{v}$ is contractible for all $x:A$ and $v:Q(x)$.
\end{proof}

\index{fiberwise!transformation|)}%


\section{The object classifier}
\label{sec:object-classification}

In type theory we have a basic notion of \emph{family of types}, namely a function $B:A\to\type$.
We have seen that such families behave somewhat like \emph{fibrations} in homotopy theory, with the fibration being the projection $\proj1:\sm{a:A} B(a) \to A$.
A basic fact in homotopy theory is that every map is equivalent to a fibration.
With univalence at our disposal, we can prove the same thing in type theory.

\begin{lem}\label{thm:fiber-of-a-fibration}
  For any type family $B:A\to\type$, the fiber of $\proj1:\sm{x:A} B(x) \to A$ over $a:A$ is equivalent to $B(a)$:
  \[ \eqv{\hfib{\proj1}{a}}{B(a)} \]
\end{lem}
\begin{proof}
  We have
  \begin{align*}
    \hfib{\proj1}{a} &\defeq \sm{u:\sm{x:A} B(x)} \proj1(u)=a\\
    &\eqvsym \sm{x:A}{b:B(x)} (x=a)\\
    &\eqvsym \sm{x:A}{p:x=a} B(x)\\
    &\eqvsym B(a)
  \end{align*}
  using the left universal property of identity types.
\end{proof}

\begin{lem}\label{thm:total-space-of-the-fibers}
  For any function $f:A\to B$, we have $\eqv{A}{\sm{b:B}\hfib{f}{b}}$.
\end{lem}
\begin{proof}
  We have
  \begin{align*}
    \sm{b:B}\hfib{f}{b} &\defeq \sm{b:B}{a:A} (f(a)=b)\\
    &\eqvsym \sm{a:A}{b:B} (f(a)=b)\\
    &\eqvsym A
  \end{align*}
  using the fact that $\sm{b:B} (f(a)=b)$ is contractible.
\end{proof}

\begin{thm}\label{thm:nobject-classifier-appetizer}
For any type $B$ there is an equivalence
\begin{equation*}
\chi:\Parens{\sm{A:\type} (A\to B)}\eqvsym (B\to\type).
\end{equation*}
\end{thm}
\begin{proof}
We have to construct quasi-inverses
\begin{align*}
\chi & : \Parens{\sm{A:\type} (A\to B)}\to B\to\type\\
\psi & : (B\to\type)\to\Parens{\sm{A:\type} (A\to B)}.
\end{align*}
We define $\chi$ by $\chi((A,f),b)\defeq\hfiber{f}b$, and $\psi$ by $\psi(P)\defeq\Pairr{(\sm{b:B} P(b)),\proj1}$.
Now we have to verify that $\chi\circ\psi\htpy\idfunc{}$ and that $\psi\circ\chi \htpy\idfunc{}$.
\begin{enumerate}
\item Let $P:B\to\type$.
  By \cref{thm:fiber-of-a-fibration},
$\hfiber{\proj1}{b}\eqvsym P(b)$ for any $b:B$, so it follows immediately
that $P\htpy\chi(\psi(P))$.
\item Let $f:A\to B$ be a function. We have to find a path
\begin{equation*}
\Pairr{\tsm{b:B} \hfiber{f}b,\,\proj1}=\pairr{A,f}.
\end{equation*}
First note that by \cref{thm:total-space-of-the-fibers}, we have
$e:\sm{b:B} \hfiber{f}b\eqvsym A$ with $e(b,a,p)\defeq a$ and $e^{-1}(a)
\defeq(f(a),a,\refl{f(a)})$.
By \cref{thm:path-sigma}, it remains to show $\trans{(\ua(e))}{\proj1} = f$.
But by the computation rule for univalence and~\eqref{eq:transport-arrow}, we have $\trans{(\ua(e))}{\proj1} = \proj1\circ e^{-1}$, and the definition of $e^{-1}$ immediately yields $\proj1 \circ e^{-1} \jdeq f$.\qedhere
\end{enumerate}
\end{proof}

\noindent
\indexdef{object!classifier}%
\indexdef{classifier!object}%
\index{.infinity1-topos@$(\infty,1)$-topos}%
In particular, this implies that we have an \emph{object classifier} in the sense of higher topos theory.
Recall from \cref{def:pointedtype} that $\pointed\type$ denotes the type $\sm{A:\type} A$ of pointed types.

\begin{thm}\label{thm:object-classifier}
Let $f:A\to B$ be a function. Then the diagram
\begin{equation*}
  \vcenter{\xymatrix{
      A\ar[r]^-{\vartheta_f} \ar[d]_{f} &
      \pointed{\type}\ar[d]^{\proj1}\\
      B\ar[r]_{\chi_f} &
      \type
      }}
\end{equation*}
is a pullback\index{pullback} square (see \cref{ex:pullback}).
Here the function $\vartheta_f$ is defined by
\begin{equation*}
 \lam{a} \pairr{\hfiber{f}{f(a)},\pairr{a,\refl{f(a)}}}.
\end{equation*}
\end{thm}
\begin{proof}
Note that we have the equivalences
\begin{align*}
A & \eqvsym \sm{b:B} \hfiber{f}b\\
& \eqvsym \sm{b:B}{X:\type}{p:\hfiber{f}b= X} X\\
& \eqvsym \sm{b:B}{X:\type}{x:X} \hfiber{f}b= X\\
& \eqvsym \sm{b:B}{Y:\pointed{\type}} \hfiber{f}b = \proj1 Y\\
& \jdeq B\times_{\type}\pointed{\type}
\end{align*}
which gives us a composite equivalence $e:A\eqvsym B\times_\type\pointed{\type}$.
We may display the action of this composite equivalence step by step by
\begin{align*}
a & \mapsto \pairr{f(a),\; \pairr{a,\refl{f(a)}}}\\
& \mapsto \pairr{f(a), \; \hfiber{f}{f(a)}, \; \refl{\hfiber{f}{f(a)}}, \; \pairr{a,\refl{f(a)}}}\\
& \mapsto \pairr{f(a), \; \hfiber{f}{f(a)}, \; \pairr{a,\refl{f(a)}}, \; \refl{\hfiber{f}{f(a)}}}.
\end{align*}
Therefore, we get homotopies $f\htpy\proj1\circ e$ and $\vartheta_f\htpy \proj2\circ e$.
\end{proof}



\section{Univalence implies function extensionality}
\label{sec:univalence-implies-funext}

\index{function extensionality!proof from univalence}%
In the last section of this chapter we include a proof that the univalence axiom implies function
extensionality. Thus, in this section we work \emph{without} the function extensionality axiom.
The proof consists of two steps. First we show
in \cref{uatowfe} that the univalence
axiom implies a weak form of function extensionality, defined in \cref{weakfunext} below. The
principle of weak function extensionality in turn implies the usual function extensionality,
and it does so without the univalence axiom (\cref{wfetofe}).

\index{univalence axiom}%
Let $\type$ be a universe; we will explicitly indicate where we assume that it is univalent.

\begin{defn}\label{weakfunext}
The \define{weak function extensionality principle}
\indexdef{function extensionality!weak}%
asserts that there is a function
\begin{equation*}
\Parens{\prd{x:A}\iscontr(P(x))} \to\iscontr\Parens{\prd{x:A}P(x)}
\end{equation*}
for any family $P:A\to\type$ of types over any type $A$.
\end{defn}

The following lemma is easy to prove using function extensionality; the point here is that it also follows from univalence without assuming function extensionality separately.

\begin{lem} \label{UA-eqv-hom-eqv}
Assuming $\type$ is univalent, for any $A,B,X:\type$ and any $e:\eqv{A}{B}$, there is an equivalence
\begin{equation*}
\eqv{(X\to A)}{(X\to B)}
\end{equation*}
of which the underlying map is given by post-composition with the underlying function of $e$.
\end{lem}

\begin{proof}
  % Immediate by induction on $\eqv{}{}$ (see \cref{thm:equiv-induction}).
  As in the proof of \cref{lem:qinv-autohtpy}, we may assume that $e = \idtoeqv(p)$ for some $p:A=B$.
  Then by path induction, we may assume $p$ is $\refl{A}$, so that $e = \idfunc[A]$.
  But in this case, post-composition with $e$ is the identity, hence an equivalence.
\end{proof}

\begin{cor}\label{contrfamtotalpostcompequiv}
Let $P:A\to\type$ be a family of contractible types, i.e.\ \narrowequation{\prd{x:A}\iscontr(P(x)).}
Then the projection $\proj{1}:(\sm{x:A}P(x))\to A$ is an equivalence. Assuming $\type$ is univalent, it follows immediately that post-composition with $\proj{1}$ gives an equivalence
\begin{equation*}
\alpha : \eqv{\Parens{A\to\sm{x:A}P(x)}}{(A\to A)}.
\end{equation*}
\end{cor}

\begin{proof}
  By \cref{thm:fiber-of-a-fibration}, for $\proj{1}:\sm{x:A}P(X)\to A$ and $x:A$ we have an equivalence
  \begin{equation*}
    \eqv{\hfiber{\proj{1}}{x}}{P(x)}.
  \end{equation*}
  Therefore $\proj{1}$ is an equivalence whenever each $P(x)$ is contractible. The assertion is now a consequence of  \cref{UA-eqv-hom-eqv}.
\end{proof}

In particular, the homotopy fiber of the above equivalence at $\idfunc[A]$ is contractible. Therefore, we can show that univalence implies weak function extensionality by showing that the dependent function type $\prd{x:A}P(x)$ is a retract of $\hfiber{\alpha}{\idfunc[A]}$.

\begin{thm}\label{uatowfe}
In a univalent universe $\type$, suppose that $P:A\to\type$ is a family of contractible types
and let $\alpha$ be the function of \cref{contrfamtotalpostcompequiv}.
Then $\prd{x:A}P(x)$ is a retract of $\hfiber{\alpha}{\idfunc[A]}$. As a consequence, $\prd{x:A}P(x)$ is contractible. In other words, the univalence axiom implies the weak function extensionality principle.
\end{thm}

\begin{proof}
Define the functions
\begin{align*}
  \varphi &: (\tprd{x:A}P(x))\to\hfiber{\alpha}{\idfunc[A]},\\
  \varphi(f) &\defeq (\lam{x} (x,f(x)),\refl{\idfunc[A]}),
\intertext{and}
  \psi &: \hfiber{\alpha}{\idfunc[A]}\to \tprd{x:A}P(x), \\
  \psi(g,p) &\defeq \lam{x} \trans {\happly (p,x)}{\proj{2} (g(x))}.
\end{align*}
Then $\psi(\varphi(f))=\lam{x} f(x)$, which is $f$, by the uniqueness principle for dependent function types.
\end{proof}

We now show that weak function extensionality implies the usual function extensionality.
Recall from~\eqref{eq:happly} the function $\happly (f,g) : (f = g)\to(f\htpy g)$ which
converts equality of functions to homotopy. In the proof that follows, the univalence
axiom is not used.

\begin{thm}\label{wfetofe}
  \index{function extensionality}%
Weak function extensionality implies the function extensionality \cref{axiom:funext}.
\end{thm}

\begin{proof}
We want to show that
\begin{equation*}
\prd{A:\type}{P:A\to\type}{f,g:\prd{x:A}P(x)}\isequiv(\happly (f,g)).
\end{equation*}
Since a fiberwise map induces an equivalence on total spaces if and only if it is fiberwise an equivalence by \cref{thm:total-fiber-equiv}, it suffices to show that the function of type
\begin{equation*}
\Parens{\sm{g:\prd{x:A}P(x)}(f= g)} \to \sm{g:\prd{x:A}P(x)}(f\htpy g)
\end{equation*}
induced by $\lam{g:\prd{x:A}P(x)} \happly (f,g)$ is an equivalence.
Since the type on the left is contractible by \cref{thm:contr-paths}, it suffices to show that the type on the right:
\begin{equation}\label{eq:uatofesp}
\sm{g:\prd{x:A}P(x)}\prd{x:A}f(x)= g(x)
\end{equation}
is contractible.
Now \cref{thm:ttac} says that this is equivalent to
\begin{equation}\label{eq:uatofeps}
\prd{x:A}\sm{u:P(x)}f(x)= u.
\end{equation}
The proof of \cref{thm:ttac} uses function extensionality, but only for one of the composites.
Thus, without assuming function extensionality, we can conclude that~\eqref{eq:uatofesp} is a retract\index{retract!of a type} of~\eqref{eq:uatofeps}.
And~\eqref{eq:uatofeps} is a product of contractible types, which is contractible by the weak function extensionality principle; hence~\eqref{eq:uatofesp} is also contractible.
\end{proof}

\sectionNotes

The fact that the space of continuous maps equipped with quasi-inverses has the wrong homotopy type to be the ``space of homotopy equivalences'' is well-known in algebraic topology.
In that context, the ``space of homotopy equivalences'' $(\eqv AB)$ is usually defined simply as the subspace of the function space $(A\to B)$ consisting of the functions that are homotopy equivalences.
In type theory, this would correspond most closely to $\sm{f:A\to B} \brck{\qinv(f)}$; see \cref{ex:brck-qinv}.

The first definition of equivalence given in homotopy type theory was the one that we have called $\iscontr(f)$, which was due to Voevodsky.
The possibility of the other definitions was subsequently observed by various people.
The basic theorems about adjoint equivalences\index{adjoint!equivalence} such as \cref{lem:coh-equiv,thm:equiv-iso-adj} are adaptations of standard facts in higher category theory and homotopy theory.
Using bi-invertibility as a definition of equivalences was suggested by Andr\'e Joyal.

The properties of equivalences discussed in \cref{sec:mono-surj,sec:equiv-closures} are well-known in homotopy theory.
Most of them were first proven in type theory by Voevodsky.

The fact that every function is equivalent to a fibration is a standard fact in homotopy theory.
The notion of object classifier
\index{object!classifier}%
\index{classifier!object}%
in $(\infty,1)$-category
\index{.infinity1-category@$(\infty,1)$-category}%
theory (the categorical analogue of \cref{thm:nobject-classifier-appetizer}) is due to Rezk (see~\cite{Rezk05,lurie:higher-topoi}).

Finally, the fact that univalence implies function extensionality (\cref{sec:univalence-implies-funext}) is due to Voevodsky.
Our proof is a simplification of his.
\cref{ex:funext-from-nondep} is also due to Voevodsky.

\sectionExercises

\begin{ex}\label{ex:two-sided-adjoint-equivalences}
  Consider the type of ``two-sided adjoint equivalence\index{adjoint!equivalence} data'' for $f:A\to B$,
  \begin{narrowmultline*}
    \sm{g:B\to A}{\eta: g \circ f \htpy \idfunc[A]}{\epsilon:f \circ g \htpy \idfunc[B]}
    \narrowbreak
    \Parens{\prd{x:A} \map{f}{\eta x} = \epsilon(fx)} \times
    \Parens{\prd{y:B} \map{g}{\epsilon y} = \eta(gy) }.
  \end{narrowmultline*}
  By \cref{lem:coh-equiv}, we know that if $f$ is an equivalence, then this type is inhabited.
  Give a characterization of this type analogous to \cref{lem:qinv-autohtpy}.

  Can you give an example showing that this type is not generally a mere proposition?
  (This will be easier after \cref{cha:hits}.)
\end{ex}

\begin{ex}\label{ex:symmetric-equiv}
  Show that for any $A,B:\UU$, the following type is equivalent to $\eqv A B$.
  \begin{equation*}
    \sm{R:A\to B\to \type}
    \Parens{\prd{a:A} \iscontr\Parens{\sm{b:B} R(a,b)}} \times
    \Parens{\prd{b:B} \iscontr\Parens{\sm{a:A} R(a,b)}}.
  \end{equation*}
  Can you extract from this a definition of a type satisfying the three desiderata of $\isequiv(f)$?
\end{ex}

\begin{ex} \label{ex:qinv-autohtpy-no-univalence}
  Reformulate the proof of \cref{lem:qinv-autohtpy} without using univalence.
\end{ex}

\begin{ex}[The unstable octahedral axiom]\label{ex:unstable-octahedron}
  \index{axiom!unstable octahedral}%
  \index{octahedral axiom, unstable}%
  Suppose $f:A\to B$ and $g:B\to C$ and $b:B$.
  \begin{enumerate}
  \item Show that there is a natural map $\hfib{g\circ f}{g(b)} \to \hfib{g}{g(b)}$ whose fiber over $(b,\refl{g(b)})$ is equivalent to $\hfib f b$.
  \item Show that $\eqv{\hfib{g\circ f}{c}}{\sm{w:\hfib{g}{c}} \hfib f {\proj1 w}}$.
  \end{enumerate}
\end{ex}

\begin{ex}\label{ex:2-out-of-6}
  \index{2-out-of-6 property}%
  Prove that equivalences satisfy the \emph{2-out-of-6 property}: given $f:A\to B$ and $g:B\to C$ and $h:C\to D$, if $g\circ f$ and $h\circ g$ are equivalences, so are $f$, $g$, $h$, and $h\circ g\circ f$.
  Use this to give a higher-level proof of \cref{thm:paths-respects-equiv}.
\end{ex}

\begin{ex}\label{ex:qinv-univalence}
  For $A,B:\UU$, define
  \[ \mathsf{idtoqinv}_{A,B} :(A=B) \to \sm{f:A\to B}\qinv(f) \]
  by path induction in the obvious way.
  Let \textbf{\textsf{qinv}-univalence} denote the modified form of the univalence axiom which asserts that for all $A,B:\UU$ the function $\mathsf{idtoqinv}_{A,B}$ has a quasi-inverse.
  \begin{enumerate}
  \item Show that \qinv-univalence can be used instead of univalence in the proof of function extensionality in \cref{sec:univalence-implies-funext}.
  \item Show that \qinv-univalence can be used instead of univalence in the proof of \cref{thm:qinv-notprop}.
  \item Show that \qinv-univalence is inconsistent (i.e.\ allows construction of an inhabitant of $\emptyt$).
    Thus, the use of a ``good'' version of $\isequiv$ is essential in the statement of univalence.
  \end{enumerate}
\end{ex}

\begin{ex}\label{ex:embedding-cancellable}
  Show that a function $f:A\to B$ is an embedding if and only if the following two conditions hold:
  \begin{enumerate}
  \item $f$ is \emph{left cancellable}, i.e.\ for any $x,y:A$, if $f(x)=f(y)$ then $x=y$.\label{item:ex:ec1}
  \item For any $x:A$, the map $\apfunc f: \Omega(A,x) \to \Omega(B,f(x))$ is an equivalence.\label{item:ex:ec2}
  \end{enumerate}
  (In particular, if $A$ is a set, then $f$ is an embedding if and only if it is left-cancellable and $\Omega(B,f(x))$ is contractible for all $x:A$.)
  Give examples to show that neither of~\ref{item:ex:ec1} or~\ref{item:ex:ec2} implies the other.
\end{ex}

\begin{ex}\label{ex:cancellable-from-bool}
  Show that the type of left-cancellable functions $\bool\to B$ (see \cref{ex:embedding-cancellable}) is equivalent to $\sm{x,y:B}(x\neq y)$.
  Give a similar explicit characterization of the type of embeddings $\bool\to B$.
\end{ex}

\begin{ex}\label{ex:funext-from-nondep}
  The \textbf{na\"{i}ve non-dependent function extensionality axiom} says that for $A,B:\type$ and $f,g:A\to B$ there is a function $(\prd{x:A} f(x)=g(x)) \to (f=g)$.
  \indexdef{function extensionality!non-dependent}%
  Modify the argument of \cref{sec:univalence-implies-funext} to show that this axiom implies the full function extensionality axiom (\cref{axiom:funext}).
\end{ex}

% Local Variables:
% TeX-master: "hott-online"
% End:
