\chapter{形式类型论 (Formal Type Theory)}
\label{cha:rules}

\index{formal!type theory|(形式!类型论|(}%
\index{type theory!formal|(类型论!形式|(}%
\index{rules of type theory|(类型论规则|(}%

正如可以在不显式使用策梅洛-弗兰克尔集合论的公理的情况下在集合论中发展数学一样,在本书中,我们在不明确提及同伦类型论形式系统的情况下,在单值基础上发展了数学。然而,\emph{拥有}对同伦类型论作为一个形式系统的精确定义是很重要的,例如,为了

\begin{itemize}
  \item 陈述并证明其元理论性质,包括逻辑一致性,
  \item 构造模型,例如在单纯集合、模型范畴、更高拓扑斯等中,\item 将其在 \Coq 或 \Agda 等证明助手中实现。
  \index{proof!assistant证明!助手}
\end{itemize}
%
即使是同伦类型论的逻辑一致性\index{consistency一致性},即在空上下文中没有术语 $a:\emptyt$,也是不明显的:如果我们错误地选择了一个定义,使得 $\eqv{\emptyt}{\unit}$,那么单值性将暗示 $\emptyt$ 有一个元素,因为 $\unit$ 有一个元素。我们的 $\Sn^1$ 作为一个高阶归纳类型的定义是否能够表现得像普通的圆,也是不明显的。

在处理这些问题之前,我们必须明确类型论的两个方面。回想一下,在引言中我们提到,类型论包含一组规则,用来规定何时判断 $a:A$ 和 $a\jdeq a':A$ 成立——例如,积的规则是:当 $a:A$ 和 $b:B$ 时,$(a,b):A\times B$。要使这一点精确,我们首先必须精确定义术语的语法——这些判断所涉及的对象 $a,a',A,\dots$;然后,我们必须精确定义判断及其推理规则——即如何从其他判断中推导出判断。

在本附录中,我们将介绍 Martin-L\"{o}f 类型论及构成同伦类型论的扩展的两种表述。第一种表述 (\cref{sec:syntax-informally}) 描述了作为无类型 $\lambda$-演算扩展的术语的语法和判断的形式,而推理规则则保持非正式状态。第二种表述 (\cref{sec:syntax-more-formally}) 以自然演绎的方式递归地定义了术语、判断和推理规则,这在许多类型论文献中是常见的做法。

\section*{预备知识 (Preliminaries)}
\label{sec:formal-prelim}

在 \cref{cha:typetheory} 中,我们介绍了类型论的两个基本\define{判断 (judgments)}\index{judgment判断}。第一个判断 $a:A$ 断言一个术语 $a$ 具有类型 $A$。第二个判断 $a\jdeq b:A$ 断言两个术语 $a$ 和 $b$ 在类型 $A$ 下是\define{判断等同的 (judgmentally equal)}\index{equality!judgmental等同性!判断上的}\index{judgmental equality判断等同性}。这些判断是由 \cref{sec:syntax-more-formally} 中描述的一组推理规则递归定义的。

构造类型 $A$ 的元素 $a$ 就是推导出 $a:A$;在本书中,我们给出了描述 $a$ 构造的非正式论证,但在正式场合中,必须具体指定一个术语 $a$ 以及 $a:A$ 的完整推导。

然而,本书中的类型论表述与本附录中的表述之间的主要区别在于,这里判断是在显式的\define{上下文 (context)}\index{context上下文}中进行表述的,或假设的列表,形式为
\[
  x_1:A_1, x_2:A_2,\dots,x_n:A_n.
\]
上下文中的元素 $x_i : A_i$ 表示变量 $x_i$ 拥有类型 $A_i$ 的假设。上下文中出现的变量 $x_1, \ldots, x_n$ 必须是不同的。我们用字母 $\Gamma$ 和 $\Delta$ 来表示上下文。

上下文 $\Gamma$ 中的判断 $a:A$ 记作
\[ \oftp\Gamma aA \]
表示在 $\Gamma$ 列出的假设下 $a:A$。当假设列表为空时,我们仅写作
\[ \oftp{}aA \]
或
\[ \oftp\emptyctx aA \]
其中 $\emptyctx$ 表示空上下文。相同的适用于等同性判断
\[
  \jdeqtp\Gamma{a}{b}{A}
\]

然而,这些判断只有在\define{上下文良构 (well-formed)}\index{context!well-formed上下文!良构}时才有意义,这一概念由我们最后的判断表达
\[
  \wfctx{(x_1:A_1, x_2:A_2,\dots,x_n:A_n)}
\]
表示在上下文 $x_1:A_1, x_2:A_2,\dots,x_{i-1}:A_{i-1}$ 中,每个 $A_i$ 都是一个类型。因此,特别地,如果 $\oftp\Gamma aA$ 且 $\wfctx\Gamma$,那么我们知道每个 $A_i$ 仅包含变量 $x_1,\dots,x_{i-1}$,而 $a$ 和 $A$ 仅包含变量 $x_1,\dots,x_n$。
\index{variable!in context变量!在上下文中}

在非正式的数学表述中,上下文是隐含的。在证明的每个阶段,数学家都知道哪些变量是可用的,以及它们的类型是什么,要么是基于历史惯例($n$ 通常是一个数,$f$ 是一个函数,等等),要么是因为变量是通过类似“设 $x$ 为一个实数”这样的句子明确引入的。我们将在 \cref{sec:more-formal-pi,sec:more-formal-sigma} 中讨论使用显式上下文的一些好处。

我们用 $B[a/x]$ 表示术语 $a$ 代替术语 $B$ 中自由出现的变量 $x$ 的\define{替换 (substitution)}\index{substitution替换},同时可能会进行避免捕获的绑定变量重命名\index{variable!and substitution变量!与替换}\index{variable!bound变量!绑定的},如 \cref{sec:function-types} 中所讨论。替换的一般形式
%
\[
  B[a_1,\dots,a_n/x_1,\dots,x_n]
\]
%
将表达式 $a_1,\dots,a_n$ 同时替换为变量 $x_1,\dots,x_n$。

\define{在表达式 $B$ 中绑定变量 $x$ (bind a variable $x$ in an expression $B$)}\indexdef{variable!bound变量!绑定}意味着将两者结合到一个更大的表达式中,称为\define{抽象 (abstraction)}\indexdef{abstraction抽象},其目的是表达 $x$ 是“局部的”于 $B$,即不应与其他地方出现的 $x$ 混淆。绑定变量对程序员来说是熟悉的,但对数学家来说却不那么熟悉。绑定有各种记法,例如 $x \mapsto B$、$\lam x B$ 和 $x \,.\, B$,具体情况视场景而定。我们可以用 $C[a]$ 表示在抽象表达式中将术语 $a$ 替换为变量的替换,即我们可以定义 $(x.B)[a]$ 为 $B[a/x]$。如 \cref{sec:function-types} 中所讨论的,改变一个表达式内的绑定变量名(“$\alpha$-转换”)\index{alpha-conversion@$\alpha $-conversion $\alpha$-转换}不会改变表达式。因此,非常精确地说,一个表达式是一个语法形式的等价类,它们在绑定变量名上有所不同。

我们还可以认为判断
\[
  x_1:A_1, x_2:A_2,\dots,x_n:A_n \vdash a : A
\]
中的每个变量 $x_i$ 都绑定在其\define{作用域 (scope)}\indexdef{variable!scope of变量!作用域的}\index{scope作用域}中,该作用域由表达式 $A_{i+1}, \ldots, A_n$、$a$ 和 $A$ 组成。
\section{第一次表述 (The First Presentation)}
\label{sec:syntax-informally}

我们的类型论中的对象和类型可以使用以下语法作为术语来表示,这是一种扩展了$\lambda$-演算的语法,包含了\emph{变量} $x, x',\dots$、
\index{variable变量}%
\emph{原始常量 (primitive constants)}
\index{primitive!constant原始!常量}%
\index{constant!primitive常量!原始的}%
$c,c',\dots$、\emph{定义常量 (defined constants)}\index{constant!defined常量!定义的} $f,f',\dots$ 以及术语形成操作:
%
\[
  t \production x \mid \lam{x} t \mid t(t') \mid c \mid f
\]
%
此处使用的符号表示术语 $t$ 要么是变量 $x$,要么是形式为 $\lam{x} t$ 的表达式,其中 $x$ 是一个变量,$t$ 是一个术语;或者它是形式为 $t(t')$ 的表达式,其中 $t$ 和 $t'$ 都是术语;或者它是一个原始常量 $c$;或者它是一个定义常量 $f$。语法标记 '$\lambda$'、'('、')' 和 '.' 是为了指导人的眼睛的标点符号。

我们使用 $t(t_1,\dots,t_n)$ 作为重复应用 $t(t_1)(t_2)\dots (t_n)$ 的缩写。我们还可以使用\emph{中缀 (infix)}\index{infix notation中缀符号}表示法,当 $\star$ 是一个原始或定义的常量时,写作 $t_1\;\star\;t_2$ 表示 $\star(t_1,t_2)$。

每个定义常量有零个、一个或多个\define{定义方程 (defining equations)}。
\index{equation, defining方程, 定义的}%
\index{defining equation定义方程}%
定义常量有两种类型。显式\index{constant!explicit常量!显式的}定义常量 $f$ 具有一个定义方程
\[ f(x_1,\dots,x_n)\defeq t,\]
其中 $t$ 不涉及 $f$。
%
例如,我们可以引入显式定义常量 $\circ$,其定义方程为
\[ \circ (x,y)(z) \defeq x(y(z)),\]
并使用中缀符号 $x\circ y$ 表示 $\circ(x,y)$。当然,这只是函数的复合。

第二种定义常量用于指定一个(参数化的)映射 $f(x_1,\dots,x_n,x)$,其中 $x$ 范围为元素由零个或多个原始常量生成的类型。对于每个这样的原始常量 $c$,都有一个形式为
\[
  f(x_1,\dots,x_n,c(y_1,\dots,y_m)) \defeq t,
\]
的定义方程,其中 $f$ 可以出现在 $t$ 中,但只能以确定方程定义了一个完全定义的函数的方式出现。这类定义函数的范例是自然数上的原始递归定义的函数。我们可以称这种函数的定义为\emph{完全递归定义 (total recursive definition)}。
\index{total!recursive definition完全!递归定义}%
在计算机科学和逻辑学中,这种在递归数据类型上定义函数的方式被称为\define{结构递归定义 (definition by structural recursion)}。
\index{definition!by structural recursion定义!通过结构递归}%
\index{structural!recursion结构!递归}%
\index{recursion!structural递归!结构的}%

\define{术语的可转换性 (Convertibility)}\index{convertibility of terms术语的可转换性}%
\index{term!convertibility of术语!可转换性}%
$t \conv t'$ 是由常量的定义方程、计算规则\index{computation rule!for function types计算规则!函数类型的}生成的等价关系
%
\[
  (\lam{x} t)(u) \defeq t[u/x],
\]
%
以及使其成为应用和$\lambda$-抽象\index{lambda abstraction@$\lambda$-abstraction $\lambda$-抽象}的\emph{同余}的规则生成的等价关系:
%
\begin{itemize}
  \item 如果 $t \conv t'$ 且 $s \conv s'$,则 $t(s) \conv t'(s')$,并且
  \item 如果 $t \conv t'$,则 $(\lam{x} t) \conv (\lam{x} t')$。
\end{itemize}
\noindent
等同性判断 $t \jdeq u : A$ 可以通过以下单一规则得出:
%
\begin{itemize}
  \item 如果 $t:A$,$u:A$,并且 $t \conv u$,那么 $t \jdeq u : A$。
\end{itemize}
%
判断等同性是一个等价关系。

注意,本节中使用的类型论与正文中使用的类型论在不包含函数的判断唯一性原则 $f \jdeq (\lam{x} f(x))$ 方面有所不同。这种等同性要求判断等同性对所涉及术语的类型敏感,因为这种等同性只有在 $f$ 被认为是一个函数时才有意义,而在本节中的可转换关系与类型无关。\cref{sec:syntax-more-formally}中的第二种表述包含了唯一性原则。


\subsection{类型宇宙 (Type Universes)}

我们假设有一个\define{宇宙 (universes)}\index{type!universe类型!宇宙}的层次结构,由原始常量表示
%
\begin{equation*}
  \UU_0, \quad \UU_1, \quad  \UU_2, \quad \ldots
\end{equation*}
%
关于宇宙的前两个规则表明它们形成了一个累积的类型层次结构:
%
\begin{itemize}
  \item $\UU_m : \UU_n$ 对于 $m < n$,
  \item 如果 $A:\UU_m$ 且 $m \le n$,那么 $A:\UU_n$,
\end{itemize}
%
第三个规则表示宇宙中的一个对象可以作为类型并在判断中出现在冒号的右边:
%
\begin{itemize}
  \item 如果 $\Gamma \vdash A : \UU_n$,并且 $x$ 是一个新变量,\footnote{“新”表示它没有出现在 $\Gamma$ 或 $A$ 中。}那么 $\vdash (\Gamma, x:A)\; \ctx$。
\end{itemize}
%
在正文中,类型 $A$ 和 $B$ 之间的等同性判断 $A \jdeq B : \UU_n$ 通常简写为 $A \jdeq B$。这是一种典型的歧义\index{typical ambiguity典型的歧义},因为我们可以随时切换到更大的宇宙,但这不会影响判断的有效性。

以下转换规则允许我们在类型判断中用一个类型替换为与之等同的类型:
%
\begin{itemize}
  \item 如果 $a:A$ 且 $A \jdeq B$,那么 $a:B$。
\end{itemize}

\subsection{依赖函数类型 (\texorpdfstring{$\Pi$}{Π}-types)}

我们引入一个原始常量 $c_\Pi$,但将 $c_\Pi(A,\lam{x} B)$ 写作 $\tprd{x:A}B$。通过以下规则引入了关于此类表达式和形式为 $\lam{x} b$ 的表达式的判断:
%
\begin{itemize}
  \item 如果 $\Gamma \vdash A:\UU_n$ 且 $\Gamma,x:A \vdash B:\UU_n$,那么 $\Gamma \vdash \tprd{x:A}B : \UU_n$
  \item 如果 $\Gamma, x:A \vdash b:B$,那么 $\Gamma \vdash (\lam{x} b) : (\tprd{x:A} B)$
  \item 如果 $\Gamma\vdash g:\tprd{x:A} B$ 且 $\Gamma\vdash t:A$,那么 $\Gamma\vdash g(t):B[t/x]$
\end{itemize}
%
如果 $x$ 在 $B$ 中不是自由的,我们将 $\tprd{x:A} B$ 简写为非依赖函数类型 $A\rightarrow B$,并推导出以下规则:
%
\begin{itemize}
  \item 如果 $\Gamma\vdash g:A \rightarrow B$ 且 $\Gamma\vdash t:A$,那么 $\Gamma\vdash g(t):B$
\end{itemize}
使用非依赖函数类型并隐式保留上下文 $\Gamma$,可以将上述规则写成以下替代风格,我们将在附录的本节其余部分中使用:
%
\begin{itemize}
  \item 如果 $A:\UU_n$ 且 $B:A\to\UU_n$,那么 $\tprd{x:A}B(x) : \UU_n$
  \item 如果 $x:A \vdash b:B(x)$ 那么 $ \lam{x} b : \tprd{x:A} B(x)$
  \item 如果 $g:\tprd{x:A} B(x)$ 且 $t:A$ 那么 $g(t):B(t)$
\end{itemize}
%

\subsection{依赖对类型 (\texorpdfstring{$\Sigma$}{Σ}-types)}

我们引入原始常量 $c_\Sigma$ 和 $c_{\mathsf{pair}}$。形式为 $c_\Sigma(A,\lam{a} B)$ 的表达式写作 $\sm{a:A}B$,
形式为 $c_{\mathsf{pair}}(a,b)$ 的表达式写作 $\tup a b$。如果 $x$ 在 $B$ 中不是自由的,我们将 $\sm{x:A} B$ 写作 $A\times B$。

通过以下规则引入了关于此类表达式的判断:
%
\begin{itemize}
  \item 如果 $A:\UU_n$ 且 $B: A \rightarrow \UU_n$,那么 $\sm{x:A}B(x) : \UU_n$
  \item 此外,如果 $a:A$ 且 $b:B(a)$,那么 $\tup a b:\sm{x:A}B(x)$
\end{itemize}
%
如果我们有如上的 $A$ 和 $B$,$C : (\sm{x:A}B(x)) \rightarrow \UU_m$,并且
\[
  d:\tprd{x:A}{y:B(x)} C(\tup x y)
\]
我们可以引入一个定义常量
\[
  f:\tprd{p:\sm{x:A}B(x)} C(p)
\]
其定义方程为
\[
  f(\tup x y)\defeq d(x,y)。
\]
%
注意,$C$、$d$、$x$ 和 $y$ 可能包含一些额外的隐式参数 $x_1,\ldots,x_n$,如果它们是在某个非空上下文中获得的;因此,完全显式的递归模式为
%
\begin{narrowmultline*}
  f(x_1,\dots,x_n,\tup{x(x_1,\dots,x_n)}{y(x_1,\dots,x_n)}) \defeq
  \narrowbreak
  d(x_1,\dots,x_n,\tup{x(x_1,\dots,x_n)}{y(x_1,\dots,x_n)})。
\end{narrowmultline*}

\subsection{余积类型 (Coproduct Types)}

我们引入原始常量 $c_+$、$c_\inlsym$ 和 $c_\inrsym$。我们用 $A+B$ 代替 $c_+(A,B)$,用 $\inl(a)$ 代替 $c_\inlsym(a)$,用 $\inr(a)$ 代替 $c_\inrsym(a)$:
%
\begin{itemize}
  \item 如果 $A,B : \UU_n$,则 $A + B : \UU_n$
  \item 此外,$\inl: A \rightarrow A+B$ 和 $\inr: B \rightarrow A+B$
\end{itemize}
%
如果我们有上述的 $A$ 和 $B$,$C : A+B \rightarrow \UU_m$,
$d:\tprd{x:A} C(\inl(x))$ 和 $e:\tprd{y:B} C(\inr(y))$,
那么我们可以引入一个定义常量 $f:\tprd{z:A+B}C(z)$,其定义方程为
%
\begin{equation*}
  f(\inl(x)) \defeq d(x)
  \qquad\text{和}\qquad
  f(\inr(y)) \defeq e(y)。
\end{equation*}

\subsection{有限类型 (The Finite Types)}

我们引入原始常量 $\ttt$、$\emptyt$ 和 $\unit$,满足以下规则:
%
\begin{itemize}
  \item $\emptyt : \UU_0$,$\unit : \UU_0$
  \item $\ttt:\unit$
\end{itemize}

给定 $C : \emptyt \rightarrow \UU_n$,我们可以引入一个定义常量 $f:\tprd{x:\emptyt} C(x)$,没有定义方程。

给定 $C : \unit \rightarrow \UU_n$ 和 $d : C(\ttt)$,我们可以引入一个定义常量 $f:\tprd{x:\unit} C(x)$,其定义方程为 $f(\ttt) \defeq d$。

\subsection{自然数 (Natural Numbers)}

自然数的类型通过引入原始常量 $\N$、$0$ 和 $\suc$ 获得,并满足以下规则:
%
\begin{itemize}
  \item $\N : \UU_0$,
  \item $0:\N$,
  \item $\suc:\N\rightarrow \N$。
\end{itemize}
%
此外,我们可以通过原始递归定义函数。如果我们有 $C : \N \rightarrow \UU_k $,当我们有
%
\begin{align*}
  d & : C(0) \\
  e & : \tprd{x:\N}(C(x)\rightarrow C(\suc (x)))
\end{align*}
%
时,我们可以引入定义常量 $f:\tprd{x:\N}C(x)$,其定义方程为
%
\begin{equation*}
  f(0) \defeq d
  \qquqquad\text{和}\qquad
  f(\suc (x)) \defeq e(x,f(x))。
\end{equation*}

\subsection{\texorpdfstring{$W$}{W}-类型 (\texorpdfstring{$W$}{W}-Types)}

对于 $W$-类型,我们引入原始常量 $c_\wtypesym$ 和 $c_\suppsym$。
形式为 $c_\wtypesym(A,\lam{x} B)$ 的表达式写作
$\wtype{x:A}B$,形式为 $c_\suppsym(x,u)$ 的表达式写作
$\supp(x,u)$:
%
\begin{itemize}
  \item 如果 $A:\UU_n$ 和 $B: A \rightarrow \UU_n$,则 $\wtype{x:A}B(x) : \UU_n$
  \item 此外,如果 $a:A$ 和 $u:B(a)\rightarrow \wtype{x:A}B(x)$,则 $\supp(a,u):\wtype{x:A}B(x)$。
\end{itemize}
%
在这里我们也可以通过完全递归定义函数。如果我们有上述的 $A$ 和 $B$,以及 $C : (\wtype{x:A}B(x)) \rightarrow \UU_m$,那么我们可以引入一个定义常量
$f:\tprd{z:\wtype{x:A}B(x)} C(z)$,当我们有
\[
  d:\tprd{a:A}{u:B(a) \rightarrow \wtype{x:A}B(x)}((\tprd{y:B(a)}C(u(y))) \rightarrow C(\supp(a,u)))
\]
时,定义方程为
\[
  f(\supp(a,u)) \defeq d(a,u,f\circ u)。
\]

\subsection{等同类型 (Identity Types)}

我们引入原始常量 $c_\idsym$ 和 $c_{\refl{}}$。当 $a:A$ 被理解时,我们将 $\id[A] a b$ 写作 $c_\idsym(A,a,b)$,将 $\refl a$ 写作 $c_{\refl{}}(A,a)$:
%
\begin{itemize}
  \item 如果 $A : \UU_n$,$a:A$ 和 $b:A$,那么 $\id[A] a b : \UU_n$。
  \item 如果 $a:A$,那么 $\refl a :\id[A] a a $。
\end{itemize}
%
给定 $a:A$,如果 $y:A, z:\id[A] a y \vdash C : \UU_m$ 和
$\vdash d:C[a,\refl{a}/y,z]$,那么我们可以引入一个定义常量
\[
  f:\tprd{y:A}{z:\id[A] a y} C
\]
其定义方程为
\[
  f(a,\refl{a})\defeq d。
\]

\section{第二次表述 (The Second Presentation)}
\label{sec:syntax-more-formally}

在本节中,有三种类型的判断:
\begin{mathpar}
  \wfctx\Gamma
  \and
  \oftp\Gamma{a}{A}
  \and
  \jdeqtp\Gamma{a}{a'}{A}
\end{mathpar}
我们通过提供推理规则来指定它们。一个典型的\define{推理规则 (inference rule)}\indexsee{inference rule}{rule规则}%
\indexdef{rule规则}%
形式如下:
%
\begin{equation*}
  \inferrule*[right=\textsc{Name}]
  {\mathcal{J}_1 \\ \cdots \\ \mathcal{J}_k}
  {\mathcal{J}}
\end{equation*}
%
它表示我们可以得出\define{结论 (conclusion)}$\mathcal{J}$,前提是我们已经得出了\define{前提 (hypotheses)}$\mathcal{J}_1, \ldots, \mathcal{J}_k$。
(注意,由于这些是判断而不是类型,它们不是\emph{内部}于类型论的假设,正如\cref{sec:types-vs-sets}中提到的;它们是推理系统中的假设,即元理论中的假设。)
在右边我们写下规则的\textsc{Name},并且在应用该规则之前可能需要检查一些额外的附加条件。

判断的\define{推导 (derivation)}\index{derivation推导}%
是一棵由这种推理规则构建的树,树的根节点是判断。例如,使用下面给出的规则,以下是$\oftp{\emptyctx}{\lamu{x:\unit} x}{\unit\to\unit}$的推导过程。
%
\begin{mathpar}
  \inferrule*[right=$\Pi$-\rintro]
  {\inferrule*[right=$\Vble$]
  {\inferrule*[right=\ctx-\textsc{ext}]
  {\inferrule*[right=$\unit$-\rform]
  {\inferrule*[right=\ctx-\textsc{emp}]
  {\ }
  {\wfctx {\emptyctx}}}
  {\oftp{}{\unit}{\UU_0}}}
  {\wfctx {\tmtp x\unit}}}
  {\oftp{\tmtp x\unit}{x}{\unit}}}
  {\oftp{\emptyctx}{\lamu{x:\unit} x}{\unit\to\unit}}
\end{mathpar}

\subsection{上下文 (Contexts)}
\label{subsec:contexts}

\index{context上下文}%
上下文是一个列表:
%
\begin{equation*}
  \tmtp{x_1}{A_1}, \tmtp{x_2}{A_2}, \ldots, \tmtp{x_n}{A_n}
\end{equation*}
%
它表示不同的变量\index{variable变量}%
$x_1, \ldots, x_n$ 分别假设具有类型 $A_1, \ldots, A_n$。该列表可以为空。我们用字母 $\Gamma$ 和 $\Delta$ 来缩写上下文,并且可以通过将它们并置来形成更大的上下文。

判断 $\wfctx{\Gamma}$ 正式表示 $\Gamma$ 是一个良构上下文,并且受以下推理规则的约束:
%
\begin{mathpar}
  \inferrule*[right=\ctx-\textsc{emp}]
  {\ }
  {\wfctx\emptyctx}
  \and
  \inferrule*[right=\ctx-\textsc{ext}]
  {\oftp{\tmtp{x_1}{A_1}, \ldots, \tmtp{x_{n-1}}{A_{n-1}}}{A_n}{\UU_i}}
  {\wfctx{(\tmtp{x_1}{A_1}, \ldots, \tmtp{x_n}{A_n})}}
\end{mathpar}
%
第二个规则的附加条件是:变量 $x_n$ 必须与变量 $x_1, \ldots, x_{n-1}$ 不同。
注意,$\ctx$-\textsc{ext}的前提和结论是不同形式的判断:前提表示在变量 $x_1, \ldots, x_{n-1}$ 的上下文中,表达式 $A_n$ 具有类型 $\UU_i$;而结论表示扩展后的上下文 $(\tmtp{x_1}{A_1}, \ldots, \tmtp{x_n}{A_n})$ 是良构的。

这是系统的元理论性质,即如果可以推导出任何形式为 $\oftp{\Gamma}{a}{A}$ 或 $\jdeqtp\Gamma{a}{a'}{A}$ 的判断,那么也可以推导出上下文 $\Gamma$ 是良构的判断 $\wfctx\Gamma$。
所有规则的前提都被选择为包含足够的良构性假设,以使该性质可证明,但没有更多的假设。
例如,不需要在 $\ctx$-\textsc{ext} 中假设 $(\tmtp{x_1}{A_1}, \ldots, \tmtp{x_{n-1}}{A_{n-1}})$ 是良构的,因为这将从其前提的可推导性中得出;但在下一节中的 $\Vble$ 规则中必须假设上下文是良构的。
这种选择只是精确制定类型论的许多可能方法之一,但详细研究这些问题超出了本附录的范围。

\subsection{结构规则 (Structural Rules)}

\index{structural!rules结构!规则|(}%
\index{rule!structural规则!结构的|(}%

上下文中持有假设的事实通过以下规则表达,该规则表示我们可以推导出上下文中列出的那些类型判断:
%
\begin{mathpar}
  \inferrule*[right=$\Vble$]
  {\wfctx {(\tmtp{x_1}{A_1}, \ldots, \tmtp{x_n}{A_n})} }
  {\oftp{\tmtp{x_1}{A_1}, \ldots, \tmtp{x_n}{A_n}}{x_i}{A_i}}
\end{mathpar}
%
与 $\ctx$-\textsc{ext} 一样,规则 $\Vble$ 的前提和结论是不同形式的判断,只不过现在它们是反过来的:我们从一个良构的上下文开始,并推导出一个类型判断。

以下重要原则,称为\define{替换 (substitution)}\indexdef{rule!of substitution规则!替换的}%
和\define{弱化 (weakening)}\indexdef{rule!of weakening规则!弱化的}%
,不需要显式假设。相反,可以通过对所有可能推导的结构进行归纳证明,每当这些规则的前提是可推导的,其结论也是可推导的。\footnote{这些规则被称为\define{可容许的规则 (admissible)}\indexdef{rule!admissible规则!可容许的}\indexsee{admissible!rule可容许的!规则}{rule, admissible规则, 可容许的}。}
对于类型判断,这些原则表现为:
%
\begin{mathpar}
  \inferrule*[right=$\Subst_1$]
  {\oftp\Gamma{a}{A} \\ \oftp{\Gamma,\tmtp xA,\Delta}{b}{B}}
  {\oftp{\Gamma,\Delta[a/x]}{b[a/x]}{B[a/x]}}
  \and
  \inferrule*[right=$\Weak_1$]
  {\oftp\Gamma{A}{\UU_i} \\ \oftp{\Gamma,\Delta}{b}{B}}
  {\oftp{\Gamma,\tmtp xA,\Delta}{b}{B}}
\end{mathpar}
对于判断等同性,它们表现为:
\begin{mathpar}
  \inferrule*[right=$\Subst_2$]
  {\oftp\Gamma{a}{A} \\ \jdeqtp{\Gamma,\tmtp xA,\Delta}{b}{c}{B}}
  {\jdeqtp{\Gamma,\Delta[a/x]}{b[a/x]}{c[a/x]}{B[a/x]}}
  \and
  \inferrule*[right=$\Subst_3$]
  {\jdeqtp\Gamma{a}{b}{A} \\ \oftp{\Gamma,\tmtp xA,\Delta}{c}{C}}
  {\jdeqtp{\Gamma,\Delta[a/x]}{c[a/x]}{c[b/x]}{C[a/x]}}
  \and
  \inferrule*[right=$\Weak_2$]
  {\oftp\Gamma{A}{\UU_i} \\ \jdeqtp{\Gamma,\Delta}{b}{c}{B}}
  {\jdeqtp{\Gamma,\tmtp xA,\Delta}{b}{c}{B}}
\end{mathpar}
%
除了为每个类型构造器给出的判断等同性规则外,我们还假设判断等同性是一个由类型支持的等价关系。
\begin{mathparpagebreakable}
  \inferrule*{\oftp\Gamma{a}{A}}{\jdeqtp\Gamma{a}{a}{A}}
  \and
  \inferrule*{\jdeqtp\Gamma{a}{b}{A}}{\jdeqtp\Gamma{b}{a}{A}}
  \and
  \inferrule*{\jdeqtp\Gamma{a}{b}{A} \\ \jdeqtp\Gamma{b}{c}{A}}{\jdeqtp\Gamma{a}{c}{A}}
  \and
  \inferrule*{\oftp\Gamma{a}{A} \\ \jdeqtp\Gamma{A}{B}{\UU_i}}{\oftp\Gamma{a}{B}}
  \and
  \inferrule*{\jdeqtp\Gamma{a}{b}{A} \\ \jdeqtp\Gamma{A}{B}{\UU_i}}{\jdeqtp\Gamma{a}{b}{B}}
\end{mathparpagebreakable}
%
最后,我们假设判断等同性是一个由类型支持的同余关系,即每个类型和术语构造器在其所有参数中保持判断等同性。例如,与$\Pi$-\rintro规则一起,我们假设规则
\[
  \inferrule*[right=$\Pi$-\rintro-eq]
  {\oftp\Gamma{A}{\UU_i} \\
  \oftp{\Gamma,\tmtp xA}{B}{\UU_i} \\
  \jdeqtp{\Gamma,\tmtp xA}{b}{b'}{B}}
  {\jdeqtp\Gamma{\lamu{x:A} b}{\lamu{x:A'} b'}{\tprd{x:A} B}}
\]
完成依赖函数类型的情况,另外两条相似的规则为$\Pi$-\textsc{form-eq}和$\Pi$-\textsc{elim-eq}。
这些局部原则(在每个类型上)一起蕴含了上述的全局同余原则$\Subst_2$和$\Subst_3$。为了简洁起见,我们将省略这些局部规则。

\index{rule!structural规则!结构的|)}%
\index{structural!rules结构!规则|)}%

\subsection{类型宇宙 (Type Universes)}

\index{type!universe类型!宇宙}%

我们假设一个无限层级的类型宇宙:
%
\begin{equation*}
  \UU_0, \quad \UU_1, \quad  \UU_2, \quad \ldots
\end{equation*}
%
每个宇宙都包含在下一个宇宙中,$\UU_i$ 中的任何类型也在 $\UU_{i+1}$ 中:
%
\begin{mathpar}
  \inferrule*[right=\UU-\textsc{intro}]
  {\wfctx \Gamma }
  {\oftp\Gamma{\UU_i}{\UU_{i+1}}}
  \and
  \inferrule*[right=\UU-\textsc{cumul}]
  {\oftp\Gamma{A}{\UU_i}}
  {\oftp\Gamma{A}{\UU_{i+1}}}
\end{mathpar}
%
我们将设置类型论的规则,使得 $\oftp\Gamma{a}{A}$ 蕴含 $\oftp\Gamma{A}{\UU_i}$ 对某个 $i$ 成立。换句话说,如果 $A$ 扮演类型的角色,那么它位于某个宇宙中。我们系统的另一个性质是 $\jdeqtp\Gamma{a}{b}{A}$ 蕴含 $\oftp\Gamma{a}{A}$ 和 $\oftp\Gamma{b}{A}$。

\subsection{依赖函数类型 (\texorpdfstring{$\Pi$}{Π}-types)}
\label{sec:more-formal-pi}

\index{type!dependent function类型!依赖函数}%
\index{type!function类型!函数}%

在\cref{sec:function-types}中,我们引入了非依赖函数$A\to B$,以定义类型族作为函数$\lam{x:A} B:A\to\UU_i$,进而产生依赖函数类型$\tprd{x:A} B$。但通过显式上下文,我们可以将$\lam{x:A} B:A\to\UU_i$替换为判断
%
\begin{equation*}
  \oftp{\tmtp xA}{B}{\UU_i}。
\end{equation*}
%
因此,我们可以直接定义依赖函数,而不需要参考非依赖函数。这样我们遵循了每个类型构造器都应独立于其他类型构造器引入的总体原则。
%
实际上,从现在起,每个类型构造器将通过以下方式系统地引入:
\begin{itemize}
  \item \define{构造规则 (formation rule)},说明何时可以应用类型构造器;
  \index{formation rule构造规则}\index{rule!formation规则!构造}
  \item 一些\define{引入规则 (introduction rules)},说明如何构造该类型的元素;
  \index{introduction rule引入规则}\index{rule!introduction规则!引入}
  \item \define{消去规则 (elimination rules)},或归纳原理,说明如何使用该类型的元素;
  \index{induction principle归纳原理}\index{eliminator消去}
  \item \define{计算规则 (computation rules)},这些是判断等同性,解释当对引入规则的结果应用消去规则时会发生什么;
  \index{computation rule计算规则}
  \indexsee{rule!computation规则!计算}{computation rule计算规则}
  \item 可选的\define{唯一性原则 (uniqueness principles)},这些是判断等同性,解释如何通过对该类型的元素应用消去规则来唯一地确定它们。
  \index{uniqueness!principle唯一性!原则}
  \indexsee{principle!uniqueness原则!唯一性}{uniqueness principle唯一性原则}
\end{itemize}
(另见\cref{rmk:introducing-new-concepts}。)

对于依赖函数类型,这些规则为:
%
\begin{mathparpagebreakable}
  \def\premise{\oftp{\Gamma}{A}{\UU_i} \and \oftp{\Gamma,\tmtp xA}{B}{\UU_i}}
  \inferrule*[right=$\Pi$-\rform]
  \premise
  {\oftp\Gamma{\tprd{x:A}B}{\UU_i}}
  \and
  \inferrule*[right=$$\Pi$-\rintro]
    {\oftp{\Gamma,\tmtp xA}{b}{B}}
    {\oftp\Gamma{\lam{x:A} b}{\tprd{x:A} B}}
  \and
  \inferrule*[right=$\Pi$-\relim]
    {\oftp\Gamma{f}{\tprd{x:A} B} \\ \oftp\Gamma{a}{A}}
    {\oftp\Gamma{f(a)}{B[a/x]}}
  \and
  \inferrule*[right=$\Pi$-\rcomp]
    {\oftp{\Gamma,\tmtp xA}{b}{B} \\ \oftp\Gamma{a}{A}}
    {\jdeqtp\Gamma{(\lam{x:A} b)(a)}{b[a/x]}{B[a/x]}}
  \and
  \inferrule*[right=$$\Pi$-\runiq]
  {\oftp\Gamma{f}{\tprd{x:A} B}}
  {\jdeqtp\Gamma{f}{(\lamu{x:A}f(x))}{\tprd{x:A} B}}
\end{mathparpagebreakable}

表达式$\lam{x:A} b$在$b$中绑定$x$的自由出现,同样$\tprd{x:A} B$也在$B$中绑定$x$。

当$x$在$B$中不自由出现,从而$B$不依赖于$A$时,我们得到一个特例,即普通函数类型$A\to B \defeq \tprd{x:A} B$。我们将其视为$\to$的\emph{定义}。

我们可以将表达式$\lam{x:A} b$缩写为$\lamu{x:A} b$,理解为在类型检查之前,应该适当地填入省略的类型$A$。

\subsection{依赖对类型 (\texorpdfstring{$\Sigma$}{Σ}-types)}
\label{sec:more-formal-sigma}

\index{type!dependent pair类型!依赖对}%
\index{type!product类型!积}%

在\cref{sec:sigma-types}中,我们需要$\to$和$\prdsym$类型,以定义$\smsym$的引入和消去规则;与$\prdsym$一样,上下文允许我们独立地陈述$\smsym$的规则。回顾一下,对于正类型如$\Sigma$,其消去规则称为\emph{归纳 (induction)},用$\ind{}$表示。
%
\begin{mathparpagebreakable}
  \def\premise{\oftp{\Gamma}{A}{\UU_i} \and \oftp{\Gamma,\tmtp xA}{B}{\UU_i}}
  \inferrule*[right=$\Sigma$-\rform]
  \premise
  {\oftp\Gamma{\tsm{x:A} B}{\UU_i}}
  \and
  \inferrule*[right=$\Sigma$-\rintro]
  {\oftp{\Gamma, \tmtp x A}{B}{\UU_i} \\
  \oftp\Gamma{a}{A} \\ \oftp\Gamma{b}{B[a/x]}}
  {\oftp\Gamma{\tup ab}{\tsm{x:A} B}}
  \and
  \inferrule*[right=$\Sigma$-\relim]
  {\oftp{\Gamma, \tmtp z {\tsm{x:A} B}}{C}{\UU_i} \\
  \oftp{\Gamma,\tmtp x A,\tmtp y B}{g}{C[\tup x y/z]} \\
  \oftp\Gamma{p}{\tsm{x:A} B}}
  {\oftp\Gamma{\ind{\tsm{x:A} B}(z.C,x.y.g,p)}{C[p/z]}}
  \and
  \inferrule*[right=$\Sigma$-\rcomp]
  {\oftp{\Gamma, \tmtp z {\tsm{x:A} B}}{C}{\UU_i} \\
  \oftp{\Gamma, \tmtp x A, \tmtp y B}{g}{C[\tup x y/z]} \\\\
  \oftp\Gamma{a}{A} \\ \oftp\Gamma{b}{B[a/x]}}
  {\jdeqtp\Gamma{\ind{\tsm{x:A} B}(z.C,x.y.g,\tup{a}{b})}{g[a,b/x,y]}{C[\tup {a} {b}/z]}}
\end{mathparpagebreakable}
%
表达式$\tsm{x:A} B$在$B$中绑定$x$的自由出现。此外,由于$\ind{\tsm{x:A} B}$有一些超出$\Gamma$的自由变量参与的参数,我们在$C$中绑定$z$,在$g$中绑定$x$和$y$。这些绑定写作$z.C$和$x.y.g$,以指示被绑定变量的名称。
\index{variable!bound变量!绑定的}%
特别地,我们将$\ind{\tsm{x:A} B}$视为一个原语,其中的两个参数包含绑定变量;这表面上类似于,但不同于,$\ind{\tsm{x:A} B}$是一个以函数作为参数的函数。

当$B$不包含$x$的自由出现时,我们得到一个特例,即笛卡尔积$A \times B \defeq \tsm{x:A} B$。我们将其视为笛卡尔积的\emph{定义}。

注意,我们没有为$\Sigma$-类型假定一个判断的唯一性原则,即使我们可以这样做;参见\cref{thm:eta-sigma}以获取对应命题唯一性原则的证明。

\subsection{余积类型 (Coproduct Types)}

\index{type!coproduct类型!余积}%

\begin{mathparpagebreakable}
  \inferrule*[right=$+$-\rform]
  {\oftp\Gamma{A}{\UU_i} \\ \oftp\Gamma{B}{\UU_i}}
  {\oftp\Gamma{A+B}{\UU_i}}
  \\
  \inferrule*[right=$+$-\rintro${}_1$]
  {\oftp\Gamma{A}{\UU_i} \\ \oftp\Gamma{B}{\UU_i} \\\\ \oftp\Gamma{a}{A}}
  {\oftp\Gamma{\inl(a)}{A+B}}
  \and
  \inferrule*[right=$+$-\rintro${}_2$]
  {\oftp\Gamma{A}{\UU_i} \\ \oftp\Gamma{B}{\UU_i} \\\\ \oftp\Gamma{b}{B}}
  {\oftp\Gamma{\inr(b)}{A+B}}
  \\
  \inferrule*[right=$+$-\relim]
  {\oftp{\Gamma,\tmtp z{(A+B)}}{C}{\UU_i} \\\\
  \oftp{\Gamma,\tmtp xA}{c}{C[\inl(x)/z]} \\
  \oftp{\Gamma,\tmtp yB}{d}{C[\inr(y)/z]} \\\\
  \oftp\Gamma{e}{A+B}}
  {\oftp\Gamma{\ind{A+B}(z.C,x.c,y.d,e)}{C[e/z]}}
  \and
  \inferrule*[right=$+$-\rcomp${}_1$]
  {\oftp{\Gamma,\tmtp z{(A+B)}}{C}{\UU_i} \\
  \oftp{\Gamma,\tmtp xA}{c}{C[\inl(x)/z]} \\
  \oftp{\Gamma,\tmtp yB}{d}{C[\inr(y)/z]} \\\\
  \oftp\Gamma{a}{A}}
  {\jdeqtp\Gamma{\ind{A+B}(z.C,x.c,y.d,\inl(a))}{c[a/x]}{C[\inl(a)/z]}}
  \and
  \inferrule*[right=$+$-\rcomp${}_2$]
  {\oftp{\Gamma,\tmtp z{(A+B)}}{C}{\UU_i} \\
  \oftp{\Gamma,\tmtp xA}{c}{C[\inl(x)/z]} \\
  \oftp{\Gamma,\tmtp yB}{d}{C[\inr(y)/z]} \\\\
  \oftp\Gamma{b}{B}}
  {\jdeqtp\Gamma{\ind{A+B}(z.C,x.c,y.d,\inr(b))}{d[b/y]}{C[\inr(b)/z]}}
\end{mathparpagebreakable}
%
在$\ind{A+B}$中,$z$在$C$中绑定,$x$在$c$中绑定,$y$在$d$中绑定。

\subsection{空类型 \texorpdfstring{$\emptyt$}{0} (The Empty Type \texorpdfstring{$\emptyt$}{0})}

\index{type!empty类型!空|(}%

\begin{mathparpagebreakable}
  \inferrule*[right=$\emptyt$-\rform]
  {\wfctx\Gamma}
  {\oftp\Gamma\emptyt{\UU_i}}
  \and
  \inferrule*[right=$\emptyt$-\relim]
  {\oftp{\Gamma,\tmtp x\emptyt}{C}{\UU_i} \\ \oftp\Gamma{a}{\emptyt}}
  {\oftp\Gamma{\ind{\emptyt}(x.C,a)}{C[a/x]}}
\end{mathparpagebreakable}
%
在$\ind{\emptyt}$中,$x$在$C$中绑定。空类型没有引入规则和计算规则。

\index{type!empty类型!空|)}%

\subsection{单位类型 \texorpdfstring{$\unit$}{1} (The Unit Type \texorpdfstring{$\unit$}{1})}
\label{sec:more-formal-unit}

\index{type!unit类型!单位|(}%

\begin{mathparpagebreakable}
  \inferrule*[right=$\unit$-\rform]
  {\wfctx\Gamma}
  {\oftp\Gamma\unit{\UU_i}}
  \and
  \inferrule*[right=$\unit$-\rintro]
  {\wfctx\Gamma}
  {\oftp\Gamma{\ttt}{\unit}}
  \and
  \inferrule*[right=$\unit$-\relim]
  {\oftp{\Gamma,\tmtp x\unit}{C}{\UU_i} \\
  \oftp{\Gamma}{c}{C[\ttt/x]} \\
  \oftp\Gamma{a}{\unit}}
  {\oftp\Gamma{\ind{\unit}(x.C,c,a)}{C[a/x]}}
  \and
  \inferrule*[right=$\unit$-\rcomp]
  {\oftp{\Gamma,\tmtp x\unit}{C}{\UU_i} \\
  \oftp{\Gamma}{c}{C[\ttt/x]}}
  {\jdeqtp\Gamma{\ind{\unit}(x.C,c,\ttt)}{c}{C[\ttt/x]}}
\end{mathparpagebreakable}
%
在$\ind{\unit}$中,变量$x$在$C$中绑定。

注意,我们没有为单位类型假设判断的唯一性原则;参见\cref{sec:finite-product-types}以获取对应命题唯一性陈述的证明。

\index{type!unit类型!单位|)}%

\subsection{自然数类型 (The Natural Number Type)}

\index{natural numbers自然数|(}%

我们给出自然数的规则,遵循\cref{sec:inductive-types}。

\begin{mathparpagebreakable}
  \def\premise{
    \oftp{\Gamma,\tmtp x{\N}}{C}{\UU_i} \\
    \oftp\Gamma{c_0}{C[0/x]} \\
    \oftp{\Gamma,\tmtp{x}\N,\tmtp y C}{c_s}{C[\suc(x)/x]}}
  %
  \inferrule*[right=$\N$-\rform]
  {\wfctx\Gamma}
  {\oftp\Gamma{\N}{\UU_i}}
  \and
  \inferrule*[right=$\N$-\rintro${}_1$]
  {\wfctx\Gamma}
  {\oftp\Gamma{0}{\N}}
  \and
  \inferrule*[right=$\N$-\rintro${}_2$]
  {\oftp\Gamma{n}{\N}}
  {\oftp\Gamma{\suc(n)}{\N}}
  \and
  \inferrule*[right=$\N$-\relim]
  {\premise \\ \oftp\Gamma{n}{\N}}
  {\oftp\Gamma{\ind{\N}(x.C,c_0,x.y.c_s,n)}{C[n/x]}}
  \and
  \inferrule*[right=$\N$-\rcomp${}_1$]
  {\premise}
  {\jdeqtp\Gamma{\ind{\N}(x.C,c_0,x.y.c_s,0)}{c_0}{C[0/x]}}
  \and
  \inferrule*[right=$$\N$-\rcomp${}_2$]
    {\premise \\ \oftp\Gamma{n}{\N}}
    {\Gamma\vdash
  {\begin{aligned}[t]
     &\ind{\N}(x.C,c_0,x.y.c_s,\suc(n)) \\
     &\quad \jdeq c_s[n,\ind{\N}(x.C,c_0,x.y.c_s,n)/x,y] : C[\suc(n)/x]
  \end{aligned}}}
\end{mathparpagebreakable}
%
在$\ind{\N}$中,$x$在$C$中绑定,$x$和$y$在$c_s$中绑定。

其他归纳定义的类型遵循相同的通用方案。

\index{natural numbers自然数|)}%

\subsection{恒等类型 (Identity Types)}

\label{sec:more-formal-identity}

\index{type!identity类型!恒等|(}%

这里的表述对应于\cref{sec:identity-types}中恒等类型的(非基)路径归纳原理。

\begin{mathparpagebreakable}
\inferrule*[right=$\idsym$-\rform]
{\oftp\Gamma{A}{\UU_i} \\ \oftp\Gamma{a}{A} \\ \oftp\Gamma{b}{A}}
{\oftp\Gamma{\id[A]{a}{b}}{\UU_i}}
\and
\inferrule*[right=$\idsym$-\rintro]
{\oftp\Gamma{A}{\UU_i} \\ \oftp\Gamma{a}{A}}
{\oftp\Gamma{\refl a}{\id[A]aa}}
\and
\inferrule*[right=$\idsym$-\relim]
{\oftp{\Gamma,\tmtp xA,\tmtp yA,\tmtp p{\id[A]xy}}{C}{\UU_i} \\
\oftp{\Gamma,\tmtp zA}{c}{C[z,z,\refl z/x,y,p]} \\
\oftp\Gamma{a}{A} \\ \oftp\Gamma{b}{A} \\ \oftp\Gamma{p'}{\id[A]ab}}
{\oftp\Gamma{\indid{A}(x.y.p.C,z.c,a,b,p')}{C[a,b,p'/x,y,p]}}
\and
\inferrule*[right=$$\idsym$-\rcomp]
{\oftp{\Gamma,\tmtp xA,\tmtp yA,\tmtp p{\id[A]xy}}{C}{\UU_i} \\
\oftp{\Gamma,\tmtp zA}{c}{C[z,z,\refl z/x,y,p]} \\
\oftp\Gamma{a}{A}}
{\jdeqtp\Gamma{\indid{A}(x.y.p.C,z.c,a,a,\refl a)}{c[a/z]}{C[a,a,\refl a/x,y,p]}}
\end{mathparpagebreakable}
%
在$\indid{A}$中,$x$,$y$,和$p$在$C$中绑定,$z$在$c$中绑定。

\index{type!identity类型!恒等|)}%

\subsection{定义 (Definitions)}

\index{definition定义}%

尽管我们列出的规则已经允许我们直接构造所需的一切,但我们仍然希望能够使用命名常量,如$\isequiv$,作为便利手段。非正式地,我们可以将这些常量视为简写,但在形式化中情况稍微复杂一些。

例如,考虑函数组合,它将$f:A\to B$和$g:B\to C$映射到$g\circ f:A\to C$。有些出乎意料的是,为了使这种形式化工作,$\circ$必须不仅接受$f$和$g$,还要接受它们的类型$A$,$B$,$C$作为参数:
%
\begin{narrowmultline*}
{\circ} \defeq \lam{A:\UU_i}{B:\UU_i}{C:\UU_i}
\narrowbreak
\lam{g:B\to C}{f:A\to B}{x:A} g(f(x))。
\end{narrowmultline*}
%
从实际角度看,我们不希望每次应用$\circ$时都注释上$A$,$B$和$C$,因为它们通常可以很容易地从上下文中推断出来。我们希望简单地写$g\circ f$。
然后,严格来说,$g \circ f$并不是$\lam{x : A} g(f(x))$的简写,因为它涉及我们希望省略的附加\define{隐含参数 (implicit arguments)}。
\index{implicit argument隐含参数}

隐含参数的推断、典型模糊性\index{typical ambiguity典型模糊性}(\cref{sec:universes})、确保符号只定义一次等,统称为\define{精炼 (elaboration)}。\index{elaboration, in type theory类型论中的精炼}
精炼必须在检查推导之前进行,因此通常不被作为核心类型论的一部分来介绍。然而,几乎不可能使用不进行精炼的类型论实现;参见\cite{Coq,norell2007towards}获取进一步讨论。

\section{同伦类型论 (Homotopy Type Theory)}
\label{sec:hott-features}

在本节中,我们陈述了同伦类型论的附加公理,这些公理将它与标准的Martin-Löf类型论区分开来:函数外延性、等价性公理(univalence axiom)以及高阶归纳类型。我们将以第二种形式的方式 (\cref{sec:syntax-more-formally}) 来陈述它们,尽管第一种形式 (\cref{sec:syntax-informally}) 也同样适用。

\subsection{函数外延性与等价性 (Function Extensionality and Univalence)}

引入不涉及新语法或判断等式的公理(如函数外延性和等价性)有两种基本方式:要么添加一个原语常量来充当公理的实例,要么通过假设一个变量来占据公理的位置并证明所有依赖于该公理的定理,参见\cref{sec:axioms}。虽然这两者本质上是等价的,但我们选择前者,因为我们认为同伦类型论的公理是核心理论的重要组成部分。

\index{function extensionality函数外延性}%
\cref{axiom:funext} 通过引入一个常量$\funext$来形式化,该常量断言$\happly$是一个等价:
%
\begin{mathparpagebreakable}
\inferrule*[right=$\Pi$-\textsc{ext}]
{\oftp\Gamma{f}{\tprd{x:A} B} \\
\oftp\Gamma{g}{\tprd{x:A} B}}
{\oftp\Gamma{\funext(f,g)}{\isequiv(\happly_{f,g})}}
\end{mathparpagebreakable}
%
$\happly$和$\isequiv$的定义可以分别在~\eqref{eq:happly}和\cref{sec:concluding-remarks}中找到。

\index{univalence axiom等价性公理}%
\cref{axiom:univalence}也以类似的方式形式化:
%
\begin{mathparpagebreakable}
\inferrule*[right=$\UU_i$-\textsc{univ}]
{\oftp\Gamma{A}{\UU_i} \\
\oftp\Gamma{B}{\UU_i}}
{\oftp\Gamma{\univalence(A,B)}{\isequiv(\idtoeqv_{A,B})}}
\end{mathparpagebreakable}
%
$\idtoeqv$的定义可以在~\eqref{eq:uidtoeqv}中找到。

\subsection{圆 (The Circle)}

\index{type!circle类型!圆}%

这里我们给出一个基本的高阶归纳类型的例子;其他类型遵循相同的一般方案,尽管有所展开。

注意,以下规则并未完全遵循\cref{sec:syntax-more-formally}中普通归纳类型的模式:这些规则涉及到映射的传递和函子性 (\cref{sec:functors}) 概念,并且第二个计算规则是命题等式,而不是判断等式。这些差异在\cref{sec:dependent-paths}中进行了讨论。

\begin{mathparpagebreakable}
\inferrule*[right=$\Sn^1$-\rform]
{\wfctx\Gamma}
{\oftp\Gamma{\Sn^1}{\UU_i}}
\and
\inferrule*[right=$$\Sn^1$-\rintro${}_1$]
{\wfctx\Gamma}
{\oftp\Gamma{\base}{\Sn^1}}
\and
\inferrule*[right=$$\Sn^1$-\rintro${}_2$]
{\wfctx\Gamma}
{\oftp\Gamma{\lloop}{\id[\Sn^1]{\base}{\base}}}
\and
\inferrule*[right=$$\Sn^1$-\relim]
{\oftp{\Gamma,\tmtp x{\Sn^1}}{C}{\UU_i} \\
\oftp{\Gamma}{b}{C[\base/x]} \\
\oftp{\Gamma}{\ell}{\dpath C \lloop b b} \\
\oftp\Gamma{p}{\Sn^1}}
{\oftp\Gamma{\ind{\Sn^1}(x.C,b,\ell,p)}{C[p/x]}}
\and
\inferrule*[right=$$\Sn^1$-\rcomp${}_1$]
{\oftp{\Gamma,\tmtp x{\Sn^1}}{C}{\UU_i} \\
\oftp{\Gamma}{b}{C[\base/x]} \\
\oftp{\Gamma}{\ell}{\dpath C \lloop b b}}
{\jdeqtp\Gamma{\ind{\Sn^1}(x.C,b,\ell,\base)}{b}{C[\base/x]}}
\and
\inferrule*[right=$$\Sn^1$-\rcomp${}_2$]
{\oftp{\Gamma,\tmtp x{\Sn^1}}{C}{\UU_i} \\
\oftp{\Gamma}{b}{C[\base/x]} \\
\oftp{\Gamma}{\ell}{\dpath C \lloop b b}}
{\oftp\Gamma{\Sn^1\text{-}\mathsf{loopcomp}}
{\id {\apd{(\lamu{y:\Sn^1} \ind{\Sn^1}(x.C,b,\ell,y))}{\lloop}} {\ell}}}
\end{mathparpagebreakable}
%
在$\ind{\Sn^1}$中,$x$在$C$中绑定。关于依赖路径 (dependent paths) 的记号${\dpath C \lloop b b}$在\cref{sec:dependent-paths}中介绍。
\index{rules of type theory类型论的规则|)}%

\section{基本元理论 (Basic Metatheory)}
\index{metatheory元理论|(}%

本节讨论了\cref{sec:syntax-informally}中呈现的类型论的元理论性质,类似的结果也适用于\cref{sec:syntax-more-formally}。当我们添加\cref{sec:hott-features}中的特性时,确定这些性质是否仍然成立很快会引出一些开放问题\index{open!problem开放!问题},正如本节末尾所讨论的那样。

回想一下,\cref{sec:syntax-informally}将类型论的项定义为无类型$\lambda$-演算的扩展。$\lambda$-演算有其自身的计算概念,即计算规则\index{computation rule!for function types计算规则!用于函数类型的}:
\[
(\lam{x} t)(u) \defeq t[u/x]。
\]
该规则与定义常量的定义方程一起形成\emph{重写规则 (rewriting rules)}\index{rewriting rule重写规则}\index{rule!rewriting规则!重写},这些规则确定了重写系统的简化步骤。这些步骤提供了计算的概念,因为每个规则都有一个自然的方向:通过计算函数的参数来简化$(\lam{x} t)(u)$。

此外,该系统是\emph{合流的 (confluent)}\index{confluence合流性},即,如果$a$在若干步骤中简化为$a'$和$a''$,则存在某个$b$,使得$a'$和$a''$最终都简化为$b$。因此,我们可以定义$t\conv u$表示$t$和$u$简化为相同的项。

(在\cref{sec:syntax-more-formally}中的情况类似:尽管我们在那里将计算规则表示为无方向的等式$\jdeq$,但我们可以通过说将消除子应用于引入形式的结果简化为其等式而非相反来给出操作语义。)

使用类型论的标准技术,可以证明\cref{sec:syntax-informally}中的系统具有以下性质:

\begin{thm}\label{thm:conversion-preserves-typing}
如果$A : \UU$且$A \conv A'$,则$A' : \UU$。
如果$t:A$且$t \conv t'$,则$t':A$。
\end{thm}

我们说一个项是\define{可归约的 (normalizable)}\indexdef{term!normalizable项!可归约的}%
\index{normalization归约性}%
\indexdef{normalizable term可归约项}%
(分别是\define{强可归约的 (strongly normalizable)}\indexdef{term!strongly normalizable项!强可归约的}%
\index{normalization!strong强归约性}%
\index{strong!normalization强归约性}%
),如果从该项开始的某些(分别是所有)重写步骤序列终止。

\begin{thm}\label{thm:strong-normalization}
如果$A : \UU$,则$A$是强可归约的。
如果$t:A$,则$A$和$t$都是强可归约的。
\end{thm}

我们说一个项处于\define{规范形式 (normal form)}\index{normal form规范形式}%
\index{term!normal form of项!规范形式}%
如果它不能被进一步简化,并且我们说一个项是\define{闭合的 (closed)}\index{closed!term闭合!项}%
\index{term!closed闭合项}%
如果其中没有自由变量出现在其中。闭合的规范类型必须是原语类型,即,形式为$c(\vec{v})$的某个原语常量(其中闭合规范项列表$\vec{v}$可能为空,此时可以省略,例如$\N$)。事实上,我们可以明确描述所有规范形式:

\begin{lem}\label{lem:normal-forms}
规范形式的项可以通过以下语法描述:
%
\begin{align*}
v & \production  k \mid \lam{x} v \mid c(\vec{v}) \mid f(\vec{v}),\\
k &\production x \mid k(v) \mid f(\vec{v})(k),
\end{align*}
%
其中$f(\vec{v})$表示已定义函数$f$的部分应用。
特别地,规范形式的类型是$k$或$c(\vec{v})$形式。
\end{lem}

\begin{thm}
如果$A$处于规范形式,则判断$A : \UU$是可判定的。如果$A : \UU$且$t$处于规范形式,则判断$t:A$是可判定的。
\end{thm}

\index{consistency一致性} (of the system in \cref{sec:syntax-informally}) follows immediately: if we had $a:\emptyt$ in the empty context, then by \cref{thm:conversion-preserves-typing,thm:strong-normalization}, $a$ simplifies to a normal term $a':\emptyt$. But by \cref{lem:normal-forms} no such term exists.

\begin{cor}
The system in \cref{sec:syntax-informally} is logically consistent.
\end{cor}

Similarly, we have the \emph{canonicity}\indexdef{canonicity} property that if $a:\N$ in the empty context, then $a$ simplifies to a normal term $\suc^k(0)$ for some numeral $k$.

\begin{cor}
The system in \cref{sec:syntax-informally} has the canonicity property.
\end{cor}

Finally, if $a,A$ are in normal form, it is \emph{decidable} whether $a:A$; in other words, because type-checking amounts to verifying the correctness of a proof, this means we can always ``recognize a correct proof when we see one''.

\begin{cor}
The property of being a proof in the system in \cref{sec:syntax-informally} is decidable.
\end{cor}

然而,以上结果并不适用于扩展了同伦类型论特性的系统(即,通过\cref{sec:hott-features}扩展的上述系统),因为等价性公理和高阶归纳类型构造器的出现从未简化,打破了\cref{lem:normal-forms}。能否通过简化这些常量的应用以恢复规范性仍然是一个开放问题\index{open!problem开放问题}。我们也没有描述所有允许的高阶归纳类型的模式,也不确定如何正确制定它们的规则(例如,高阶构造器的计算规则是否应该是判断等式)。

Martin-Löf类型论扩展到等价性和高阶归纳类型的系统的一致性\index{consistency一致性}可以通过发明一种适当的归约过程来证明,但目前这些系统一致性的唯一证明是通过语义模型实现的——对于等价性,一个由Voevodsky提出的Kan复形\index{Kan complex Kan复形}模型 \cite{klv:ssetmodel},而对于高阶归纳类型,一个由Lumsdaine和Shulman提出的模型 \cite{ls:hits}。

其他元理论问题以及我们当前结果的总结,在本书引言的“构造性”和“开放问题”部分中有更详细的讨论。

\index{metatheory元理论|)}%
\sectionNotes\label{subsec:general-remarks}

% 本节内容强烈受到1972年和1973年Martin-Löf的启发。

引入原语常量的规则体系,以及消除和计算规则(定义常量),受到Gentzen自然演绎的启发。对存在量词消除规则的增强可能性在 \cite{howard:pat} 中提到。对析取的公理增强出现在 \cite{Martin-Lof-1972} 中,对荒谬消除和等式类型的增强出现在 \cite{Martin-Lof-1973} 中。$W$-类型首次出现在 \cite{Martin-Lof-1979} 中。它们推广了 \cite{Tait-1968} 引入的树的概念。
\index{Martin-Löf}%

% 灵感来自Spector的未发表工作。

自然数和序数的原始递归的广义形式出现在 \cite{Hilbert-1925} 中。这激发了G\"odel系统$T$的诞生,\cite{Goedel-T-1958},该系统由 \cite{Tait-1966} 进行了分析,他使用了 \cite{Goedel-T-1958} 中的术语“定义等式”(definitional equality)来表示归约:如果两个项通过一系列应用归约规则归约到一个共同的项,则它们被称为\emph{判断等同的}(judgmentally equal)。这种术语也被de Bruijn在其关于\emph{AUTOMATH}的呈现中使用 \cite{deBruijn-1973}。
\index{AUTOMATH}%

我们的第二种形式呈现包含了相当标准的内涵Martin-Löf类型论,并且增加了一些在同伦类型论中必需的特性。与 \cite{hofmann:syntax-and-semantics} 的参考呈现相比,本书中的类型论有以下几个非关键性差异:
%
\begin{itemize}
\item Russell风格的宇宙(universes),即\cite{martin-lof:bibliopolis}中的概念;以及
\item 对$\Pi$类型的判断$\eta$等式和函数外延性;
\end{itemize}
以及几个对于同伦类型论至关重要的特性:
\begin{itemize}
\item 等价性公理;以及
\item 高阶归纳类型。
\end{itemize}
%
作为一种便利手段,本书主要通过\emph{模式匹配}(pattern matching)定义函数。
\index{pattern matching模式匹配}%
\index{definition!by pattern matching通过模式匹配定义}%
事实上,可以像在\cref{sec:syntax-informally}中那样形式化模式匹配的概念。然而,标准的类型理论呈现方式(在\cref{sec:syntax-more-formally}中采用)是为每个类型构造引入一个单一的\emph{依赖消除子}(dependent eliminator),由此可以从该类型定义出函数。这种方法在语法和语义上都更容易形式化,因为它等同于类型构造的通用属性。这两种方法是等价的;参见\cref{sec:pattern-matching}以获取更详细的讨论。

\index{type theory!formal形式化类型论|)}%
\index{formal!type theory形式化!类型论|)}%
%%% Local Variables:
%%% mode: latex
%%% TeX-master: "hott-online"
%%% End:
