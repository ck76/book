\chapter{实数 (Real numbers)}
\label{cha:real-numbers}

\index{real numbers|(实数|(}%
任何称得上是数学基础的理论最终都必须解决实数的构造问题,这个问题在数学分析中被理解为完备的阿基米德有序域。
\index{ordered field (有序域)}%
完备性有两种定义。一种由柯西提出,要求实数在柯西序列的极限下是封闭的\index{Cauchy!sequence (柯西!序列)},而Dedekind提出的更强的定义要求实数在Dedekind分割下是封闭的\index{cut!Dedekind (分割!Dedekind)}。
这两种完备性定义分别引出了两种构造实数的方法,我们将在\cref{sec:dedekind-reals}和\cref{sec:cauchy-reals}中分别进行研究。在\cref{RD-final-field,RC-initial-Cauchy-complete}中,我们从泛性质的角度对这两种构造进行了刻画:Dedekind实数是最终的阿基米德有序域,而柯西实数是初始的柯西完备阿基米德有序域。

在传统的构造性数学中,
\index{mathematics!constructive (数学!构造性)}%
实数总是需要某些妥协。例如,Dedekind实数在幂集或其他形式的不可判定性存在时效果较好,而柯西实数在可数选择的存在下效果较好。
\index{axiom!of choice!countable (选择公理!可数)}%
然而,我们给出了柯西实数的新构造,将其作为一种更高阶的归纳-归纳类型,这种构造似乎是第三种可能性,它既不需要幂集也不需要可数选择。

在~\cref{sec:comp-cauchy-dedek}中,我们比较了这两种实数构造。柯西实数包含在Dedekind实数中。如果排中律或可数选择成立,它们是一致的,但通常情况下,这种包含可能是严格的。

在~\cref{sec:compactness-interval}中,我们讨论了闭区间~$[0,1]$的三种紧致性定义。我们首先证明~$[0,1]$在度量意义上是紧致的\indexdef{metrically compact (度量紧致)}\indexdef{compactness!metric (紧致性!度量)},即它是完备且全有界的,并且度量紧致空间上的一致连续映射表现得如预期那样。相对而言,Bolzano--Weierstra\ss{}性质,即每个序列都有一个收敛的子序列,意味着有限全知原理,这是排中律的一种形式。最后,我们讨论了Heine--Borel紧致性。有限子覆盖性质的一个朴素的表述是不可行的,但证明相关的归纳覆盖是可行的。
本节基本上属于标准的构造性分析。

在同伦类型论中,实数和分析的发展可以很容易地与经典数学兼容。通过假设排中律和选择公理,我们得到了标准的经典分析\index{mathematics!classical (数学!经典的)}\index{classical!analysis (经典!分析)}:Dedekind实数和柯西实数一致,关于Dedekind实数的不可判定性性质的基础问题消失了,并且区间是尽可能紧致的。

我们在\cref{sec:surreals}中通过构造Conway的超现实数作为一种更高阶的归纳-归纳类型来结束本章;
这种构造在一元类型论中比在经典集合论中更自然。

除了\cref{cha:basics,cha:logic}中的基本理论外,如上所述,我们还使用了“更高阶的归纳-归纳类型”来构造柯西实数和超现实数:这些结合了\cref{cha:hits}中的思想和\cref{sec:generalizations}中提到的归纳-归纳类型的概念。
我们还经常使用\cref{subsec:prop-trunc}中描述的传统逻辑符号,以及(在\cref{sec:piw-pretopos}中证明的)我们的“集合”表现如预期的事实。

请注意,圆的通用覆盖的总空间,在
\cref{subsec:pi1s1-homotopy-theory}中扮演了类似于“实数”在经典代数拓扑中的角色,但并\emph{不是}我们寻找的实数类型。该
类型是可缩的,因此等价于单类型,因此它不能被赋予非平凡的代数结构。



\section{有理数域 (The field of rational numbers)}
\label{sec:field-rati-numb}

\indexdef{rational numbers (有理数)}%
\indexsee{number!rational (数!有理数)}{rational numbers (有理数)}%
我们首先构造有理数 \Q,因为实数可以被视为~\Q 的完备化。
一个专家可能会指出,\Q 可以被任何近似域替代,
\indexdef{field!approximate (域!近似)}%
即 \Q 的一个子环,其中存在任意精确的近似逆元
\index{inverse!approximate (逆元!近似)}%
。一个例子是
二进制有理数环,
\index{rational numbers!dyadic (有理数!二进制)}%
这些有理数的形式为 $n/2^k$。
如果我们在计算机上实现构造性数学,
近似域会更适合,但我们将这种精细留给那些
关心~$\pi$ 的位数的人。

我们在\cref{sec:set-quotients}中构造了整数 \Z,作为 $\N\times
\N$ 的商,并观察到该商由幂等生成。在
\cref{sec:free-algebras}中我们看到 \Z 是在 \unit 上的自由群;我们可以类似地
证明它是 \emptyt 上的自由交换环\index{ring (环)}。有理数域 \Q 是
沿着相同的思路构造的,即为商
%
\[ \Q \defeq (\Z \times \N)/{\approx} \]
%
其中
\[ (u,a) \approx (v,b) \defeq (u (b + 1) = v (a + 1))。 \]
%
换句话说,一个对 $(u, a)$ 代表有理数 $u / (1 + a)$。这里不存在除以零的问题,因为我们巧妙地在分母~$a$ 上加了一。这里我们
也有一个规范的代表选择,即最简分数。因此我们可以
应用\cref{lem:quotient-when-canonical-representatives}来获得集合 \Q,它
再次具有可判定的等式。
\index{decidable!equality (可判定!等式)}%

我们不打算写下 \Q 上的算术运算,因为我们相信我们的读者
知道如何计算分数,即使是在分母上加一的情况下。
让我们只是记录结论,即有理数域 \Q 的构造完全没有问题,它具有可判定的等式和可判定的顺序。
它也可以被刻画为初始有序域。
\index{initial!ordered field (初始!有序域)}%

\symlabel{positive-rationals (正有理数)}
\indexdef{rational numbers!positive (有理数!正)}%
\indexdef{positive!rational numbers (正!有理数)}%
最后,我们将 $\Qp \defeq \setof{ q : \Q | q > 0 }$ 记作正有理数的类型。

\section{Dedekind 实数 (Dedekind reals)}
\label{sec:dedekind-reals}

\index{real numbers!Dedekind|(实数!Dedekind|(}%
让我们首先回顾一下Dedekind构造的基本思想。我们使用双侧的Dedekind
分割,而不是经常使用的单侧版本,因为对称性使
构造更优雅,并且它在构造性数学和经典数学中都能工作。
\index{mathematics!constructive (数学!构造性)}%
一个\emph{Dedekind分割}\index{cut!Dedekind (分割!Dedekind)}由一对 $(L, U)$ 组成,其中 $L, U \subseteq \Q$,
分别称为
\emph{下分割 (lower cut)} 和 \emph{上分割 (upper cut)},满足:
%
\begin{enumerate}
  \item \emph{非空 (inhabited):} 存在 $q \in L$ 和 $r \in U$,
  \item \emph{圆整 (rounded):} $q \in L \Leftrightarrow \exis {r \in \Q} q < r \land r \in L$
  和 $r \in U \Leftrightarrow \exis {q \in \Q} q \in U \land q < r$,
  \index{rounded!Dedekind cut (圆整!Dedekind 分割)}
  \item \emph{不相交 (disjoint):} $\lnot (q \in L \land q \in U)$,并且
  \item \emph{确定性 (located):} $q < r \Rightarrow q \in L \lor r \in U$。
  \index{locatedness (确定性)}%
\end{enumerate}
%
从左到右阅读圆整性条件告诉我们分割是\emph{开集 (open)},
\index{open!cut (开!分割)}%
而从右到左它们分别是\emph{下集 (lower)} 和 \emph{上集 (upper)}。确定性条件表明 $L$ 和 $U$ 之间没有大间隙。由于分割始终是开集,它们永远不会包含“中间的点”,即使它是有理数。一个典型的Dedekind分割如下图所示:
%
\begin{center}
  \begin{tikzpicture}[x=\textwidth]
    \draw[<-),line width=0.75pt] (0,0) -- (0.297,0) node[anchor=south east]{$L\ $};
    \draw[(->,line width=0.75pt] (0.300, 0) node[anchor=south west]{$\ U$} -- (0.9, 0) ;
  \end{tikzpicture}
\end{center}
%
我们可能会天真地将非正式定义翻译成类型论,认为分割
是一对映射 $L, U : \Q \to \prop$。但是我们在\cref{subsec:prop-subsets}中看到
$\prop$ 是 $\prop_{\UU_i}$ 的一种含糊的\index{typical ambiguity (典型模糊性)}符号,其中~$\UU_i$ 是一个宇宙。一旦我们
使用特定的 $\UU_i$ 来定义分割,实数类型将位于下一个
宇宙 $\UU_{i+1}$ 中,实数的性质位于更高的宇宙 $\UU_{i+2}$ 中,实数
子集的性质位于更高的宇宙 $\UU_{i+3}$ 中,依此类推。原则上,我们应该能够
跟踪宇宙层级\index{universe level (宇宙层级)},特别是在证明助手的帮助下,但这样做只会
让我们负担更多的繁琐工作,因此我们更愿意避免。我们因此将
作一个简化的假设,即单一命题类型 $\Omega$ 足以满足我们所有的需求。

事实上,Dedekind 实数的构造对逻辑操作相当有弹性。
我们可以有几种方法来理解使用单一类型
$\Omega$ 的含义:
%
\begin{enumerate}

  \item 我们可以将 $\Omega$ 识别为含糊的 $\prop$ 并跟踪所有出现在定义和构造中的宇宙。

  \item 我们可以假设命题重设公理,
  \index{propositional!resizing (命题!重设)}%
  如\cref{subsec:prop-subsets}所述,这本质上将所有 $\prop_{\UU_i}$ 的层级折叠为
  最低层级\index{universe level (宇宙层级)},我们称之为 $\Omega$。

  \item 对于一个不关心类型论宇宙的复杂性或计算的经典数学家,他可以简单地假设对于单纯命题成立的排中律~\eqref{eq:lem},
  \index{excluded middle (排中律)}%
  这样 $\Omega \jdeq \bool$。
  这不仅消除了对
  $\prop$ 的层级\index{universe level (宇宙层级)}问题,还将我们所做的一切转化为标准的经典\index{mathematics!classical (数学!经典的)}实数构造。

  \item 在另一个极端,人们可能会要求一个使构造工作所需的最小要求。定义一个命题为Dedekind分割的条件
  只使用合取,析取和对~\Q 的存在量词\index{quantifier!existential (量词!存在)},\Q 是一个可数集。因此我们可以将 $\Omega$ 视为初始\emph{$\sigma$-框架 (sigma-frame)},
  \index{initial!sigma-frame@$\sigma$-frame (初始!$\sigma$-框架)}%
  \index{sigma-frame@$\sigma$-frame!initial|defstyle ($\sigma$-框架!初始|定义风格)}%
  即一个具有可数上确界的格\index{lattice (格)},其中二元下确界分布在可数
  上确界上。(初始 $\sigma$-框架不能是两点格 $\bool$,因为
  $\bool$ 不闭合于可数上确界,除非我们假设排中律。) 这
  将导致 $\Omega$ 的构造作为一种更高阶的归纳-归纳类型,但在\cref{sec:cauchy-reals}中进行这种实验已经足够了。
\end{enumerate}

在所有上述情况下,$\Omega$ 是一个集合。
%
事不宜迟,我们将非正式定义翻译成类型论。
在本章中,我们使用了
来自\cref{defn:logical-notation}中的
逻辑符号。

\begin{defn} \label{defn:dedekind-reals}
一个\define{Dedekind 分割 (Dedekind cut)}
\indexsee{Dedekind!cut (Dedekind!分割)}{cut, Dedekind (分割, Dedekind)}%
\indexdef{cut!Dedekind (分割!Dedekind)}%
是一对仅仅命题的 $L : \Q \to \Omega$ 和 $U
: \Q \to \Omega$,满足:
%
\begin{enumerate}
  \item \label{defn:dedekind-reals-inhabited}
  \emph{非空 (inhabited) (即有界):} $\exis{q : \Q} L(q)$ 和 $\exis{r : \Q} U(r)$,
  \item \emph{圆整 (rounded):} 对于所有 $q, r : \Q$,
  \index{rounded!Dedekind cut (圆整!Dedekind 分割)}
  %
  \begin{align*}
    L(q) &\Leftrightarrow \exis{r : \Q} (q < r) \land L(r)
    \qquad\text{并且}\\
    U(r) &\Leftrightarrow \exis{q : \Q} (q < r) \land U(q),
  \end{align*}
  \item \emph{不相交 (disjoint):} 对于所有 $q : \Q$,$\lnot (L(q) \land U(q))$,
  \item \emph{确定性 (located):} 对于所有 $q, r : \Q$,$(q < r) \Rightarrow L(q) \lor U(r)$。
  \index{locatedness (确定性)}%
\end{enumerate}
%
我们用 $\dcut(L, U)$ 表示这些条件的合取。定义
\define{Dedekind 实数 (Dedekind reals)} 的类型为
\indexsee{Dedekind!real numbers (Dedekind!实数)}{real numbers, Dedekind (实数, Dedekind)}%
\indexdef{real numbers!Dedekind (实数!Dedekind)}%
%
\begin{equation*}
  \RD \defeq \setof{ (L, U) : (\Q \to \Omega) \times (\Q \to \Omega) | \dcut(L,U)}。
\end{equation*}
\end{defn}

显然,$\dcut(L, U)$ 是一个仅仅命题,并且由于 $\Q \to \Omega$ 是一个
集合,Dedekind 实数也形成了一个集合。参见
\cref{ex:RD-extended-reals,ex:RD-lower-cuts,ex:RD-interval-arithmetic},了解Dedekind
分割的变体,它们导致了扩展实数、下实数和上实数以及区间
域。

存在一个嵌入 $\Q \to \RD$,它将每个有理数 $q : \Q$ 关联到分割
$(L_q, U_q)$,其中
%
\begin{equation*}
  L_q(r) \defeq (r < q)
  \qquad\text{并且}\qquad
  U_q(r) \defeq (q < r)。
\end{equation*}
%
我们将简单地用 $q$ 表示与有理数相关联的分割 $(L_q, U_q)$。

\subsection{Dedekind 实数的代数结构 (The algebraic structure of Dedekind reals)}
\label{sec:algebr-struct-dedek}

在直觉主义逻辑中,Dedekind 实数的代数和序理论结构的构造
如往常一样进行。我们不打算详细讨论,而是指出经典\index{mathematics!classical (数学!经典的)}和直觉主义设置之间的差异。用 $L_x$ 和 $U_x$ 表示实数 $x : \RD$ 的下分割和上分割,我们定义加法为%
%
\indexdef{addition!of Dedekind reals (加法!Dedekind 实数)}%
\begin{align*}
  L_{x + y}(q) &\defeq \exis{r, s : \Q} L_x(r) \land L_y(s) \land q = r + s, \\
  U_{x + y}(q) &\defeq \exis{r, s : \Q} U_x(r) \land U_y(s) \land q = r + s,
\end{align*}
%
并定义加法逆元为
%
\begin{align*}
  L_{-x}(q) &\defeq \exis{r : \Q} U_x(r) \land q = - r, \\
  U_{-x}(q) &\defeq \exis{r : \Q} L_x(r) \land q = - r。
\end{align*}
%
通过这些操作,$(\RD, 0, {+}, {-})$ 是一个阿贝尔群\index{group!abelian (群!阿贝尔)}。乘法稍微复杂一点:
%
\indexdef{multiplication!of Dedekind reals (乘法!Dedekind 实数)}%
\begin{align*}
  L_{x \cdot y}(q) &\defeq
  \begin{aligned}[t]
    \exis{a, b, c, d : \Q} & L_x(a) \land U_x(b) \land L_y(c) \land U_y(d) \land {}\\
    & \qquad q < \min (a \cdot c, a \cdot d, b \cdot c, b \cdot d),
  \end{aligned} \\
  U_{x \cdot y}(q) &\defeq
  \begin{aligned}[t]
    \exis{a, b, c, d : \Q} & L_x(a) \land U_x(b) \land L_y(c) \land U_y(d) \land {}\\
    & \qquad \max (a \cdot c, a \cdot d, b \cdot c, b \cdot d) < q。
  \end{aligned}
\end{align*}
%
\index{interval!arithmetic (区间!算术)}%
这些公式与区间算术中区间的乘法有关,其中区间 $[a,b]$ 和 $[c,d]$ 的端点是有理数,乘积为区间
%
\begin{equation*}
[a,b] \cdot [c,d] =
[\min(a c, a d, b c, b d), \max(a c, a d, b c, b d)]。
\end{equation*}
%
例如,下分割的公式可以解释为当存在包含 $x$ 和 $y$ 的区间 $[a,b]$ 和 $[c,d]$ 时,$q < x \cdot y$
说明 $q$ 位于 $[a,b] \cdot [c,d]$ 的左侧。通常将满足 $L_x(a)$ 和 $U_x(b)$ 的区间 $[a,b]$ 视为~$x$ 的近似是有用的,参见
\cref{ex:RD-interval-arithmetic}。

现在我们有了一个具有单位元的交换环\index{ring (环)},单位元为
\index{unit!of a ring (单位元!环)}%
$(\RD, 0, 1, {+}, {-}, {\cdot})$。为了处理乘法逆元,我们首先引入顺序。定义 $\leq$ 和 $<$ 为
%
\begin{align*}
(x \leq y) &\ \defeq \ \fall{q : \Q} L_x(q) \Rightarrow L_y(q), \\
(x < y)    &\ \defeq \ \exis{q : \Q} U_x(q) \land L_y(q)。
\end{align*}

\begin{lem} \label{dedekind-in-cut-as-le}
对于所有 $x : \RD$ 和 $q : \Q$,$L_x(q) \Leftrightarrow (q < x)$ 并且 $U_x(q)
\Leftrightarrow (x < q)$。
\end{lem}

\begin{proof}
  如果 $L_x(q)$,则通过圆整性,存在仅仅一个 $r > q$ 使得 $L_x(r)$,并且由于
  $U_q(r)$,因此得出 $q < x$。反之,如果 $q < x$,则存在 $r : \Q$ 使得 $U_q(r)$ 并且 $L_x(r)$,因此由于 $L_x$ 是下集,因此 $L_x(q)$。其余证明对称成立。
\end{proof}

\index{partial order (偏序)}%
\index{transitivity!of . for reals@of $<$ for reals (传递性!实数的 $<$)}%
\index{transitivity!of . for reals@of $\leq$ for reals (传递性!实数的 $\leq$)}%
\index{relation!irreflexive (关系!反自反)}%
\index{irreflexivity!of . for reals@of $<$ for reals (反自反性!实数的 $<$)}%
关系 $\leq$ 是偏序,并且 $<$ 是传递的和反自反的。线性性
\index{order!linear (顺序!线性)}%
\index{linear order (线性顺序)}%
%
\begin{equation*}
(x < y) \lor (y \leq x)
\end{equation*}
%
在假设排中律时成立,但在不假设排中律的情况下,我们得到弱线性性
%
\index{order!weakly linear (顺序!弱线性)}
\index{weakly linear order (弱线性顺序)}
\begin{equation} \label{eq:RD-linear-order}
(x < y) \Rightarrow (x < z) \lor (z < y)。
\end{equation}
%
乍一看,\eqref{eq:RD-linear-order} 可能不清楚与线性顺序的关系。但如果我们令 $x \jdeq u - \epsilon$ 和 $y \jdeq u + \epsilon$,其中
$\epsilon > 0$,那么我们得到
%
\begin{equation*}
(u - \epsilon < z) \lor (z < u + \epsilon)。
\end{equation*}
%
这是“加上一个小数误差”的线性性,即,由于不合理地期望我们可以实际用无限精度计算,我们不应感到惊讶,我们只能在计算的任何有限精度范围内决定~$<$。

要看出~\eqref{eq:RD-linear-order} 成立,假设 $x < y$。那么仅仅存在 $q : \Q$ 使得 $U_x(q)$ 并且
$L_y(q)$。通过圆整性,存在仅仅 $r, s : \Q$ 使得 $r < q < s$,$U_x(r)$
并且 $L_y(s)$。然后,通过确定性 $L_z(r)$ 或 $U_z(s)$。在第一种情况下,我们得到 $x < z$,在第二种情况下,得到 $z < y$。

在经典情况下,乘法逆元存在于所有不等于零的数中。
然而,在没有排中律的情况下,需要一个更强的条件。称 $x, y : \RD$
彼此\define{相异 (apart)},
\indexdef{apartness (相异)}%
记作 $x \apart y$,当 $(x < y) \lor (y < x)$ 时:
%
\symlabel{apart (相异)}
\begin{equation*}
(x \apart y) \defeq (x < y) \lor (y < x)。
\end{equation*}
%
如果 $x \apart y$,则 $\lnot (x = y)$。
如果假设排中律,反过来也是成立的,但在构造性数学中不可证。
\index{mathematics!constructive (数学!构造性)}%
事实上,如果 $\lnot (x = y)$ 蕴含 $x\apart y$,那么排中律的一小部分将成立;参见\cref{ex:reals-apart-neq-MP}。

\begin{thm} \label{RD-inverse-apart-0}
当且仅当一个实数与 $0$ 相异时,它是可逆的。
\end{thm}

\begin{rmk}
  我们观察到一个实数是可逆的,当且仅当它是仅仅可逆的。实际上,在任何环中都是如此,\index{ring (环)} 因为一个环是一个集合,如果存在,乘法逆元是唯一的。 参见\cref{cor:UC}后的讨论。
\end{rmk}

\begin{proof}
  假设 $x \cdot y = 1$。那么仅仅存在 $a, b, c, d : \Q$ 使得
  $a < x < b$,$c < y < d$ 并且 $0 < \min (a c, a d, b c, b d)$。由于 $0 < a c$ 和 $0 < b c$,可得
  $a$,$b$ 和 $c$ 要么全为正数,要么全为负数。
  因此,要么 $0 < a < x$,要么 $x < b < 0$,因此 $x \apart 0$。

  反之,如果 $x \apart 0$,则 $x > 0$ 或 $x < 0$。
  如果 $x > 0$,我们定义 $x^{-1}$ 如下:
  %
  \begin{align*}
    L_{x^{-1}}(q) &\defeq
    (q > 0) \Rightarrow \exis{r : \Q} U_x(r) \land (q r < 1),
    \\
    U_{x^{-1}}(q) &\defeq
    (q > 0) \land \exis{r : \Q} L_x(r) \land (q r > 1)。
  \end{align*}
  %
  如果 $x < 0$,则我们通过以下方式定义它:
  %
  \begin{align*}
    L_{x^{-1}}(q) &\defeq
    (q < 0) \land \exis{r : \Q} U_x(r) \land (q r > 1),
    \\
    U_{x^{-1}}(q) &\defeq
    (q < 0) \Rightarrow \exis{r : \Q} L_x(r) \land (q r < 1)。\qedhere
  \end{align*}
\end{proof}

\index{ordered field!archimedean (有序域!阿基米德)}%
\index{dense (稠密)}%
\indexsee{order-dense (顺序稠密)}{dense (稠密)}%
阿基米德原理可以用几种方式表达。我们认为最具启发性的是
形式,它说 $\Q$ 在 $\RD$ 中是稠密的。

\begin{thm}[阿基米德原理对于 $\RD$] \label{RD-archimedean}
%
对于所有 $x, y : \RD$,如果 $x < y$,则仅仅存在 $q : \Q$ 使得
$x < q < y$。
\end{thm}

\begin{proof}
  根据 $<$ 的定义。
\end{proof}

在处理 Dedekind 实数的完备性之前,让我们准确地描述它们具有的代数
结构。在以下定义中,我们的目标不是最小公理化,而是一个有用的结构和性质。

\begin{defn} \label{ordered-field} 一个\define{有序域 (ordered field)}
\indexdef{ordered field (有序域)}%
\indexsee{field!ordered (域!有序)}{ordered field (有序域)}%
是一个集合 $F$,它与
常数 $0$,$1$,运算 $+$,$-$,$\cdot$,$\min$,$\max$,和仅仅关系
$\leq$,$<$,$\apart$ 使得:
%
\begin{enumerate}
  \item $(F, 0, 1, {+}, {-}, {\cdot})$ 是一个交换环,带有单位元;
  \index{unit!of a ring (单位元!环)}%
  \index{ring (环)}%
  \item 当且仅当 $x \apart 0$ 时,$x : F$ 是可逆的;
  \item $(F, {\leq}, {\min}, {\max})$ 是一个格;
  \item 严格顺序 $<$ 是传递的,反自反的,
  \index{relation!irreflexive (关系!反自反)}
  \index{irreflexivity!of . in a field@of $<$ in a field (反自反性!$<$ 在一个域中)}%
  并且弱线性 ($x < y \Rightarrow x < z \lor z < y$);\index{transitivity!of . in a field@of $<$ in a field (传递性!$<$ 在一个域中)}
  \index{order!weakly linear (顺序!弱线性)}
  \index{weakly linear order (弱线性顺序)}
  \index{strict!order (严格!顺序)}%
  \index{order!strict (顺序!严格)}%
  \item 相异性 $\apart$ 是反自反的,对称的,并且是对传递的 ($x \apart y \Rightarrow x \apart z \lor y \apart z$);
  \index{relation!irreflexive (关系!反自反)}
  \index{irreflexivity!of apartness (反自反性!相异性)}%
  \indexdef{relation!cotransitive (关系!对传递)}%
  \index{cotransitivity of apartness (相异性的对传递)}%
  \item 对于所有 $x, y, z : F$:
  %
  \begin{align*}
    x \leq y &\Leftrightarrow \lnot (y < x), &
    x < y \leq z &\Rightarrow x < z, \\
    x \apart y &\Leftrightarrow (x < y) \lor (y < x), &
    x \leq y < z &\Rightarrow x < z, \\
    x \leq y &\Leftrightarrow x + z \leq y + z, &
    x \leq y \land 0 \leq z &\Rightarrow x z \leq y z, \\
    x < y &\Leftrightarrow x + z < y + z, &
    0 < z \Rightarrow (x < y &\Leftrightarrow x z < y z), \\
    0 < x + y &\Rightarrow 0 < x \lor 0 < y, &
    0 &< 1。
  \end{align*}
\end{enumerate}
%
每个这样的域都有一个规范的嵌入 $\Q \to F$。一个有序域是
\define{阿基米德 (archimedean)}
\indexdef{ordered field!archimedean (有序域!阿基米德)}%
\indexsee{archimedean property (阿基米德性质)}{ordered field, archimedean (有序域, 阿基米德)}%
当对于所有 $x, y : F$,如果 $x < y$,则仅仅存在 $q :
\Q$ 使得 $x < q < y$。
\end{defn}

\begin{thm} \label{RD-archimedean-ordered-field}
Dedekind 实数形成了一个有序的阿基米德域。
\end{thm}

\begin{proof}
  我们省略了证明,因为我们已经展示的内容使得该定理
  看起来是合理的。
\end{proof}

\subsection{Dedekind 实数是柯西完备的 (Dedekind reals are Cauchy complete)}
\label{sec:RD-cauchy-complete}

回顾一下,$x : \N \to \Q$ 是一个\emph{柯西序列 (Cauchy sequence)}\indexdef{Cauchy!sequence (柯西!序列)}当它满足
%
\begin{equation} \label{eq:cauchy-sequence}
\prd{\epsilon : \Qp} \sm{n : \N} \prd{m, k \geq n} |x_m - x_k| < \epsilon。
\end{equation}
%
请注意,我们并\emph{没有} 截断内部存在的存在量词,因为我们实际上希望
计算收敛速度——没有误差估计的近似几乎没有什么有用的信息。通过\cref{thm:ttac},\eqref{eq:cauchy-sequence}产生一个函数 $M
: \Qp \to \N$,称为\emph{收敛模 (modulus of convergence)}\indexdef{modulus!of convergence (模!收敛)},使得 $m, k \geq M(\epsilon)$
时,$|x_m - x_k| < \epsilon$。由此我们得到 $|x_{M(\delta/2)} - x_{M(\epsilon/2)}|<
\delta + \epsilon$ 对于所有 $\delta, \epsilon : \Qp$。事实上,映射 $(\epsilon \mapsto
x_{M(\epsilon/2)}) : \Qp \to \Q$ 携带的极限信息与
原始的柯西条件~\eqref{eq:cauchy-sequence} 相同。我们将处理这些
近似函数而不是柯西序列。

\begin{defn} \label{defn:cauchy-approximation}
一个\define{柯西近似 (Cauchy approximation)}
\indexdef{Cauchy!approximation (柯西!近似)}%
是一个映射 $x : \Qp \to \RD$,满足
%
\begin{equation}
  \label{eq:cauchy-approx}
  \fall{\delta, \epsilon :\Qp} |x_\delta - x_\epsilon| < \delta + \epsilon。
\end{equation}
%
一个柯西近似 $x : \Qp \to \RD$ 的\define{极限 (limit)}
\index{limit!of a Cauchy approximation (极限!柯西近似)}%
是一个实数 $\ell : \RD$,使得
%
\begin{equation*}
  \fall{\epsilon, \theta : \Qp} |x_\epsilon - \ell| < \epsilon + \theta。
\end{equation*}
\end{defn}

\begin{thm} \label{RD-cauchy-complete}
$\RD$ 中的每一个柯西近似都有一个极限。
\end{thm}

\begin{proof}
  请注意,我们正在展示极限的存在,而不是仅仅存在。
  给定一个柯西近似 $x : \Qp \to \RD$,定义
  %
  \begin{align*}
    L_y(q) &\defeq \exis{\epsilon, \theta : \Qp} L_{x_\epsilon}(q + \epsilon + \theta),\\
    U_y(q) &\defeq \exis{\epsilon, \theta : \Qp} U_{x_\epsilon}(q - \epsilon - \theta)。
  \end{align*}
  %
  很明显 $L_y$ 和 $U_y$ 是非空的,圆整的,不相交的。为了证明
  确定性,考虑任意的 $q, r : \Q$ 满足 $q < r$。存在 $\epsilon : \Qp$ 满足
  $5 \epsilon < r - q$。由于 $q + 2 \epsilon < r - 2 \epsilon$ 仅仅
  $L_{x_\epsilon}(q + 2 \epsilon)$ 或 $U_{x_\epsilon}(r - 2 \epsilon)$。在第一种情况下,
  我们有 $L_y(q)$,在第二种情况下我们有 $U_y(r)$。

  为了表明 $y$ 是 $x$ 的极限,考虑任意的 $\epsilon, \theta : \Qp$。由于
  $\Q$ 在 $\RD$ 中是稠密的,存在仅仅的 $q, r : \Q$ 满足
  %
  \begin{narrowmultline*}
    x_\epsilon - \epsilon - \theta/2 < q < x_\epsilon - \epsilon - \theta/4
    < x_\epsilon < \\
    x_\epsilon + \epsilon + \theta/4 < r < x_\epsilon + \epsilon + \theta/2,
  \end{narrowmultline*}
  %
  因此 $q < y < r$。现在要么 $y < x_\epsilon + \theta/2$,要么 $x_\epsilon - \theta/2 < y$。
  在第一种情况下,我们有
  %
  \begin{equation*}
    x_\epsilon - \epsilon - \theta/2 < q < y < x_\epsilon + \theta/2,
  \end{equation*}
  %
  在第二种情况下
  %
  \begin{equation*}
    x_\epsilon - \theta/2 < y < r < x_\epsilon + \epsilon + \theta/2。
  \end{equation*}
  %
  无论哪种情况,都得出 $|y - x_\epsilon| < \epsilon + \theta$。
\end{proof}

为了完整性,我们记录了经典的表达方式。

\begin{cor}
  假设 $x : \N \to \RD$ 满足柯西条件~\eqref{eq:cauchy-sequence}。那么
  存在 $y : \RD$ 满足
  %
  \begin{equation*}
    \prd{\epsilon : \Qp} \sm{n : \N} \prd{m \geq n} |x_m - y| < \epsilon。
  \end{equation*}
\end{cor}

\begin{proof}
  通过\cref{thm:ttac}存在 $M : \Qp \to \N$ 使得 $\bar{x}(\epsilon) \defeq
  x_{M(\epsilon/2)}$ 是一个柯西近似。令 $y$ 为其极限,该极限通过
  \cref{RD-cauchy-complete} 存在。给定任意 $\epsilon : \Qp$,令 $n \defeq M(\epsilon/4)$
  并观察到,对于任何 $m \geq n$,
  %
  \begin{narrowmultline*}
    |x_m - y| \leq |x_m - x_n| + |x_n - y| =
    |x_m - x_n| + |\bar{x}(\epsilon/2) - y| < \narrowbreak
    \epsilon/4 + \epsilon/2 + \epsilon/4 = \epsilon。\qedhere
  \end{narrowmultline*}
\end{proof}

\subsection{Dedekind 实数是 Dedekind 完备的 (Dedekind reals are Dedekind complete)}
\label{sec:RD-dedekind-complete}

我们得到了 $\RD$ 作为 $\Q$ 上的 Dedekind 分割的类型。但是我们也可以从任意的阿基米德有序域 $F$ 开始,并构造 $F$ 上的 Dedekind 分割\index{cut!Dedekind (分割!Dedekind)}。这些将
再次形成一个阿基米德有序域 $\bar{F}$,称为\define{Dedekind 完备的 $F$},
\index{completion!Dedekind (完备!Dedekind)}%
\indexsee{Dedekind!completion (Dedekind!完备)}{completion, Dedekind (完备, Dedekind)},
其中 $F$ 被包含为一个子域。如果我们将此构造应用于
$\RD$,我们会得到更多的实数吗?答案是否定的。事实上,我们将证明一个更强的结果:$\RD$ 是最终的。

称一个有序域~$F$ 是\define{对 $\Omega$ 可接受的 (admissible for $\Omega$)}
\indexsee{admissible!ordered field (可接受的!有序域)}{ordered field, admissible (有序域, 可接受的)}%
\indexdef{ordered field!admissible (有序域!可接受的)}%
当严格顺序
$<$ 在~$F$ 上是一个映射 ${<} : F \to F \to \Omega$。

\begin{thm} \label{RD-final-field}
每一个对 $\Omega$ 可接受的阿基米德有序域都是~$\RD$ 的一个子域。
\end{thm}

\begin{proof}
  令 $F$ 为一个阿基米德有序域。对于每一个 $x : F$ 定义 $L_x, U_x : \Q \to
  \Omega$ 如下
  %
  \begin{equation*}
    L_x(q) \defeq (q < x)
    \qquad\text{并且}\qquad
    U_x(q) \defeq (x < q)。
  \end{equation*}
  %
  (我们刚刚使用了 $F$ 是对 $\Omega$ 可接受的这个假设。)
  然后 $(L_x, U_x)$ 是一个 Dedekind 分割。事实上,这些分割是非空且圆整的,因为
  $F$ 是阿基米德的,$<$ 是传递的,不相交是因为 $<$ 是反自反的,并且
  确定性是因为 $<$ 是弱线性顺序。令 $e : F \to \RD$ 为映射 $e(x) \defeq (L_x,
  U_x)$。

  我们声称 $e$ 是一个保持并反映顺序的域嵌入。首先,
  注意 $e(q) = q$ 对于一个有理数 $q$。其次,我们有等价关系,
  对于所有 $x, y : F$,
  %
  \begin{narrowmultline*}
    x < y \Leftrightarrow
    (\exis{q : \Q} x < q < y) \Leftrightarrow \narrowbreak
    (\exis{q : \Q} U_x(q) \land L_y(q)) \Leftrightarrow
    e(x) < e(y),
  \end{narrowmultline*}
  %
  因此 $e$ 确实保持并反映顺序。$e(x + y) = e(x) + e(y)$ 成立
  是因为对于所有 $q : \Q$,
  %
  \begin{equation*}
    q < x + y \Leftrightarrow
    \exis{r, s : \Q} r < x \land s < y \land q = r + s。
  \end{equation*}
  %
  从右到左的蕴含是显然的。对于另一方向,如果 $q < x +
  y$,则仅仅存在 $r : \Q$ 使得 $q - y < r < x$,通过取 $s \defeq
  q - r$ 我们得到所需的 $r$ 和 $s$。我们将保留 $e$ 的乘法的证明
  作为练习。
\end{proof}

为了证明 $\RD$ 上的 Dedekind 分割不会给我们带来任何新内容,我们需要再证明一个引理。

\begin{lem} \label{lem:cuts-preserve-admissibility}
如果 $F$ 对 $\Omega$ 是可接受的,那么它的 Dedekind 完备也是如此。
\index{completion!Dedekind (完备!Dedekind)}%
\end{lem}

\begin{proof}
  令 $\bar{F}$ 为 $F$ 的 Dedekind 完备。$\bar{F}$ 上的严格顺序
  定义为
  %
  \begin{equation*}
    ((L,U) < (L',U')) \defeq \exis{q : \Q} U(q) \land L'(q)。
  \end{equation*}
  %
  由于 $U(q)$ 和 $L'(q)$ 是 $\Omega$ 的元素,该引理成立只要 $\Omega$
  在合取和可数存在下是封闭的,这是我们一开始就假设的。
\end{proof}


\begin{cor} \label{RD-dedekind-complete}
%
\indexdef{complete!ordered field, Dedekind (完备!有序域, Dedekind)}%
\indexdef{Dedekind!completeness (Dedekind!完备性)}%
Dedekind 实数是 Dedekind 完备的:对于每一个实数值的 Dedekind 分割 $(L, U)$,
存在唯一的 $x : \RD$ 使得 $L(y) = (y < x)$ 并且 $U(y) = (x < y)$ 对于所有 $y :
\RD$ 成立。
\end{cor}

\begin{proof}
  通过\cref{lem:cuts-preserve-admissibility},$\RD$ 的 Dedekind 完备 $\barRD$
  对 $\Omega$ 是可接受的,因此通过\cref{RD-final-field} 我们有一个嵌入 $\barRD
  \to \RD$,以及一个嵌入 $\RD \to \barRD$。但这些嵌入必须是
  同构,因为它们的组合是保持顺序的域同态\index{homomorphism!field (同态!域)},它们
  固定了稠密子域~$\Q$,这意味着它们是恒等映射。推论现在
  立即得出,因为 $\barRD \to \RD$ 是一个同构。
\end{proof}

\index{real numbers!Dedekind|) (实数!Dedekind|)}%

\section{柯西实数 (Cauchy Reals)}
\label{sec:cauchy-reals}

\index{实数!柯西|(}%
\index{完备化!柯西|(}%
\indexsee{柯西!完备化}{完备化, 柯西}%
柯西实数是通过极限方式构造 \Q 的完备化 (completion)。在经典的柯西实数构造中,我们考虑所有柯西序列 (Cauchy sequences) 的集合 $\mathcal{C}$,然后形成适当的商 $\mathcal{C}/{\approx}$。为了证明 $\mathcal{C}/{\approx}$ 是柯西完备的,我们考虑一个柯西序列 $x : \N \to \mathcal{C}/{\approx}$,将其提升为序列 $\bar{x} : \N \to \mathcal{C}$,并使用 $\bar{x}$ 构造 $x$ 的极限。然而,将 $x$ 提升为 $\bar{x}$ 的过程使用了可数选择公理 (axiom of countable choice) 或排中律 (law of excluded middle),我们可能希望避免这种情况。
\indexdef{公理!选择!可数}%
任何最后一步是商的实数构造都存在这个问题。构造性数学中有三种常见的解决方案:
\index{数学!构造性}%
\begin{enumerate}
  \item 假装实数是一个集合 $(\mathcal{C}, {\approx})$,即附带有重合 (coincidence) 关系的柯西序列 $\mathcal{C}$。实数序列就简化为表示它们的柯西序列。
  \item 接受可数选择公理的诱惑。毕竟,该公理在大多数基于计算观点的构造性数学模型中是有效的,比如可实现性模型。
  \item 宣布柯西实数不够完美,转而构造 Dedekind 实数。在某些上下文中,这样的判断是完全有效的,比如在 sheaf-theoretic 模型的构造性数学中。然而,正如我们在 \cref{sec:dedekind-reals} 中看到的,构造性 Dedekind 实数也有自己的问题。
\end{enumerate}

然而,使用更高阶归纳类型 (higher inductive types),有第四种解决方案,这种方法优于以上任一种,甚至对经典数学家也是有趣的。其核心思想是柯西实数应该是 \Q 生成的 \emph{自由完备度量空间} (free complete metric space)。通常,构造自由结构需要多次将结构操作应用于生成元。例如,自由群上的元素不仅仅是元素 $X$ 的二元乘积和逆元,还需要通过迭代乘积和逆元的构造形成单词。因此,我们可能自然地期望同样的过程适用于柯西完备化,相关的“操作”是“取柯西序列的极限”。(在这种情况下,即使在无穷次操作后仍然会有新的柯西序列,因此迭代必须是超限的。)

上述论证表明,如果排中律或可数选择成立,那么柯西完备化是非常特殊的:在构造空间的完备化时,只需要在“一步”操作后停止。这可以类比于自由单子和自由群可以通过(化简的)单词给出明确描述。然而,我们在 \cref{sec:free-algebras} 中看到,更高阶归纳类型允许我们\emph{直接}构造自由结构,而不管是否存在明确的描述。在本节中,我们将展示对于柯西实数也是如此(类似技术可以构造任何度量空间的柯西完备化;参见 \cref{ex:metric-completion})。具体而言,更高阶归纳类型允许我们\emph{同时}添加柯西序列的极限并商取重合关系,从而可以避免将实数序列提升为代表序列的问题。
\index{完备化!柯西|)}%

\subsection{柯西实数的构造 (Construction of Cauchy Reals)}
\label{sec:constr-cauchy-reals}

作为一个更高阶归纳类型,柯西实数 $\RC$ 的构造比 \cref{sec:free-algebras} 中讨论的自由代数结构更为复杂。我们打算包括一个“取极限”的构造器,其输入是一个实数的柯西序列,但“实数柯西序列”的概念依赖于度量“距离”的方法。当然,两个实数之间的距离将是另一个实数,从而导致一个潜在的问题。

然而,我们实际上所需要的不是一般意义上的“距离”,而是表达“两个实数之间的距离小于 $\epsilon$”的方式,对于任意 $\epsilon:\Qp$。这可以通过一组二元关系 $\mathord{\close\epsilon} : \RC\to\RC\to \prop$ 来表示。$\mathord{\close\epsilon}$ 的意图是表达 $|x - y| < \epsilon$,但由于我们尚未定义减法、绝对值或不等式(毕竟,我们刚刚定义 $\RC$),我们必须在定义 $\RC$ 的同时定义这些关系 $\close\epsilon$。由于 $\close\epsilon$ 是由两个 $\RC$ 和一个 $\Qp$ 索引的类型族,我们不能使用普通的归纳(higher)定义;相反,我们必须使用 \emph{更高阶归纳归纳定义} (higher inductive-inductive definition)。
\index{归纳归纳类型!更高阶}

\cref{sec:generalizations} 中提到的普通归纳归纳定义允许我们通过同时归纳定义一个类型及其索引的类型族。当然,“更高阶”版本允许类型和族具有路径构造器以及点构造器。我们不会尝试制定任何一般的更高阶归纳归纳定义理论,但希望我们对 $\RC$ 和 $\close\epsilon$ 的描述能够使这一思想变得清晰。

\begin{rmk}
  我们可能还考虑一种 \emph{更高阶归纳递归定义},其中 $\close\epsilon$ 是使用 $\RC$ 的 \emph{递归} 原则定义的,同时与 $\RC$ 的 \emph{归纳} 定义一起进行。然而,我们选择归纳归纳路线有两个原因。首先,更高阶归纳递归定义在同伦语义中更难以证明。其次,更重要的是,归纳归纳定义产生了更强大的归纳原则,这是我们在发展柯西实数的基本理论时所需要的。
\end{rmk}

最后,正如我们在 \cref{sec:RD-cauchy-complete} 中讨论柯西完备性的 Dedekind 实数时所做的那样,我们将使用 \emph{柯西逼近} (\cref{defn:cauchy-approximation}) 而不是柯西序列。当然,我们的柯西逼近现在将由柯西实数构成,而不是 Dedekind 实数或有理数。

\begin{defn}\label{defn:cauchy-reals}
让 $\RC$ 和关系 $\closesym:\Qp \times \RC \times \RC \to \type$ 成为以下更高阶归纳归纳类型族。柯西实数的类型 $\RC$
\indexdef{实数!柯西}%
由以下构造生成:
\begin{itemize}
  \item \emph{有理点 (rational points):}
  对于任意 $q : \Q$,存在一个实数 $\rcrat(q)$。
  \index{有理数!作为柯西实数}%
  \item \emph{极限点 (limit points):}
  对于任意 $x : \Qp \to \RC$,如果满足
  %
  \begin{equation}
    \label{eq:RC-cauchy}
    \fall{\delta, \epsilon : \Qp} x_\delta \close{\delta + \epsilon} x_\epsilon
  \end{equation}
  %
  则存在一个点 $\rclim(x) : \RC$。我们称 $x$ 为一个 \define{柯西逼近 (Cauchy approximation)}。
  \indexdef{柯西!逼近}%
  \index{极限!柯西逼近}%
  %
  \item \emph{路径 (paths):}
  对于 $u, v : \RC$,如果满足
  %
  \begin{equation}
    \label{eq:RC-path}
    \fall{\epsilon : \Qp} u \close\epsilon v
  \end{equation}
  %
  则存在一个路径 $\rceq(u, v) : \id[\RC]{u}{v}$。
\end{itemize}
同时,类型族 $\closesym:\RC\to\RC\to\Qp \to\type$ 由以下构造生成。
其中 $q$ 和 $r$ 表示有理数;$\delta$、$\epsilon$ 和 $\eta$ 表示正有理数;$u$ 和 $v$ 表示柯西实数;$x$ 和 $y$ 表示柯西逼近:
\begin{itemize}
  \item 对于任意 $q,r,\epsilon$,如果 $-\epsilon < q - r < \epsilon$,则 $\rcrat(q) \close\epsilon \rcrat(r)$,
  \item 对于任意 $q,y,\epsilon,\delta$,如果 $\rcrat(q) \close{\epsilon - \delta} y_\delta$,则 $\rcrat(q) \close{\epsilon} \rclim(y)$,
  \item 对于任意 $x,r,\epsilon,\delta$,如果 $x_\delta \close{\epsilon - \delta} \rcrat(r)$,则 $\rclim(x) \close\epsilon \rcrat(r)$,
  \item 对于任意 $x,y,\epsilon,\delta,\eta$,如果 $x_\delta \close{\epsilon - \delta - \eta} y_\eta$,则 $\rclim(x) \close\epsilon \rclim(y)$,
  \item 对于任意 $u,v,\epsilon$,如果 $\xi,\zeta : u \close{\epsilon} v$,则 $\xi=\zeta$(命题截断)。
\end{itemize}
\end{defn}

\mentalpause

$\RC$ 的第一个构造器表示任何有理数都可以看作实数。第二个构造器表示从任何柯西逼近到一个实数的极限,我们可以得到一个新的实数。第三个构造器表达了如果两个柯西逼近重合,那么它们的极限相等。

$\closesym$ 的前四个构造器指定了何时两个有理数接近,何时一个有理数接近一个极限,以及何时两个极限接近。对于两个有理数的情况,这只是有理数 $\epsilon$-接近的通常概念,而其他情况可以通过注意每个逼近 $x_\delta$ 应该在极限 $\rclim(x)$ 的 $\delta$ 范围内来推导。

我们提醒自己关于证明相关的内容:一个通过 $\rclim$ 获得的实数不仅由一个柯西逼近 $x$ 表示,还需要一个证明 $p$ 证明 \eqref{eq:RC-cauchy},所以我们在技术上应该写 $\rclim(x,p)$ 而不是简单地写 $\rclim(x)$。类似地,$\rceq$ 和 \eqref{eq:RC-path} 的情况也类似,但我们将简单地写 $\rceq : u = v$ 而不是 $\rceq(u, v, p) : u = v$。这些符号的滥用通过忽略命题并省略容易猜测的信息来减轻其影响。同样,$\mathord{\close\epsilon}$ 的最后一个构造器解释了我们为何将其他四个构造器留作无名。

我们立即可以用许多实数填充 $\RC$。因为假设 $x : \N \to \Q$ 是一个传统的有理数柯西序列,并且设 $M : \Qp \to \N$ 是其收敛模数。那么 $\rcrat \circ x \circ M : \Qp \to \RC$ 是一个柯西逼近,使用 $\closesym$ 的第一个构造器产生所需的证明。因此,$\rclim(\rcrat \circ x \circ m)$ 是一个实数。各种著名的实数如 $\sqrt{2}$、$\pi$、$e$ 等都是这种有理数柯西序列的极限。

\subsection{柯西实数的归纳和递归 (Induction and Recursion on Cauchy Reals)}
\label{sec:induct-recurs-cauchy}

当然,要对 $\RC$ 做有用的事情,我们需要给出其归纳原则。每当我们归纳定义两个或多个对象时,$\RC$ 和 $\closesym$ 的基本归纳原则都需要同时对两者进行归纳。因此,我们应期望其说的是,假设有两个类型族分别位于 $\RC$ 和 $\closesym$ 之上,并具有与每个构造器对应的数据,那么就存在这两个类型族的截面函数。然而,由于 $\closesym$ 依赖于两个 $\RC$ 和一个 $\Qp$,因此这些族的确切依赖关系有点微妙。归纳原则将适用于任意一对类型族:
\begin{align*}
  A&:\RC\to\type\\
  B&:\prd{x,y:\RC} A(x) \to A(y) \to \prd{\epsilon:\Qp} (x\close\epsilon y) \to \type.
\end{align*}
$A$ 的类型显而易见,但 $B$ 的类型需要仔细考虑。由于 $B$ 必须依赖于 $\closesym$,但 $\closesym$ 依赖于两个 $\RC$ 和一个 $\Qp$,所以很明显 $B$ 也必须依赖于变量 $x,y:\RC$ 和 $\epsilon:\Qp$ 以及 $(x\close\epsilon y)$ 的一个元素。稍微不太明显的是,$B$ 还必须依赖于 $A(x)$ 和 $A(y)$。

如果我们考虑非依赖型情况(递归原则),可能更容易理解,这时 $A$ 是一个简单类型(而不是类型族)。在这种情况下,我们期望 $B$ 不依赖于 $x,y:\RC$ 或 $(x\close\epsilon y)$。但是,递归原则(及其相关的唯一性原则)应该表明 $\RC$ 和 $\close\epsilon$ 是某种范畴中的“初始对象”,因此在这种情况下 $A$ 和 $B$ 的依赖结构应与 $\RC$ 和 $\close\epsilon$ 的依赖结构一致:即我们应该有 $B:A\to A\to \Qp \to \type$。结合这一观察,并考虑到在依赖型情况下 $B$ 必须依赖于 $x,y:\RC$ 和 $(x\close\epsilon y)$,最终导致了上述 $B$ 的类型。

\symlabel{RC-recursion}
将 $B$ 看作 $\epsilon$ 索引的关系族是有帮助的,这些关系位于类型 $A(x)$ 和 $A(y)$ 之间。这样,我们可以将 $B(x,y,a,b,\epsilon,\xi)$ 写为 $(x,a) \bsim_\epsilon^\xi (y,b)$。由于 $\xi:x\close\epsilon y$ 是唯一的(如果存在),我们通常省略 $\xi$,写作 $(x,a) \bsim_\epsilon (y,b)$;只要我们记住这一关系仅在 $x\close\epsilon y$ 时定义,这样的省略是无害的。我们有时还可以进一步简化,写作 $a\bsim_\epsilon b$,其中 $x$ 和 $y$ 从 $a$ 和 $b$ 的类型中推断,但有时为了清晰起见需要包含它们。

\index{归纳原则!用于柯西实数}%
现在,给定一个类型族 $A:\RC\to\type$ 和一个关系族 $\bsim$,归纳原则的假设包括以下数据,每个对应 $\RC$ 或 $\closesym$ 的一个构造器:
\begin{itemize}
  \item 对于任意 $q : \Q$,有一个元素 $f_q:A(\rcrat(q))$。
  \item 对于任意柯西逼近 $x$,以及任何 $a:\prd{\epsilon:\Qp} A(x_\epsilon)$,满足
  \begin{equation}
    \fall{\delta, \epsilon : \Qp}
    (x_\delta,a_\delta) \bsim_{\delta+\epsilon} (x_\epsilon,a_\epsilon),
    \label{eq:depCauchyappx}
  \end{equation}
  的元素 $f_{x,a}:A(\rclim(x))$。我们称这样的 $a$ 为 \define{依赖柯西逼近 (dependent Cauchy approximation)}。
  \indexdef{柯西!逼近!依赖}%
  \indexsee{逼近, 柯西}{柯西逼近}%
  \indexdef{依赖!柯西逼近}%
  在 $x$ 上。
  \item 对于 $u, v : \RC$,如果 $h:\fall{\epsilon : \Qp} u \close\epsilon v$,以及所有 $a:A(u)$ 和 $b:A(v)$ 满足 $\fall{\epsilon:\Qp} (u,a) \bsim_\epsilon (v,b)$,则有 $\dpath{A}{\rceq(u,v)}{a}{b}$。
  \item 对于 $q,r:\Q$ 和 $\epsilon:\Qp$,如果 $-\epsilon < q - r < \epsilon$,则
  \narrowequation{(\rcrat(q),f_q) \bsim_\epsilon (\rcrat(r),f_r)。}
  \item 对于 $q:\Q$ 和 $\delta,\epsilon:\Qp$ 以及 $y$ 柯西逼近,和 $b$ 依赖柯西逼近 $y$,如果 $\rcrat(q) \close{\epsilon - \delta} y_\delta$,那么
  \[(\rcrat(q),f_q) \bsim_{\epsilon-\delta} (y_\delta,b_\delta)
  \;\Rightarrow\;
  (\rcrat(q),f_q) \bsim_\epsilon (\rclim(y),f_{y,b})。\]
  \item 类似地,对于 $r:\Q$ 和 $\delta,\epsilon:\Qp$ 和 $x$ 柯西逼近,和 $a$ 依赖柯西逼近 $x$,如果 $x_\delta \close{\epsilon - \delta} \rcrat(r)$,那么
  \[(x_\delta,a_\delta) \bsim_{\epsilon-\delta} (\rcrat(r),f_r)
  \;\Rightarrow\;
  (\rclim(x),f_{x,a}) \bsim_\epsilon (\rcrat(q),f_r)。
  \]
  \item 对于 $\epsilon,\delta,\eta:\Qp$ 和 $x,y$ 柯西逼近,和 $a$ 和 $b$ 依赖柯西逼近 $x$ 和 $y$,如果我们有 $x_\delta \close{\epsilon - \delta - \eta} y_\eta$,那么
  \[ (x_\delta,a_\delta) \bsim_{\epsilon - \delta - \eta} (y_\eta,b_\eta)
  \;\Rightarrow\;
  (\rclim(x),f_{x,a}) \bsim_\epsilon (\rclim(y),f_{y,b})。\]
  \item 对于 $\epsilon:\Qp$ 和 $x,y:\RC$ 和 $\xi,\zeta:x\close{\epsilon} y$,和 $a:A(x)$ 和 $b:A(y)$,$(x,a) \bsim_\epsilon^\xi (y,b)$ 和 $(x,a) \bsim_\epsilon^\zeta (y,b)$ 中的任意两个元素在 $\xi=\zeta$ 上依赖相等。注意,这与要求 $\bsim$ 取值于单纯命题是等价的。
\end{itemize}
在这些假设下,我们得到函数
\begin{align*}
  f&:\prd{x:\RC} A(x)\\
  g&:\prd{x,y:\RC}{\epsilon:\Qp}{\xi:x\close{\epsilon} y}
  (x,f(x)) \bsim_\epsilon^\xi (y,f(y))
\end{align*}
其计算如预期:
\begin{align}
  f(\rcrat(q)) &\defeq f_q, \label{eq:rcsimind1}\\
  f(\rclim(x)) &\defeq f_{x,(f,g)[x]}。 \label{eq:rcsimind2}
\end{align}
这里 $(f,g)[x]$ 表示将 $f$ 和 $g$ 应用于柯西逼近 $x$ 以获得 $x$ 上的依赖柯西逼近的结果。也就是说,我们定义 $(f,g)[x]_\epsilon \defeq f(x_\epsilon) : A(x_\epsilon)$,然后对于任何 $\epsilon,\delta:\Qp$,我们有 $g(x_\epsilon,x_\delta,\epsilon+\delta,\xi)$ 来证明 $(f,g)[x]$ 是依赖柯西逼近,其中 $\xi: x_\epsilon \close{\epsilon+\delta} x_\delta$ 来源于 $x$ 是柯西逼近的假设。

我们不会再使用这种符号表示,所以不用记住它。通常我们使用模式匹配约定,其中 $f$ 由类似于 \eqref{eq:rcsimind1} 和 \eqref{eq:rcsimind2} 的方程定义,其中 \eqref{eq:rcsimind2} 的右侧可能涉及符号 $f(x_\epsilon)$ 和假设它们形成依赖柯西逼近。

然而,这个归纳原则确实仍然相当复杂。为了帮助理解它,我们观察到它包含了两个关于 $\RC$ 和 $\closesym$ 的单独的归纳原则的特殊情况。首先,假设只给定一个类型族 $A:\RC\to\type$,并将 $\bsim$ 定义为常量 \unit。那么许多必要的数据就变得微不足道,我们剩下:
\begin{itemize}
  \item 对于任意 $q : \Q$,一个元素 $f_q:A(\rcrat(q))$,
  \item 对于任意柯西逼近 $x$,和任意 $a:\prd{\epsilon:\Qp} A(x_\epsilon)$,一个元素 $f_{x,a}:A(\rclim(x))$,
  \item 对于 $u, v : \RC$ 和 $h:\fall{\epsilon : \Qp} u \close\epsilon v$,以及 $a:A(u)$ 和 $b:A(v)$,我们有 $\dpath{A}{\rceq(u,v)}{a}{b}$。
\end{itemize}
在这些数据的基础上,归纳原则生成一个函数 $f:\prd{x:\RC} A(x)$,使得
\begin{align*}
  f(\rcrat(q)) &\defeq f_q,\\
  f(\rclim(x)) &\defeq f_{x,f(x)}。
\end{align*}
我们称此原则为 \define{$\RC$-归纳 ($\RC$-induction)};它基本上表明,如果我们认为 $\close\epsilon$ 是已知的,那么 $\RC$ 是通过其构造器归纳生成的。

注意,如果 $A$ 是单纯命题,那么 $\RC$-归纳中的第三个假设是自动成立的(我们将很快看到这些是等价的命题)。因此,我们可以通过简单地证明对有理数和柯西逼近的极限来证明实数的单纯命题。以下是一个例子。

\begin{lem} \label{lem:close-reflexive}
对于任意 $u:\RC$ 和 $\epsilon:\Qp$,我们有 $u\close\epsilon u$。
\end{lem}
\begin{proof}
  定义 $A(u) \defeq \fall{\epsilon:\Qp} (u\close\epsilon u)$。由于这是一个单纯命题(根据 $\closesym$ 的最后一个构造器),根据 $\RC$-归纳,我们只需要证明当 $u$ 是 $\rcrat(q)$ 和 $u$ 是 $\rclim(x)$ 时的情况。在第一个情况下,我们显然有 $|q-q|<\epsilon$ 对于任意 $\epsilon$,因此根据 $\closesym$ 的第一个构造器,$\rcrat(q) \close\epsilon \rcrat(q)$ 是成立的。
  %
  在第二种情况下,我们可以假设归纳地 $x_\delta \close\epsilon x_\delta$ 对于所有 $\delta,\epsilon:\Qp$ 成立。然后特别地,我们有 $x_{\epsilon/3} \close{\epsilon/3} x_{\epsilon/3}$,因此根据 $\closesym$ 的第四个构造器,$\rclim(x) \close{\epsilon} \rclim(x)$ 成立。
\end{proof}

从 \cref{lem:close-reflexive},我们推断,如果类型族 $A:\RC\to\type$ 是一个单纯命题,那么直接应用 $\RC$-归纳只有在这种情况下才有可能成功。为了证明这一点,固定 $u:\RC$。将 $v$ 设为 $u$,$\RC$-归纳的第三个假设告诉我们,对于任意 $a : A(u)$,我们有 $\dpath{A}{\rceq(u,u)}{a}{a}$。再给定一个点 $b : A(u)$,我们也得到 $\dpath{A}{\rceq(u,u)}{a}{b}$。根据依赖路径类型的定义,我们得出这些路径组合的结论是 $a = b$,即 $A(u)$ 中的所有点都是相等的。

\begin{thm}\label{thm:Cauchy-reals-are-a-set}
$\RC$ 是一个集合 (set)。
\end{thm}
\begin{proof}
  我们刚刚证明了
  \narrowequation{P(u,v) \defeq \fall{\epsilon:\Qp} (u\close\epsilon v)}
  是自反的。由于它暗含了恒等性,$\RC$ 的路径构造器表明结果可由 \cref{thm:h-set-refrel-in-paths-sets} 推导。
\end{proof}

我们还可以证明,尽管 $\RC$ 可能不是有理数柯西序列集合的商,但它仍然是实数柯西序列集合的商。(当然,这不是 $\RC$ 的有效\emph{定义},但它是一个有用的性质。)我们将柯西逼近的类型定义为
%
\symlabel{cauchy-approximations}%
\index{柯西!逼近!类型}%
\begin{equation*}
  \CAP \defeq
  \setof{ x : \Qp \to \RC |
  \fall{\epsilon, \delta : \Qp} x_\delta \close{\delta + \epsilon} x_\epsilon
  }.
\end{equation*}
$\RC$ 的第二个构造器提供了函数 $\rclim:\CAP\to\RC$。

\begin{lem} \label{RC-lim-onto}
每个实数 (real) 仅仅是一个极限点:$\fall{u : \RC} \exis{x : \CAP} u = \rclim(x)$。换句话说,$\rclim:\CAP\to\RC$ 是满射。
\end{lem}
\begin{proof}
  通过 $\RC$-归纳,我们可以对 $u$ 进行分情况讨论。当然,如果 $u$ 是一个极限 $\rclim(x)$,那么该命题是显然成立的。所以假设 $u$ 是一个有理点 $\rcrat(q)$;我们声称 $u$ 等于 $\rclim(\lam{\epsilon} \rcrat(q))$。通过 $\RC$ 的路径构造器,只需证明 $\rcrat(q) \close\epsilon \rclim(\lam{\epsilon} \rcrat(q))$ 对所有 $\epsilon:\Qp$ 成立。而根据 $\closesym$ 的第二个构造器,为此只需找到 $\delta:\Qp$ 使得 $\rcrat(q)\close{\epsilon-\delta} \rcrat(q)$。但根据 $\closesym$ 的第一个构造器,我们可以取任意 $\delta:\Qp$ 使得 $\delta<\epsilon$。
\end{proof}


%

\begin{lem} \label{RC-lim-factor}
如果 $A$ 是一个集合并且 $f : \CAP \to A$ 满足Cauchy近似的重合性\index{coincidence!of Cauchy approximations},即
%
\begin{equation*}
  \fall{x, y : \CAP} \rclim(x) = \rclim(y) \Rightarrow f(x) = f(y),
\end{equation*}
%
那么 $f$ 唯一地通过 $\rclim : \CAP \to \RC$ 因子化。
\end{lem}
\begin{proof}
  由于 $\rclim$ 是满射,根据 \cref{lem:images_are_coequalizers},$\RC$ 是 $\CAP$ 通过 $\rclim$ 的核对偶\index{kernel!pair} 的商集。
  这正是引理的表述。
\end{proof}

对于归纳原理的第二个特例,假设我们将 $A$ 取为常量 $\unit$。
在这种情况下,$\bsim$ 只是对 $\epsilon$ 近似对实数对的 $\epsilon$ 索引的关系族,因此我们可以写 $u\bsim_\epsilon v$ 代替 $(u,\ttt)\bsim_\epsilon (v,\ttt)$。
那么所需的数据简化为如下内容,其中 $q, r$ 表示有理数,$\epsilon, \delta, \eta$ 是正有理数,而 $x, y$ 是 Cauchy 近似:
\begin{itemize}
  \item 如果 $-\epsilon < q - r < \epsilon$,那么
  $\rcrat(q) \bsim_\epsilon \rcrat(r)$,
  \item 如果 $\rcrat(q) \close{\epsilon - \delta} y_\delta$ 并且
  $\rcrat(q)\bsim_{\epsilon-\delta} y_\delta$,
  那么 $\rcrat(q) \bsim_\epsilon \rclim(y)$,
  \item 如果 $x_\delta \close{\epsilon - \delta} \rcrat(r)$ 并且
  $x_\delta \bsim_{\epsilon-\delta} \rcrat(r)$,
  那么 $\rclim(y) \bsim_\epsilon \rcrat(q)$,
  \item 如果 $x_\delta \close{\epsilon - \delta - \eta} y_\eta$ 并且
  $x_\delta\bsim_{\epsilon - \delta - \eta} y_\eta$,
  那么 $\rclim(x) \bsim_\epsilon \rclim(y)$。
\end{itemize}
由此得出的结论是 $\fall{u,v:\RC}{\epsilon:\Qp} (u\close\epsilon v) \to (u \bsim_\epsilon v)$。
我们称这个原理为\define{$\closesym$-归纳};它本质上表明,如果我们将 $\RC$ 视为已知,那么 $\close\epsilon$ 由\emph{它的}构造函数归纳生成(作为类型族)。
例如,我们可以使用它来证明 $\closesym$ 是对称的。

\begin{lem}\label{thm:RCsim-symmetric}
对于任意的 $u,v:\RC$ 和 $\epsilon:\Qp$,我们有 $(u\close\epsilon v) = (v\close\epsilon u)$。
\end{lem}
\begin{proof}
  由于两者都是单纯命题,通过对称性我们只需证明一个方向的蕴涵。
  因此,令 $(u\bsim_\epsilon v) \defeq (v\close\epsilon u)$。
  通过 $\closesym$-归纳,我们可以将问题归结为 $u\close\epsilon v$ 由 $\closesym$ 的四个有趣构造函数之一导出。
  在第一种情况下,当 $u$ 和 $v$ 都是有理数时,结果是显然的(我们可以再次应用第一个构造函数)。
  在其他三种情况下,归纳假设(以及 $\Q$ 中加法的交换性)正好得出 $\closesym$ 的另一个构造函数的输入(第二个和第三个构造函数互换,而第四个保持不变)。
\end{proof}

一般归纳原理,我们可以称之为\define{$(\RC,\closesym)$-归纳},因此是一种联合的 $\RC$-归纳和 $\closesym$-归纳。
例如,考虑它的非依赖版本,我们称之为\define{$(\RC,\closesym)$-递归},这是我们最常用的。
\index{recursion principle!for Cauchy reals}%
普通的 $\RC$-递归告诉我们,要定义一个函数 $f : \RC \to A$,只需:
\begin{enumerate}
  \item 对每个 $q : \Q$ 构造 $f(\rcrat(q)) : A$,
  \item 对每个 Cauchy 近似 $x : \Qp \to \RC$,构造 $f(x) : A$,
  假设已为所有 $\epsilon : \Qp$ 定义了 $f(x_\epsilon)$,
  \item 证明对于所有满足 $\fall{\epsilon:\Qp} u\close\epsilon v$ 的 $u, v : \RC$,有 $f(u) = f(v)$。\label{item:rcrec3}
\end{enumerate}
然而,在不了解 $f$ 如何作用于 $\epsilon$ 近似的 Cauchy 实数的情况下,通常很难证明~\ref{item:rcrec3}。
$(\RC,\closesym)$-递归的增强原理弥补了这一缺陷,使我们能够指定一个 \emph{任意的} ``$f$ 作用于 $\epsilon$ 近似的 Cauchy 实数的方式'',然后我们可以通过与 $f$ 的定义同时归纳来证明这一点。
这就是关系族 $\bsim$。
由于 $A$ 独立于 $\RC$,为简单起见,我们可以假设 $\bsim$ 仅依赖于 $A$ 和 $\Qp$,因此在写 $a\bsim_\epsilon b$ 时没有歧义,而不是 $(u,a) \bsim_\epsilon (v,b)$。
在这种情况下,通过 $(\RC,\closesym)$-递归定义函数 $f:\RC\to A$ 需要以下情况(我们现在使用模式匹配约定来写)。
\begin{itemize}
  \item 对于每个 $q : \Q$,构造 $f(\rcrat(q)) : A$。
  \item 对于每个 Cauchy 近似 $x : \Qp \to \RC$,构造 $f(\rclim(x)) : A$,假设归纳地已经为所有 $\epsilon : \Qp$ 定义了 $f(x_\epsilon)$ 并形成一个 ``相对于 $\bsim$ 的 Cauchy 近似'',即 $\fall{\epsilon,\delta:\Qp} (f(x_\epsilon) \bsim_{\epsilon+\delta} f(x_\delta))$。
  \item 证明这些关系 $\bsim$ 是\emph{分离的},即对于任意 $a,b:A$,
  \indexdef{relation!separated family of}%
  \indexdef{separated family of relations}%
  \narrowequation{(\fall{\epsilon:\Qp} a\bsim_\epsilon b) \Rightarrow (a=b)。}
  \item 证明如果 $-\epsilon< q-r <\epsilon$ 对于 $q,r:\Q$,那么 $f(\rcrat(q))\bsim_\epsilon f(\rcrat(r))$。
  \item 对于任意 $q:\Q$ 和任意 Cauchy 近似 $y$,证明
  \narrowequation{f(\rcrat(q)) \bsim_\epsilon f(\rclim(y)),} 假设归纳地 $\rcrat(q)\close{\epsilon-\delta} y_\delta$ 并且 $f(\rcrat(q)) \bsim_{\epsilon-\delta} f(y_\delta)$ 对于某个 $\delta:\Qp$,并且 $\eta \mapsto f(x_\eta)$ 是相对于 $\bsim$ 的 Cauchy 近似。
  \item 对于任意 Cauchy 近似 $x$ 和任意 $r:\Q$,证明
  \narrowequation{f(\rclim(x)) \bsim_\epsilon f(\rcrat(r)),}
  假设归纳地 $x_\delta \close{\epsilon-\delta} \rcrat(r)$ 并且 $f(x_\delta) \bsim_{\epsilon-\delta} f(\rcrat(r))$ 对于某个 $\delta:\Qp$,并且 $\eta\mapsto f(x_\eta)$ 是相对于 $\bsim$ 的 Cauchy 近似。
  \item 对于任意 Cauchy 近似 $x,y$,证明
  \narrowequation{f(\rclim(x)) \bsim_\epsilon f(\rclim(y)),}
  假设归纳地 $x_\delta \close{\epsilon-\delta-\eta} y_\eta$ 并且 $f(x_\delta) \bsim_{\epsilon-\delta-\eta} f(y_\eta)$ 对于某个 $\delta,\eta:\Qp$,并且 $\theta\mapsto f(x_\theta)$ 和 $\theta\mapsto f(y_\theta)$ 是相对于 $\bsim$ 的 Cauchy 近似。
\end{itemize}
注意,在最后四个证明中,我们可以自由使用在前两个数据中给出的 $f(\rcrat(q))$ 和 $f(\rclim(x))$ 的具体定义。
然而,分离性的证明必须适用于 $A$ 的\emph{任意} 两个元素,而与 $f$ 无关:它是一种对关系族 $\bsim$ 的``可接纳性''条件。
因此,我们通常首先验证它,在定义 $\bsim$ 之后立即进行,然后再继续定义 $f(\rcrat(q))$ 和 $f(\rclim(x))$。

在上述假设下,$(\RC,\closesym)$-递归产生一个函数 $f:\RC\to A$,使得 $f(\rcrat(q))$ 和 $f(\rclim(x))$ 在判断上等于在前两个子句中给出的定义。
此外,我们还可以得出
\begin{equation}
  \fall{u,v:\RC}{\epsilon:\Qp} (u\close\epsilon v) \to (f(u) \bsim_\epsilon f(v)).\label{eq:RC-sim-recursion-extra}
\end{equation}

作为一个典型的例子,$(\RC,\closesym)$-递归允许我们将定义在 $\Q$ 上的函数扩展到所有 $\RC$ 上,只要它们足够连续。
\index{function!continuous}%

\begin{defn}\label{defn:lipschitz}
函数 $f:\Q\to\RC$ 是 \define{Lipschitz}
\indexdef{function!Lipschitz}%
\indexdef{Lipschitz!function}%
\indexdef{Lipschitz!constant}%
\indexdef{constant!Lipschitz}%
的,如果存在 $L:\Qp$(\define{Lipschitz常数}),使得
\[ |q - r|<\epsilon \Rightarrow (f(q) \close{L\epsilon} f(r)) \]
对于所有 $\epsilon:\Qp$ 和 $q,r:\Q$。
%
类似地,$g:\RC\to\RC$ 是 \define{Lipschitz} 的,如果存在 $L:\Qp$ 使得
\[ (u\close\epsilon v) \Rightarrow (g(u) \close{L\epsilon} g(v)) \]
对于所有 $\epsilon:\Qp$ 和 $u,v:\RC$。
\end{defn}

特别地,注意到通过 $\closesym$ 的第一个构造函数,如果 $f:\Q\to\Q$ 在显然的意义上是 Lipschitz 的,那么复合函数 $\Q\xrightarrow{f} \Q \to \RC$ 也是如此。

\begin{lem}\label{RC-extend-Q-Lipschitz}
假设 $f : \Q \to \RC$ 是常数为 $L : \Qp$ 的 Lipschitz 函数。
那么存在一个 Lipschitz 映射 $\bar{f} : \RC \to \RC$,其常数也是 $L$,并且对于所有 $q:\Q$,有 $\bar{f}(\rcrat(q)) \jdeq f(q)$。
\end{lem}

\begin{proof}
    % 唯一性直接由 \cref{RC-continuous-eq} 推出。
  我们通过 $(\RC,\closesym)$-递归定义 $\bar{f}$,其余象为 $A\defeq \RC$。
  我们定义关系 $\mathord{\bsim}: \RC \to \RC \to \Qp \to \prop$ 为
  \begin{align*}
  (u \bsim_\epsilon v) &\defeq (u \close{L\epsilon} v)。
  \end{align*}
  对于 $q : \Q$,我们定义
  %
  \begin{equation*}
    \bar{f}(\rcrat(q)) \defeq \rcrat(f(q))。
  \end{equation*}
  %
  对于 Cauchy 近似 $x : \Qp \to \RC$,我们定义
  %
  \begin{equation*}
    \bar{f}(\rclim(x)) \defeq \rclim(\lamu{\epsilon : \Qp} \bar{f}(x_{\epsilon/L}))。
  \end{equation*}
  %
  为了使这有意义,我们必须验证 $y \defeq \lamu{\epsilon : \Qp} \bar{f}(x_{\epsilon/L})$ 是一个 Cauchy 近似。
  然而,这一步的归纳假设是,对于任何 $\delta,\epsilon:\Qp$,我们有 $\bar{f}(x_\delta) \bsim_{\delta+\epsilon} \bar{f}(x_\epsilon)$,即 $\bar{f}(x_\delta) \close{L\delta+L\epsilon} \bar{f}(x_\epsilon)$。
  因此我们有
  \[y_\delta \jdeq f(x_{\delta/L}) \close{\delta + \epsilon} f(x_{\epsilon/L})   \jdeq y_\epsilon。 \]

  为了证明分离性,我们简单地观察到 $\fall{\epsilon:\Qp} a\bsim_\epsilon b$ 意味着 $\fall{\epsilon:\Qp} a\close{L\epsilon} b$,这意味着 $\fall{\epsilon:\Qp}a\close\epsilon b$,从而 $a=b$。

  为了完成 $(\RC,\closesym)$-递归,还需要验证 $\bsim$ 上的四个条件。
  这基本上相当于证明 $\bar f$ 对 $\closesym$ 的所有四个构造函数都是 Lipschitz 的。
  \begin{enumerate}
    \item 当 $u$ 是 $\rcrat(q)$ 并且 $v$ 是 $\rcrat(r)$ 时,如果 $-\epsilon < |q-r| <\epsilon$,假设 $f$ 是 Lipschitz 的得出 $f(q) \close{L\epsilon} f(r)$,因此 $\bar{f}(\rcrat(q)) \bsim_\epsilon \bar{f}(\rcrat(r))$ 通过定义。
    \item 当 $u$ 是 $\rclim(x)$ 并且 $v$ 是 $\rcrat(q)$,如果 $x_\eta \close{\epsilon - \eta} \rcrat(q)$,那么归纳假设是 $\bar{f}(x_\eta) \close{L \epsilon - L \eta} \rcrat(f(q))$,这证明了
    \narrowequation{\bar{f}(\rclim(x)) \close{L \epsilon} \bar{f}(\rcrat(q))。}
    \item 当 $u$ 是有理数而 $v$ 是极限时,对称情况基本相同。
    \item 当 $u$ 是 $\rclim(x)$ 并且 $v$ 是 $\rclim(y)$,其中 $\delta, \eta : \Qp$ 使得 $x_\delta \close{\epsilon - \delta - \eta} y_\eta$,
    归纳假设是 $\bar{f}(x_\delta) \close{L \epsilon - L \delta - L \eta} \bar{f}(y_\eta)$,这证明了 $\bar{f}(\rclim(x)) \close{L
    \epsilon} \bar{f}(\rclim(y))$ 通过 $\closesym$ 的第四个构造函数。
  \end{enumerate}
  这完成了 $(\RC,\closesym)$-递归,从而完成了 $\bar f$ 的构造。
  所需的等式 $\bar f(\rcrat(q))\jdeq f(q)$ 正是 $(\RC,\closesym)$-递归的第一个计算规则,附加条件~\eqref{eq:RC-sim-recursion-extra} 正是表明 $\bar f$ 是常数为 $L$ 的 Lipschitz 的。
\end{proof}

在这一点上,如果没有对 $\closesym$ 的更好表征,我们已经做到了尽可能多的事情。
我们在 $\closesym$ 的构造函数中指定了我们希望何种形式的 Cauchy 实数在 $\epsilon$-近似时是 $\epsilon$-close。
然而,我们如何知道在所生成的归纳-归纳类型族中,这些是 \emph{唯一的} 证明此事实的见证者?
我们已经看到,归纳类型族(如恒等类型,参见 \cref{sec:identity-systems})和更高的归纳类型有包含 ``超出它们的内容'' 的倾向,因此这不是一个空洞的问题。

为了更精确地表征 $\closesym$,我们将定义一个关系族 $\approx_\epsilon$ 在 $\RC$ 上\emph{递归}地计算,使其能够在构造函数上计算,并证明这个关系族等价于 $\close\epsilon$。

\begin{thm}\label{defn:RC-approx}
存在一个关系族 $\mathord\approx:\RC\to\RC\to\Qp\to\prop$,使得
\begin{align}
(\rcrat(q) \approx_\epsilon \rcrat(r))  &\defeq
(-\epsilon < q - r < \epsilon)\label{eq:RCappx1}\\
(\rcrat(q) \approx_\epsilon \rclim(y)) &\defeq
\exis{\delta : \Qp} \rcrat(q) \approx_{\epsilon - \delta} y_\delta\label{eq:RCappx2}\\
(\rclim(x) \approx_\epsilon \rcrat(r)) &\defeq
\exis{\delta : \Qp} x_\delta \approx_{\epsilon - \delta} \rcrat(r)\label{eq:RCappx3}\\
(\rclim(x) \approx_\epsilon \rclim(y)) &\defeq
\exis{\delta, \eta : \Qp} x_\delta \approx_{\epsilon - \delta - \eta} y_\eta。\label{eq:RCappx4}
\end{align}
此外,我们有
\begin{gather}
(u \approx_\epsilon v) \Leftrightarrow \exis{\theta:\Qp} (u \approx_{\epsilon-\theta} v) \label{RC-sim-rounded}\\
(u \approx_\epsilon v) \to (v\close\delta w) \to (u\approx_{\epsilon+\delta} w)\label{eq:RC-sim-rtri}\\
(u \close\epsilon v) \to (v\approx_\delta w) \to (u\approx_{\epsilon+\delta} w)\label{eq:RC-sim-ltri}。
\end{gather}
\end{thm}

附加条件~\eqref{RC-sim-rounded}--\eqref{eq:RC-sim-ltri} 证明了使得递归定义能够进行。
条件~\eqref{RC-sim-rounded} 被称为\define{圆滑的}。
\indexsee{relation!rounded}{rounded relation}%
\indexdef{rounded!relation}%
从右向左阅读得出 $\approx$ 的 \define{单调性},
\index{monotonicity}%
\index{relation!monotonic}%
%
\begin{equation*}
(\delta < \epsilon) \land (u \approx_\delta v) \Rightarrow (u \approx_\epsilon v)
\end{equation*}
%
而从左向右阅读则得出 $\approx$ 的 \define{开放性},
\index{open!relation}%
\index{relation!open}%
%
\begin{equation*}
(u \approx_\epsilon v) \Rightarrow \exis{\delta : \Qp} (\delta < \epsilon) \land (u \approx_\delta v)。
\end{equation*}
%
条件~\eqref{eq:RC-sim-rtri} 和~\eqref{eq:RC-sim-ltri} 是三角不等式的形式,表明 $\approx$ 在两侧是 $\closesym$ 的``模''。

\begin{proof}
  我们将通过双 $(\RC,\closesym)$-递归定义 $\mathord\approx:\RC\to\RC\to\Qp\to\prop$。
  首先我们将应用 $(\RC,\closesym)$-递归,其余象是 $\RC\to\Qp\to\prop$ 的子集,包含那些圆滑并满足一种适当形式的三角不等式的谓词族。
  将这些谓词视为二元关系的一半,我们将它们写为 $(u,\epsilon) \mapsto (\hapx_\epsilon u)$,其中符号 $\hapname$ 指整个关系。
  现在我们可以精确地写出 $A$
  \begin{multline*}
    A \defeq\; \Bigg\{ \hapname :\RC\to\Qp\to\prop \;\bigg|\; \\
    \Big(\fall{u:\RC}{\epsilon:\Qp}
    \big((\hapx_\epsilon u) \Leftrightarrow \exis{\theta:\Qp} (\hapx_{\epsilon-\theta} u)\big)\Big)  \\
    \land \Big(\fall{u,v:\RC}{\eta,\epsilon:\Qp} (u\close\epsilon v) \to\\
    \big((\hapx_\eta u) \to (\hapx_{\eta+\epsilon} v) \big) \land \big((\hapx_\eta v) \to (\hapx_{\eta+\epsilon} u) \big)\Big)\Bigg\}
  \end{multline*}
  与通常的子集一样,我们将使用相同的符号来表示 $A$ 的一个元素及其第一个分量 $\hapname$。
  作为 $(\RC,\closesym)$-递归所需的关系族,我们考虑以下内容,这将确保三角不等式的另一种形式:
  \begin{narrowmultline*}
  (\hapname \bsim_\epsilon \hapbname ) \defeq \narrowbreak
  \fall{u:\RC}{\eta:\Qp} ((\hapx_\eta u) \to (\hapxb_{\epsilon+\eta} u))
  \land \narrowbreak
  ((\hapxb_\eta u) \to (\hapx_{\epsilon+\eta} u))。
  \end{narrowmultline*}
  我们观察到这些关系是分离的。
  因为假设
  \narrowequation{\fall{\epsilon:\Qp} (\hapname \bsim_\epsilon \hapbname),}
  要证明 $\hapname = \hapbname$,只需证明对于所有 $u:\RC$,$(\hapx_\epsilon u) \Leftrightarrow (\hapxb_\epsilon u)$。
  但 $\hapx_\epsilon u$ 意味着对于某个 $\theta$,$\hapx_{\epsilon-\theta} u$,根据圆滑性,并且 $\hapname \bsim_\epsilon \hapbname$ 意味着 $\hapxb_\epsilon u$;反之亦然。

  现在,递归原则所需的前两个数据如下。
  \begin{itemize}
    \item 对于任何 $q:\Q$,我们必须给出 $A$ 的一个元素,我们将其记作 $(\rcrat(q)\approx_{(\blank)} \blank)$。
    \item 对于任何 Cauchy 近似 $x$,如果我们假设定义了函数 $\Qp \to A$,我们将其记作 $\epsilon \mapsto (x_\epsilon \approx_{(\blank)} \blank)$,其具有以下属性:
    % \[ \fall{u,v:\RC}{\delta,\epsilon,\eta:\Qp} (x_\delta \approx_\eta u) \to (u\close{\delta+\epsilon} v) \to (x_\epsilon \approx_{\eta+\delta+\epsilon} v) \]
    \begin{equation}
      \fall{u:\RC}{\delta,\epsilon,\eta:\Qp} (x_\delta \approx_\eta u) \to (x_\epsilon \approx_{\eta+\delta+\epsilon} u),\label{eq:appxrec2}
    \end{equation}
    我们必须给出 $A$ 的一个元素,我们将其写作 $(\rclim(x)\approx_{(\blank)} \blank)$。
  \end{itemize}
  在这两种情况下,我们通过使用嵌套 $(\RC,\closesym)$-递归给出所需定义,其余象为由圆滑的命题构成的 $\Qp\to\prop$ 的子集。
  将这些命题视为二元关系的零一半,我们将它们写作 $\epsilon \mapsto (\tap{\epsilon})$,符号 $\tapname$ 指整个关系族。
  现在我们可以精确地写出这些内递归的余象:
  \begin{narrowmultline*}
    C \defeq
    \bigg\{ \tapname :\Qp\to\prop \;\;\Big|\;\; \narrowbreak
    \fall{\epsilon:\Qp} \Big((\tap\epsilon) \Leftrightarrow \exis{\theta:\Qp} (\tap{\epsilon-\theta})\Big)\bigg\}
  \end{narrowmultline*}
  我们将所需的关系族定义为三角不等式的剩余部分:
  \begin{narrowmultline*}
  (\tapname \bbsim_\epsilon \tapbname) \defeq
  \fall{\eta:\Qp} ((\tap\eta) \to (\tapb{\epsilon+\eta})) \land
  \narrowbreak
  ((\tapb\eta) \to (\tap{\epsilon+\eta}))。
  \end{narrowmultline*}
  通过与 $\bsim$ 类似的论证,这些关系是分离的,使用 $C$ 的所有元素的圆滑性。

  注意,如果这样的内递归成功,它将生成一个谓词族 $\hapname : \RC\to\Qp\to \prop$,这些谓词族是圆滑的
  (因为它们在 $\Qp\to\prop$ 中的象属于 $C$)并且满足
  \[ \fall{u,v:\RC}{\epsilon:\Qp} (u\close\epsilon v) \to \big((\hapx_{(\blank)} u) \bbsim_\epsilon (\hapx_{(\blank)} u)\big)。 \]
  展开 $\bbsim$ 的定义,这正是使得 $\hapname$ 属于 $A$ 的第三个条件;因此,这正是我们需要的。

  此时我们可以给出定义~\eqref{eq:RCappx1}--\eqref{eq:RCappx4},作为两个内递归的前两个子句,分别对应于有理点和极限。
  在每种情况下,我们必须验证关系是圆滑的,因此属于 $C$。
  在有理-有理的情况下~\eqref{eq:RCappx1},这是显然的,而在其他情况下则来自归纳假设。
  (在~\eqref{eq:RCappx2} 中,相关的归纳假设是 $(\rcrat(q) \approx_{(\blank)} y_\delta) : C$,而在~\eqref{eq:RCappx3} 和~\eqref{eq:RCappx4} 中,它是 $(x_\delta \approx_{(\blank)} \blank) : A$。)

  余下的子递归数据包括证明~\eqref{eq:RCappx1}--\eqref{eq:RCappx4} 满足相对于 $\closesym$ 构造函数的右侧三角不等式。
  有八种情况 —— 每个子递归中的四个 —— 对应于 $u$、$v$ 和 $w$ 在~\eqref{eq:RC-sim-rtri} 中可以选择为有理点或极限的八种方式。
  首先考虑 $u$ 是 $\rcrat(q)$ 的情况。
  \begin{enumerate}
    \item 假设 $\rcrat(q)\approx_\phi \rcrat(r)$ 并且 $-\epsilon<|r-s|<\epsilon$,我们必须证明 $\rcrat(q)\approx_{\phi+\epsilon} \rcrat(s)$。
    但根据 $\approx$ 的定义,这简化为有理数的三角不等式。
    \item 我们假设 $\phi,\epsilon,\delta:\Qp$ 使得 $\rcrat(q)\approx_\phi \rcrat(r)$ 并且 $\rcrat(r) \close{\epsilon-\delta} y_\delta$,并且归纳地
    \begin{equation}
      \fall{\psi:\Qp}(\rcrat(q) \approx_{\psi} \rcrat(r)) \to (\rcrat(q) \approx_{\psi+\epsilon-\delta} y_\delta)。\label{eq:RCappx-rtri-rrl1}
    \end{equation}
    我们还假设 $\psi,\delta\mapsto (\rcrat(q) \approx_{\psi} y_\delta)$ 是相对于 $\bbsim$ 的 Cauchy 近似,即
    \begin{equation}
      \fall{\psi,\xi、zeta:\Qp} (\rcrat(q) \approx_{\psi} y_\xi) \to (\rcrat(q) \approx_{\psi+\xi+\zeta} y_\zeta),\label{eq:RCappx-rtri-rrl2}
    \end{equation}
    尽管在这种情况下我们不需要这种假设。
    事实上,\eqref{eq:RCappx-rtri-rrl1} 直接给出了 $\rcrat(q) \approx_{\phi+\epsilon-\delta} y_\delta$,从而通过 $\approx$ 的定义 $\rcrat(q) \approx_{\phi+\epsilon} \rclim(y)$。
    \item 我们假设 $\phi,\epsilon,\delta:\Qp$ 使得 $\rcrat(q)\approx_\phi \rclim(y)$ 并且 $y_\delta \close{\epsilon-\delta} \rcrat(r)$,并且归纳地
    \begin{gather}
      \fall{\psi:\Qp}(\rcrat(q) \approx_{\psi} y_\delta) \to (\rcrat(q) \approx_{\psi+\epsilon-\delta} \rcrat(r))。\label{eq:RCappx-rtri-rlr1}\\
      \fall{\psi,\xi,\zeta:\Qp} (\rcrat(q) \approx_{\psi} y_\xi) \to (\rcrat(q) \approx_{\psi+\xi+\zeta} y_\zeta)。\label{eq:RCappx-rtri-rlr2}
    \end{gather}
    根据定义,$\rcrat(q)\approx_\phi \rclim(y)$ 意味着我们有 $\theta:\Qp$ 使得 $\rcrat(q) \approx_{\phi-\theta} y_\theta$。
    根据假设~\eqref{eq:RCappx-rtri-rlr2},因此我们还得出 $\rcrat(q) \approx_{\phi+\delta} y_\delta$,然后根据~\eqref{eq:RCappx-rtri-rlr1} 得出 $\rcrat(q) \approx_{\phi+\epsilon} \rcrat(r)$,如所需。
    \item 我们假设 $\phi,\epsilon,\delta、eta:\Qp$ 使得 $\rcrat(q)\approx_\phi \rclim(y)$ 并且 $y_\delta \close{\epsilon-\delta-\eta} z_\eta$,并且归纳地
    \begin{gather}
      \fall{\psi:\Qp}(\rcrat(q) \approx_{\psi} y_\delta) \to (\rcrat(q) \approx_{\psi+\epsilon-\delta-\eta} z_\eta), \label{eq:RCappx-rtri-rll1}\\
      \fall{\psi,\xi,\zeta:\Qp} (\rcrat(q) \approx_{\psi} y_\xi) \to (\rcrat(q) \approx_{\psi+\xi+\zeta} y_\zeta), \label{eq:RCappx-rtri-rll2}\\
      \fall{\psi,\xi,\zeta:\Qp} (\rcrat(q) \approx_{\psi} z_\xi) \to (\rcrat(q) \approx_{\psi+\xi+\zeta} z_\zeta)。 \label{eq:RCappx-rtri-rll3}
    \end{gather}
    同样地,$\rcrat(q)\approx_\phi \rclim(y)$ 意味着我们有 $\xi:\Qp$ 使得 $\rcrat(q) \approx_{\phi-\xi} y_\xi$,而~\eqref{eq:RCappx-rtri-rll2} 然后意味着 $\rcrat(q) \approx_{\phi+\delta} y_\delta$ 并且~\eqref{eq:RCappx-rtri-rll1} 意味着 $\rcrat(q) \approx_{\phi+\epsilon-\eta} z_\eta$。
    但根据 $\approx$ 的定义,这意味着 $\rcrat(q) \approx_{\phi+\epsilon} \rclim(z)$ 如所需。
  \end{enumerate}
  现在我们继续讨论 $u$ 是 $\rclim(x)$ 的情况,其中 $x$ 是 Cauchy 近似。
  在这种情况下,$(\rclim(x) \approx_{(\blank)} {\blank}) : A$ 的定义的背景归纳假设是我们有 $(x_\delta \approx_{(\blank)} {\blank}) : A$,因此除了是圆滑的外,它们还满足右侧三角不等式。
  \begin{enumerate}\setcounter{enumi}{4}
  \item 假设 $\rclim(x)\approx_\phi \rcrat(r)$ 并且 $-\epsilon<|r-s|<\epsilon$,我们必须证明 $\rclim(x)\approx_{\phi+\epsilon} \rcrat(s)$。
  根据 $\approx$ 的定义,前者意味着 $x_\delta \approx_{\phi-\delta} \rcrat(r)$,因此上述三角不等式得出 $x_\delta \approx_{\epsilon+\phi-\delta} \rcrat(s)$,从而 $\rclim(x)\approx_{\phi+\epsilon} \rcrat(s)$ 如所需。
  \item 我们假设 $\phi,\epsilon、delta:\Qp$ 使得 $\rclim(x)\approx_\phi \rcrat(r)$ 并且 $\rcrat(r) \close{\epsilon-\delta} y_\delta$,并且有两个不需要的归纳假设。
  %
  根据定义,我们有 $\eta:\Qp$ 使得 $x_\eta \approx_{\phi-\eta} \rcrat(r)$,因此归纳三角不等式得出 $x_\eta \approx_{\phi+\epsilon-\eta-\delta} y_\delta$。
  然后 $\approx$ 的定义立即得出 $\rclim(x) \approx_{\phi+\epsilon} \rclim(y)$。
  \item 我们假设 $\phi,\epsilon、delta:\Qp$ 使得 $\rclim(x)\approx_\phi \rclim(y)$ 并且 $y_\delta \close{\epsilon-\delta} \rcrat(r)$,并且有两个不需要的归纳假设。
  根据定义,我们有 $\xi、theta:\Qp$ 使得 $x_\xi \approx_{\phi-\xi-\theta} y_\theta$。
  由于 $y$ 是 Cauchy 近似,我们有 $y_\theta \close{\theta+\delta} y_\delta$,因此归纳三角不等式得出 $x_\xi \approx_{\phi+\delta-\xi} y_\delta$ 然后 $x_\xi \close{\phi+\epsilon-\xi} \rcrat(r)$。
  然后 $\approx$ 的定义得出 $\rclim(x) \approx_{\phi+\epsilon}\rcrat(r)$,如所需。
  \item 最后,我们假设 $\phi,\epsilon、delta、eta:\Qp$ 使得 $\rclim(x)\approx_\phi \rclim(y)$ 并且 $y_\delta \close{\epsilon-\delta-\eta} z_\eta$。
  然后如前所述,我们有 $\xi、theta:\Qp$ 使得 $x_\xi \approx_{\phi-\xi-\theta} y_\theta$,并且如前所述的两个三角不等式足够了。
  \end{enumerate}

  这完成了两个内递归,从而完成了 $(\rclim(x) \approx_{(\blank)} {\blank}) : A$ 的定义。
  由于所有都是 $A$ 的元素,它们是圆滑的,并且满足相对于 $\closesym$ 的右侧三角不等式。
% , 并且满足~\eqref{eq:appxrec2}。
  余下的任务是验证与 $\bsim$ 相关的条件,即这些关系满足相对于 $\closesym$ 构造函数的\emph{左侧}三角不等式。
  四种情况对应于在~\eqref{eq:RC-sim-ltri} 中为 $u$ 和 $v$ 选择有理点或极限的四种选择,并且由于它们都是纯命题,我们可以应用 $\RC$ 归纳,并假设 $w$ 也是有理点或极限。
  这产生了另外八种情况,其证明基本与前述证明相同;因此我们不再让读者经历它们。
\end{proof}

我们现在可以证明:

\begin{thm}\label{thm:RC-sim-characterization}
对于任意的 $u,v:\RC$ 和 $\epsilon:\Qp$,我们有 $(u\close\epsilon v) = (u\approx_\epsilon v)$。
\end{thm}
\begin{proof}
  由于两者都是纯命题,足以证明双向蕴涵。
  对于左到右方向,我们使用 $\closesym$-归纳,应用于 $C(u,v,\epsilon)\defeq (u\approx_\epsilon v)$。
  因此,足以考虑 $\closesym$ 的四个构造函数。
  在每种情况下,$u$ 和 $v$ 专门化为有理点或极限,因此 $\approx$ 的定义计算并且归纳假设总是适用。

  对于右到左方向,我们使用 $\RC$-归纳来假设 $u$ 和 $v$ 是有理点或极限,允许 $\approx$ 计算。
  但现在 $\approx$ 的定义以及归纳假设提供了 $\closesym$ 构造函数所需的数据。
\end{proof}

\index{encode-decode method}%
略微夸张地说,可以将 $\approx$ 称为 $\closesym$ 的``编码'',上述证明的两个方向分别是 \encode 和 \decode。
根据 $\approx$ 的定义,从 \cref{thm:RC-sim-characterization} 我们得出等价关系
\begin{align*}
(\rcrat(q) \close\epsilon \rcrat(r))  &=
(-\epsilon < q - r < \epsilon)\\
(\rcrat(q) \close\epsilon \rclim(y)) &=
\exis{\delta : \Qp} \rcrat(q) \close{\epsilon - \delta} y_\delta\\
(\rclim(x) \close\epsilon \rcrat(r)) &=
\exis{\delta : \Qp} x_\delta \close{\epsilon - \delta} \rcrat(r)\\
(\rclim(x) \close\epsilon \rclim(y)) &=
\exis{\delta, \eta : \Qp} x_\delta \close{\epsilon - \delta - \eta} y_\eta。
\end{align*}
我们的证明还提供了以下附加信息。

\begin{cor}
  \index{triangle!inequality for R@inequality for $\RC$}%
  \indexsee{inequality!triangle}{triangle inequality}%
  $\closesym$ 是圆滑的\index{rounded!relation}并且满足三角不等式:
  \begin{gather}
    \eqvspaced{
      (u \close\epsilon v)
    }{
      \exis{\theta : \Qp} u \close{\epsilon - \theta} v
    }\\
    (u\close\epsilon v) \to (v\close\delta w) \to (u\close{\epsilon+\delta} w)。 \label{item:RC-sim-triangle}
  \end{gather}
\end{cor}
% \begin{proof}
%   $\approx$ 的构造同时证明了它是圆滑的,并且满足``三角不等式'',如
%   \[ (u\approx_\epsilon v) \to (v\close\delta w) \to (u\approx_{\epsilon+\delta} w)。 \]
%   因此,这两个性质由 \cref{thm:RC-sim-characterization} 得出。
% \end{proof}

有了三角不等式,我们可以证明 Cauchy 近似的``极限''实际上表现得像极限。

\begin{lem}\label{thm:RC-sim-lim}
对于任意的 $u:\RC$,Cauchy 近似 $y$,和 $\epsilon,\delta:\Qp$,如果 $u\close\epsilon y_\delta$ 那么 $u\close{\epsilon+\delta} \rclim(y)$。
\end{lem}
\begin{proof}
  我们对 $u$ 使用 $\RC$-归纳。
  如果 $u$ 是 $\rcrat(q)$,那么这正是 $\closesym$ 的第二个构造函数。
  现在假设 $u$ 是 $\rclim(x)$,并且每个 $x_\eta$ 具有以下性质:对于任意 $y,\epsilon,\delta$,如果 $x_\eta\close\epsilon y_\delta$ 那么 $x_\eta \close{\epsilon+\delta} \rclim(y)$。
  特别地,对于 $y\defeq x$ 和 $\delta\defeq\eta$ 在此假设下,我们得出对于任意 $\eta,\theta$,$x_\eta \close{\eta+\theta} \rclim(x)$。

  现在令 $y,\epsilon,\delta$ 任意,并假设 $\rclim(x) \close\epsilon y_\delta$。
  通过圆滑性,存在一个 $\theta$ 使得 $\rclim(x) \close{\epsilon-\theta} y_\delta$。
  然后,根据上述观察,对于任何 $\eta$,我们有 $x_\eta \close{\eta+\theta/2} \rclim(x)$,因此通过三角不等式得出 $x_\eta \close{\epsilon+\eta-\theta/2} y_\delta$。
  因此,$\closesym$ 的第四个构造函数得出 $\rclim(x) \close{\epsilon+2\eta+\delta-\theta/2} \rclim(y)$。
  因此,如果我们选择 $\eta \defeq \theta/4$,则结果成立。
\end{proof}

\begin{lem}\label{thm:RC-sim-lim-term}
对于任意的 Cauchy 近似 $y$ 和任意的 $\delta,\eta:\Qp$ 我们有 $y_\delta \close{\delta+\eta} \rclim(y)$。
\end{lem}
\begin{proof}
  在前一个引理中取 $u\defeq y_\delta$ 和 $\epsilon\defeq \eta$。
\end{proof}

\begin{rmk}
  我们可能期望有 $y_\delta \close{\delta} \rclim(y)$,但在某些例子中这会失败。
  例如,考虑定义为 $x_\epsilon \defeq \epsilon$ 的 $x$。
  它的极限显然是 $0$,但我们没有 $|\epsilon - 0 |<\epsilon$,只有 $\le$。
\end{rmk}

作为一个应用,\cref{thm:RC-sim-lim-term} 使我们能够证明 \cref{RC-extend-Q-Lipschitz} 中 Lipschitz 函数的扩展是唯一的。

\begin{lem}\label{RC-continuous-eq}
\index{function!continuous}%
令 $f,g:\RC\to\RC$ 是连续的,意思是
\[ \fall{u:\RC}{\epsilon:\Qp}\exis{\delta:\Qp}\fall{v:\RC} (u\close\delta v) \to (f(u) \close\epsilon f(v)) \]
并且类似地对于 $g$。
如果对于所有 $q:\Q$ 有 $f(\rcrat(q))=g(\rcrat(q))$,那么 $f=g$。
\end{lem}
\begin{proof}
  我们通过 $\RC$-归纳证明对于所有 $u$ 有 $f(u)=g(u)$。
  有理数情况只是假设。
  因此,假设对于所有 $\delta$ 有 $f(x_\delta)=g(x_\delta)$。
  我们将证明对于所有 $\epsilon$,$f(\rclim(x))\close\epsilon g(\rclim(x))$,因此 $\RC$ 的路径构造函数适用。

  由于 $f$ 和 $g$ 是连续的,存在 $\theta,\eta$ 使得对于所有 $v$,我们有
  \begin{align*}
  (\rclim(x)\close\theta v) &\to (f(\rclim(x)) \close{\epsilon/2} f(v))\\
  (\rclim(x)\close\eta v) &\to (g(\rclim(x)) \close{\epsilon/2} g(v))。
  \end{align*}
  选择 $\delta < \min(\theta,\eta)$,通过 \cref{thm:RC-sim-lim-term} 我们有 $\rclim(x)\close\theta y_\delta$ 和 $\rclim(x)\close\eta y_\delta$。
  因此
  \[ f(\rclim(x)) \close{\epsilon/2} f(y_\delta) = g(y_\delta) \close{\epsilon/2} g(\rclim(x))\]
  因此通过三角不等式 $f(\rclim(x))\close\epsilon g(\rclim(x))$。
\end{proof}

\subsection{Cauchy 实数的代数结构}
\label{sec:algebr-struct-cauchy}

我们首先定义加法结构 $(\RC, 0, {+}, {-})$。显然,加法单位元素
$0$ 就是 $\rcrat(0)$,而加法逆元 ${-} : \RC \to \RC$ 是通过加法逆元 ${-} : \Q \to \Q$ 的扩展得到的,使用 \cref{RC-extend-Q-Lipschitz}
Lipschitz 常数为 $1$。对于加法我们必须做更多工作。

\begin{lem} \label{RC-binary-nonexpanding-extension}
假设 $f : \Q \times \Q \to \Q$ 满足,对于所有 $q, r, s : \Q$,
%
\begin{equation*}
  |f(q, s) - f(r, s)| \leq |q - r|
  \qquad\text{和}\qquad
  |f(q, r) - f(q, s)| \leq |r - s|。
\end{equation*}
%
那么存在一个函数 $\bar{f} : \RC \times \RC \to \RC$,使得
对于所有 $q, r : \Q$,有 $\bar{f}(\rcrat(q), \rcrat(r)) = f(q,r)$。此外,
对于所有 $u, v, w : \RC$ 和 $q : \Qp$,
%
\begin{equation*}
  u \close\epsilon v \Rightarrow \bar{f}(u,w) \close\epsilon \bar{f}(v,w)
  \quad\text{和}\quad
  v \close\epsilon w \Rightarrow \bar{f}(u,v) \close\epsilon \bar{f}(u,w)。
\end{equation*}
\end{lem}


\begin{proof}
  我们使用 $(\RC, {\closesym})$-递归来构造 $\bar{f}$ 的柯里化形式,作为映射
  $\RC \to A$,其中 $A$ 是非扩展\index{function!non-expanding}\index{non-expanding function}实值函数的空间:
  %
  \begin{equation*}
    A \defeq
    \setof{ h : \RC \to \RC |
    \fall{\epsilon : \Qp} \fall{u, v : \RC}
    u \close\epsilon v \Rightarrow h(u) \close\epsilon h(v)
    }.
  \end{equation*}
  %
  我们还需要一个合适的 $\bsim_\epsilon$ 在 $A$ 上,我们定义为
  %
  \begin{equation*}
  (h \bsim_\epsilon k) \defeq \fall{u : \RC} h(u) \close\epsilon k(u)。
  \end{equation*}
  %
  显然,如果 $\fall{\epsilon : \Qp} h \bsim_\epsilon k$ 那么对所有 $u : \RC$ 有 $h(u) = k(u)$,因此 $\bsim$ 是分离的。

  对于基准情况,我们定义 $\bar{f}(\rcrat(q)) : A$,其中 $q : \Q$,作为 Lipschitz 映射 $\lam{r} f(q,r)$ 从 $\Q \to \Q$ 到 $\RC \to \RC$ 的扩展,使用 Lipschitz 常数为~$1$ 的 \cref{RC-extend-Q-Lipschitz} 构造。接下来,对于一个 Cauchy 近似 $x$,我们定义 $\bar{f}(\rclim(x)) : \RC \to \RC$ 为
  %
  \begin{equation*}
    \bar{f}(\rclim(x))(v) \defeq \rclim (\lam{\epsilon} \bar{f}(x_\epsilon)(v))。
  \end{equation*}
  %
  为了使这个定义有效,$\lam{\epsilon} \bar{f}(x_\epsilon)(v)$ 应该是一个 Cauchy 近似,因此考虑任意 $\delta, \epsilon : \Q$。然后根据假设 $\bar{f}(x_\delta) \bsim_{\delta + \epsilon} \bar{f}(x_\epsilon)$,因此 $\bar{f}(x_\delta)(v) \close{\delta + \epsilon} \bar{f}(x_\epsilon)(v)$。此外,$\bar{f}(\rclim(x))$ 是非扩展的,因为根据归纳假设 $\bar{f}(x_\epsilon)$ 是这样的。事实上,如果 $u \close\epsilon v$,那么对于所有 $\epsilon : \Q$,
  %
  \begin{equation*}
    \bar{f}(x_{\epsilon/3})(u) \close{\epsilon/3} \bar{f}(x_{\epsilon/3})(v),
  \end{equation*}
  %
  因此通过 $\closesym$ 的第四个构造函数 $\bar{f}(\rclim(x))(u) \close\epsilon \bar{f}(\rclim(x))(v)$。

  我们还需要检查四个条件,我们来说明其中一个。假设 $\epsilon : \Qp$ 并且对于某个 $\delta : \Qp$ 有 $\rcrat(q) \close{\epsilon - \delta} y_\delta$ 和 $\bar{f}(\rcrat(q)) \bsim_{\epsilon - \delta} \bar{f}(y_\delta)$。为了证明 $\bar{f}(\rcrat(q)) \bsim_\epsilon \bar{f}(\rclim(y))$,考虑任意 $v : \RC$ 并观察到
  %
  \begin{equation*}
    \bar{f}(\rcrat(q))(v) \close{\epsilon - \delta} \bar{f}(y_\delta)(v)。
  \end{equation*}
  %
  因此,通过 $\closesym$ 的第二个构造函数,我们有
  \narrowequation{\bar{f}(\rcrat(q))(v) \close\epsilon \bar{f}(\rclim(y))(v)}
  如所需。
\end{proof}

我们可以将 \cref{RC-binary-nonexpanding-extension} 应用于任何在每个变量中分别是非扩展的二元有理函数。加法就是这样的函数,因此我们得到了 ${+} : \RC \times \RC \to \RC$。
\indexdef{addition!of Cauchy reals}%
此外,只要我们要求它在每个变量中是非扩展的,这个扩展就是唯一的,并且与一元情况下相同,有理数上的恒等式扩展为实数上的恒等式。由于非扩展映射的复合再次是非扩展的,我们可以得出加法满足通常的性质,例如交换律和结合律。
\index{associativity!of addition!of Cauchy reals}%
因此,$(\RC, 0, {+}, {-})$ 是一个交换群。

我们还可以将 \cref{RC-binary-nonexpanding-extension} 应用于函数 $\min : \Q \times \Q \to \Q$ 和 $\max : \Q \times \Q \to \Q$,这将 $\RC$ 变成一个格。$\RC$ 上的偏序 $\leq$ 以 $\max$ 的形式定义为
%
\symlabel{leq-RC}
\index{order!non-strict}%
\index{non-strict order}%
\begin{equation*}
(u \leq v) \defeq (\max(u, v) = v)。
\end{equation*}
%
关系 $\leq$ 是偏序,因为它在 $\Q$ 上是这样的,并且偏序的公理可以用 $\min$ 和 $\max$ 表示为方程,因此它们转移到 $\RC$。

\index{absolute value}%
另一个通过相同方法扩展到 $\RC$ 的函数是绝对值 $|{\blank}|$。
同样,它具有预期的性质,因为它们从 $\Q$ 转移到 $\RC$。

\symlabel{lt-RC}
从 $\leq$ 我们得到严格顺序 $<$,其定义为
\index{strict!order}%
\index{order!strict}%
%
\begin{equation*}
(u < v) \defeq \exis{q, r : \Q} (u \leq \rcrat(q)) \land (q < r) \land (\rcrat(r) \leq v)。
\end{equation*}
%
即,当仅存在一对有理数 $q < r$ 使得 $x \leq \rcrat(q)$ 且 $\rcrat(r) \leq v$ 时,$u < v$ 成立。不难检查 $<$ 是反身和传递的,并且具有有序域预期的其他性质。
阿基米德原则直接来自于 $<$ 的定义。

\index{ordered field!archimedean}%
\begin{thm}[RC 的阿基米德原理] \label{RC-archimedean}
%
对于每个 $u, v : \RC$ 使得 $u < v$,仅存在 $q : \Q$ 使得 $u < q < v$。
\end{thm}

\begin{proof}
  从 $u < v$ 我们仅得到 $r, s : \Q$ 使得 $u \leq r < s \leq v$,我们可以取 $q \defeq (r + s) / 2$。
\end{proof}

我们现在有足够的结构来用标准概念表达 $u \close\epsilon v$。

\begin{lem}\label{thm:RC-le-grow}
如果 $q:\Q$ 和 $u:\RC$ 满足 $u\le \rcrat(q)$,那么对于任何 $v:\RC$ 和 $\epsilon:\Qp$,如果 $u\close\epsilon v$ 那么 $v\le \rcrat(q+\epsilon)$。
\end{lem}
\begin{proof}
  注意函数 $\max(\rcrat(q),\blank):\RC\to\RC$ 是 Lipschitz 的,常数为 $1$。
  首先考虑 $u=\rcrat(r)$ 为有理数的情况。
  对于此,我们使用归纳法 $v$。
  如果 $v$ 是有理数,那么该陈述是显然的。
  如果 $v$ 是 $\rclim(y)$,我们假设归纳地,对于任何 $\epsilon,\delta$,如果 $\rcrat(r)\close\epsilon y_\delta$ 那么 $y_\delta \le \rcrat(q+\epsilon)$,即 $\max(\rcrat(q+\epsilon),y_\delta)=\rcrat(q+\epsilon)$。

  现在假设 $\epsilon$ 并且 $\rcrat(r)\close\epsilon \rclim(y)$,我们有 $\theta$ 使得 $\rcrat(r)\close{\epsilon-\theta} \rclim(y)$,因此 $\rcrat(r)\close\epsilon y_\delta$ 只要 $\delta<\theta$。
  因此,归纳假设给出了 $\max(\rcrat(q+\epsilon),y_\delta)=\rcrat(q+\epsilon)$ 对于这样的 $\delta$。
  但根据定义,
  \[\max(\rcrat(q+\epsilon),\rclim(y)) \jdeq \rclim(\lam{\delta} \max(\rcrat(q+\epsilon),y_\delta))。\]
  由于最终常数 Cauchy 近似的极限是那个常数,我们有
  \[\max(\rcrat(q+\epsilon),\rclim(y)) = \rcrat(q+\epsilon),\] 因此 $\rclim(y)\le \rcrat(q+\epsilon)$。

  现在考虑一个一般的 $u:\RC$。
  由于 $u\le \rcrat(q)$ 意味着 $\max(\rcrat(q),u)=\rcrat(q)$,假设 $u\close\epsilon v$ 和 $\max(\rcrat(q),-)$ 的 Lipschitz 性质意味着 $\max(\rcrat(q),v) \close\epsilon \rcrat(q)$。
  因此,$\rcrat(q)\le \rcrat(q)$ 的第一个情况意味着 $\max(\rcrat(q),v) \le \rcrat(q+\epsilon)$,因此通过 $\le$ 的传递性 $v\le \rcrat(q+\epsilon)$。
\end{proof}

\begin{lem}\label{thm:RC-lt-open}
假设 $q:\Q$ 和 $u:\RC$ 满足 $u<\rcrat(q)$。 那么:
\begin{enumerate}
  \item 对于任何 $v:\RC$ 和 $\epsilon:\Qp$,如果 $u\close\epsilon v$ 那么 $v< \rcrat(q+\epsilon)$。\label{item:RCltopen1}
  \item 存在 $\epsilon:\Qp$ 使得对于任何 $v:\RC$,如果 $u\close\epsilon v$ 我们有 $v<\rcrat(q)$。\label{item:RCltopen2}
\end{enumerate}
\end{lem}
\begin{proof}
  根据定义,$u<\rcrat(q)$ 意味着存在 $r:\Q$ 使得 $r<q$ 并且 $u\le \rcrat(r)$。
  然后根据 \cref{thm:RC-le-grow},对于任何 $\epsilon$,如果 $u\close\epsilon v$ 那么 $v\le \rcrat(r+\epsilon)$。
  结论~\ref{item:RCltopen1} 立即得出,因为 $r+\epsilon<q+\epsilon$,而对于~\ref{item:RCltopen2} 我们可以取任何 $\epsilon <q-r$。
\end{proof}

我们现在可以证明辅助关系 $\closesym$ 确实如我们所想的那样。

\begin{thm} \label{RC-sim-eqv-le}
\index{distance}%
对于所有 $u, v : \RC$ 和 $\epsilon : \Qp$,$\eqv{(u \close\epsilon v)}{(|u - v| < \rcrat(\epsilon))}$。
\end{thm}
\begin{proof}
  减法和绝对值的 Lipschitz 性质意味着如果 $u\close\epsilon v$,那么 $|u-v| \close\epsilon |u-u| = 0$。
  因此,对于从左到右的方向,足以证明如果 $u\close\epsilon 0$,那么 $|u|<\rcrat(\epsilon)$。
  我们通过 $\RC$-归纳法证明 $u$。

  如果 $u$ 是有理数,那么该陈述立即得出,因为绝对值和顺序扩展了 $\Qp$ 上的标准。
  如果 $u$ 是 $\rclim(x)$,那么通过圆滑性我们有 $\theta:\Qp$,使得 $\rclim(x)\close{\epsilon-\theta} 0$。
  因此,通过三角不等式,我们有 $x_{\theta/3} \close{\epsilon-2\theta/3} 0$,因此归纳假设得出 $|x_{\theta/3}|<\rcrat(\epsilon-2\theta/3)$。
  但是 $x_{\theta/3} \close{2\theta/3} \rclim(x)$,因此根据 Lipschitz 性质 $|x_{\theta/3}| \close{2\theta/3} |\rclim(x)|$,因此 \cref{thm:RC-lt-open}\ref{item:RCltopen1} 意味着 $|\rclim(x)|<\rcrat(\epsilon)$。

  另一个方向上,我们使用 $\RC$-归纳法证明 $u$ 和 $v$。
  如果两者都是有理数,这是 $\closesym$ 的第一个构造函数。

  如果 $u$ 是 $\rcrat(q)$ 而 $v$ 是 $\rclim(y)$,我们归纳假设,对于任何 $\epsilon,\delta$,如果 $|\rcrat(q)-y_\delta|<\rcrat(\epsilon)$ 那么 $\rcrat(q) \close{\epsilon} y_\delta$。
  固定一个 $\epsilon$ 使得 $|\rcrat(q) - \rclim(y)|<\rcrat(\epsilon)$。
  由于 $\Q$ 在 $\RC$ 中是序致密的,存在 $\theta<\epsilon$ 使得 $|\rcrat(q) - \rclim(y)|<\rcrat(\theta)$。
  现在对于任何 $\delta,\eta$,我们有 $\rclim(y)\close{2\delta} y_\delta$,因此根据 Lipschitz 性质
  \[ |\rcrat(q) - \rclim(y)| \close{\delta+\eta} |\rcrat(q) - y_\delta|。 \]
  因此,根据 \cref{thm:RC-lt-open}\ref{item:RCltopen1},我们有 $|\rcrat(q) - y_\delta| < \rcrat(\theta+2\delta)$。
  因此,根据归纳假设,$\rcrat(q) \close{\theta+2\delta} y_\delta$,因此通过三角不等式 $\rcrat(q)\close{\theta+4\delta} \rclim(y)$。
  因此,足以选择 $\delta \defeq (\epsilon-\theta)/4$。

  剩下的两个情况完全类似。
\end{proof}

\indexdef{multiplication!of Cauchy reals}%
接下来,我们想为 $\RC$ 配备乘法结构。对于每个 $q : \Q$,映射 $r \mapsto q \cdot r$ 是 Lipschitz 的,常数\footnote{我们将 Lipschitz 常数定义为 \emph{正}有理数。} 为 $|q| + 1$,因此我们可以将其扩展到实数上的乘法。因此 $\RC$ 是一个 $\Q$ 上的向量空间\index{vector!space}。
通常,我们可以将实数乘法定义为
%
\begin{equation}
  u \cdot v \defeq
  {\textstyle \frac{1}{2}} \cdot ((u + v)^2 - u^2 - v^2),\label{mult-from-square}
\end{equation}
%
因此我们只需要平方\index{squaring function} $u \mapsto u^2$ 作为一个映射 $\RC \to \RC$。平方不是一个 Lipschitz 映射,但它在每个有界域上是 Lipschitz 的,这使得我们可以将其拼接在一起。定义开区间和闭区间
%
\indexdef{interval!open and closed}%
\indexdef{open!interval}%
\indexdef{closed!interval}%
\begin{equation*}
[u,v] \defeq \setof{ x : \RC | u \leq x \leq v }
\qquad\text{和}\qquad
(u,v) \defeq \setof{ x : \RC | u < x < v }。
\end{equation*}
%
尽管技术上 $[u,v]$ 或 $(u,v)$ 的元素是一个 Cauchy 实数,连同一个证明,因为后者居住在一个无关紧要的命题中,所以它是无趣的。
因此,与子集类型常见的情况一样,我们通常仅当 $x:\RC$ 满足 $u\leq x \leq v$ 时写作 $x:[u,v]$,同样。

\begin{thm} \label{RC-squaring}
%
存在一个唯一的函数 ${(\blank)}^2 : \RC \to \RC$ 扩展了有理数的平方 $q \mapsto q^2$ 并满足
%
\begin{equation*}
  \fall{n : \N}
  \fall{u, v : [-n, n]}
  |u^2 - v^2| \leq 2 \cdot n \cdot |u - v|。
\end{equation*}
\end{thm}

\begin{proof}
  我们首先观察到,对于每个 $u : \RC$ 仅存在 $n : \N$ 使得 $-n \leq u \leq n$,参见 \cref{ex:traditional-archimedean},因此映射
  %
  \begin{equation*}
    e : \Parens{\sm{n : \N} [-n, n]} \to \RC
    \qquad定义为\qquad
    e(n, x) \defeq x
  \end{equation*}
  %
  是满射。接下来,对于每个 $n : \N$,平方映射
  %
  \begin{equation*}
    s_n : \setof{ q : \Q | -n \leq q \leq n } \to \Q
    \qquad定义为\qquad
    s_n(q) \defeq q^2
  \end{equation*}
  %
  是 Lipschitz 的,常数为 $2 n$,因此我们可以使用 \cref{RC-extend-Q-Lipschitz} 将其扩展到 Lipschitz 常数为 $2 n$ 的映射 $\bar{s}_n : [-n, n] \to \RC$,参见 \cref{RC-Lipschitz-on-interval} 了解详细信息。这些映射 $\bar{s}_n$ 是兼容的:如果 $m < n$ 对于某些 $m, n : \N$,那么 $s_n$ 限制到 $[-m, m]$ 必须与 $s_m$ 一致,因为两者都是 Lipschitz 的,因此在 \cref{RC-continuous-eq} 的意义上是连续的。因此,根据 \cref{lem:images_are_coequalizers},映射
  %
  \begin{equation*}
    \Parens{\sm{n : \N} [-n, n]} \to \RC,
    \qquad给定为\qquad
    (n, x) \mapsto s_n(x)
  \end{equation*}
  %
  唯一地分解为 $\RC$ 给我们所需的函数。
\end{proof}

此时我们得到了实数的环结构和阿基米德顺序。要将 $\RC$ 作为一个阿基米德有序域,我们仍然需要逆元。

\begin{thm}
  \index{apartness}%
  Cauchy 实数是可逆的,当且仅当它远离零。
\end{thm}

\begin{proof}
  首先,假设 $u : \RC$ 有一个逆元 $v : \RC$ 根据阿基米德原则存在 $q :
  \Q$ 使得 $|v| < q$。那么 $1 = |u v| < |u| \cdot v < |u| \cdot q$,因此 $|u| >
  1/q$,也就是说 $u \apart 0$。

  相反,我们通过拼接函数来构造逆映射
  %
  \begin{equation*}
  ({\blank})^{-1} : \setof{ u : \RC | u \apart 0 } \to \RC。
  \end{equation*}
  %
  只给出主要步骤。对于每个 $q : \Q$,定义
  %
  \begin{equation*}
  [q, \infty) \defeq \setof{u : \RC | q \leq u}
  \qquad和\qquad
  (-\infty, q] \defeq \setof{u : \RC | u \leq -q}。
  \end{equation*}
  %
  然后,当 $q$ 取遍 $\Qp$ 时,类型 $(-\infty, q]$ 和 $[q, \infty)$ 共同覆盖
  $\setof{u : \RC | u \apart 0}$。在每个 $[q, \infty)$ 和 $(-\infty, q]$ 上,逆函数通过应用 \cref{RC-extend-Q-Lipschitz} 与 Lipschitz 常数 $1/q^2$ 得到。最后,\cref{lem:images_are_coequalizers} 保证逆函数唯一地分解为 $\setof{ u : \RC | u apart 0 }$。
\end{proof}

我们用定理总结 $\RC$ 的代数结构。

\begin{thm} \label{RC-archimedean-ordered-field}
Cauchy 实数组成一个阿基米德有序域。
\end{thm}

\subsection{Cauchy 实数是 Cauchy 完备的}
\label{sec:cauchy-reals-cauchy-complete}

我们通过将 $\Q$ 在 Cauchy 近似的极限下闭合来构造 $\RC$,所以 $\RC$ 应该是 Cauchy 完备的。根据 \cref{RC-sim-eqv-le},Cauchy 近似 $x : \Qp \to \RC$ 的定义与 $\RC$ 构造中的定义一致,并且与 \cref{defn:cauchy-approximation}(适用于 $\RC$)中的 Cauchy 近似一致。

因此,给定一个 Cauchy 近似 $x : \Qp \to \RC$,我们很自然地期望 $\rclim(x)$ 是它的极限,极限的概念定义如 \cref{defn:cauchy-approximation}。但事实如此,通过 \cref{RC-sim-eqv-le} 和 \cref{thm:RC-sim-lim-term}。我们已经证明了:

\begin{thm}
  每个 $\RC$ 中的 Cauchy 近似都有一个极限。
\end{thm}

一个阿基米德有序域中每个 Cauchy 近似都有一个极限的域称为\define{Cauchy 完备}。
\indexdef{Cauchy!completeness}%
\indexdef{complete!ordered field, Cauchy}%
\index{ordered field}%
Cauchy 实数是最小的此类域。

\begin{thm} \label{RC-initial-Cauchy-complete}
Cauchy 实数嵌入到每个 Cauchy 完备的阿基米德有序域中。
\end{thm}

\begin{proof}
  \index{limit!of a Cauchy approximation}%
  假设 $F$ 是一个 Cauchy 完备的阿基米德有序域。因为极限是唯一的,所以存在一个运算符 $\lim$,它将 $F$ 中的 Cauchy 近似映射到它们的极限。我们定义嵌入 $e : \RC \to F$ 通过 $(\RC, {\closesym})$-递归作为
  %
  \begin{equation*}
    e(\rcrat(q)) \defeq q
    \qquad和\qquad
    e(\rclim(x)) \defeq \lim (e \circ x)。
  \end{equation*}
  %
  在 $F$ 上合适的 $\bsim$ 是
  %
  \begin{equation*}
  (a \bsim_\epsilon b) \defeq |a - b| < \epsilon。
  \end{equation*}
  %
  这是一个分离的关系,因为 $F$ 是阿基米德的。对于 $(\RC, {\closesym})$-递归的其余条款很容易验证。还需要检查 $e$ 是一个固定有理数的有序域的嵌入。
\end{proof}

\index{real numbers!Cauchy|)}%

\section{柯西实数与戴德金实数的比较 (Comparison of Cauchy and Dedekind reals)}
\label{sec:comp-cauchy-dedek}

\index{real numbers!Dedekind|(实数!戴德金|(}%
\index{real numbers!Cauchy|(实数!柯西|(}%
\index{depression|(抑郁|(}

我们来讨论一下柯西实数 ($\RC$) 和戴德金实数 ($\RD$) 之间的关系。根据\cref{RC-archimedean-ordered-field},$\RC$ 是阿基米德有序域 (archimedean ordered field)。它对于 $\Omega$ 也是可容许的(admissible),这一点很容易验证。(如果 $\Omega$ 是初始 $\sigma$-框架 (initial $\sigma$-frame),则只需简单的归纳法即可验证;在其他情况下,这一点是直接显然的。)因此,根据\cref{RD-final-field},有一个有序域的嵌入 (embedding of ordered fields):

\begin{equation*}
  \RC \to \RD
\end{equation*}

这个嵌入固定了有理数 (rational numbers)。
(我们也可以从\cref{RC-initial-Cauchy-complete,RD-cauchy-complete} 中得出这一结论。)
通常情况下,如果没有进一步的假设,我们并不期望 $\RC$ 和 $\RD$ 会相等。

\begin{lem} \label{lem:untruncated-linearity-reals-coincide}
如果对于每个 $x : \RD$ ,仅存在一个满足以下条件的 $c$:
\begin{equation}
  \label{eq:untruncated-linearity}
  c : \prd{q, r : \Q} (q < r) \to (q < x) + (x < r)
\end{equation}
那么柯西实数 (Cauchy reals) 和戴德金实数 (Dedekind reals) 是一致的。
\end{lem}

\begin{proof}
  请注意,\eqref{eq:untruncated-linearity} 中的类型是~\eqref{eq:RD-linear-order} 的非截断版本,后者表明~$<$ 是一个弱线性序 (weak linear order)。
  我们已经知道 $\RC$ 嵌入了 $\RD$,因此只需证明每个戴德金实数 (Dedekind real) 都仅是一个有理数柯西序列 (Cauchy sequence) 的极限即可。

  考虑任意 $x : \RD$。根据假设,存在一个如引理陈述中所描述的 $c$,并且根据切分 (cuts) 的居留性 (inhabitation),存在有理数 $a, b : \Q$ 满足 $a < x < b$。
  我们通过递归构造一个序列 $f : \N \to \setof{ \pairr{q, r} \in \Q \times \Q | q < r }$:
  \begin{enumerate}
    \item 设 $f(0) \defeq \pairr{a, b}$。
    \item 假设 $f(n)$ 已经定义为 $\pairr{q_n, r_n}$,且 $q_n < r_n$。定义 $s \defeq (2 q_n + r_n)/3$ 和 $t \defeq (q_n + 2 r_n)/3$。然后 $c(s,t)$ 决定了 $s < x$ 还是 $x < t$。如果决定 $s < x$,那么我们设置 $f(n+1) \defeq \pairr{s, r_n}$,否则设置 $f(n+1) \defeq \pairr{q_n, t}$。
  \end{enumerate}

  让我们用 $\pairr{q_n, r_n}$ 表示序列 $f$ 的第 $n$ 项。那么很容易看出,对于所有 $n : \N$,都有 $q_n < x < r_n$ 并且 $|q_n - r_n| \leq (2/3)^n \cdot |q_0 - r_0|$。因此,$q_0, q_1, \ldots$ 和 $r_0, r_1, \ldots$ 都是收敛于戴德金实数 $x$ 的柯西序列 (Cauchy sequences)。
  我们已经证明,对于每个 $x : \RD$,仅存在一个收敛于 $x$ 的柯西序列。
\end{proof}

该引理表明,无论是可数选择公理 (countable choice) 还是排中律 (excluded middle) 都足以保证 $\RC$ 和 $\RD$ 的一致性。

\begin{cor} \label{when-reals-coincide}
如果排中律 (excluded middle) 或者可数选择公理 (countable choice) 成立,那么 $\RC$ 和 $\RD$ 是等价的。
\end{cor}

\begin{proof}
  如果排中律成立,那么 $(x < y) \to (x < z) + (z < y)$ 可以被证明:要么 $x < z$,要么 $\lnot (x < z)$。在前一种情况下,我们完成证明;而在后一种情况下,我们得出 $z < y$,因为 $z \leq x < y$。因此,我们得到~\eqref{eq:untruncated-linearity},可以应用\cref{lem:untruncated-linearity-reals-coincide}。

  假设可数选择公理成立。集合 $S = \setof{ \pairr{q, r} \in \Q \times \Q | q < r }$ 与 $\N$ 是等价的,因此我们可以将可数选择公理应用于 $x$ 的定位 (located),即:
  \begin{equation*}
    \fall{\pairr{q, r} : S} (q < x) \lor (x < r)。
  \end{equation*}
  请注意,$(q < x) \lor (x < r)$ 可以表示为一个存在量词语句 (existential statement) $\exis{b : \bool} (b = \bfalse \to q < x) \land (b = \btrue \to x < r)$。选择函数的柯里形式 (curried form) 就是~\eqref{eq:untruncated-linearity},因此\cref{lem:untruncated-linearity-reals-coincide} 再次适用。
\end{proof}

\index{real numbers!Dedekind|)实数!戴德金|)}%
\index{real numbers!Cauchy|)实数!柯西|)}%
\index{real numbers!agree实数!一致性}%

\index{depression|)抑郁|)}

\section{区间的紧致性 (Compactness of the interval)}
\label{sec:compactness-interval}

\index{mathematics!classical|(数学!经典|(}%
\index{mathematics!constructive|(数学!构造性|(}%

我们已经指出,我们对于实数的构造与经典逻辑 (classical logic) 完全兼容。因此,假设排中律 (law of excluded middle) \eqref{eq:lem} 和选择公理 (axiom of choice) \eqref{eq:ac} 成立,我们可以发展经典分析 (classical analysis),\index{classical!analysis经典!分析}\index{analysis!classical分析!经典}这实际上相当于复制任何标准的分析书籍。

\index{analysis!constructive分析!构造性}%
\index{constructive!analysis构造性!分析}%
然而,对于任何对计算感兴趣的人,例如数值分析学家,应该对在一个计算上有意义的环境中发展分析感到好奇。构造性环境中的分析是可能的,这一点已经由 \cite{Bishop1967} 证明了。作为经典分析和构造性分析之间差异和相似性的一个例子,我们将简要讨论一个主题——闭区间 $[0,1]$ 的紧致性 (compactness) 以及围绕这一概念的一些定理。

在构造性数学中,经典上等价的概念常常会分裂,紧致性也不例外。最常用的三种紧致性概念是:
%
\indexdef{compactness紧致性}%
\begin{enumerate}
  \item \define{度量紧致性 (metrically compact):} “柯西完备 (Cauchy complete) 和全有界 (totally bounded)”,
  \indexdef{metrically compact度量紧致性}%
  \indexdef{compactness!metric紧致性!度量}%
  \item \define{波尔查诺-魏尔斯特拉斯紧致性 (Bolzano--Weierstra\ss{} compact):} “每个序列都有一个收敛的子序列”,
  \index{compactness!Bolzano--Weierstrass@Bolzano--Weierstra\ss{}紧致性!波尔查诺-魏尔斯特拉斯}%
  \indexsee{Bolzano--Weierstrass@Bolzano--Weierstra\ss{}紧致性}{compactness紧致性}%
  \index{sequence序列}%
  \item \define{海涅-博雷尔紧致性 (Heine--Borel compact):} “每个开覆盖 (open cover) 都有一个有限子覆盖 (finite subcover)”。
  \index{compactness!Heine--Borel紧致性!海涅-博雷尔}%
  \indexsee{Heine--Borel海涅-博雷尔}{compactness紧致性}%
\end{enumerate}
%
这些在经典数学中都是等价的。
让我们看看它们在同伦类型论 (homotopy type theory) 中的表现。我们可以使用戴德金实数 (Dedekind reals) 或柯西实数 (Cauchy reals),因此我们将实数记作 $\R$。首先我们回顾一些基本定义。

\indexsee{space!metric空间!度量}{metric space度量空间}
\index{metric space|(度量空间|(}%

\begin{defn} \label{defn:metric-space}
\define{度量空间 (metric space)}
\indexdef{metric space度量空间}%
$(M, d)$ 是一个集合 $M$,其上有一个映射 $d : M \times M \to \R$,
满足对所有 $x, y, z : M$,
%
\begin{align*}
  d(x,y) &\geq 0, &
  d(x,y) &= d(y,x), \\
  d(x,y) &= 0 \Leftrightarrow x = y, &
  d(x,z) &\leq d(x,y) + d(y,z).
\end{align*}
  %
\end{defn}

\begin{defn} \label{defn:complete-metric-space}
\define{柯西逼近 (Cauchy approximation)}
\index{Cauchy!approximation柯西!逼近}%
在 $M$ 中是一个序列 $x : \Qp \to M$,满足
%
\begin{equation*}
  \fall{\delta, \epsilon} d(x_\delta, x_\epsilon) < \delta + \epsilon.
\end{equation*}
%
\index{limit!of a Cauchy approximation柯西逼近的极限}%
一个柯西逼近 $x : \Qp \to M$ 的 \define{极限 (limit)} 是一个点 $\ell : M$,满足
%
\begin{equation*}
  \fall{\epsilon, \theta : \Qp} d(x_\epsilon, \ell) < \epsilon + \theta.
\end{equation*}
%
\indexdef{metric space!complete度量空间!完备}%
\indexdef{complete!metric space完备!度量空间}%
一个 \define{完备度量空间 (complete metric space)} 是其中每个柯西逼近都有极限的空间。
\end{defn}

\begin{defn} \label{defn:total-bounded-metric-space}
对于一个正有理数 $\epsilon$,一个 \define{$\epsilon$-网 ($\epsilon$-net)}
\indexdef{epsilon-net@$\epsilon$-net}%
在度量空间 $(M, d)$ 中是一个元素
%
\begin{equation*}
  \sm{n : \N}{x_1, \ldots, x_n : M}
  \fall{y : M} \exis{k \leq n} d(x_k, y) < \epsilon.
\end{equation*}
%
换句话说,这是一个有限的点序列 $x_1, \ldots, x_n$,使得 $M$ 中的每个点仅在某个 $x_k$ 的 $\epsilon$ 内。

一个度量空间 $(M, d)$ 是 \define{全有界 (totally bounded)} 的,
\indexdef{totally bounded metric space全有界度量空间}%
\indexdef{metric space!totally bounded度量空间!全有界}%
当它有所有大小的 $\epsilon$-网时:
%
\begin{equation*}
  \prd{\epsilon : \Qp}
  \sm{n : \N}{x_1, \ldots, x_n : M}
  \fall{y : M} \exis{k \leq n} d(x_k, y) < \epsilon.
\end{equation*}
\end{defn}

\begin{rmk}
  在全有界性的定义中,我们使用了不严格的记法 $\sm{n : \N}{x_1, \ldots, x_n : M}$。正式来说,我们应该写成 $\sm{x : \lst{M}}$,其中 $\lst{M}$ 是来自 \cref{sec:bool-nat} 的有限列表的归纳类型 (inductive type of finite lists)\index{type!of lists类型!列表}。
  然而,这会使得表达其余部分的陈述更加麻烦。
\end{rmk}

注意,在全有界性的定义中,我们要求纯粹存在 (pure existence) 的 $\epsilon$-网,而不是仅仅存在 (mere existence)。这样我们得到一个函数,该函数为每个 $\epsilon : \Qp$ 分配一个特定的 $\epsilon$-网。这样的函数可以称为“全有界性的模数 (modulus of total boundedness)”。通常,在将经典的度量概念移植到同伦类型论时,我们应该谨慎地使用命题截断 (propositional truncation),通常这样做是为了避免要求从 $\R$ 到 $\Q$ 或 $\N$ 的非常数映射。例如,以下是均匀连续性的“正确”定义。

\begin{defn} \label{defn:uniformly-continuous}
在度量空间上的映射 $f : M \to \R$ 是 \define{均匀连续 (uniformly continuous)} 的,
\indexdef{function!uniformly continuous函数!均匀连续}%
\indexdef{uniformly continuous function均匀连续函数}%
当
%
\begin{equation*}
  \prd{\epsilon : \Qp}
  \sm{\delta : \Qp}
  \fall{x, y : M}
  d(x,y) < \delta \Rightarrow |f(x) - f(y)| < \epsilon.
\end{equation*}
%
特别地,一个均匀连续映射有一个均匀连续性的模数 (modulus of uniform continuity)\indexdef{modulus!of uniform continuity模数!均匀连续性},
这是一个为每个 $\epsilon$ 分配相应 $\delta$ 的函数。
\end{defn}

让我们证明 $[0,1]$ 在第一种意义上是紧致的。

\begin{thm} \label{analysis-interval-ctb}
\index{compactness!metric紧致性!度量}%
\index{interval!open and closed区间!开和闭}%
闭区间 $[0,1]$ 是完备的且全有界的。
\end{thm}

\begin{proof}
  给定 $\epsilon : \Qp$,存在 $k : \N$ 使得 $2/k < \epsilon$,所以我们可以取 $\epsilon$-网 $x_i = i/k$,其中 $i = 0, \ldots, k$。这是一个 $\epsilon$-网,因为对于每个 $y : [0,1]$,仅存在某个 $i$,使得 $0 \leq i \leq k$ 且 $(i - 1)/k < y < (i+1)/k$,因此 $|y - x_i| < 2/k < \epsilon$。

  对于 $[0,1]$ 的完备性,考虑一个柯西逼近 $x : \Qp \to [0,1]$ 并让 $\ell$ 是它在 $\R$ 中的极限。由于 $\max$ 和 $\min$ 是 Lipschitz 映射,由 $r(x) \defeq \max(0, \min(1, x))$ 定义的从 $\R$ 到 $[0,1]$ 的缩回 (retraction) $r$ 与柯西逼近的极限交换,因此
  %
  \begin{equation*}
    r(\ell) =
    r (\lim x) =
    \lim (r \circ x) =
    \lim x =
    \ell,
  \end{equation*}
  %
  这意味着 $0 \leq \ell \leq 1$,这是我们所要求的。
\end{proof}

因此,我们在同伦类型论中至少有一种好的紧致性概念。不幸的是,它仅限于度量空间,因为全有界性是一个度量概念。我们很快将考虑其他两个概念,但首先我们证明在全有界空间上的均匀连续映射具有 \define{上确界 (supremum)}\indexsee{least upper bound上确界}{supremum上确界},即一个小于或等于所有其他上界的上界。

\begin{thm} \label{ctb-uniformly-continuous-sup}
\indexdef{supremum!of uniformly continuous function均匀连续函数的上确界}%
在一个全有界度量空间 $(M, d)$ 上的均匀连续映射 $f : M \to \R$ 有一个上确界 $m : \R$。对于每个 $\epsilon : \Qp$,存在 $u : M$,使得 $|m - f(u)| < \epsilon$。
\end{thm}

\begin{proof}
  设 $h : \Qp \to \Qp$ 是 $f$ 的均匀连续性模数 (modulus of uniform continuity)。
  我们如下定义一个逼近 $x : \Qp \to \R$:对于任意 $\epsilon : \Q$,$M$ 的全有界性给出了一个 $h(\epsilon)$-网 $y_0, \ldots, y_n$。定义
  %
  \begin{equation*}
    x_\epsilon \defeq \max (f(y_0), \ldots, f(y_n))。
  \end{equation*}
  %
  我们声称 $x$ 是一个柯西逼近。考虑任意 $\epsilon, \eta : \Q$,使得
  %
  \begin{equation*}
    x_\epsilon \jdeq \max (f(y_0), \ldots, f(y_n))
    \quad\text{和}\quad
    x_\eta \jdeq \max (f(z_0), \ldots, f(z_m))
  \end{equation*}
  %
  对于某个 $h(\epsilon)$-网 $y_0, \ldots, y_n$ 和 $h(\eta)$-网 $z_0, \ldots, z_m$。
  每个 $z_i$ 仅与某个 $y_j$ 的 $h(\epsilon)$-近,因此 $|f(z_i) - f(y_j)| < \epsilon$,我们可以得出
  %
  \begin{equation*}
    f(z_i) < \epsilon + f(y_j) \leq \epsilon + x_\epsilon,
  \end{equation*}
  %
  因此 $x_\eta < \epsilon + x_\epsilon$。对称地,我们得到 $x_\eta < \eta + x_\eta$,因此 $|x_\eta - x_\epsilon| < \eta + \epsilon$。

  我们声称 $m \defeq \lim x$ 是 $f$ 的上确界。为了证明 $f(x) \leq m$ 对所有 $x : M$ 成立,只需证明 $\lnot (m < f(x))$。假设相反的情况,即 $m < f(x)$。存在 $\epsilon : \Qp$ 使得 $m + \epsilon < f(x)$。但是现在仅对定义 $x_\epsilon$ 的某个 $y_i$,我们得到 $|f(x) - f(y_i)| < \epsilon$,因此 $m < f(x) - \epsilon < f(y_i) \leq m$,这与假设矛盾。

  最后,我们通过证明 $m$ 满足定理的第二部分来结束证明,因为它自动是一个最小的上界。给定任意 $\epsilon : \Qp$,一方面 $|m - f(x_{\epsilon/2})| < 3 \epsilon/4$,另一方面 $|f(x_{\epsilon/2}) - f(y_i)| < \epsilon/4$ 仅对定义 $x_{\epsilon/2}$ 的某个 $y_i$ 成立,因此通过取 $u \defeq y_i$,我们通过三角不等式 (triangle inequality) 得到 $|m - f(u)| < \epsilon$。
\end{proof}

现在,如果在 \cref{ctb-uniformly-continuous-sup} 中我们也知道 $M$ 是完备的,我们可以希望将均匀连续性的假设减弱为连续性 (continuity),并将结论加强为存在一个点在该点上达到上确界。通常的证明这些改进依赖于以下事实:在一个完备的全有界空间中
%
\begin{enumerate}
  \item 连续性意味着均匀连续性,
  \item 每个序列都有一个收敛的子序列。
\end{enumerate}
%
第一个陈述很容易从海涅-博雷尔紧致性推导出来,第二个则只是波尔查诺-魏尔斯特拉斯紧致性。
\index{compactness!Bolzano--Weierstrass@Bolzano--Weierstra\ss{}紧致性!波尔查诺-魏尔斯特拉斯}%
不幸的是,这两者都有些问题。首先我们证明波尔查诺-魏尔斯特拉斯紧致性蕴含了一个排中律的实例,称为\define{全知原理的有限形式 (limited principle of omniscience)}:
\indexsee{axiom!limited principle of omniscience公理!全知原理的有限形式}{limited principle of omniscience全知原理的有限形式}%
\indexdef{limited principle of omniscience全知原理的有限形式}%
对于每个 $\alpha : \N \to \bool$,
%
\begin{equation} \label{eq:lpo}
\Parens{\sm{n : \N} \alpha(n) = \btrue} +
\Parens{\prd{n : \N} \alpha(n) = \bfalse}.
\end{equation}
%
从计算的角度来看,我们不会期望这个原理成立,因为它要求我们决定一个函数是否有无穷多个值为 $\bfalse$。

\begin{thm} \label{analysis-bw-lpo}
波尔查诺-魏尔斯特拉斯紧致性 (Bolzano--Weierstra\ss{} compactness) 对于 $[0,1]$ 蕴含了全知原理的有限形式。
\index{compactness!Bolzano--Weierstrass@Bolzano--Weierstra\ss{}紧致性!波尔查诺-魏尔斯特拉斯}%
\end{thm}

\begin{proof}
  给定任意 $\alpha : \N \to \bool$,定义序列 (sequence) $x : \N \to [0,1]$ 为
  %
  \begin{equation*}
    x_n \defeq
    \begin{cases}
      0 & \text{如果对所有 $k < n$,$\alpha(k) = \bfalse$,}\\
      1 & \text{如果对某个 $k < n$,$\alpha(k) = \btrue$}.
    \end{cases}
  \end{equation*}
  %
  如果波尔查诺-魏尔斯特拉斯性质成立,存在一个严格递增的 $f : \N \to \N$,使得 $x \circ f$ 是一个柯西序列 (Cauchy sequence)。对于一个足够大的 $n : \N$,第 $n$ 项 $x_{f(n)}$ 距离其极限小于 $1/6$。要么 $x_{f(n)} < 2/3$,要么 $x_{f(n)} > 1/3$。如果 $x_{f(n)} < 2/3$,则 $x_n$ 收敛到 $0$,因此 $\prd{n : \N} \alpha(n) = \bfalse$。如果 $x_{f(n)} > 1/3$,则 $x_{f(n)} = 1$,因此 $\sm{n : \N} \alpha(n) = \btrue$。
\end{proof}

虽然我们可能不会太在意波尔查诺-魏尔斯特拉斯紧致性,但没有海涅-博雷尔紧致性似乎更难以接受,正如经典数学和布劳威尔直觉主义 (Brouwer's Intuitionism) 都接受了它一样。由于我们不想深入一般拓扑学,我们将使用基本开集 (basic open sets) 来工作。在 $\R$ 的情况下,这些是具有有理数端点的开区间。一个由类型 $I$ 索引的此类区间的族将是一个映射
%
\begin{equation*}
  \mathcal{F} : I \to \setof{(q, r) : \Q \times \Q | q < r},
\end{equation*}
%
其想法是,一个有理数对 $(q, r)$ 与 $q < r$ 确定类型 $\setof{ x : \R | q < x < r}$。允许退化区间稍微更方便一些,因此我们取一个 \define{基本区间族 (family of basic intervals)} \indexdef{family!of basic intervals族!基本区间的}%
\indexdef{interval!family of basic区间!基本的族}%
为一个映射
%
\begin{equation*}
  \mathcal{F} : I \to \Q \times \Q。
\end{equation*}
%
要非常精确,一个族是一个依赖对 $(I, \mathcal{F})$,而不仅仅是 $\mathcal{F}$。一个 \define{有限基本区间族 (finite family of basic intervals)} 是由 $\setof{ m : \N | m < n}$ 对某个 $n : \N$ 索引的族。我们通常用有限列表 $[(q_0, r_0), \ldots, (q_{n-1}, r_{n-1})]$ 来表示它。最后,一个 $(I, \mathcal{F})$ 的 \define{有限子族 (finite subfamily)} \indexdef{subfamily, finite, of intervals子族,有限的,区间的} 是由索引列表 $[i_1, \ldots, i_n]$ 给出的,这些索引决定了有限族 $[\mathcal{F}(i_1), \ldots, \mathcal{F}(i_n)]$。

只要我们知道对 $(q, r)$ 和对应的区间 $\setof{ x : \R | q < x < r}$ 之间的区别,我们可以安全地对两者使用相同的符号 $(q, r)$。区间的交集 (intersections)\indexdef{intersection!of intervals交集!区间的} 和包含 (inclusions)\indexdef{inclusion!of intervals包含!区间的}\indexdef{containment!of intervals包含!区间的} 可以用它们的端点表示:
%
\symlabel{interval-intersection}
\symlabel{interval-subset}
\begin{align*}
(q, r) \cap (s, t) &\ \defeq\  (\max(q, s), \min(r, t)),\\
(q, r) \subseteq (s, t) &\ \defeq\ (q < r \Rightarrow s \leq q < r \leq t)。
\end{align*}
%
我们说 $\intfam{i}{I}{(q_i, r_i)}$ \define{(逐点)覆盖 $[a,b]$ ((pointwise) covers $[a,b]$)},\indexdef{interval!pointwise cover区间!逐点覆盖的}%
\indexdef{cover!pointwise覆盖!逐点的}%
\indexdef{pointwise!cover逐点!覆盖的}%
当
%
\begin{equation} \label{eq:cover-pointwise-truncated}
\fall{x : [a,b]} \exis{i : I} q_i < x < r_i。
\end{equation}
%
$[0,1]$ 的 \define{海涅-博雷尔紧致性 (Heine--Borel compactness)}\indexdef{compactness!Heine--Borel紧致性!海涅-博雷尔}%
表示每个覆盖 $[0,1]$ 的区间族都有一个有限子族仍然覆盖 $[0,1]$。

\index{depression}
\begin{thm} \label{classical-Heine-Borel}
\index{excluded middle排中律}%
如果排中律成立,那么 $[0,1]$ 是海涅-博雷尔紧致的。
\end{thm}

\begin{proof}
  假设为了达到矛盾,一个族 $\intfam{i}{I}{(a_i, b_i)}$ 覆盖 $[0,1]$,但没有有限子族覆盖它。我们构造一系列闭区间 $[q_n, r_n]$,这些区间是嵌套的,它们的大小收缩到 $0$,且其中没有一个被 $\intfam{i}{I}{(a_i, b_i)}$ 的有限子族覆盖。

  我们设 $[q_0, r_0] \defeq [0,1]$。假设 $[q_n, r_n]$ 已经构造,设 $s \defeq (2 q_n + r_n)/3$ 和 $t \defeq (q_n + 2 r_n)/3$。$[q_n, t]$ 和 $[s, r_n]$ 都被 $\intfam{i}{I}{(a_i, b_i)}$ 覆盖,但它们不能同时有一个有限子覆盖,否则 $[q_n, r_n]$ 也会有一个有限子覆盖。要么 $[q_n, t]$ 有一个有限子覆盖,要么它没有。如果有,我们设 $[q_{n+1}, r_{n+1}] \defeq [s, r_n]$,否则我们设 $[q_{n+1}, r_{n+1}] \defeq [q_n, t]$。

  序列 $q_0, q_1, \ldots$ 和 $r_0, r_1, \ldots$ 都是柯西的,它们收敛到 $[0,1]$ 中的一个点 $x$,该点包含在每个 $[q_n, r_n]$ 中。
  仅存在 $i : I$ 使得 $a_i < x < b_i$。由于区间 $[q_n, r_n]$ 的大小收缩到零,存在 $n : \N$ 使得 $a_i < q_n \leq x \leq r_n < b_i$,但这意味着 $[q_n, r_n]$ 被单个区间 $(a_i, b_i)$ 覆盖,而同时它没有有限子覆盖。这是矛盾。
\end{proof}

没有排中律,或布劳威尔直觉主义 (Brouwerian Intuitionism) 的一点点帮助,我们似乎陷入了困境。然而,在构造性环境中,$[0,1]$ 的海涅-博雷尔紧致性 \emph{可以} 得到恢复,并且仍然与经典数学兼容!为此,我们需要重新审视覆盖的概念。\eqref{eq:cover-pointwise-truncated} 的问题在于截断的存在使得一个空间可以以任何随意的方式被覆盖,从计算的角度来看,我们几乎没有希望仅提取一个有限子覆盖。通过移除截断,我们得到
%
\begin{equation} \label{eq:cover-pointwise}
\prd{x : [0,1]} \sm{i : I} q_i < x < r_i,
\end{equation}
%
这可能有所帮助,但它对覆盖的要求太高了。使用这个定义,我们甚至无法证明 $(0,3)$ 和 $(2,5)$ 覆盖 $[1,4]$,因为这相当于展示一个从 $[1,4]$ 到 $\bool$ 的非常数映射,参见 \cref{ex:reals-non-constant-into-Z}。在这里,我们可以从“无点拓扑 (pointfree topology)”\index{pointfree topology无点拓扑}%
\index{topology!pointfree拓扑!无点的}%
(即位置理论 (locale theory))\index{locale位置}%
中吸取教训:覆盖的概念应该用开集来表达,而不涉及点。这样一种对空间的“整体”观点将允许我们分析覆盖的概念,我们将能够恢复海涅-博雷尔紧致性。位置理论使用幂集 (power sets),\index{power set幂集}%
我们可以通过假设命题调整 (propositional resizing) 获得;\index{propositional!resizing命题!调整}%
但我们可以从位置理论的预测性 (predicative) 近亲中偷取想法,这被称为“形式拓扑 (formal topology)”\index{formal!topology形式!拓扑}%
\index{mathematics!predicative数学!预测性}。

\index{acceptance|(接受|(}

假设我们有一个族 $\pairr{I, \mathcal{F}}$ 和一个区间 $(a, b)$。我们如何表达 $(a,b)$ 被该族覆盖的事实,而不涉及点呢?这里有一种方法:如果 $(a, b)$ 等于某个 $\mathcal{F}(i)$,那么它就被该族覆盖了。还有一种方法:如果 $(a,b)$ 被另一个族 $(J, \mathcal{G})$ 覆盖,而每个 $\mathcal{G}(j)$ 都被 $\pairr{I, \mathcal{F}}$ 覆盖,那么 $(a,b)$ 就被 $\pairr{I, \mathcal{F}}$ 覆盖。注意,我们正在列出可以用来推导出 $\pairr{I, \mathcal{F}}$ 覆盖 $(a,b)$ 的\emph{规则}。我们应该找到足够好的规则并将它们转化为一个归纳定义。

\begin{defn} \label{defn:inductive-cover}
\define{归纳覆盖 $\cover$ (inductive cover $\cover$)}\indexdef{inductive!cover归纳!覆盖}%
\indexdef{cover!inductive覆盖!归纳的}%
是一个单纯关系 (mere relation)
%
\begin{equation*}
{\cover} : (\Q \times \Q) \to \Parens{\sm{I : \type} (I \to \Q \times \Q)} \to \prop
\end{equation*}
%
通过以下规则进行归纳定义,其中 $q, r, s, t$ 是有理数,$\pairr{I, \mathcal{F}}$ 和 $\pairr{J, \mathcal{G}}$ 是基本区间族:
%
\begin{enumerate}

  \item \emph{自反性 (reflexivity):}\index{reflexivity!of inductive cover自反性!归纳覆盖的}%
  对于所有 $i : I$,$\mathcal{F}(i) \cover \pairr{I, \mathcal{F}}$,

  \item \emph{传递性 (transitivity):}\index{transitivity!of inductive cover传递性!归纳覆盖的}%
  如果 $(q, r) \cover \pairr{J, \mathcal{G}}$ 并且 $\fall{j : J} \mathcal{G}(j) \cover \pairr{I,\mathcal{F}}$,那么 $(q, r) \cover \pairr{I, \mathcal{F}}$,

  \item \emph{单调性 (monotonicity):}\index{monotonicity!of inductive cover单调性!归纳覆盖的}%
  如果 $(q, r) \subseteq (s, t)$ 并且 $(s,t) \cover \pairr{I, \mathcal{F}}$,那么 $(q, r) \cover \pairr{I, \mathcal{F}}$,

  \item \emph{局部化 (localization):}\index{localization of inductive cover局部化归纳覆盖的}%
  如果 $(q, r) \cover (I, \mathcal{F})$,那么 $(q, r) \cap (s, t) \cover \intfam{i}{I}{(\mathcal{F}(i) \cap (s, t))}$。

  \item \label{defn:inductive-cover-interval-1}
  如果 $q < s < t < r$,那么 $(q, r) \cover [(q, t), (r, s)]$,

  \item \label{defn:inductive-cover-interval-2}
  $(q, r) \cover \intfam{u}{\setof{ (s,t) : \Q \times \Q | q < s < t < r}}{u}$。
\end{enumerate}
\end{defn}

该定义应被视为一个高阶归纳类型 (higher-inductive type),其中列出的规则是点构造器 (point constructors),并且该类型是 $(-1)$ 截断的。前四个条款是一般性的,应该是直观的。最后两条是特定于实数的:一条说,如果它们重叠,则一个区间可以被两个区间覆盖,而另一条说,一个区间可以从内部覆盖。如果 $r \leq q$,则根据最后一条规则,$(q, r)$ 被空族覆盖。

归纳覆盖享有海涅-博雷尔性质,其证明需要一个引理。

\begin{lem} \label{reals-formal-topology-locally-compact}
假设 $q < s < t < r$ 并且 $(q, r) \cover \pairr{I, \mathcal{F}}$。那么仅存在 $\pairr{I, \mathcal{F}}$ 的一个有限子族,它归纳覆盖 $(s, t)$。
\end{lem}

\begin{proof}
  我们通过归纳证明 $(q, r) \cover \pairr{I, \mathcal{F}}$。有六种情况:
  %
  \begin{enumerate}

    \item 自反性:如果 $(q, r) = \mathcal{F}(i)$,那么根据单调性 $(s, t)$ 被有限子族 $[\mathcal{F}(i)]$ 覆盖。

    \item 传递性:
    假设 $(q, r) \cover \pairr{J, \mathcal{G}}$ 并且 $\fall{j : J} \mathcal{G}(j) \cover \pairr{I, \mathcal{F}}$。根据归纳假设,仅存在 $[\mathcal{G}(j_1), \ldots, \mathcal{G}(j_n)]$ 覆盖 $(s, t)$。
    再次根据归纳假设,它们中的每一个都被 $\pairr{I, \mathcal{F}}$ 的一个有限子族覆盖,并且我们可以将这些子族收集成一个有限子族覆盖 $(s, t)$。

    \item 单调性:
    如果 $(q, r) \subseteq (u, v)$ 并且 $(u, v) \cover \pairr{I, \mathcal{F}}$,那么我们可以应用归纳假设到 $(u, v) \cover \pairr{I, \mathcal{F}}$,因为 $u < s < t < v$。

    \item 局部化:
    假设 $(q', r') \cover \pairr{I, \mathcal{F}}$ 并且 $(q, r) = (q', r') \cap (a, b)$。
    因为 $q' < s < t < r'$,根据归纳假设,存在一个有限子覆盖 $[\mathcal{F}(i_1), \ldots, \mathcal{F}(i_n)]$ 覆盖 $(s, t)$。我们还知道 $a < s < t < b$,因此 $(s, t) = (s, t) \cap (a, b)$ 被 $[\mathcal{F}(i_1) \cap (a,b), \ldots, \mathcal{F}(i_n) \cap (a,b)]$ 覆盖,这是 $\intfam{i}{I}{(\mathcal{F}(i) \cap (a, b))}$ 的一个有限子族。

    \item 如果 $(q, r) \cover [(q, v), (u, r)]$ 对于某个 $q < u < v < r$,那么根据单调性 $(s, t) \cover [(q, v), (u, r)]$。

    \item 最后,通过自反性 $(s, t) \cover \intfam{z}{\setof{ (u,v):\Q \times \Q | q < u < v < r}}{z}$。 \qedhere
  \end{enumerate}
\end{proof}

说 $\pairr{I, \mathcal{F}}$ 归纳覆盖 $[a, b]$ 当仅存在 $\epsilon : \Qp$,使得 $(a - \epsilon, b + \epsilon) \cover \pairr{I, \mathcal{F}}$。

\begin{cor} \label{interval-Heine-Borel}
\index{compactness!Heine-Borel紧致性!海涅-博雷尔}%
\index{interval!open and closed区间!开和闭}%
闭区间对归纳覆盖是海涅-博雷尔紧致的。
\end{cor}

\begin{proof}
  假设 $[a, b]$ 被 $\pairr{I, \mathcal{F}}$ 归纳覆盖,因此仅存在 $\epsilon : \Qp$,使得 $(a - \epsilon, b + \epsilon) \cover \pairr{I, \mathcal{F}}$。根据 \cref{reals-formal-topology-locally-compact} 存在一个有限子覆盖 $(a - \epsilon/2, b + \epsilon/2)$,因此是 $[a, b]$ 的有限子覆盖。
\end{proof}

形式拓扑的经验 (Experience from formal topology)\index{topology!formal拓扑!形式} 表明,归纳覆盖的规则对于构造性的无点拓扑 (pointfree topology) 发展是足够的。但我们也可以提供自己的证据,证明它们是一个合理的概念。

\begin{thm} \label{inductive-cover-classical}
\mbox{}
%
\begin{enumerate}
  \item 一个归纳覆盖也是一个逐点覆盖。
  \item 假设排中律,一个逐点覆盖也是一个归纳覆盖。
\end{enumerenumerate}
\end{thm}

\begin{proof}
  \mbox{}
  %
  \begin{enumerate}

    \item
    考虑一个基本区间族 $\pairr{I, \mathcal{F}}$,其中我们写 $(q_i, r_i) \defeq \mathcal{F}(i)$,一个由 $\pairr{I, \mathcal{F}}$ 逐点覆盖的区间 $(a,b)$ 和 $x$,使得 $a < x < b$。
    我们通过 $(a,b) \cover \pairr{I, \mathcal{F}}$ 的归纳证明,仅存在 $i : I$,使得 $q_i < x < r_i$。大多数情况都很明显,所以我们只展示两个。如果 $(a,b) \cover \pairr{I, \mathcal{F}}$ 是通过自反性覆盖的,那么仅存在某个 $i : I$,使得 $(a,b) = (q_i, r_i)$,因此 $q_i < x < r_i$。如果 $(a,b) \cover \pairr{I, \mathcal{F}}$ 是通过 $\intfam{j}{J}{(s_j, t_j)}$ 的传递性覆盖的,那么根据归纳假设,仅存在 $j : J$,使得 $s_j < x < t_j$,然后因为 $(s_j, t_j) \cover \pairr{I, \mathcal{F}}$,再次根据归纳假设,仅存在 $i : I$,使得 $q_i < x < r_i$。其他情况同样激动人心。

    \item 假设 $\intfam{i}{I}{(q_i, r_i)}$ 逐点覆盖 $(a, b)$。根据 \cref{defn:inductive-cover-interval-2} 和 \cref{defn:inductive-cover} 的定义,只需证明 $\intfam{i}{I}{(q_i, r_i)}$ 归纳覆盖 $(c, d)$,只要 $a < c < d < b$,所以考虑这样的 $c$ 和 $d$。根据 \cref{classical-Heine-Borel},存在一个有限子族 $[i_1, \ldots, i_n]$,它已经逐点覆盖 $[c, d]$,因此 $(c,d)$。设 $\epsilon : \Qp$ 为 $[c, d]$ 的 $[(q_{i_1}, r_{i_1}), \ldots, (q_{i_n}, r_{i_n})]$ 的 Lebesgue 数 (Lebesgue number)\index{Lebesgue number}%
    ,如 \cref{ex:finite-cover-lebesgue-number} 中所示。存在一个正数 $k : \N$,使得 $2 (d - c)/k < \min(1, \epsilon)$。对于 $0 \leq i \leq k$,设
    %
    \begin{equation*}
      c_k \defeq ((k - i) c + i d) / k。
    \end{equation*}
    %
    区间 $(c_0, c_2)$、$(c_1, c_3)$、……、$(c_{k-2}, c_k)$ 通过反复使用传递性和 \cref{defn:inductive-cover-interval-1} 归纳覆盖 $(c,d)$。由于它们的宽度低于 $\epsilon$,因此每个都包含在某个 $(q_i, r_i)$ 中,并且我们可以使用传递性和单调性得出 $\intfam{i}{I}{(q_i, r_i)}$ 归纳覆盖 $(c, d)$。 \qedhere
  \end{enumerenumerate}
\end{proof}

前述定理的结果是,就经典数学而言,逐点覆盖和归纳覆盖之间没有区别。特别是,由于在同伦类型论中假设排中律是自洽的,我们不能展示一个归纳覆盖无法逐点覆盖。或者换句话说,逐点覆盖和归纳覆盖之间的区别不在于它们覆盖什么,而在于它们\emph{证明}它们覆盖的内容。

我们可以写另一本书来继续讨论这些内容,但让我们在这里停止,希望我们已经提供了充分的理由来证明分析可以在同伦类型论中进行发展。好奇的读者应参阅 \cref{ex:mean-value-theorem},以了解中值定理 (intermediate value theorem) 的构造性版本。

\index{acceptance|接受|)}

\index{mathematics!classical|数学!经典|}%
\index{mathematics!constructive|数学!构造性|}%

\section{超现实数 (The surreal numbers)}
\label{sec:surreals}

\index{surreal numbers|(}%

在本节中,我们将讨论另一种高阶归纳-归纳类型的例子,它汇集了我们的许多线索:Conway 的超现实数领域 \NO~\cite{conway:onag}。
超现实数是(Dedekind)实数(参见 \cref{sec:dedekind-reals})和序数(参见 \cref{sec:ordinals})的自然公共推广。
Conway 在具有排中律和选择公理的经典\index{mathematics!classical}数学中,定义了一个超现实数为一对超现实数集合,记为 $\surr L R$,其中 $L$ 中的每个元素都严格小于 $R$ 中的每个元素。
这显然看起来像是一个归纳定义,但将其视为归纳定义存在三个问题。

首先,该定义需要超现实数之间的(严格)不等关系,因此该关系必须与超现实数类型 \NO 同时定义。
(Conway 通过首先定义 \emph{游戏}\index{game!Conway} 来避免这个问题,游戏类似于超现实数,但省略了对 $L$ 和 $R$ 的兼容性条件。)
与 Cauchy 实数的 $\closesym$ 关系一样,这种同时定义可以先验地是归纳-归纳或归纳-递归的。
我们将选择将其做成归纳-归纳类型,原因与我们为 $\closesym$ 做出该选择的原因相同。

此外,我们将分别定义超现实数的严格不等关系 $<$ 和非严格不等关系 $\le$(并且互相归纳定义)。
Conway 用 $\le$ 来定义 $<$,这种方法在经典数学中是合理的,但在构造性数学中却不合理。
\index{mathematics!constructive}%
此外,$<$ 的负定义将使其作为高阶归纳类型构造器的假设无法接受(参见 \cref{sec:strictly-positive})。

其次,Conway 说在 $\surr L R$ 中的 $L$ 和 $R$ 应该是“超现实数集合”,但将其直观地理解为谓词 $\NO\to\prop$ 并不正性,因此不能作为归纳构造器的输入。
然而,这也不是 Conway 真正想表达的类型论翻译,因为在集合论中,超现实数形成一个适当的类,而 $L$ 和 $R$ 是实际的(小)集合,而不是 \NO 的任意子类。
在类型论中,这意味着 \NO 将相对于一个宇宙 \UU 定义,但它本身将属于下一个更高的宇宙 $\UU'$,如序数和基数的集合 \ord 和 \card、累积层次结构 $V$,甚至在没有命题调整的情况下的 Dedekind 实数。
\index{propositional!resizing}%
然后我们将要求超现实数的“集合” $L$ 和 $R$ 是 \UU-小的,因此用一些 \UU-小类型索引的超现实数族来表示它们是自然的。
(这与我们在 \cref{sec:cumulative-hierarchy} 中对累积层次结构所做的完全相同。)
也就是说,超现实数的构造器将具有类型
\[ \prd{\LL,\RR:\UU} (\LL\to\NO) \to (\RR\to \NO) \to (\text{some condition}) \to \NO \]
并且确实是严格正性的。\index{strict!positivity}

最后,在给出 \NO 及其排序的相互定义之后,Conway 宣布如果 $x\le y$ 且 $y\le x$,那么两个超现实数 $x$ 和 $y$ 是\emph{相等的}。
这自然被理解为通过等价关系对“预超现实数”集合进行商。
%(在集合论基础中,必须使用额外的技巧来处理大的等价类。)
然而,在没有选择公理的情况下,这样的商在构造 Cauchy 实数的通常过程中提出了同样的问题:它将不再是超现实数族 $L$ 和 $R$ 可以产生一个新的超现实数 $\surr L R$,因为我们无法必然地“提升” $L$ 和 $R$ 到预超现实数族。
当然,我们可以像对 Cauchy 实数一样,通过使用\emph{高阶}归纳-归纳定义来解决这个问题。

\begin{defn}\label{defn:surreals}
\define{超现实数}的类型 \NO,
\indexdef{surreal numbers}%
\indexsee{number!surreal}{surreal numbers}%
以及关系 $\mathord<:\NO\to\NO\to\type$ 和 $\mathord\le:\NO\to\NO\to\type$,定义为如下的高阶归纳-归纳类型。
类型 \NO 具有以下构造器。
\begin{itemize}
  \item 对于任何 $\LL,\RR:\UU$ 和函数 $\LL\to \NO$ 和 $\RR\to \NO$,其值分别记为 $x^L$ 和 $x^R$,其中 $L:\LL$ 和 $R:\RR$,如果 $\fall{L:\LL}{R:\RR} x^L<x^R$,则存在一个超现实数 $x$。
  \item 对于任何 $x,y:\NO$,如果 $x\le y$ 且 $y\le x$,则我们有 $\noeq(x,y):x=y$。
\end{itemize}
我们将第一个构造器的输入称为\define{分割 (cut)}。
\indexdef{cut!of surreal numbers}%
如果 $x$ 是由一个分割构造的超现实数,那么符号 $x^L$ 将隐含地假设 $L:\LL$,类似地,$x^R$ 将假设 $R:\RR$。
通过这种方式,我们通常可以避免命名索引类型 $\LL$ 和 $\RR$,这在讨论许多不同的分割时非常方便。
按照 Conway 的说法,我们称 $x^L$ 为 $x$ 的\emph{左选项 (left option)}\indexdef{option of a surreal number},$x^R$ 为\emph{右选项 (right option)}。

路径构造器意味着不同的分割可以定义相同的超现实数。
因此,除非我们也知道 $x$ 是由一个特定的分割定义的,否则讲一个任意超现实数 $x$ 的左选项或右选项是没有意义的。
因此,在下面的内容中,我们会说“给定定义超现实数 $x$ 的一个分割”,而不是“给定一个超现实数 $x$”。

关系 $\le$ 具有以下构造器。
\index{non-strict order}%
\index{order!non-strict}%
\begin{itemize}
  \item 给定定义两个超现实数 $x$ 和 $y$ 的分割,如果对所有 $L$,$x^L<y$,并且对所有 $R$,$x<y^R$,则 $x\le y$。
  \item 命题截断:
  对于任何 $x,y:\NO$,如果 $p,q:x\le y$,则 $p=q$。
\end{itemize}
关系 $<$ 具有以下构造器。
\index{strict!order}%
\index{order!strict}%
\begin{itemize}
  % 在第一个和第二个条件中技术上不需要 x 和 y 定义为分割?
  \item 给定定义两个超现实数 $x$ 和 $y$ 的分割,如果存在一个 $L$ 使得 $x\le y^L$,则 $x<y$。
  \item 给定定义两个超现实数 $x$ 和 $y$ 的分割,如果存在一个 $R$ 使得 $x^R\le y$,则 $x<y$。
  \item 命题截断:对于任何 $x,y:\NO$,如果 $p,q:x<y$,则 $p=q$。
\end{itemize}
\end{defn}

\noindent
我们将此与 Conway 的定义进行比较:
\begin{itemize}\footnotesize
\item[-] 如果 $L,R$ 是任意两个数集,并且 $L$ 的任何成员都不大于 $R$ 的任何成员,那么存在一个数 $\surr L R$。
所有的数都是这样构造的。
\item[-] $x\ge y$ 当且仅当(没有 $x^R\le y$ 且 $x\le$ 没有 $y^L$)。
\item[-] $x=y$ 当且仅当($x \ge y$ 且 $y\ge x$)。
\item[-] $x>y$ 当且仅当($x\ge y$ 且 $y\not\ge x$)。
\end{itemize}
在 $<$ 的定义中包括 $x\ge y$ 是不必要的,如果所有对象都是[超现实]数而不是“游戏”\index{game!Conway}。
因此,Conway 的 $<$ 只是其 $\ge$ 的否定,因此 $\surr L R$ 成为超现实数的条件与我们的相同。
通过取 negating Conway 的 $\le$ 并取消双重否定,我们得出了我们对 $<$ 的定义,然后我们可以在没有否定的情况下重新定义他的 $\le$。

我们可以立即用许多超现实数填充 \NO。
像 Conway 一样,我们写作
\symlabel{surreal-cut}
\[\surr{x,y,z,\dots}{u,v,w,\dots}\]
表示由一个分割定义的超现实数,其中 $\LL\to\NO$ 和 $\RR\to\NO$ 是由 $x,y,z,\dots$ 和 $u,v,w,\dots$ 描述的族。
当然,如果 $\LL$ 或 $\RR$ 是 $\emptyt$,我们会在记号中省略相应部分。
存在一个不幸的冲突,即与子集的标准记号 $\setof{x:A | P(x)}$ 冲突,但我们在本节中不会使用后者。
\begin{itemize}
  \item 我们递归地定义 $\iota_{\nat}:\nat\to\NO$ 为
  \begin{align*}
    \iota_{\nat}(0) &\defeq \surr{}{},\\
    \iota_\nat(\suc(n)) &\defeq \surr{\iota_\nat(n)}{}。
  \end{align*}
  即,$\iota_\nat(0)$ 由 $\emptyt\to\NO$ 和 $\emptyt\to\NO$ 组成的分割定义。
  类似地,$\iota_\nat(\suc(n))$ 由 $\unit\to\NO$ 定义(选择 $\iota_\nat(n)$)和 $\emptyt\to\NO$。
  \item 类似地,我们使用符号情况递归原理(\cref{thm:sign-induction})定义 $\iota_{\Z}:\Z\to\NO$:
  \begin{align*}
    \iota_{\Z}(0) &\defeq \surr{}{},\\
    \iota_\Z(n+1) &\defeq \surr{\iota_\Z(n)}{} & &\text{当 $n\ge 0$ 时,}\\
    \iota_\Z(n-1) &\defeq \surr{}{\iota_\Z(n)} & &\text{当 $n\le 0$ 时。}
  \end{align*}
  \item 通过\define{二进制有理数}
  \indexdef{rational numbers!dyadic}%
  \indexsee{dyadic rational}{rational numbers, dyadic}%
  我们指的是一个 $(a,n)$ 的对,其中 $a:\Z$ 且 $n:\nat$,并且如果 $n>0$,则 $a$ 是奇数。
  我们将其记为 $a/2^n$,并将其与相应的有理数等同。
  如果 $\Q_D$ 表示二进制有理数的集合,我们通过对 $n$ 进行归纳定义 $\iota_{\Q_D}:\Q_D\to\NO$:
  \begin{align*}
    \iota_{\Q_D}(a/2^0) &\defeq \iota_\Z(a),\\
    \iota_{\Q_D}(a/2^n) &\defeq \surr{\iota_{\Q_D}(a/2^n - 1/2^n)}{\iota_{\Q_D}(a/2^n + 1/2^n)},
    \quad \text{当 $n>0$ 时。}
  \end{align*}
  这里我们使用了这样一个事实,即如果 $n>0$ 且 $a$ 是奇数,那么 $a/2^n \pm 1/2^n$ 是一个分母比 $a/2^n$ 更小的二进制有理数。
  \item 我们定义 $\iota_{\RD}:\RD\to\NO$,\label{reals-into-surreals} 其中 $\RD$ 是(任意版本的)Dedekind 实数,参见 \cref{sec:dedekind-reals},定义为
  \begin{align*}
    \iota_{\RD}(x) &\defeq
    \surr{q\in\Q_D \text{ 满足 } q<x}{q\in\Q_D \text{ 满足 } x<q}。
  \end{align*}
  与前面的情况不同,当我们将二进制有理数视为 Dedekind 实数时,显然这扩展了 $\iota_{\Q_D}$。
  这可以通过简单性定理(\cref{thm:NO-simplicity})得出。
  \item 回忆 \cref{sec:ordinals} 中定义的\emph{序数}\index{ordinal}类型 \ord,它由 $<$ 关系良好排序,其中 $A<B$ 意味着 $A = \ordsl B b$ 对某些 $b:B$ 成立。
  我们通过序数上的良基递归(\cref{thm:wfrec})定义 $\iota_{\ord}:\ord\to\NO$\label{ord-into-surreals}:
  \begin{equation*}
    \iota_{\ord}(A) \defeq
    \surr{\iota_\ord(\ordsl A a) \text{ 对所有 } a:A}{}。
  \end{equation*}
  它也将从简单性定理中得出,$\iota_\ord$ 限制在有限序数上时与 $\iota_\nat$ 一致。
  (然而,我们提醒读者,与上述例子不同,$\iota_\ord$ 在不限制为较小的序数类时在构造上不可注入;参见 \cref{ex:ord-into-surreals,ex:hiit-plump}。)
  \item 以下是从 Conway 中选取的一些有趣例子:
  \begin{align*}
    \omega &\defeq \surr{0,1,2,3,\dots}{} \qquad\text{(也是一个序数)}\\
    -\omega &\defeq \surr{}{\dots,-3,-2,-1,0}\\
    1/\omega &\defeq \textstyle\surr{0}{1,\frac12,\frac14,\frac18,\dots}\\
    \omega-1 &\defeq \surr{0,1,2,3,\dots}{\omega}\\
    \omega/2 &\defeq \surr{0,1,2,3,\dots}{\dots,\omega-2,\omega-1,\omega}。
  \end{align*}
\end{itemize}

在识别由不同分割表示的超现实数时,以下简单的观察是有用的。

\begin{thm}[Conway 的简单性定理]\label{thm:NO-simplicity}
\index{simplicity theorem}%
\index{theorem!Conway's simplicity}%
假设 $x$ 和 $z$ 是由分割定义的超现实数,并且满足以下条件。
\begin{itemize}
  \item 对于所有的 $L$ 和 $R$,$x^L < z < x^R$。
  \item 对于 $z$ 的每个左选项 $z^L$,存在一个左选项 $x^{L'}$ 使得 $z^L\le x^{L'}$。
  \item 对于 $z$ 的每个右选项 $z^R$,存在一个右选项 $x^{R'}$ 使得 $x^{R'}\le z^R$。
\end{itemize}
则 $x=z$。
\end{thm}
\begin{proof}
  应用 \NO 的路径构造器,我们必须证明 $x\le z$ 和 $z\le x$。
  首先要求对所有 $L$,$x^L<z$,这已假定,并且对所有 $R$,$x<z^R$。
  但根据假设,对于任何 $z^R$,都有一个 $x^{R'}$ 使得 $x^{R'}\le z^R$,因此 $x<z^R$ 如所需。
  因此 $x\le z$;$z\le x$ 的证明是对称的。
\end{proof}

\index{induction principle!for surreal numbers}
然而,为了进一步了解超现实数,我们需要它们的归纳原理。
$(\NO,\le,<)$ 的相互归纳原理适用于三种类型的族:
\begin{align*}
  A &: \NO\to\type\\
  B &: \prd{x,y:\NO}{a:A(x)}{b:A(y)} (x\le y) \to \type\\
  C &: \prd{x,y:\NO}{a:A(x)}{b:A(y)} (x<y) \to \type。
\end{align*}
与 Cauchy 实数的归纳原理一样,考虑 $B$ 和 $C$ 作为类型 $A(x)$ 和 $A(y)$ 之间的关系族是有帮助的。
\symlabel{NO-recursion}
因此,我们将 $B(x,y,a,b,\xi)$ 写为 $(x,a) \ble^\xi (y,b)$,将 $C(x,y,a,b,\xi)$ 写为 $(x,a) \blt^\xi (y,b)$。
同样,我们通常省略 $\xi$,因为它居住在一个简单的命题中,因此并不有趣,并且我们可能经常省略 $x$ 和 $y$,简单地写作 $a\ble b$ 或 $a\blt b$。
有了这些符号,归纳原理的假设如下。
\begin{itemize}
  \item 对于定义超现实数 $x$ 的任何分割,以及
  \begin{enumerate}
    \item 对于每个 $L$,一个元素 $a^L:A(x^L)$,以及
    \item 对于每个 $R$,一个元素 $a^R:A(x^R)$,使得
    \item 对于所有 $L$ 和 $R$ 我们有 $(x^L,a^L) \blt (x^R,a^R)$
  \end{enumerate}
  有一个指定的元素 $f_a:A(x)$。
  我们称这些数据为定义 $x$ 的分割上的\define{依赖分割 (dependent cut)}。
  \indexdef{cut!of surreal numbers!dependent}%
  \indexdef{dependent!cut}%
  \item 对于 $x,y:\NO$ 及 $a:A(x)$ 和 $b:A(y)$,如果 $x\le y$ 且 $y\le x$ 且 $(x,a) \ble (y,b)$ 并且 $(y,b) \ble (x,a)$,则 $\dpath{A}{\noeq}{a}{b}$。
  \item 给定定义两个超现实数 $x$ 和 $y$ 的分割,以及 $x$ 上的依赖分割 $a$ 和 $y$ 上的依赖分割 $b$,如果对于所有 $L$,我们有 $x^L<y$ 且 $(x^L,a^L)\blt (y,f_b)$,并且对于所有 $R$,我们有 $x<y^R$ 且 $(x,f_a) \blt (y^R,b^R)$,则 $(x,f_a) \ble (y,f_b)$。
  \item $\ble$ 的值是简单命题。
  \item 给定定义两个超现实数 $x$ 和 $y$ 的分割,$x$ 上的依赖分割 $a$ 和 $y$ 上的依赖分割 $b$,以及一个 $L_0$,使得 $x\le y^{L_0}$ 且 $(x,f_a) \ble (y^{L_0},b^{L_0})$,则我们有 $(x,f_a) \blt (y,f_b)$。
  \item 给定定义两个超现实数 $x$ 和 $y$ 的分割,$x$ 上的依赖分割 $a$ 和 $y$ 上的依赖分割 $b$,以及一个 ${R_0}$,使得 $x^{R_0}\le y$ 以及 $(x^{R_0},a^{R_0}),\ble (y,f_b)$,则我们有 $(x,f_a) \blt (y,f_b)$。
  \item $\blt$ 的值是简单命题。
\end{itemize}
在这些假设下,我们推导出一个函数 $f:\prd{x:\NO} A(x)$,使得
\begin{align}
  f(x) &\;\jdeq\; f_{f[x]} \label{eq:noind1}\\
  (x\le y) &\;\Rightarrow\; (x,f(x)) \ble (y,f(y)) \notag\\
  (x< y) &\;\Rightarrow\; (x,f(x)) \blt (y,f(y))。 \notag
\end{align}
在点构造器的计算规则~\eqref{eq:noind1} 中,$x$ 是由一个分割定义的超现实数,$f[x]$ 表示通过应用 $f$(并使用 $f$ 取 $<$ 到 $\blt$ 的事实)定义的 $x$ 上的依赖分割。
通常,我们将使用模式匹配符号,其中对分割 $\surr{x^L}{x^R}$ 的 $f$ 的定义可以使用符号 $f(x^L)$ 和 $f(x^R)$ 并假设它们形成依赖分割。

与 Cauchy 实数一样,我们有一些通过将 $A$,$\ble$ 和~$\blt$ 简化而得到的特殊情况。
取 $\ble$ 和 $\blt$ 为常数 \unit,我们有 \define{\NO-归纳},为简单起见,我们仅为简单属性状态:
\begin{itemize}
  \item 给定 $P:\NO\to\prop$,如果 $P(x)$ 在由分割定义的超现实数 $x$ 且对所有 $L$ 和 $R$ 满足 $P(x^L)$ 和 $P(x^R)$ 时成立,则 $P(x)$ 对所有 $x:\NO$ 成立。
\end{itemize}
这应该与 Conway 的评论进行比较:
\begin{quote}\footnotesize
一般来说,当我们希望为所有数 $x$ 建立一个命题 $P(x)$ 时,我们将通过从所有命题 $P(x^L)$ 和 $P(x^R)$ 的真理中推导出 $P(x)$ 来进行归纳证明。
我们认为短语“所有数都是这样构造的”证明了这一过程的合法性。
\end{quote}
通过 \NO-归纳,我们可以证明

\begin{thm}[Conway 定理 0]\label{thm:NO-refl-opt}\
\index{theorem!Conway's 0}%
\begin{enumerate}
  \item 对于任何 $x:\NO$,我们有 $x\le x$。\label{item:NO-le-refl}
  \item 对于由分割定义的任何 $x:\NO$,我们有 $x^L <x$ 和 $x<x^R$ 对所有 $L$ 和 $R$ 成立。\label{item:NO-lt-opt}
\end{enumerate}
\end{thm}
\begin{proof}
  首先注意,如果 $x\le x$,那么每当 $x$ 作为某个分割 $y$ 的左选项出现时,我们有 $x<y$,通过 $<$ 的第一个构造器,每当 $x$ 作为 $y$ 的右选项出现时,我们有 $y<x$。
  特别是,~\ref{item:NO-le-refl}$\Rightarrow$\ref{item:NO-lt-opt}。

  我们通过 \NO-归纳来证明~\ref{item:NO-le-refl}。
  因此,假设 $x$ 是由一个分割定义的,使得对所有 $L$ 和 $R$,$x^L\le x^L$ 和 $x^R \le x^R$ 成立。
  但通过上面的观察,这些假设意味着 $x^L<x$ 和 $x<x^R$ 对所有 $L$ 和 $R$ 成立,通过 $\le$ 的构造器,我们得到 $x\le x$。
\end{proof}

\begin{cor}\label{thm:NO-set}
\NO 是一个 0-类型。
%  (与 $V$ 一样,说它是一个“集合”可能会令人困惑。)
\end{cor}
\begin{proof}
  简单关系 $R(x,y)\defeq (x\le y) \land (y\le x)$ 通过 \NO 的路径构造器意味着恒等,并且通过 \cref{thm:NO-refl-opt}\ref{item:NO-le-refl} 包含对角线。
  因此,\cref{thm:h-set-refrel-in-paths-sets} 适用。
\end{proof}

相比之下,Conway 的定理 1($\le$ 的传递性)用我们的定义稍难建立;参见 \cref{thm:NO-unstrict-transitive}。

% 当然,我们也有:

% \begin{lem}
%   每个超现实数都是由分割定义的。
% \end{lem}
% \begin{proof}
%   通过 \NO-归纳显然成立。
% \end{proof}

我们还需要联合递归原理,\define{$(\NO,\le,<)$-递归}。
方便起见,将其表示如下。
假设 $A$ 是一个带有关系 $\mathord\ble:A\to A\to\prop$ 和 $\mathord\blt:A\to A\to\prop$ 的类型。
然后我们可以通过执行以下操作来定义 $f:\NO\to A$。
\begin{enumerate}
  \item 对于由一个分割定义的任意 $x$,假设 $f(x^L)$ 和 $f(x^R)$ 已定义,使得 $f(x^L)\blt f(x^R)$ 对所有 $L$ 和 $R$ 成立,我们必须定义 $f(x)$。 (我们将其称为递归的\emph{主要条款})。\label{item:NO-rec-primary}
  \item 证明 $\ble$ 是\emph{反对称的}\index{relation!antisymmetric}:如果 $a\ble b$ 且 $b\ble a$,则 $a=b$。
  \item 对于由分割定义的 $x,y$,假设对所有 $L$,$x^L<y$,对所有 $R$,$x<y^R$,并假设归纳地 $f(x^L)\blt f(y)$ 对所有 $L$ 成立,$f(x)\blt f(y^R)$ 对所有 $R$ 成立,并且 $f(x^L)\blt f(x^R)$ 和 $f(y^L)\blt f(y^R)$ 对所有 $L$ 和 $R$ 成立,必须证明 $f(x)\ble f(y)$。
  \item 对于由分割定义的 $x,y$ 和一个 $L_0$ 使得 $x\le y^{L_0}$,假设归纳地 $f(x)\ble f(y^{L_0})$,并且 $f(x^L)\blt f(x^R)$ 和 $f(y^L)\blt f(y^R)$ 对所有 $L$ 和 $R$ 成立,我们必须证明 $f(x)\blt f(y)$。
  \item 对于由分割定义的 $x,y$ 和一个 ${R_0}$ 使得 $x^{R_0}\le y$,假设归纳地 $f(x^{R_0})\ble f(y)$,并且 $f(x^L)\blt f(x^R)$ 和 $f(y^L)\blt f(y^R)$ 对所有 $L$ 和 $R$ 成立,我们必须证明 $f(x)\blt f(y)$。 \label{item:NO-rec-last}
\end{enumerate}
最后三条可以更简洁地描述为我们必须证明 $f$(如在第一个条款中定义)将 $\le$ 映射为 $\ble$,并且 $<$ 映射为 $\blt$。
我们将通过说\emph{$f$ 保持不等性}来引用这些属性。
此外,在证明 $f$ 保持不等性的过程中,我们可以假设构造器的输入是不等性 $<$ 或 $\le$ 的特定实例,并且我们也可以使用 $f$ 保持构造器输入中出现的所有不等性的归纳假设。

如果我们在上述~\ref{item:NO-rec-primary}--\ref{item:NO-rec-last} 中取得成功,那么我们将得到 $f:\NO\to A$,它在分割上的计算如~\ref{item:NO-rec-primary} 所规定,并且保持所有不等性:
%
\begin{narrowmultline*}
  \fall{x,y:\NO}\Big((x\le y) \to (f(x)\ble f(y))\Big) \land
  \narrowbreak
  \Big((x< y) \to (f(x)\blt f(y))\Big)。
\end{narrowmultline*}
%
就像 Cauchy 实数的 $(\RC,\closesym)$-递归一样,这种递归原理在定义 \NO 上的函数时仍然是必不可少的,因为我们无法先定义一个“预超现实数”上的函数,然后再证明它符合等同性的概念。

\begin{eg}
  我们定义 \emph{否定} 函数 $\NO\to\NO$。
  我们将联合递归原理应用于 $A\defeq\NO$,其中 $(x\ble y)\defeq (y\le x)$,$(x\blt y)\defeq (y< x)$。
  显然,这个 $\ble$ 是反对称的。

  对于递归的主要条款,我们假设 $x$ 是由一个分割定义的,并且 $-x^L$ 和 $-x^R$ 已定义,使得 $-x^L \blt -x^R$ 对所有 $L$ 和 $R$ 成立。
  根据定义,这意味着对所有 $L$ 和 $R$,$-x^R< -x^L$,因此我们可以通过分割 $\surr{-x^R}{-x^L}$ 定义 $-x$。
  这个符号,跟随 Conway,指的是一个分割,其中左选项由 $x$ 的右选项的索引类型 $\RR$ 索引,而右选项由左选项的索引类型 $\LL$ 索引,并且相应的族 $\RR\to\NO$ 和 $\LL\to\NO$ 通过与否定组合定义。

  现在我们必须验证 $f$ 保持不等性。
  \begin{itemize}
    \item 对于 $x\le y$,我们可以假设 $x^L<y$ 对所有 $L$ 和 $x < y^R$ 对所有 $R$ 成立,并证明 $-y\le -x$。
    但根据归纳假设,我们可以假设 $-y <-x^L$ 和 $-y^R<-x$,这给出了所需的结果,依据 $-y$、$-x$ 的定义和 $\le$ 的构造器。
    \item 对于 $x<y$,在它从某个 $x\le y^{L_0}$ 产生的第一种情况,我们可以归纳地假设 $-y^{L_0} \le -x$,在这种情况下,$-y<-x$ 通过 $<$ 的构造器得出。
    \item 类似地,如果 $x<y$ 是由 $x^{R_0}\le y$ 产生的,那么归纳假设是 $-y \le -x^R$,因此再次得到 $-y<-x$。
  \end{itemize}
\end{eg}

要做更多的事情,然而,我们需要像对 Cauchy 实数一样更明确地描述 $\le$ 和 $<$ 的关系,如在 \cref{thm:RC-sim-characterization} 中那样。
同样,在那里,为了使归纳进行下去,我们必须同时证明这些关系的一些基本性质。

\begin{thm}\label{defn:No-codes}
存在关系 $\mathord\preceq:\NO\to\NO\to\prop$ 和 $\mathord\prec:\NO\to\NO\to\prop$,使得如果 $x$ 和 $y$ 是由分割定义的超现实数,则
\begin{align*}
(x\preceq y) &\defeq
\big(\fall{L} x^L\prec y\big) \land \big(\fall{R} x\prec y^R\big)\\
(x\prec y) &\defeq
\big(\exis{L} x\preceq y^L\big) \lor \big(\exis{R} x^R \preceq y\big)。
\end{align*}
此外,我们有
\begin{equation}\label{eq:NO-codes-unstrict}
(x\prec y) \to (x\preceq y)
\end{equation}
并且所有合理的传递性质使得 $\prec$ 和 $\preceq$ 成为 $\le$ 和 $<$ 上的“双模”\index{bimodule}:
\begin{equation}\label{eq:NO-codes-transitivity}
\begin{array}{c@{\hspace{1cm}}c}
(x \le y) \to (y\preceq z) \to (x\preceq z) &
(x \preceq y) \to (y\le z) \to (x\preceq z) \\
(x \le y) \to (y\prec z) \to (x\prec z) &
(x \preceq y) \to (y< z) \to (x\prec z) \\
(x < y) \to (y\preceq z) \to (x\prec z) &
(x \prec y) \to (y\le z) \to (x\prec z)。
\end{array}
\end{equation}
\end{thm}

\begin{proof}
  我们通过对 $x,y$ 进行双重 $(\NO,\le,<)$-归纳来定义 $\preceq$ 和 $\prec$。
  第一个归纳是一个简单的递归,其余域是 $(\NO\to\prop)\times (\NO\to\prop)$ 的子集 $A$,其中包含的命题对满足“右传递性”条件,即~\eqref{eq:NO-codes-unstrict} 和~\eqref{eq:NO-codes-transitivity} 的右列,并将 $(x\preceq \blank)$ 和 $(x\prec \blank)$ 替换为给定的两个命题。
  正如在 \cref{defn:RC-approx} 的证明中,我们将这些命题视为二元关系的一半,将它们写作 $y\mapsto (\hle y)$ 和 $y\mapsto (\hlt y)$,其中 $\hlname$ 表示命题对。
  % $A$ 的精确定义为
  % \begin{align*}
  %   A\defeq \bigg\{ \hlname : (\NO\to\prop)\times (\NO\to\prop) \;\bigg|\;\\
  %   \begin{split}
  %     \fall{y,z:\NO}
  %     &\Big( (\hle y) \to (y\le z) \to (\hle z) \Big)\\
  %     \land\; &\Big( (\hle y) \to (y< z) \to (\hlt z) \Big)\\
  %     \land\; &\Big( (\hlt y) \to (y\le z) \to (\hlt z) \Big)\\
  %     \land\; &\Big( (\hlt y) \to (y< z) \to (\hlt z) \Big) \bigg\}
  %   \end{split}
  % \end{align*}
  我们给 $A$ 配备以下两种关系:
  \begin{align*}
  (\hlname \ble \hlbname) &\defeq
  \fall{y:\NO} \Big( (\hleb y) \to (\hle y) \Big) \land
  \Big( (\hltb y) \to (\hlt y) \Big),\\
  (\hlname \blt \hlbname) &\defeq
  \fall{y:\NO} \Big( (\hleb y) \to (\hlt y) \Big)。
    %\land \Big( (\hltb y) \to (\hlt y) \Big)
  \end{align*}
  %(这些是一般归纳原理中的类型族 $B$ 和 $C$。)
  注意 $\ble$ 是反对称的,因为如果 $\hlname \ble \hlbname$ 和 $\hlbname \ble \hlname$,则 $(\hleb y) \Leftrightarrow (\hle y)$ 和 $(\hltb y) \Leftrightarrow (\hlt y)$ 对所有 $y$ 成立,因此根据简单命题的一致性和函数扩展性,$\hlname=\hlbname$。
  此外,将函数 $\NO\to A$ 保持不等性是说,当它被视为一对二元关系时,它满足“左传递性”(~\eqref{eq:NO-codes-transitivity} 的左列)。

  现在对于递归的主要条款,我们假设 $x$ 是由一个分割定义的,并且对所有 $L$ 和 $R$,关系 $(x^L \prec \blank)$,$(x^R \prec \blank)$,$(x^L \preceq \blank)$ 和 $(x^R \preceq \blank)$ 已定义,严格的关系暗示非严格的关系,并满足右传递性,并且
  \begin{equation}\label{eq:NO-prec-outer-IH}
  \fall{L,R}{y:\NO}\Big( (x^R\preceq y) \to (x^L \prec y) \Big)。
    % \land\Big( (x^R \prec y) \to (x^L \prec y) \Big)
  \end{equation}
  现在我们必须定义对所有 $y$ 的 $(x\prec y)$ 和 $(x\preceq y)$。
  在与 \cref{defn:RC-approx} 相反的情况下,而不是嵌套递归,我们使用嵌套归纳,以便能够使用归纳通过 $x^L<x$ 和 $x<x^R$ 的不等关系。
  定义 $A':\NO\to\type$,其中 $A'(y)$ 是 $\prop\times\prop$ 的子集 $A'$,由两个简单命题组成,分别表示 $\tle y$ 和 $\tlt y$(用 $\tlname:A'(y)$ 表示),并且
  \begin{gather}
  (\tlt y) \to (\tle y),\\
  \fall{L} (\tle y)\to (x^L\prec y) \label{eq:NO-prec-IHL},\\
  \fall{R} (x^R \preceq y) \to (\tlt y) \label{eq:NO-prec-IHR}。
  \end{gather}
  使用与 $\ble$ 和 $\blt$ 类似的符号,我们为 $A'$ 配备了由以下定义的两个关系,针对 $\tlname:A'(y)$ 和 $\tlbname:A'(z)$:
  \begin{align*}
  (\tlname \bble \tlbname) &\defeq
  \Big((\tle y) \to (\tleb z)\Big) \land \Big((\tlt y) \to (\tltb z)\Big)\\
  (\tlname \bblt \tlbname) &\defeq
  \Big((\tle y) \to (\tltb z)\Big)。 % \land \Big(\tlt \to \tltb\Big)。
  \end{align*}
  %(这些是一般归纳原理中的类型族 $B$ 和 $C$。)
  再次地,$\bble$ 显然是反对称的。
  此外,映射到 $A'(y)$ 的函数 $\prd{y:\NO} A'(y)$ 保持不等性,是指 $x^L<x$ 和 $x<x^R$ 的不等关系在左侧保持,$y^L<y^R$ 的不等关系在右侧保持。
  因此,这种内部归纳将提供我们需要的来完成外部递归的主要条款。

  对于内部归纳的主要条款,我们还假设 $y$ 是由一个分割定义的,并且 $(x\prec y^L)$,$(x\prec y^R)$,$(x\preceq y^L)$ 和 $(x\preceq y^R)$ 的性质已定义,对所有 $L$ 和 $R$ 满足,严格的关系暗示非严格的关系,并且左侧的 $x^L<x$ 和 $x<x^R$ 的传递性以及右侧的 $y^L<y^R$ 的传递性。
  % \begin{equation}
  %   \fall{L,R}\Big((x \preceq y^L) \to (x \prec y^R)\Big) % \land \Big((x \prec y^L) \to (x\prec y^R)\Big)。
  %   \label{eq:NO-prec-inner-IH}
  % \end{equation}
  我们现在可以给出定理陈述中指定的定义:
  \begin{align}
  (x\preceq y) &\defeq
  (\fall{L} x^L\prec y) \land (\fall{R} x\prec y^R), \label{eq:NO-preceq-def}\\
  (x\prec y) &\defeq
  (\exis{L} x\preceq y^L) \lor (\exis{R} x^R \preceq y).\label{eq:NO-prec-def}
  \end{align}
  为了定义 $A'(y)$ 的一个元素,我们首先必须证明 $(x\prec y) \to (x\preceq y)$。
  假设 $x\prec y$ 有两种情况。
  一方面,如果存在 $L_0$ 使得 $x\preceq y^{L_0}$,则根据右边的传递性并考虑 $y^{L_0}<y^R$,我们得到对于所有的 $R$,$x\prec y^R$。
  此外,根据左边的传递性并考虑 $x^L<x$,我们有对于任何 $L$,$x^L \prec y^{L_0}$,因此根据右边的传递性,我们有 $x^L\prec y$。
  因此,$x\preceq y$。

  另一方面,如果存在 $R_0$ 使得 $x^{R_0}\preceq y$,那么根据~\eqref{eq:NO-prec-outer-IH},我们有对于所有的 $L$,$x^L \prec y$。
  根据左边和右边的传递性并考虑 $x<x^{R_0}$ 和 $y<y^R$,我们对于任何 $R$ 都有 $x\prec y^R$。
  因此,$x\preceq y$。

  我们还需要证明这些定义在左边对于 $x^L<x$ 和 $x<x^R$ 是传递的。
  但是如果 $x\preceq y$,那么根据定义,所有的 $L$ 都有 $x^L\prec y$;如果 $x^R\preceq y$,那么根据定义 $x\prec y$。

  因此,~\eqref{eq:NO-preceq-def} 和~\eqref{eq:NO-prec-def} 确实定义了 $A'(y)$ 的一个元素。
  我们现在必须验证这个定义是否保持了不等式,作为到 $A'$ 的一个依赖函数,即这些关系在右边是否是传递的。
  请记住,在每种情况下,我们可以归纳假设它们在右边对于由不等式构造器产生的所有不等式是传递的。
  \begin{itemize}
    \item 假设 $x\preceq y$ 且 $y\le z$,后者来自 $y^L<z$ 和 $y<z^R$ 对于所有的 $L$ 和 $R$。
    然后,应用到 $y<z^R$ 的内递归的归纳假设,对于任何 $R$,我们得到 $x\prec z^R$。
    此外,根据定义 $x\preceq y$ 意味着对于任何 $L$,$x^L \prec y$,所以通过外递归的归纳假设,我们得到 $x^L \prec z$。
    因此,$x\preceq z$。
    \item 假设 $x\preceq y$ 且 $y<z$。
    首先,假设 $y<z$ 来自 $y\le z^{L_0}$。
    然后应用到 $y\le z^{L_0}$ 的内递归的归纳假设,我们得到 $x \preceq z^{L_0}$,因此 $x\prec z$。

    其次,假设 $y<z$ 来自 $y^{R_0}\le z$。
    然后根据定义,$x\preceq y$ 意味着 $x\prec y^{R_0}$,然后对于 $y^{R_0}\le z$ 的内递归的归纳假设,我们得到 $x\prec z$。
    \item 假设 $x\prec y$ 且 $y\le z$,后者来自 $y^L<z$ 和 $y<z^R$ 对于所有的 $L$ 和 $R$。
    根据定义,$x\prec y$ 仅意味着存在 $R_0$ 使得 $x^{R_0}\preceq y$ 或存在 $L_0$ 使得 $x\preceq y^{L_0}$。
    如果 $x^{R_0}\preceq y$,那么外递归的归纳假设得出 $x^{R_0}\preceq z$,因此 $x\prec z$。
    如果 $x\preceq y^{L_0}$,那么对于 $y^{L_0}<z$ 的内递归的归纳假设(该假设由 $y\le z$ 的构造器成立)得出 $x\prec z$。
  \end{itemize}
  这完成了内递归。
  因此,对于由分割定义的任何 $x$,我们有 $(x\prec \blank)$ 和 $(x\preceq \blank)$ 由~\eqref{eq:NO-preceq-def} 和~\eqref{eq:NO-prec-def} 定义,并在右边是传递的。

  要完成外递归,我们需要验证这些定义在左边也是传递的。
  在对 $z$ 进行 $\NO$-归纳后,我们得到了三个基本上与上述右边传递性相同的情况。
  因此,我们省略它们。
\end{proof}

\begin{thm}\label{thm:NO-encode-decode}
对于任何 $x,y:\NO$,我们有 $(x<y)=(x\prec y)$ 和 $(x\le y)=(x\preceq y)$。
\end{thm}
\begin{proof}
  从左到右,我们使用 $(\NO,\le,<)$-归纳,其中 $A(x)\defeq\unit$,$\preceq$ 和 $\prec$ 提供了关系 $\ble$ 和 $\blt$。
  在所有构造器的情况下,$x$ 和 $y$ 是通过分割定义的,因此 $\preceq$ 和 $\prec$ 的定义得到评估,归纳假设适用。

  从右到左,我们使用 $\NO$-归纳假设 $x$ 和 $y$ 是通过分割定义的。
  但现在 $\preceq$ 和 $\prec$ 的定义,以及归纳假设,正好提供了 $\le$ 和 $<$ 的相关构造器所需的数据。
\end{proof}

\begin{cor}\label{thm:NO-unstrict-transitive}
在 $\NO$ 上的关系 $\le$ 和 $<$ 满足
\[ \fall{x,y:\NO} (x<y) \to (x\le y) \]
并且是传递的:
\index{transitivity!of . for surreals@of $<$ for surreals}
\index{transitivity!of . for surreals@of $\leq$ for surreals}
\begin{gather*}
(x\le y) \to (y\le z) \to (x\le z)\\
(x\le y) \to (y< z) \to (x< z)\\
(x< y) \to (y\le z) \to (x< z).
\end{gather*}
\end{cor}

与柯西实数相似,联合 $(\NO,\le,<)$-递归原则在定义所有 $\NO$ 上的操作时仍然是必要的。

\begin{eg}\label{eg:surreal-addition}
\index{addition!of surreal numbers}%
我们通过对第一个参数进行递归,然后对第二个参数进行归纳来定义 $\mathord+:\NO\to\NO\to\NO$。
对于外递归,我们将余域设为由函数 $g$ 组成的 $\NO\to\NO$ 的子集,使得对于所有 $x,y$,$(x<y) \to (g(x)<g(y))$ 和 $(x\le y) \to (g(x)\le g(y))$ 成立。
对于这样的 $g,h$,我们定义 $(g\ble h)\defeq \fall{x:\NO} g(x)\le h(x)$ 和 $(g\blt h)\defeq \fall{x:\NO} g(x)< h(x)$。
显然,$\ble$ 是反对称的。

对于递归的主要条款,我们假设 $x$ 是通过分割定义的,函数 $(x^L+\blank)$ 和 $(x^R+\blank)$ 已定义、保持不等式,并满足 $x^L+y<x^R+y$,然后我们定义 $(x+\blank)$。
与 \cref{defn:No-codes} 中一样,我们不使用内递归,而是通过进入 $A:\NO\to\type$ 的家族进行内归纳,其中 $A(y)$ 是那些满足每个 $x^L + y < z$ 和每个 $x^R + y > z$ 的 $z:\NO$ 的子集。
我们给 $A$ 配备由 $\NO$ 诱导的不等式 $\le$ 和 $<$,因此反对称性是显而易见的。
对于内递归的主要条款,我们还假设 $y$ 是通过分割定义的,并且每个 $x+y^L$ 和 $x+y^R$ 都已定义并满足 $x^L+y^L < x+y^L$,$x^L+y^R < x+y^R$,$x+y^L < x^R + y^L$ 和 $x+y^R < x^R+y^R$(这些来自施加在 $A(y)$ 元素上的附加条件),并且 $x+y^L < x+y^R$(因为 $x+y^L$ 和 $x+y^R$ 是 $A(y)$ 的依赖分割)。
现在我们给出康威的定义:
\[ x+y \defeq \surr{x^L+y, x+y^L}{x^R+y,x+y^R}. \]
换句话说,$x+y$ 的左选项是形式为 $x^L+y$ 的所有数字,其中 $x^L$ 是左选项,或者 $x+y^L$,其中 $y^L$ 是左选项。
我们必须证明这些左选项中的每一个都小于这些右选项中的每一个:
\begin{itemize}
  \item 根据外归纳假设,$x^L+y < x^R+y$。
  \item 根据 $(x^L+\blank)$ 保持不等式的事实,$x^L+y < x^L + y^R < x + y^R$,而第二个不等式由于 $x+y^R : A(y^R)$ 成立。
  \item 根据 $x+y^L : A(y^L)$ 和 $(x^R+\blank)$ 保持不等式的事实,$x+y^L < x^R+ y^L < x^R + y$。
  \item 根据内归纳假设(特别是我们有一个依赖分割的事实),$x+y^L < x+y^R$。
\end{itemize}
我们还必须证明这样定义的 $x+y$ 位于 $A(y)$ 中,即 $x^L + y < x+y$ 和 $x+y < x^R + y$;但这由 \cref{thm:NO-refl-opt}\ref{item:NO-lt-opt} 成立。

接下来我们必须验证 $(x+\blank)$ 的定义是否保持不等式:
\begin{itemize}
  \item 如果 $y\le z$ 来自知道 $y^L<z$ 和 $y<z^R$ 对于所有的 $L$ 和 $R$,那么内归纳假设得出 $x+y^L<x+z$ 和 $x+y < x+z^R$,而外归纳假设得出 $x^L+y \le x^L+z$ 和 $x^R+ y \le x^R+z$。
  此外,由于 $x^R+y$ 是 $x+y$ 的一个右选项,我们有 $x+y < x^R+y$。
  类似地,我们发现 $x^L+z$ 是 $x+z$ 的一个左选项,因此 $x^L+z < x+z$。
  因此,使用传递性,我们有 $x^L+y < x+z$ 和 $x+y < x^R+z$;所以我们可以通过 $\le$ 的构造器得出 $x+y \le x+z$。
  \item 如果 $y<z$ 来自 $L_0$ 使得 $y\le z^{L_0}$,那么归纳得出 $x+y \le x+z^{L_0}$,因此 $x+y<x+z$,因为 $x+z^{L_0}$ 是 $x+z$ 的一个右选项。
  \item 类似地,如果 $y<z$ 来自 $y^{R_0}\le z$,那么 $x+y<x+z$ 因为 $x+y^{R_0}\le x+z$。
\end{itemize}
这完成了内递归。
对于外递归,我们必须验证 $+$ 在左边也保持不等式。
在 $\NO$-归纳后,这个过程与之前的完全相同。
\end{eg}

\index{acceptance|(}%
\index{mathematics!formalized}%
在~\cite{conway:onag} 的零部分附录中,康威讨论了如何在 ZFC 集合论中形式化超现实数:通过沿着序数迭代并转向每个等价类的最低等级的代表集,或者通过用“符号扩展”来表示数字。
然后他指出
\begin{quote}\footnotesize
这些构造的奇异复杂性更多地告诉我们有关 ZF 中形式化的性质,而不是我们数字系统的性质\dots
\end{quote}
并继续倡导一种“允许的构造类型”的一般理论,该理论应包括
\begin{enumerate}\footnotesize
\item 可以以任何合理的构造方式从早期对象创建新对象。\label{item:conway1}
\item 在创建对象之间的平等可以是任何所需的等价关系。\label{item:conway2}
\end{enumerate}
\noindent
条件~\ref{item:conway1} 可以自然地理解为证明\emph{归纳定义}的一般原则是合理的,例如在 \cref{sec:strictly-positive,sec:generalizations} 中提出的那些原则。
特别是,构造器的严格正性条件可以被视为“合理构造”的形式化。
然后条件~\ref{item:conway2} 建议我们应将其扩展到各种\emph{高阶归纳定义},其中我们可以引入路径构造器以使对象以任何合理的方式相等。
例如,在下一段中康威说:
\begin{quote}\footnotesize
\dots 我们还可以自由创建一个新对象 $(x,y)$ 并将其称为 $x$ 和 $y$ 的有序对。
我们还可以创建一个与 $(x,y)$ 不同但与之共存的有序对 $[x,y]$\dots
如果我们想将 $(x,y)$ 变成无序对,我们可以通过等价关系 $(x,y)=(z,t)$ 来定义平等,当且仅当 $x=z,y=t$ 或者 $x=t,y=z$。
\end{quote}
引入具有新名称的新对象的自由,生成特定形式的构造器,正是我们在归纳定义理论中所拥有的自由。
正如我们在 \cref{sec:appetizer-univalence} 中的自然数 $\nat$ 和 $\nat'$ 的两个副本,如果我们写下一个与笛卡尔积类型 $A\times B$ 相同的定义,我们将获得一个不同的积类型 $A\times' B$,其规范元素我们可以自由地写作 $[x,y]$。
并且我们可以通过添加合适的路径构造器使其中一个成为无序对的类型。%(也许进行 0 截断)。

当然,康威的观点不是抱怨 ZF,特别是,而是反对所有的基础理论:
\begin{quote}\footnotesize
\dots 这个提议不是作为 ZF 的任何特定理论的替代品\dots{}
提议的是我们赋予自己创造这些种类的任意数学理论的自由,但证明一个元定理,该定理一劳永逸地确保任何这种理论都可以用任何标准的基础理论形式化。
\end{quote}
有人可能会回应说,实际上,单一性的基础不是康威所设想的“标准基础理论”之一,而是我们可以表达我们创造新理论的能力的\emph{元理论},并且关于它我们可以证明康威的元定理。
例如,超现实数是康威心中的“数学理论”之一,我们已经看到它们可以在单一性基础内构造并得到证明。
同样,康威早些时候提到
\begin{quote}\footnotesize
\dots 集合论将是这样的一种理论,集合通过对应于通常公理的过程从早期集合中构造出来,平等关系就是具有相同成员的关系。
\end{quote}
这个描述与 \cref{sec:cumulative-hierarchy} 中累积层次的高阶归纳构造非常吻合。
康威的元定理将对应于我们多次提到的事实,即我们可以在 ZFC 内构造单一性基础的模型(这超出了本书的范围)。

然而,单一性基础本身如此丰富和强大,以至于仅将其归为构造类似集合的理论的元理论是愚蠢的。
我们已经看到,即使在集合(0-类型)的层面,单一性基础中的高阶归纳类型通过它们的通用属性(\cref{sec:free-algebras})直接构造对象,例如柯西完备性的构造性理论(\cref{sec:cauchy-reals})。
但最重要的是,直接在基础系统中建模同伦论和范畴论的潜力(\cref{cha:homotopy,cha:category-theory})赋予单一性基础以任何集合论基础都无法比拟的优势。
\index{acceptance|)}%

\index{surreal numbers|)}%

\sectionNotes

定义Dedekind实数上的代数运算,特别是乘法,既有点棘手又有点繁琐。
有几种方法可以启动算术:每种方法都有其优势,但它们似乎都需要一些技术工作。
例如,Richman~\cite{Richman:reals} 首先在正切割上定义了 Dedekind 实数的乘法,然后将其代数扩展到所有 Dedekind 切割,而 Conway~\cite{conway:onag} 则观察到,超现实数的乘法定义对于 Dedekind 实数来说效果很好。

我们对 Dedekind 实数的处理借鉴了~\cite{BauerTaylor09} 的许多想法,在该文献中,Dedekind 实数是在抽象 Stone 对偶性的背景下构造的。
\index{Abstract Stone Duality}%
这是一种(受限的)简单类型的 $\lambda$-演算,其中有一个区分的对象 $\Sigma$ 用于分类开集,并且通过对偶也分类闭集。在~\cite{BauerTaylor09} 中,您还可以找到算术运算基本性质的详细证明。

$\RC$ 是最小的柯西完备阿基米德有序域这一事实,如在 \cref{RC-initial-Cauchy-complete} 中证明的那样,表明我们的柯西实数可能与 Escard{\'o}-Simpson 实数~\cite{EscardoSimpson:01} 一致。
\index{real numbers!Escardo-Simpson@Escard\'o-Simpson}%
检查这是否真的如此将很有趣\index{open!problem}。Escard{\'o}-Simpson 实数或更准确地说是相应的闭区间的概念很有趣,因为它可以在任何具有有限积的范畴中表述。

在通过“正规扩展公理”增强的构造性集合论中,您还可以尝试通过在柯西序列的极限下进行闭合并进行超限迭代来定义柯西完备性。
检查此构造是否与我们的构造一致也将很有趣。

众所周知,柯西实数与 Dedekind 实数的一致性需要依赖选择,但不太为人所知的是,可数选择就足够了。请记住\define{依赖选择}
\indexdef{axiom!of choice!dependent}%
\index{axiom!of choice!countable}%
\index{total!relation}%
声明,对于 $A$ 上的总关系 $R$,我们意味着 $\fall{x : A} \exis{y : A} R(x,y)$,并且对于任何 $a : A$,仅存在 $f : \N \to A$ 使得 $f(0) = a$ 并且对于所有 $n : \N$,$R(f(n), f(n+1))$ 成立。我们的 \cref{when-reals-coincide} 使用了将依赖选择的应用转换为使用可数选择的典型技巧。即,我们使用可数选择来预先做出所有可能出现的选择,然后使用选择函数来避免依赖选择。

在构造性背景下,各种紧致性概念之间的复杂关系在 \cite{bridges2002compactness} 中有所讨论。Palmgren~\cite{Palmgren:FT} 对逐点分析和点自由拓扑进行了良好的比较。

超现实数是由~\cite{conway:onag} 定义的,使用了一种归纳定义的形式,但没有在任何基础系统中显式证明。
出于这个原因,一些后来的作者倾向于使用符号扩展或其他更明确的表示,这些表示可以更明显地编码到集合论中。
在类型论中表示它们的想法最早由 Hancock 提出,而
Setzer 和 Forsberg~\cite{forsbergfinite} 注意到超现实数及其不等式关系 $<$ 和 $\le$ 自然地形成了一个归纳-归纳定义。
这里提出的\emph{高阶}归纳-归纳版本,内置了超现实数的正确平等概念,是新的。

\sectionExercises

\begin{ex}\label{ex:alt-dedekind-reals}
给出 Dedekind 实数的替代定义,首先定义平方,然后使用 \cref{mult-from-square}。
检查是否获得了一个交换环。
\end{ex}

\begin{ex} \label{ex:RD-extended-reals}
%
假设我们去除了 \cref{defn:dedekind-reals} 中的有界性条件 \ref{defn:dedekind-reals-inhabited}。
那么我们得到\define{扩展实数}
\indexdef{real numbers!extended}%
\indexdef{extended real numbers}%
它们包含 $-\infty \defeq
(\emptyt, \Q)$ 和 $\infty \defeq (\Q, \emptyt)$。 哪些切割上算术运算的定义仍然适用于扩展实数?我们得到什么代数结构?
\end{ex}

\begin{ex} \label{ex:RD-lower-cuts}
%
通过考虑单侧切割,我们得到\define{下}和\define{上} Dedekind 实数,
\indexdef{real numbers!Dedekind!upper}%
\indexdef{real numbers!Dedekind!lower}%
\indexdef{lower Dedekind reals}%
\indexdef{upper Dedekind reals}%
\index{cut!Dedekind}%
分别。例如,下实数由谓词 $L : \Q \to \Omega$ 给出
%
\begin{enumerate}
  \item \emph{非空:} $\exis{q : \Q} L(q)$ 和
  \item \emph{舍入:} $L(q) = \exis{r : \Q} q < r \land L(r)$。
  \index{rounded!Dedekind cut}
\end{enumerate}
%
(我们还可以要求 $\exis{r : \Q} \lnot L(r)$ 以排除切割 $\infty \defeq
\Q$。)您可以在下实数上定义哪些算术运算?特别是,发生了什么加法逆运算?
\end{ex}

\begin{ex} \label{ex:RD-interval-arithmetic}
%
\index{interval!arithmetic}%
假设我们去除了 \cref{defn:dedekind-reals} 中的定点性条件。
那么我们得到\define{区间域}
\indexdef{interval!domain}%
$\mathbb{I}$ 因为切割允许
存在“间隙”,即只是区间。通过定义偏序 $\sqsubseteq$ 在
$\mathbb{I}$ 上通过
%
\begin{narrowmultline*}
  ((L, U) \sqsubseteq (L', U'))
  \defeq \narrowbreak
  (\fall{q : \Q} L(q) \Rightarrow L'(q)) \land
  (\fall{q : \Q} U(q) \Rightarrow U'(q))。
\end{narrowmultline*}
%
什么是相对于 $\mathbb{I}$ 的 $\mathbb{I}$ 的最大元素?定义
“端点”操作,它将区间域的元素分配到它的下端点和
上端点。端点是实数、下实数还是上实数(参见
\cref{ex:RD-lower-cuts})?哪些切割上的算术运算定义仍然适用于区间域?
\end{ex}

\begin{ex} \label{ex:RD-lt-vs-le}
显示对于所有 $x, y : \RD$,
%
\begin{equation*}
  \lnot (x < y) \Rightarrow y \leq x
\end{equation*}
%
和
%
\begin{equation*}
  \eqv{(x \leq y)}{\Parens{\prd{\epsilon : \Qp} x < y + \epsilon}}。
\end{equation*}
%
$\lnot (x \leq y)$ 是否意味着 $y < x$?
\end{ex}

\begin{ex} \label{ex:reals-non-constant-into-Z}
\mbox{}
%
\begin{enumerate}
  \item
  假设排中律,构造一个非常数映射 $\RD \to \Z$。
  \item
  假设 $f : \RD \to \Z$ 是一个映射使得 $f(0) = 0$ 并且对于所有 $x >
  0$,$f(x) \neq 0$。从中推导出有限全知原理~\eqref{eq:lpo}。
  \index{limited principle of omniscience}%
\end{enumerate}
\end{ex}

\begin{ex} \label{ex:traditional-archimedean}
\index{ordered field!archimedean}%
显示在一个有序域 $F$ 中,$\Q$ 的密度和传统的阿基米德公理
是等价的:
%
\begin{equation*}
(\fall{x, y : F} x < y \Rightarrow \exis{q : \Q} x < q < y)
  \Leftrightarrow
  (\fall{x : F} \exis{k : \Z} x < k)。
\end{equation*}
\end{ex}

\begin{ex} \label{RC-Lipschitz-on-interval} 假设 $a, b : \Q$ 和 $f : \setof{ q : \Q |
a \leq q \leq b } \to \RC$ 是常数为 $L$ 的 Lipschitz 映射。显示存在一个唯一的
延拓 $\bar{f} : [a,b] \to \RC$ 的 Lipschitz 映射
常数 $L$。提示:不要重新执行 \cref{RC-extend-Q-Lipschitz} 对于封闭
区间,请注意有一个缩回 $r : \RC \to [-n,n]$ 并应用
\cref{RC-extend-Q-Lipschitz} 对 $f \circ r$。
\end{ex}

\begin{ex} \label{ex:metric-completion}
\index{completion!of a metric space}%
将 $\RC$ 的构造推广到任何度量空间的柯西完备性。首先,考虑哪种实数概念最自然作为度量空间距离的余域。它重要吗?接下来,详细研究两种构造:
%
\begin{enumerate}
  \item 按照柯西实数的构造,将度量空间的完备性定义为在柯西序列的极限下闭合的归纳-归纳类型。
  \item 使用以下构造,由 Lawvere~\cite{lawvere:metric-spaces} 和 Richman~\cite{Richman00thefundamental} 提出,其中度量空间 $(M, d)$ 的完备性由\define{位置}的类型给出。
  \indexdef{location}%
  位置是一个函数 $f : M \to \R$ 使得
  %
  \begin{enumerate}
    \item $f(x) \geq |f(y) - d(x,y)|$ 对于所有 $x, y : M$,并且
    \item $\inf_{x \in M} f(x) = 0$,我们意味着 $\fall{\epsilon : \Qp} \exis{x : M} |f(x)| < \epsilon$ 和 $\fall{x : M} f(x) \geq 0$。
  \end{enumerate}
  %
  这个想法是 $f$ 看起来像是在测量距离从一个点。
\end{enumerate}
%
\index{universal!property!of metric completion}%
最后,证明度量完备性的以下通用属性:从度量空间到柯西完备度量空间的局部一致连续映射可以唯一地延拓到完备性上的局部一致连续映射。 (我们说映射是\define{局部一致连续的}
\indexdef{function!locally uniformly continuous}%
\indexdef{locally uniformly continuous map}%
如果它在开放球上是一致连续的。)
\end{ex}

\index{metric space|)}%

\begin{ex} \label{ex:reals-apart-neq-MP}
\define{马尔可夫原理}
\indexdef{axiom!Markov's principle}%
\indexdef{Markov's principle}%
表示对于所有 $f : \nat \to \bool$,
%
\begin{equation*}
  (\lnot \lnot \exis{n : \nat} f(n) = \btrue)
  \Rightarrow
  \exis{n : \nat} f(n) = \btrue。
\end{equation*}
%
这是双重否定律的一个特例~\eqref{eq:ldn}。显示
$\fall{x, y: \RD} x \neq y \Rightarrow x \apart y$ 意味着马尔可夫原理。是
反过来也成立?
\end{ex}

\begin{ex} \label{ex:reals-apart-zero-divisors}
\index{apartness}%
验证以下“无零因子”性质是否适用于实数:
$x y \apart 0 \Leftrightarrow x \apart 0 \land y \apart 0$。
\end{ex}

\begin{ex} \label{ex:finite-cover-lebesgue-number}
%
假设 $(q_1, r_1), \ldots, (q_n, r_n)$ 逐点覆盖 $(a, b)$。那么有
$\epsilon : \Qp$ 使得每当 $a < x < y < b$ 并且 $|x - y| < \epsilon$
然后仅存在 $i$ 使得 $q_i < x < r_i$ 并且 $q_i < y < r_i$。这样的
$\epsilon$ 被称为\define{勒贝格数}
\indexdef{Lebesgue number}%
对于给定的覆盖。
\end{ex}

\begin{ex} \label{ex:mean-value-theorem}
%
证明中值定理的以下近似版本:
%
\begin{quote}
  \emph{
    如果 $f : [0,1] \to \R$ 是一致连续的并且 $f(0) < 0 < f(1)$ 然后
    对于每个 $\epsilon : \Qp$ 仅存在 $x : [0,1]$ 使得 $|f(x)| <
    \epsilon$。
  }
\end{quote}
%
提示:不要尝试使用二分法,因为它会导致选择公理。
相反,用分段线性图逼近 $f$。您如何构造分段
线性图?
\end{ex}

\begin{ex}\label{ex:knuth-surreal-check}
检查是否可以使用 \cref{sec:surreals} 的高阶归纳-归纳超现实数来完成~\cite{knuth74:_surreal_number} 中的所有操作。
\end{ex}

\begin{ex}\label{ex:reals-into-surreals}
回忆在~\pageref{reals-into-surreals} 页定义的函数 $\iota_{\RD}:\RD\to\NO$。
\begin{enumerate}
  \item 显示 $\iota_{\RD}$ 是单射的。
  \item 有显然的扩展 $\iota_{\RD}$ 到扩展实数(\cref{ex:RD-extended-reals})和区间域(\cref{ex:RD-interval-arithmetic})。
  它们是单射的吗?
\end{enumerate}
\end{ex}

\begin{ex}\label{ex:ord-into-surreals}
显示在~\pageref{ord-into-surreals} 页定义的函数 $\iota_{\ord}:\ord\to\NO$ 是单射的,当且仅当 \LEM{} 成立。
\end{ex}

\begin{ex}\label{ex:hiit-plump}
定义一个类型 $\mathsf{POrd}$,配备二元关系 $\le$ 和 $<$,通过模仿 \NO 的定义,但只使用左选项。
\begin{enumerate}
  \item 构造一个映射 $j:\mathsf{POrd} \to \NO$ 并显示它是嵌入的。
  \item 显示 $\mathsf{POrd}$ 是一个序数(在下一个更高的宇宙中,如 \ord)在关系 $<$ 下。
  \item 假设命题缩放,显示 $\mathsf{POrd}$ 等价于子集
  \[\setof{A:\ord | \mathsf{isPlump}(A)}\]
  来自 \cref{ex:plump-ordinals} 的 \ord。
  结论 $\iota_{\ord}:\ord\to\NO$ 是单射的,当限制为饱满序数时。
\end{enumerate}
在没有命题缩放的情况下,我们仍然可以将 $\mathsf{POrd}$(或它们在 \NO 中的图像)称为\define{饱满序数}。\index{ordinal!plump}\index{plump!ordinal}
\end{ex}

\begin{ex}\label{ex:pseudo-ordinals}
定义一个超现实数为\define{伪序数}\index{pseudo-ordinal}\index{ordinal!pseudo-} 如果它等于一个没有右选项的分割 $\surr{x^L}{}$(但其左选项本身可能有右选项)。
显示声明“每个伪序数都是饱满序数”当且仅当等价于 \LEM{}。
\end{ex}

\begin{ex}\label{ex:double-No-recursion}
注意 \cref{defn:No-codes} 和 \cref{eg:surreal-addition} 都使用类似的模式来定义一个函数 $\NO \to \NO \to B$:一个外 \NO-递归,其余域是从 \NO 到 $B$ 的保序函数的集合,接着是内 \NO-归纳到一个家族 $A:\NO\to\type$,其中 $A(y)$ 是 $B$ 的子集,确保不等式 $x^L<x$ 和 $x<x^R$ 也被保留。
制定并证明一个“双 \NO-递归”的通用原则,它推广了这些证明。
\end{ex}

\index{real numbers|)}%

%%% Local Variables:
%%% mode: latex
%%% TeX-master: "hott-online"
%%% End:
