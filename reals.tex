\chapter{实数 (Real numbers)}
\label{cha:real-numbers}

\index{real numbers|(实数|(}%
任何称得上是数学基础的理论最终都必须解决实数的构造问题,这个问题在数学分析中被理解为完备的阿基米德有序域。
\index{ordered field (有序域)}%
完备性有两种定义。一种由柯西提出,要求实数在柯西序列的极限下是封闭的\index{Cauchy!sequence (柯西!序列)},而Dedekind提出的更强的定义要求实数在Dedekind分割下是封闭的\index{cut!Dedekind (分割!Dedekind)}。
这两种完备性定义分别引出了两种构造实数的方法,我们将在\cref{sec:dedekind-reals}和\cref{sec:cauchy-reals}中分别进行研究。在\cref{RD-final-field,RC-initial-Cauchy-complete}中,我们从泛性质的角度对这两种构造进行了刻画:Dedekind实数是最终的阿基米德有序域,而柯西实数是初始的柯西完备阿基米德有序域。

在传统的构造性数学中,
\index{mathematics!constructive (数学!构造性)}%
实数总是需要某些妥协。例如,Dedekind实数在幂集或其他形式的不可判定性存在时效果较好,而柯西实数在可数选择的存在下效果较好。
\index{axiom!of choice!countable (选择公理!可数)}%
然而,我们给出了柯西实数的新构造,将其作为一种更高阶的归纳-归纳类型,这种构造似乎是第三种可能性,它既不需要幂集也不需要可数选择。

在~\cref{sec:comp-cauchy-dedek}中,我们比较了这两种实数构造。柯西实数包含在Dedekind实数中。如果排中律或可数选择成立,它们是一致的,但通常情况下,这种包含可能是严格的。

在~\cref{sec:compactness-interval}中,我们讨论了闭区间~$[0,1]$的三种紧致性定义。我们首先证明~$[0,1]$在度量意义上是紧致的\indexdef{metrically compact (度量紧致)}\indexdef{compactness!metric (紧致性!度量)},即它是完备且全有界的,并且度量紧致空间上的一致连续映射表现得如预期那样。相对而言,Bolzano--Weierstra\ss{}性质,即每个序列都有一个收敛的子序列,意味着有限全知原理,这是排中律的一种形式。最后,我们讨论了Heine--Borel紧致性。有限子覆盖性质的一个朴素的表述是不可行的,但证明相关的归纳覆盖是可行的。
本节基本上属于标准的构造性分析。

在同伦类型论中,实数和分析的发展可以很容易地与经典数学兼容。通过假设排中律和选择公理,我们得到了标准的经典分析\index{mathematics!classical (数学!经典的)}\index{classical!analysis (经典!分析)}:Dedekind实数和柯西实数一致,关于Dedekind实数的不可判定性性质的基础问题消失了,并且区间是尽可能紧致的。

我们在\cref{sec:surreals}中通过构造Conway的超现实数作为一种更高阶的归纳-归纳类型来结束本章;
这种构造在一元类型论中比在经典集合论中更自然。

除了\cref{cha:basics,cha:logic}中的基本理论外,如上所述,我们还使用了“更高阶的归纳-归纳类型”来构造柯西实数和超现实数:这些结合了\cref{cha:hits}中的思想和\cref{sec:generalizations}中提到的归纳-归纳类型的概念。
我们还经常使用\cref{subsec:prop-trunc}中描述的传统逻辑符号,以及(在\cref{sec:piw-pretopos}中证明的)我们的“集合”表现如预期的事实。

请注意,圆的通用覆盖的总空间,在
\cref{subsec:pi1s1-homotopy-theory}中扮演了类似于“实数”在经典代数拓扑中的角色,但并\emph{不是}我们寻找的实数类型。该
类型是可缩的,因此等价于单类型,因此它不能被赋予非平凡的代数结构。



\section{有理数域 (The field of rational numbers)}
\label{sec:field-rati-numb}

\indexdef{rational numbers (有理数)}%
\indexsee{number!rational (数!有理数)}{rational numbers (有理数)}%
我们首先构造有理数 \Q,因为实数可以被视为~\Q 的完备化。
一个专家可能会指出,\Q 可以被任何近似域替代,
\indexdef{field!approximate (域!近似)}%
即 \Q 的一个子环,其中存在任意精确的近似逆元
\index{inverse!approximate (逆元!近似)}%
。一个例子是
二进制有理数环,
\index{rational numbers!dyadic (有理数!二进制)}%
这些有理数的形式为 $n/2^k$。
如果我们在计算机上实现构造性数学,
近似域会更适合,但我们将这种精细留给那些
关心~$\pi$ 的位数的人。

我们在\cref{sec:set-quotients}中构造了整数 \Z,作为 $\N\times
\N$ 的商,并观察到该商由幂等生成。在
\cref{sec:free-algebras}中我们看到 \Z 是在 \unit 上的自由群;我们可以类似地
证明它是 \emptyt 上的自由交换环\index{ring (环)}。有理数域 \Q 是
沿着相同的思路构造的,即为商
%
\[ \Q \defeq (\Z \times \N)/{\approx} \]
%
其中
\[ (u,a) \approx (v,b) \defeq (u (b + 1) = v (a + 1))。 \]
%
换句话说,一个对 $(u, a)$ 代表有理数 $u / (1 + a)$。这里不存在除以零的问题,因为我们巧妙地在分母~$a$ 上加了一。这里我们
也有一个规范的代表选择,即最简分数。因此我们可以
应用\cref{lem:quotient-when-canonical-representatives}来获得集合 \Q,它
再次具有可判定的等式。
\index{decidable!equality (可判定!等式)}%

我们不打算写下 \Q 上的算术运算,因为我们相信我们的读者
知道如何计算分数,即使是在分母上加一的情况下。
让我们只是记录结论,即有理数域 \Q 的构造完全没有问题,它具有可判定的等式和可判定的顺序。
它也可以被刻画为初始有序域。
\index{initial!ordered field (初始!有序域)}%

\symlabel{positive-rationals (正有理数)}
\indexdef{rational numbers!positive (有理数!正)}%
\indexdef{positive!rational numbers (正!有理数)}%
最后,我们将 $\Qp \defeq \setof{ q : \Q | q > 0 }$ 记作正有理数的类型。

\section{Dedekind 实数 (Dedekind reals)}
\label{sec:dedekind-reals}

\index{real numbers!Dedekind|(实数!Dedekind|(}%
让我们首先回顾一下Dedekind构造的基本思想。我们使用双侧的Dedekind
分割,而不是经常使用的单侧版本,因为对称性使
构造更优雅,并且它在构造性数学和经典数学中都能工作。
\index{mathematics!constructive (数学!构造性)}%
一个\emph{Dedekind分割}\index{cut!Dedekind (分割!Dedekind)}由一对 $(L, U)$ 组成,其中 $L, U \subseteq \Q$,
分别称为
\emph{下分割 (lower cut)} 和 \emph{上分割 (upper cut)},满足:
%
\begin{enumerate}
  \item \emph{非空 (inhabited):} 存在 $q \in L$ 和 $r \in U$,
  \item \emph{圆整 (rounded):} $q \in L \Leftrightarrow \exis {r \in \Q} q < r \land r \in L$
  和 $r \in U \Leftrightarrow \exis {q \in \Q} q \in U \land q < r$,
  \index{rounded!Dedekind cut (圆整!Dedekind 分割)}
  \item \emph{不相交 (disjoint):} $\lnot (q \in L \land q \in U)$,并且
  \item \emph{确定性 (located):} $q < r \Rightarrow q \in L \lor r \in U$。
  \index{locatedness (确定性)}%
\end{enumerate}
%
从左到右阅读圆整性条件告诉我们分割是\emph{开集 (open)},
\index{open!cut (开!分割)}%
而从右到左它们分别是\emph{下集 (lower)} 和 \emph{上集 (upper)}。确定性条件表明 $L$ 和 $U$ 之间没有大间隙。由于分割始终是开集,它们永远不会包含“中间的点”,即使它是有理数。一个典型的Dedekind分割如下图所示:
%
\begin{center}
  \begin{tikzpicture}[x=\textwidth]
    \draw[<-),line width=0.75pt] (0,0) -- (0.297,0) node[anchor=south east]{$L\ $};
    \draw[(->,line width=0.75pt] (0.300, 0) node[anchor=south west]{$\ U$} -- (0.9, 0) ;
  \end{tikzpicture}
\end{center}
%
我们可能会天真地将非正式定义翻译成类型论,认为分割
是一对映射 $L, U : \Q \to \prop$。但是我们在\cref{subsec:prop-subsets}中看到
$\prop$ 是 $\prop_{\UU_i}$ 的一种含糊的\index{typical ambiguity (典型模糊性)}符号,其中~$\UU_i$ 是一个宇宙。一旦我们
使用特定的 $\UU_i$ 来定义分割,实数类型将位于下一个
宇宙 $\UU_{i+1}$ 中,实数的性质位于更高的宇宙 $\UU_{i+2}$ 中,实数
子集的性质位于更高的宇宙 $\UU_{i+3}$ 中,依此类推。原则上,我们应该能够
跟踪宇宙层级\index{universe level (宇宙层级)},特别是在证明助手的帮助下,但这样做只会
让我们负担更多的繁琐工作,因此我们更愿意避免。我们因此将
作一个简化的假设,即单一命题类型 $\Omega$ 足以满足我们所有的需求。

事实上,Dedekind 实数的构造对逻辑操作相当有弹性。
我们可以有几种方法来理解使用单一类型
$\Omega$ 的含义:
%
\begin{enumerate}

  \item 我们可以将 $\Omega$ 识别为含糊的 $\prop$ 并跟踪所有出现在定义和构造中的宇宙。

  \item 我们可以假设命题重设公理,
  \index{propositional!resizing (命题!重设)}%
  如\cref{subsec:prop-subsets}所述,这本质上将所有 $\prop_{\UU_i}$ 的层级折叠为
  最低层级\index{universe level (宇宙层级)},我们称之为 $\Omega$。

  \item 对于一个不关心类型论宇宙的复杂性或计算的经典数学家,他可以简单地假设对于单纯命题成立的排中律~\eqref{eq:lem},
  \index{excluded middle (排中律)}%
  这样 $\Omega \jdeq \bool$。
  这不仅消除了对
  $\prop$ 的层级\index{universe level (宇宙层级)}问题,还将我们所做的一切转化为标准的经典\index{mathematics!classical (数学!经典的)}实数构造。

  \item 在另一个极端,人们可能会要求一个使构造工作所需的最小要求。定义一个命题为Dedekind分割的条件
  只使用合取,析取和对~\Q 的存在量词\index{quantifier!existential (量词!存在)},\Q 是一个可数集。因此我们可以将 $\Omega$ 视为初始\emph{$\sigma$-框架 (sigma-frame)},
  \index{initial!sigma-frame@$\sigma$-frame (初始!$\sigma$-框架)}%
  \index{sigma-frame@$\sigma$-frame!initial|defstyle ($\sigma$-框架!初始|定义风格)}%
  即一个具有可数上确界的格\index{lattice (格)},其中二元下确界分布在可数
  上确界上。(初始 $\sigma$-框架不能是两点格 $\bool$,因为
  $\bool$ 不闭合于可数上确界,除非我们假设排中律。) 这
  将导致 $\Omega$ 的构造作为一种更高阶的归纳-归纳类型,但在\cref{sec:cauchy-reals}中进行这种实验已经足够了。
\end{enumerate}

在所有上述情况下,$\Omega$ 是一个集合。
%
事不宜迟,我们将非正式定义翻译成类型论。
在本章中,我们使用了
来自\cref{defn:logical-notation}中的
逻辑符号。

\begin{defn} \label{defn:dedekind-reals}
一个\define{Dedekind 分割 (Dedekind cut)}
\indexsee{Dedekind!cut (Dedekind!分割)}{cut, Dedekind (分割, Dedekind)}%
\indexdef{cut!Dedekind (分割!Dedekind)}%
是一对仅仅命题的 $L : \Q \to \Omega$ 和 $U
: \Q \to \Omega$,满足:
%
\begin{enumerate}
  \item \label{defn:dedekind-reals-inhabited}
  \emph{非空 (inhabited) (即有界):} $\exis{q : \Q} L(q)$ 和 $\exis{r : \Q} U(r)$,
  \item \emph{圆整 (rounded):} 对于所有 $q, r : \Q$,
  \index{rounded!Dedekind cut (圆整!Dedekind 分割)}
  %
  \begin{align*}
    L(q) &\Leftrightarrow \exis{r : \Q} (q < r) \land L(r)
    \qquad\text{并且}\\
    U(r) &\Leftrightarrow \exis{q : \Q} (q < r) \land U(q),
  \end{align*}
  \item \emph{不相交 (disjoint):} 对于所有 $q : \Q$,$\lnot (L(q) \land U(q))$,
  \item \emph{确定性 (located):} 对于所有 $q, r : \Q$,$(q < r) \Rightarrow L(q) \lor U(r)$。
  \index{locatedness (确定性)}%
\end{enumerate}
%
我们用 $\dcut(L, U)$ 表示这些条件的合取。定义
\define{Dedekind 实数 (Dedekind reals)} 的类型为
\indexsee{Dedekind!real numbers (Dedekind!实数)}{real numbers, Dedekind (实数, Dedekind)}%
\indexdef{real numbers!Dedekind (实数!Dedekind)}%
%
\begin{equation*}
  \RD \defeq \setof{ (L, U) : (\Q \to \Omega) \times (\Q \to \Omega) | \dcut(L,U)}。
\end{equation*}
\end{defn}

显然,$\dcut(L, U)$ 是一个仅仅命题,并且由于 $\Q \to \Omega$ 是一个
集合,Dedekind 实数也形成了一个集合。参见
\cref{ex:RD-extended-reals,ex:RD-lower-cuts,ex:RD-interval-arithmetic},了解Dedekind
分割的变体,它们导致了扩展实数、下实数和上实数以及区间
域。

存在一个嵌入 $\Q \to \RD$,它将每个有理数 $q : \Q$ 关联到分割
$(L_q, U_q)$,其中
%
\begin{equation*}
  L_q(r) \defeq (r < q)
  \qquad\text{并且}\qquad
  U_q(r) \defeq (q < r)。
\end{equation*}
%
我们将简单地用 $q$ 表示与有理数相关联的分割 $(L_q, U_q)$。

\subsection{Dedekind 实数的代数结构 (The algebraic structure of Dedekind reals)}
\label{sec:algebr-struct-dedek}

在直觉主义逻辑中,Dedekind 实数的代数和序理论结构的构造
如往常一样进行。我们不打算详细讨论,而是指出经典\index{mathematics!classical (数学!经典的)}和直觉主义设置之间的差异。用 $L_x$ 和 $U_x$ 表示实数 $x : \RD$ 的下分割和上分割,我们定义加法为%
%
\indexdef{addition!of Dedekind reals (加法!Dedekind 实数)}%
\begin{align*}
  L_{x + y}(q) &\defeq \exis{r, s : \Q} L_x(r) \land L_y(s) \land q = r + s, \\
  U_{x + y}(q) &\defeq \exis{r, s : \Q} U_x(r) \land U_y(s) \land q = r + s,
\end{align*}
%
并定义加法逆元为
%
\begin{align*}
  L_{-x}(q) &\defeq \exis{r : \Q} U_x(r) \land q = - r, \\
  U_{-x}(q) &\defeq \exis{r : \Q} L_x(r) \land q = - r。
\end{align*}
%
通过这些操作,$(\RD, 0, {+}, {-})$ 是一个阿贝尔群\index{group!abelian (群!阿贝尔)}。乘法稍微复杂一点:
%
\indexdef{multiplication!of Dedekind reals (乘法!Dedekind 实数)}%
\begin{align*}
  L_{x \cdot y}(q) &\defeq
  \begin{aligned}[t]
    \exis{a, b, c, d : \Q} & L_x(a) \land U_x(b) \land L_y(c) \land U_y(d) \land {}\\
    & \qquad q < \min (a \cdot c, a \cdot d, b \cdot c, b \cdot d),
  \end{aligned} \\
  U_{x \cdot y}(q) &\defeq
  \begin{aligned}[t]
    \exis{a, b, c, d : \Q} & L_x(a) \land U_x(b) \land L_y(c) \land U_y(d) \land {}\\
    & \qquad \max (a \cdot c, a \cdot d, b \cdot c, b \cdot d) < q。
  \end{aligned}
\end{align*}
%
\index{interval!arithmetic (区间!算术)}%
这些公式与区间算术中区间的乘法有关,其中区间 $[a,b]$ 和 $[c,d]$ 的端点是有理数,乘积为区间
%
\begin{equation*}
[a,b] \cdot [c,d] =
[\min(a c, a d, b c, b d), \max(a c, a d, b c, b d)]。
\end{equation*}
%
例如,下分割的公式可以解释为当存在包含 $x$ 和 $y$ 的区间 $[a,b]$ 和 $[c,d]$ 时,$q < x \cdot y$
说明 $q$ 位于 $[a,b] \cdot [c,d]$ 的左侧。通常将满足 $L_x(a)$ 和 $U_x(b)$ 的区间 $[a,b]$ 视为~$x$ 的近似是有用的,参见
\cref{ex:RD-interval-arithmetic}。

现在我们有了一个具有单位元的交换环\index{ring (环)},单位元为
\index{unit!of a ring (单位元!环)}%
$(\RD, 0, 1, {+}, {-}, {\cdot})$。为了处理乘法逆元,我们首先引入顺序。定义 $\leq$ 和 $<$ 为
%
\begin{align*}
(x \leq y) &\ \defeq \ \fall{q : \Q} L_x(q) \Rightarrow L_y(q), \\
(x < y)    &\ \defeq \ \exis{q : \Q} U_x(q) \land L_y(q)。
\end{align*}

\begin{lem} \label{dedekind-in-cut-as-le}
对于所有 $x : \RD$ 和 $q : \Q$,$L_x(q) \Leftrightarrow (q < x)$ 并且 $U_x(q)
\Leftrightarrow (x < q)$。
\end{lem}

\begin{proof}
  如果 $L_x(q)$,则通过圆整性,存在仅仅一个 $r > q$ 使得 $L_x(r)$,并且由于
  $U_q(r)$,因此得出 $q < x$。反之,如果 $q < x$,则存在 $r : \Q$ 使得 $U_q(r)$ 并且 $L_x(r)$,因此由于 $L_x$ 是下集,因此 $L_x(q)$。其余证明对称成立。
\end{proof}

\index{partial order (偏序)}%
\index{transitivity!of . for reals@of $<$ for reals (传递性!实数的 $<$)}%
\index{transitivity!of . for reals@of $\leq$ for reals (传递性!实数的 $\leq$)}%
\index{relation!irreflexive (关系!反自反)}%
\index{irreflexivity!of . for reals@of $<$ for reals (反自反性!实数的 $<$)}%
关系 $\leq$ 是偏序,并且 $<$ 是传递的和反自反的。线性性
\index{order!linear (顺序!线性)}%
\index{linear order (线性顺序)}%
%
\begin{equation*}
(x < y) \lor (y \leq x)
\end{equation*}
%
在假设排中律时成立,但在不假设排中律的情况下,我们得到弱线性性
%
\index{order!weakly linear (顺序!弱线性)}
\index{weakly linear order (弱线性顺序)}
\begin{equation} \label{eq:RD-linear-order}
(x < y) \Rightarrow (x < z) \lor (z < y)。
\end{equation}
%
乍一看,\eqref{eq:RD-linear-order} 可能不清楚与线性顺序的关系。但如果我们令 $x \jdeq u - \epsilon$ 和 $y \jdeq u + \epsilon$,其中
$\epsilon > 0$,那么我们得到
%
\begin{equation*}
(u - \epsilon < z) \lor (z < u + \epsilon)。
\end{equation*}
%
这是“加上一个小数误差”的线性性,即,由于不合理地期望我们可以实际用无限精度计算,我们不应感到惊讶,我们只能在计算的任何有限精度范围内决定~$<$。

要看出~\eqref{eq:RD-linear-order} 成立,假设 $x < y$。那么仅仅存在 $q : \Q$ 使得 $U_x(q)$ 并且
$L_y(q)$。通过圆整性,存在仅仅 $r, s : \Q$ 使得 $r < q < s$,$U_x(r)$
并且 $L_y(s)$。然后,通过确定性 $L_z(r)$ 或 $U_z(s)$。在第一种情况下,我们得到 $x < z$,在第二种情况下,得到 $z < y$。

在经典情况下,乘法逆元存在于所有不等于零的数中。
然而,在没有排中律的情况下,需要一个更强的条件。称 $x, y : \RD$
彼此\define{相异 (apart)},
\indexdef{apartness (相异)}%
记作 $x \apart y$,当 $(x < y) \lor (y < x)$ 时:
%
\symlabel{apart (相异)}
\begin{equation*}
(x \apart y) \defeq (x < y) \lor (y < x)。
\end{equation*}
%
如果 $x \apart y$,则 $\lnot (x = y)$。
如果假设排中律,反过来也是成立的,但在构造性数学中不可证。
\index{mathematics!constructive (数学!构造性)}%
事实上,如果 $\lnot (x = y)$ 蕴含 $x\apart y$,那么排中律的一小部分将成立;参见\cref{ex:reals-apart-neq-MP}。

\begin{thm} \label{RD-inverse-apart-0}
当且仅当一个实数与 $0$ 相异时,它是可逆的。
\end{thm}

\begin{rmk}
  我们观察到一个实数是可逆的,当且仅当它是仅仅可逆的。实际上,在任何环中都是如此,\index{ring (环)} 因为一个环是一个集合,如果存在,乘法逆元是唯一的。 参见\cref{cor:UC}后的讨论。
\end{rmk}

\begin{proof}
  假设 $x \cdot y = 1$。那么仅仅存在 $a, b, c, d : \Q$ 使得
  $a < x < b$,$c < y < d$ 并且 $0 < \min (a c, a d, b c, b d)$。由于 $0 < a c$ 和 $0 < b c$,可得
  $a$,$b$ 和 $c$ 要么全为正数,要么全为负数。
  因此,要么 $0 < a < x$,要么 $x < b < 0$,因此 $x \apart 0$。

  反之,如果 $x \apart 0$,则 $x > 0$ 或 $x < 0$。
  如果 $x > 0$,我们定义 $x^{-1}$ 如下:
  %
  \begin{align*}
    L_{x^{-1}}(q) &\defeq
    (q > 0) \Rightarrow \exis{r : \Q} U_x(r) \land (q r < 1),
    \\
    U_{x^{-1}}(q) &\defeq
    (q > 0) \land \exis{r : \Q} L_x(r) \land (q r > 1)。
  \end{align*}
  %
  如果 $x < 0$,则我们通过以下方式定义它:
  %
  \begin{align*}
    L_{x^{-1}}(q) &\defeq
    (q < 0) \land \exis{r : \Q} U_x(r) \land (q r > 1),
    \\
    U_{x^{-1}}(q) &\defeq
    (q < 0) \Rightarrow \exis{r : \Q} L_x(r) \land (q r < 1)。\qedhere
  \end{align*}
\end{proof}

\index{ordered field!archimedean (有序域!阿基米德)}%
\index{dense (稠密)}%
\indexsee{order-dense (顺序稠密)}{dense (稠密)}%
阿基米德原理可以用几种方式表达。我们认为最具启发性的是
形式,它说 $\Q$ 在 $\RD$ 中是稠密的。

\begin{thm}[阿基米德原理对于 $\RD$] \label{RD-archimedean}
%
对于所有 $x, y : \RD$,如果 $x < y$,则仅仅存在 $q : \Q$ 使得
$x < q < y$。
\end{thm}

\begin{proof}
  根据 $<$ 的定义。
\end{proof}

在处理 Dedekind 实数的完备性之前,让我们准确地描述它们具有的代数
结构。在以下定义中,我们的目标不是最小公理化,而是一个有用的结构和性质。

\begin{defn} \label{ordered-field} 一个\define{有序域 (ordered field)}
\indexdef{ordered field (有序域)}%
\indexsee{field!ordered (域!有序)}{ordered field (有序域)}%
是一个集合 $F$,它与
常数 $0$,$1$,运算 $+$,$-$,$\cdot$,$\min$,$\max$,和仅仅关系
$\leq$,$<$,$\apart$ 使得:
%
\begin{enumerate}
  \item $(F, 0, 1, {+}, {-}, {\cdot})$ 是一个交换环,带有单位元;
  \index{unit!of a ring (单位元!环)}%
  \index{ring (环)}%
  \item 当且仅当 $x \apart 0$ 时,$x : F$ 是可逆的;
  \item $(F, {\leq}, {\min}, {\max})$ 是一个格;
  \item 严格顺序 $<$ 是传递的,反自反的,
  \index{relation!irreflexive (关系!反自反)}
  \index{irreflexivity!of . in a field@of $<$ in a field (反自反性!$<$ 在一个域中)}%
  并且弱线性 ($x < y \Rightarrow x < z \lor z < y$);\index{transitivity!of . in a field@of $<$ in a field (传递性!$<$ 在一个域中)}
  \index{order!weakly linear (顺序!弱线性)}
  \index{weakly linear order (弱线性顺序)}
  \index{strict!order (严格!顺序)}%
  \index{order!strict (顺序!严格)}%
  \item 相异性 $\apart$ 是反自反的,对称的,并且是对传递的 ($x \apart y \Rightarrow x \apart z \lor y \apart z$);
  \index{relation!irreflexive (关系!反自反)}
  \index{irreflexivity!of apartness (反自反性!相异性)}%
  \indexdef{relation!cotransitive (关系!对传递)}%
  \index{cotransitivity of apartness (相异性的对传递)}%
  \item 对于所有 $x, y, z : F$:
  %
  \begin{align*}
    x \leq y &\Leftrightarrow \lnot (y < x), &
    x < y \leq z &\Rightarrow x < z, \\
    x \apart y &\Leftrightarrow (x < y) \lor (y < x), &
    x \leq y < z &\Rightarrow x < z, \\
    x \leq y &\Leftrightarrow x + z \leq y + z, &
    x \leq y \land 0 \leq z &\Rightarrow x z \leq y z, \\
    x < y &\Leftrightarrow x + z < y + z, &
    0 < z \Rightarrow (x < y &\Leftrightarrow x z < y z), \\
    0 < x + y &\Rightarrow 0 < x \lor 0 < y, &
    0 &< 1。
  \end{align*}
\end{enumerate}
%
每个这样的域都有一个规范的嵌入 $\Q \to F$。一个有序域是
\define{阿基米德 (archimedean)}
\indexdef{ordered field!archimedean (有序域!阿基米德)}%
\indexsee{archimedean property (阿基米德性质)}{ordered field, archimedean (有序域, 阿基米德)}%
当对于所有 $x, y : F$,如果 $x < y$,则仅仅存在 $q :
\Q$ 使得 $x < q < y$。
\end{defn}

\begin{thm} \label{RD-archimedean-ordered-field}
Dedekind 实数形成了一个有序的阿基米德域。
\end{thm}

\begin{proof}
  我们省略了证明,因为我们已经展示的内容使得该定理
  看起来是合理的。
\end{proof}

\subsection{Dedekind 实数是柯西完备的 (Dedekind reals are Cauchy complete)}
\label{sec:RD-cauchy-complete}

回顾一下,$x : \N \to \Q$ 是一个\emph{柯西序列 (Cauchy sequence)}\indexdef{Cauchy!sequence (柯西!序列)}当它满足
%
\begin{equation} \label{eq:cauchy-sequence}
\prd{\epsilon : \Qp} \sm{n : \N} \prd{m, k \geq n} |x_m - x_k| < \epsilon。
\end{equation}
%
请注意,我们并\emph{没有} 截断内部存在的存在量词,因为我们实际上希望
计算收敛速度——没有误差估计的近似几乎没有什么有用的信息。通过\cref{thm:ttac},\eqref{eq:cauchy-sequence}产生一个函数 $M
: \Qp \to \N$,称为\emph{收敛模 (modulus of convergence)}\indexdef{modulus!of convergence (模!收敛)},使得 $m, k \geq M(\epsilon)$
时,$|x_m - x_k| < \epsilon$。由此我们得到 $|x_{M(\delta/2)} - x_{M(\epsilon/2)}|<
\delta + \epsilon$ 对于所有 $\delta, \epsilon : \Qp$。事实上,映射 $(\epsilon \mapsto
x_{M(\epsilon/2)}) : \Qp \to \Q$ 携带的极限信息与
原始的柯西条件~\eqref{eq:cauchy-sequence} 相同。我们将处理这些
近似函数而不是柯西序列。

\begin{defn} \label{defn:cauchy-approximation}
一个\define{柯西近似 (Cauchy approximation)}
\indexdef{Cauchy!approximation (柯西!近似)}%
是一个映射 $x : \Qp \to \RD$,满足
%
\begin{equation}
  \label{eq:cauchy-approx}
  \fall{\delta, \epsilon :\Qp} |x_\delta - x_\epsilon| < \delta + \epsilon。
\end{equation}
%
一个柯西近似 $x : \Qp \to \RD$ 的\define{极限 (limit)}
\index{limit!of a Cauchy approximation (极限!柯西近似)}%
是一个实数 $\ell : \RD$,使得
%
\begin{equation*}
  \fall{\epsilon, \theta : \Qp} |x_\epsilon - \ell| < \epsilon + \theta。
\end{equation*}
\end{defn}

\begin{thm} \label{RD-cauchy-complete}
$\RD$ 中的每一个柯西近似都有一个极限。
\end{thm}

\begin{proof}
  请注意,我们正在展示极限的存在,而不是仅仅存在。
  给定一个柯西近似 $x : \Qp \to \RD$,定义
  %
  \begin{align*}
    L_y(q) &\defeq \exis{\epsilon, \theta : \Qp} L_{x_\epsilon}(q + \epsilon + \theta),\\
    U_y(q) &\defeq \exis{\epsilon, \theta : \Qp} U_{x_\epsilon}(q - \epsilon - \theta)。
  \end{align*}
  %
  很明显 $L_y$ 和 $U_y$ 是非空的,圆整的,不相交的。为了证明
  确定性,考虑任意的 $q, r : \Q$ 满足 $q < r$。存在 $\epsilon : \Qp$ 满足
  $5 \epsilon < r - q$。由于 $q + 2 \epsilon < r - 2 \epsilon$ 仅仅
  $L_{x_\epsilon}(q + 2 \epsilon)$ 或 $U_{x_\epsilon}(r - 2 \epsilon)$。在第一种情况下,
  我们有 $L_y(q)$,在第二种情况下我们有 $U_y(r)$。

  为了表明 $y$ 是 $x$ 的极限,考虑任意的 $\epsilon, \theta : \Qp$。由于
  $\Q$ 在 $\RD$ 中是稠密的,存在仅仅的 $q, r : \Q$ 满足
  %
  \begin{narrowmultline*}
    x_\epsilon - \epsilon - \theta/2 < q < x_\epsilon - \epsilon - \theta/4
    < x_\epsilon < \\
    x_\epsilon + \epsilon + \theta/4 < r < x_\epsilon + \epsilon + \theta/2,
  \end{narrowmultline*}
  %
  因此 $q < y < r$。现在要么 $y < x_\epsilon + \theta/2$,要么 $x_\epsilon - \theta/2 < y$。
  在第一种情况下,我们有
  %
  \begin{equation*}
    x_\epsilon - \epsilon - \theta/2 < q < y < x_\epsilon + \theta/2,
  \end{equation*}
  %
  在第二种情况下
  %
  \begin{equation*}
    x_\epsilon - \theta/2 < y < r < x_\epsilon + \epsilon + \theta/2。
  \end{equation*}
  %
  无论哪种情况,都得出 $|y - x_\epsilon| < \epsilon + \theta$。
\end{proof}

为了完整性,我们记录了经典的表达方式。

\begin{cor}
  假设 $x : \N \to \RD$ 满足柯西条件~\eqref{eq:cauchy-sequence}。那么
  存在 $y : \RD$ 满足
  %
  \begin{equation*}
    \prd{\epsilon : \Qp} \sm{n : \N} \prd{m \geq n} |x_m - y| < \epsilon。
  \end{equation*}
\end{cor}

\begin{proof}
  通过\cref{thm:ttac}存在 $M : \Qp \to \N$ 使得 $\bar{x}(\epsilon) \defeq
  x_{M(\epsilon/2)}$ 是一个柯西近似。令 $y$ 为其极限,该极限通过
  \cref{RD-cauchy-complete} 存在。给定任意 $\epsilon : \Qp$,令 $n \defeq M(\epsilon/4)$
  并观察到,对于任何 $m \geq n$,
  %
  \begin{narrowmultline*}
    |x_m - y| \leq |x_m - x_n| + |x_n - y| =
    |x_m - x_n| + |\bar{x}(\epsilon/2) - y| < \narrowbreak
    \epsilon/4 + \epsilon/2 + \epsilon/4 = \epsilon。\qedhere
  \end{narrowmultline*}
\end{proof}

\subsection{Dedekind 实数是 Dedekind 完备的 (Dedekind reals are Dedekind complete)}
\label{sec:RD-dedekind-complete}

我们得到了 $\RD$ 作为 $\Q$ 上的 Dedekind 分割的类型。但是我们也可以从任意的阿基米德有序域 $F$ 开始,并构造 $F$ 上的 Dedekind 分割\index{cut!Dedekind (分割!Dedekind)}。这些将
再次形成一个阿基米德有序域 $\bar{F}$,称为\define{Dedekind 完备的 $F$},
\index{completion!Dedekind (完备!Dedekind)}%
\indexsee{Dedekind!completion (Dedekind!完备)}{completion, Dedekind (完备, Dedekind)},
其中 $F$ 被包含为一个子域。如果我们将此构造应用于
$\RD$,我们会得到更多的实数吗?答案是否定的。事实上,我们将证明一个更强的结果:$\RD$ 是最终的。

称一个有序域~$F$ 是\define{对 $\Omega$ 可接受的 (admissible for $\Omega$)}
\indexsee{admissible!ordered field (可接受的!有序域)}{ordered field, admissible (有序域, 可接受的)}%
\indexdef{ordered field!admissible (有序域!可接受的)}%
当严格顺序
$<$ 在~$F$ 上是一个映射 ${<} : F \to F \to \Omega$。

\begin{thm} \label{RD-final-field}
每一个对 $\Omega$ 可接受的阿基米德有序域都是~$\RD$ 的一个子域。
\end{thm}

\begin{proof}
  令 $F$ 为一个阿基米德有序域。对于每一个 $x : F$ 定义 $L_x, U_x : \Q \to
  \Omega$ 如下
  %
  \begin{equation*}
    L_x(q) \defeq (q < x)
    \qquad\text{并且}\qquad
    U_x(q) \defeq (x < q)。
  \end{equation*}
  %
  (我们刚刚使用了 $F$ 是对 $\Omega$ 可接受的这个假设。)
  然后 $(L_x, U_x)$ 是一个 Dedekind 分割。事实上,这些分割是非空且圆整的,因为
  $F$ 是阿基米德的,$<$ 是传递的,不相交是因为 $<$ 是反自反的,并且
  确定性是因为 $<$ 是弱线性顺序。令 $e : F \to \RD$ 为映射 $e(x) \defeq (L_x,
  U_x)$。

  我们声称 $e$ 是一个保持并反映顺序的域嵌入。首先,
  注意 $e(q) = q$ 对于一个有理数 $q$。其次,我们有等价关系,
  对于所有 $x, y : F$,
  %
  \begin{narrowmultline*}
    x < y \Leftrightarrow
    (\exis{q : \Q} x < q < y) \Leftrightarrow \narrowbreak
    (\exis{q : \Q} U_x(q) \land L_y(q)) \Leftrightarrow
    e(x) < e(y),
  \end{narrowmultline*}
  %
  因此 $e$ 确实保持并反映顺序。$e(x + y) = e(x) + e(y)$ 成立
  是因为对于所有 $q : \Q$,
  %
  \begin{equation*}
    q < x + y \Leftrightarrow
    \exis{r, s : \Q} r < x \land s < y \land q = r + s。
  \end{equation*}
  %
  从右到左的蕴含是显然的。对于另一方向,如果 $q < x +
  y$,则仅仅存在 $r : \Q$ 使得 $q - y < r < x$,通过取 $s \defeq
  q - r$ 我们得到所需的 $r$ 和 $s$。我们将保留 $e$ 的乘法的证明
  作为练习。
\end{proof}

为了证明 $\RD$ 上的 Dedekind 分割不会给我们带来任何新内容,我们需要再证明一个引理。

\begin{lem} \label{lem:cuts-preserve-admissibility}
如果 $F$ 对 $\Omega$ 是可接受的,那么它的 Dedekind 完备也是如此。
\index{completion!Dedekind (完备!Dedekind)}%
\end{lem}

\begin{proof}
  令 $\bar{F}$ 为 $F$ 的 Dedekind 完备。$\bar{F}$ 上的严格顺序
  定义为
  %
  \begin{equation*}
    ((L,U) < (L',U')) \defeq \exis{q : \Q} U(q) \land L'(q)。
  \end{equation*}
  %
  由于 $U(q)$ 和 $L'(q)$ 是 $\Omega$ 的元素,该引理成立只要 $\Omega$
  在合取和可数存在下是封闭的,这是我们一开始就假设的。
\end{proof}


\begin{cor} \label{RD-dedekind-complete}
%
\indexdef{complete!ordered field, Dedekind (完备!有序域, Dedekind)}%
\indexdef{Dedekind!completeness (Dedekind!完备性)}%
Dedekind 实数是 Dedekind 完备的:对于每一个实数值的 Dedekind 分割 $(L, U)$,
存在唯一的 $x : \RD$ 使得 $L(y) = (y < x)$ 并且 $U(y) = (x < y)$ 对于所有 $y :
\RD$ 成立。
\end{cor}

\begin{proof}
  通过\cref{lem:cuts-preserve-admissibility},$\RD$ 的 Dedekind 完备 $\barRD$
  对 $\Omega$ 是可接受的,因此通过\cref{RD-final-field} 我们有一个嵌入 $\barRD
  \to \RD$,以及一个嵌入 $\RD \to \barRD$。但这些嵌入必须是
  同构,因为它们的组合是保持顺序的域同态\index{homomorphism!field (同态!域)},它们
  固定了稠密子域~$\Q$,这意味着它们是恒等映射。推论现在
  立即得出,因为 $\barRD \to \RD$ 是一个同构。
\end{proof}

\index{real numbers!Dedekind|) (实数!Dedekind|)}%

\section{柯西实数 (Cauchy Reals)}
\label{sec:cauchy-reals}

\index{实数!柯西|(}%
\index{完备化!柯西|(}%
\indexsee{柯西!完备化}{完备化, 柯西}%
柯西实数是通过极限方式构造 \Q 的完备化 (completion)。在经典的柯西实数构造中,我们考虑所有柯西序列 (Cauchy sequences) 的集合 $\mathcal{C}$,然后形成适当的商 $\mathcal{C}/{\approx}$。为了证明 $\mathcal{C}/{\approx}$ 是柯西完备的,我们考虑一个柯西序列 $x : \N \to \mathcal{C}/{\approx}$,将其提升为序列 $\bar{x} : \N \to \mathcal{C}$,并使用 $\bar{x}$ 构造 $x$ 的极限。然而,将 $x$ 提升为 $\bar{x}$ 的过程使用了可数选择公理 (axiom of countable choice) 或排中律 (law of excluded middle),我们可能希望避免这种情况。
\indexdef{公理!选择!可数}%
任何最后一步是商的实数构造都存在这个问题。构造性数学中有三种常见的解决方案:
\index{数学!构造性}%
\begin{enumerate}
  \item 假装实数是一个集合 $(\mathcal{C}, {\approx})$,即附带有重合 (coincidence) 关系的柯西序列 $\mathcal{C}$。实数序列就简化为表示它们的柯西序列。
  \item 接受可数选择公理的诱惑。毕竟,该公理在大多数基于计算观点的构造性数学模型中是有效的,比如可实现性模型。
  \item 宣布柯西实数不够完美,转而构造 Dedekind 实数。在某些上下文中,这样的判断是完全有效的,比如在 sheaf-theoretic 模型的构造性数学中。然而,正如我们在 \cref{sec:dedekind-reals} 中看到的,构造性 Dedekind 实数也有自己的问题。
\end{enumerate}

然而,使用更高阶归纳类型 (higher inductive types),有第四种解决方案,这种方法优于以上任一种,甚至对经典数学家也是有趣的。其核心思想是柯西实数应该是 \Q 生成的 \emph{自由完备度量空间} (free complete metric space)。通常,构造自由结构需要多次将结构操作应用于生成元。例如,自由群上的元素不仅仅是元素 $X$ 的二元乘积和逆元,还需要通过迭代乘积和逆元的构造形成单词。因此,我们可能自然地期望同样的过程适用于柯西完备化,相关的“操作”是“取柯西序列的极限”。(在这种情况下,即使在无穷次操作后仍然会有新的柯西序列,因此迭代必须是超限的。)

上述论证表明,如果排中律或可数选择成立,那么柯西完备化是非常特殊的:在构造空间的完备化时,只需要在“一步”操作后停止。这可以类比于自由单子和自由群可以通过(化简的)单词给出明确描述。然而,我们在 \cref{sec:free-algebras} 中看到,更高阶归纳类型允许我们\emph{直接}构造自由结构,而不管是否存在明确的描述。在本节中,我们将展示对于柯西实数也是如此(类似技术可以构造任何度量空间的柯西完备化;参见 \cref{ex:metric-completion})。具体而言,更高阶归纳类型允许我们\emph{同时}添加柯西序列的极限并商取重合关系,从而可以避免将实数序列提升为代表序列的问题。
\index{完备化!柯西|)}%

\subsection{柯西实数的构造 (Construction of Cauchy Reals)}
\label{sec:constr-cauchy-reals}

作为一个更高阶归纳类型,柯西实数 $\RC$ 的构造比 \cref{sec:free-algebras} 中讨论的自由代数结构更为复杂。我们打算包括一个“取极限”的构造器,其输入是一个实数的柯西序列,但“实数柯西序列”的概念依赖于度量“距离”的方法。当然,两个实数之间的距离将是另一个实数,从而导致一个潜在的问题。

然而,我们实际上所需要的不是一般意义上的“距离”,而是表达“两个实数之间的距离小于 $\epsilon$”的方式,对于任意 $\epsilon:\Qp$。这可以通过一组二元关系 $\mathord{\close\epsilon} : \RC\to\RC\to \prop$ 来表示。$\mathord{\close\epsilon}$ 的意图是表达 $|x - y| < \epsilon$,但由于我们尚未定义减法、绝对值或不等式(毕竟,我们刚刚定义 $\RC$),我们必须在定义 $\RC$ 的同时定义这些关系 $\close\epsilon$。由于 $\close\epsilon$ 是由两个 $\RC$ 和一个 $\Qp$ 索引的类型族,我们不能使用普通的归纳(higher)定义;相反,我们必须使用 \emph{更高阶归纳归纳定义} (higher inductive-inductive definition)。
\index{归纳归纳类型!更高阶}

\cref{sec:generalizations} 中提到的普通归纳归纳定义允许我们通过同时归纳定义一个类型及其索引的类型族。当然,“更高阶”版本允许类型和族具有路径构造器以及点构造器。我们不会尝试制定任何一般的更高阶归纳归纳定义理论,但希望我们对 $\RC$ 和 $\close\epsilon$ 的描述能够使这一思想变得清晰。

\begin{rmk}
  我们可能还考虑一种 \emph{更高阶归纳递归定义},其中 $\close\epsilon$ 是使用 $\RC$ 的 \emph{递归} 原则定义的,同时与 $\RC$ 的 \emph{归纳} 定义一起进行。然而,我们选择归纳归纳路线有两个原因。首先,更高阶归纳递归定义在同伦语义中更难以证明。其次,更重要的是,归纳归纳定义产生了更强大的归纳原则,这是我们在发展柯西实数的基本理论时所需要的。
\end{rmk}

最后,正如我们在 \cref{sec:RD-cauchy-complete} 中讨论柯西完备性的 Dedekind 实数时所做的那样,我们将使用 \emph{柯西逼近} (\cref{defn:cauchy-approximation}) 而不是柯西序列。当然,我们的柯西逼近现在将由柯西实数构成,而不是 Dedekind 实数或有理数。

\begin{defn}\label{defn:cauchy-reals}
让 $\RC$ 和关系 $\closesym:\Qp \times \RC \times \RC \to \type$ 成为以下更高阶归纳归纳类型族。柯西实数的类型 $\RC$
\indexdef{实数!柯西}%
由以下构造生成:
\begin{itemize}
  \item \emph{有理点 (rational points):}
  对于任意 $q : \Q$,存在一个实数 $\rcrat(q)$。
  \index{有理数!作为柯西实数}%
  \item \emph{极限点 (limit points):}
  对于任意 $x : \Qp \to \RC$,如果满足
  %
  \begin{equation}
    \label{eq:RC-cauchy}
    \fall{\delta, \epsilon : \Qp} x_\delta \close{\delta + \epsilon} x_\epsilon
  \end{equation}
  %
  则存在一个点 $\rclim(x) : \RC$。我们称 $x$ 为一个 \define{柯西逼近 (Cauchy approximation)}。
  \indexdef{柯西!逼近}%
  \index{极限!柯西逼近}%
  %
  \item \emph{路径 (paths):}
  对于 $u, v : \RC$,如果满足
  %
  \begin{equation}
    \label{eq:RC-path}
    \fall{\epsilon : \Qp} u \close\epsilon v
  \end{equation}
  %
  则存在一个路径 $\rceq(u, v) : \id[\RC]{u}{v}$。
\end{itemize}
同时,类型族 $\closesym:\RC\to\RC\to\Qp \to\type$ 由以下构造生成。
其中 $q$ 和 $r$ 表示有理数;$\delta$、$\epsilon$ 和 $\eta$ 表示正有理数;$u$ 和 $v$ 表示柯西实数;$x$ 和 $y$ 表示柯西逼近:
\begin{itemize}
  \item 对于任意 $q,r,\epsilon$,如果 $-\epsilon < q - r < \epsilon$,则 $\rcrat(q) \close\epsilon \rcrat(r)$,
  \item 对于任意 $q,y,\epsilon,\delta$,如果 $\rcrat(q) \close{\epsilon - \delta} y_\delta$,则 $\rcrat(q) \close{\epsilon} \rclim(y)$,
  \item 对于任意 $x,r,\epsilon,\delta$,如果 $x_\delta \close{\epsilon - \delta} \rcrat(r)$,则 $\rclim(x) \close\epsilon \rcrat(r)$,
  \item 对于任意 $x,y,\epsilon,\delta,\eta$,如果 $x_\delta \close{\epsilon - \delta - \eta} y_\eta$,则 $\rclim(x) \close\epsilon \rclim(y)$,
  \item 对于任意 $u,v,\epsilon$,如果 $\xi,\zeta : u \close{\epsilon} v$,则 $\xi=\zeta$(命题截断)。
\end{itemize}
\end{defn}

\mentalpause

$\RC$ 的第一个构造器表示任何有理数都可以看作实数。第二个构造器表示从任何柯西逼近到一个实数的极限,我们可以得到一个新的实数。第三个构造器表达了如果两个柯西逼近重合,那么它们的极限相等。

$\closesym$ 的前四个构造器指定了何时两个有理数接近,何时一个有理数接近一个极限,以及何时两个极限接近。对于两个有理数的情况,这只是有理数 $\epsilon$-接近的通常概念,而其他情况可以通过注意每个逼近 $x_\delta$ 应该在极限 $\rclim(x)$ 的 $\delta$ 范围内来推导。

我们提醒自己关于证明相关的内容:一个通过 $\rclim$ 获得的实数不仅由一个柯西逼近 $x$ 表示,还需要一个证明 $p$ 证明 \eqref{eq:RC-cauchy},所以我们在技术上应该写 $\rclim(x,p)$ 而不是简单地写 $\rclim(x)$。类似地,$\rceq$ 和 \eqref{eq:RC-path} 的情况也类似,但我们将简单地写 $\rceq : u = v$ 而不是 $\rceq(u, v, p) : u = v$。这些符号的滥用通过忽略命题并省略容易猜测的信息来减轻其影响。同样,$\mathord{\close\epsilon}$ 的最后一个构造器解释了我们为何将其他四个构造器留作无名。

我们立即可以用许多实数填充 $\RC$。因为假设 $x : \N \to \Q$ 是一个传统的有理数柯西序列,并且设 $M : \Qp \to \N$ 是其收敛模数。那么 $\rcrat \circ x \circ M : \Qp \to \RC$ 是一个柯西逼近,使用 $\closesym$ 的第一个构造器产生所需的证明。因此,$\rclim(\rcrat \circ x \circ m)$ 是一个实数。各种著名的实数如 $\sqrt{2}$、$\pi$、$e$ 等都是这种有理数柯西序列的极限。

\subsection{柯西实数的归纳和递归 (Induction and Recursion on Cauchy Reals)}
\label{sec:induct-recurs-cauchy}

当然,要对 $\RC$ 做有用的事情,我们需要给出其归纳原则。每当我们归纳定义两个或多个对象时,$\RC$ 和 $\closesym$ 的基本归纳原则都需要同时对两者进行归纳。因此,我们应期望其说的是,假设有两个类型族分别位于 $\RC$ 和 $\closesym$ 之上,并具有与每个构造器对应的数据,那么就存在这两个类型族的截面函数。然而,由于 $\closesym$ 依赖于两个 $\RC$ 和一个 $\Qp$,因此这些族的确切依赖关系有点微妙。归纳原则将适用于任意一对类型族:
\begin{align*}
  A&:\RC\to\type\\
  B&:\prd{x,y:\RC} A(x) \to A(y) \to \prd{\epsilon:\Qp} (x\close\epsilon y) \to \type.
\end{align*}
$A$ 的类型显而易见,但 $B$ 的类型需要仔细考虑。由于 $B$ 必须依赖于 $\closesym$,但 $\closesym$ 依赖于两个 $\RC$ 和一个 $\Qp$,所以很明显 $B$ 也必须依赖于变量 $x,y:\RC$ 和 $\epsilon:\Qp$ 以及 $(x\close\epsilon y)$ 的一个元素。稍微不太明显的是,$B$ 还必须依赖于 $A(x)$ 和 $A(y)$。

如果我们考虑非依赖型情况(递归原则),可能更容易理解,这时 $A$ 是一个简单类型(而不是类型族)。在这种情况下,我们期望 $B$ 不依赖于 $x,y:\RC$ 或 $(x\close\epsilon y)$。但是,递归原则(及其相关的唯一性原则)应该表明 $\RC$ 和 $\close\epsilon$ 是某种范畴中的“初始对象”,因此在这种情况下 $A$ 和 $B$ 的依赖结构应与 $\RC$ 和 $\close\epsilon$ 的依赖结构一致:即我们应该有 $B:A\to A\to \Qp \to \type$。结合这一观察,并考虑到在依赖型情况下 $B$ 必须依赖于 $x,y:\RC$ 和 $(x\close\epsilon y)$,最终导致了上述 $B$ 的类型。

\symlabel{RC-recursion}
将 $B$ 看作 $\epsilon$ 索引的关系族是有帮助的,这些关系位于类型 $A(x)$ 和 $A(y)$ 之间。这样,我们可以将 $B(x,y,a,b,\epsilon,\xi)$ 写为 $(x,a) \bsim_\epsilon^\xi (y,b)$。由于 $\xi:x\close\epsilon y$ 是唯一的(如果存在),我们通常省略 $\xi$,写作 $(x,a) \bsim_\epsilon (y,b)$;只要我们记住这一关系仅在 $x\close\epsilon y$ 时定义,这样的省略是无害的。我们有时还可以进一步简化,写作 $a\bsim_\epsilon b$,其中 $x$ 和 $y$ 从 $a$ 和 $b$ 的类型中推断,但有时为了清晰起见需要包含它们。

\index{归纳原则!用于柯西实数}%
现在,给定一个类型族 $A:\RC\to\type$ 和一个关系族 $\bsim$,归纳原则的假设包括以下数据,每个对应 $\RC$ 或 $\closesym$ 的一个构造器:
\begin{itemize}
  \item 对于任意 $q : \Q$,有一个元素 $f_q:A(\rcrat(q))$。
  \item 对于任意柯西逼近 $x$,以及任何 $a:\prd{\epsilon:\Qp} A(x_\epsilon)$,满足
  \begin{equation}
    \fall{\delta, \epsilon : \Qp}
    (x_\delta,a_\delta) \bsim_{\delta+\epsilon} (x_\epsilon,a_\epsilon),
    \label{eq:depCauchyappx}
  \end{equation}
  的元素 $f_{x,a}:A(\rclim(x))$。我们称这样的 $a$ 为 \define{依赖柯西逼近 (dependent Cauchy approximation)}。
  \indexdef{柯西!逼近!依赖}%
  \indexsee{逼近, 柯西}{柯西逼近}%
  \indexdef{依赖!柯西逼近}%
  在 $x$ 上。
  \item 对于 $u, v : \RC$,如果 $h:\fall{\epsilon : \Qp} u \close\epsilon v$,以及所有 $a:A(u)$ 和 $b:A(v)$ 满足 $\fall{\epsilon:\Qp} (u,a) \bsim_\epsilon (v,b)$,则有 $\dpath{A}{\rceq(u,v)}{a}{b}$。
  \item 对于 $q,r:\Q$ 和 $\epsilon:\Qp$,如果 $-\epsilon < q - r < \epsilon$,则
  \narrowequation{(\rcrat(q),f_q) \bsim_\epsilon (\rcrat(r),f_r)。}
  \item 对于 $q:\Q$ 和 $\delta,\epsilon:\Qp$ 以及 $y$ 柯西逼近,和 $b$ 依赖柯西逼近 $y$,如果 $\rcrat(q) \close{\epsilon - \delta} y_\delta$,那么
  \[(\rcrat(q),f_q) \bsim_{\epsilon-\delta} (y_\delta,b_\delta)
  \;\Rightarrow\;
  (\rcrat(q),f_q) \bsim_\epsilon (\rclim(y),f_{y,b})。\]
  \item 类似地,对于 $r:\Q$ 和 $\delta,\epsilon:\Qp$ 和 $x$ 柯西逼近,和 $a$ 依赖柯西逼近 $x$,如果 $x_\delta \close{\epsilon - \delta} \rcrat(r)$,那么
  \[(x_\delta,a_\delta) \bsim_{\epsilon-\delta} (\rcrat(r),f_r)
  \;\Rightarrow\;
  (\rclim(x),f_{x,a}) \bsim_\epsilon (\rcrat(q),f_r)。
  \]
  \item 对于 $\epsilon,\delta,\eta:\Qp$ 和 $x,y$ 柯西逼近,和 $a$ 和 $b$ 依赖柯西逼近 $x$ 和 $y$,如果我们有 $x_\delta \close{\epsilon - \delta - \eta} y_\eta$,那么
  \[ (x_\delta,a_\delta) \bsim_{\epsilon - \delta - \eta} (y_\eta,b_\eta)
  \;\Rightarrow\;
  (\rclim(x),f_{x,a}) \bsim_\epsilon (\rclim(y),f_{y,b})。\]
  \item 对于 $\epsilon:\Qp$ 和 $x,y:\RC$ 和 $\xi,\zeta:x\close{\epsilon} y$,和 $a:A(x)$ 和 $b:A(y)$,$(x,a) \bsim_\epsilon^\xi (y,b)$ 和 $(x,a) \bsim_\epsilon^\zeta (y,b)$ 中的任意两个元素在 $\xi=\zeta$ 上依赖相等。注意,这与要求 $\bsim$ 取值于单纯命题是等价的。
\end{itemize}
在这些假设下,我们得到函数
\begin{align*}
  f&:\prd{x:\RC} A(x)\\
  g&:\prd{x,y:\RC}{\epsilon:\Qp}{\xi:x\close{\epsilon} y}
  (x,f(x)) \bsim_\epsilon^\xi (y,f(y))
\end{align*}
其计算如预期:
\begin{align}
  f(\rcrat(q)) &\defeq f_q, \label{eq:rcsimind1}\\
  f(\rclim(x)) &\defeq f_{x,(f,g)[x]}。 \label{eq:rcsimind2}
\end{align}
这里 $(f,g)[x]$ 表示将 $f$ 和 $g$ 应用于柯西逼近 $x$ 以获得 $x$ 上的依赖柯西逼近的结果。也就是说,我们定义 $(f,g)[x]_\epsilon \defeq f(x_\epsilon) : A(x_\epsilon)$,然后对于任何 $\epsilon,\delta:\Qp$,我们有 $g(x_\epsilon,x_\delta,\epsilon+\delta,\xi)$ 来证明 $(f,g)[x]$ 是依赖柯西逼近,其中 $\xi: x_\epsilon \close{\epsilon+\delta} x_\delta$ 来源于 $x$ 是柯西逼近的假设。

我们不会再使用这种符号表示,所以不用记住它。通常我们使用模式匹配约定,其中 $f$ 由类似于 \eqref{eq:rcsimind1} 和 \eqref{eq:rcsimind2} 的方程定义,其中 \eqref{eq:rcsimind2} 的右侧可能涉及符号 $f(x_\epsilon)$ 和假设它们形成依赖柯西逼近。

然而,这个归纳原则确实仍然相当复杂。为了帮助理解它,我们观察到它包含了两个关于 $\RC$ 和 $\closesym$ 的单独的归纳原则的特殊情况。首先,假设只给定一个类型族 $A:\RC\to\type$,并将 $\bsim$ 定义为常量 \unit。那么许多必要的数据就变得微不足道,我们剩下:
\begin{itemize}
  \item 对于任意 $q : \Q$,一个元素 $f_q:A(\rcrat(q))$,
  \item 对于任意柯西逼近 $x$,和任意 $a:\prd{\epsilon:\Qp} A(x_\epsilon)$,一个元素 $f_{x,a}:A(\rclim(x))$,
  \item 对于 $u, v : \RC$ 和 $h:\fall{\epsilon : \Qp} u \close\epsilon v$,以及 $a:A(u)$ 和 $b:A(v)$,我们有 $\dpath{A}{\rceq(u,v)}{a}{b}$。
\end{itemize}
在这些数据的基础上,归纳原则生成一个函数 $f:\prd{x:\RC} A(x)$,使得
\begin{align*}
  f(\rcrat(q)) &\defeq f_q,\\
  f(\rclim(x)) &\defeq f_{x,f(x)}。
\end{align*}
我们称此原则为 \define{$\RC$-归纳 ($\RC$-induction)};它基本上表明,如果我们认为 $\close\epsilon$ 是已知的,那么 $\RC$ 是通过其构造器归纳生成的。

注意,如果 $A$ 是单纯命题,那么 $\RC$-归纳中的第三个假设是自动成立的(我们将很快看到这些是等价的命题)。因此,我们可以通过简单地证明对有理数和柯西逼近的极限来证明实数的单纯命题。以下是一个例子。

\begin{lem} \label{lem:close-reflexive}
对于任意 $u:\RC$ 和 $\epsilon:\Qp$,我们有 $u\close\epsilon u$。
\end{lem}
\begin{proof}
  定义 $A(u) \defeq \fall{\epsilon:\Qp} (u\close\epsilon u)$。由于这是一个单纯命题(根据 $\closesym$ 的最后一个构造器),根据 $\RC$-归纳,我们只需要证明当 $u$ 是 $\rcrat(q)$ 和 $u$ 是 $\rclim(x)$ 时的情况。在第一个情况下,我们显然有 $|q-q|<\epsilon$ 对于任意 $\epsilon$,因此根据 $\closesym$ 的第一个构造器,$\rcrat(q) \close\epsilon \rcrat(q)$ 是成立的。
  %
  在第二种情况下,我们可以假设归纳地 $x_\delta \close\epsilon x_\delta$ 对于所有 $\delta,\epsilon:\Qp$ 成立。然后特别地,我们有 $x_{\epsilon/3} \close{\epsilon/3} x_{\epsilon/3}$,因此根据 $\closesym$ 的第四个构造器,$\rclim(x) \close{\epsilon} \rclim(x)$ 成立。
\end{proof}

从 \cref{lem:close-reflexive},我们推断,如果类型族 $A:\RC\to\type$ 是一个单纯命题,那么直接应用 $\RC$-归纳只有在这种情况下才有可能成功。为了证明这一点,固定 $u:\RC$。将 $v$ 设为 $u$,$\RC$-归纳的第三个假设告诉我们,对于任意 $a : A(u)$,我们有 $\dpath{A}{\rceq(u,u)}{a}{a}$。再给定一个点 $b : A(u)$,我们也得到 $\dpath{A}{\rceq(u,u)}{a}{b}$。根据依赖路径类型的定义,我们得出这些路径组合的结论是 $a = b$,即 $A(u)$ 中的所有点都是相等的。

\begin{thm}\label{thm:Cauchy-reals-are-a-set}
$\RC$ 是一个集合 (set)。
\end{thm}
\begin{proof}
  我们刚刚证明了
  \narrowequation{P(u,v) \defeq \fall{\epsilon:\Qp} (u\close\epsilon v)}
  是自反的。由于它暗含了恒等性,$\RC$ 的路径构造器表明结果可由 \cref{thm:h-set-refrel-in-paths-sets} 推导。
\end{proof}

我们还可以证明,尽管 $\RC$ 可能不是有理数柯西序列集合的商,但它仍然是实数柯西序列集合的商。(当然,这不是 $\RC$ 的有效\emph{定义},但它是一个有用的性质。)我们将柯西逼近的类型定义为
%
\symlabel{cauchy-approximations}%
\index{柯西!逼近!类型}%
\begin{equation*}
  \CAP \defeq
  \setof{ x : \Qp \to \RC |
  \fall{\epsilon, \delta : \Qp} x_\delta \close{\delta + \epsilon} x_\epsilon
  }.
\end{equation*}
$\RC$ 的第二个构造器提供了函数 $\rclim:\CAP\to\RC$。

\begin{lem} \label{RC-lim-onto}
每个实数 (real) 仅仅是一个极限点:$\fall{u : \RC} \exis{x : \CAP} u = \rclim(x)$。换句话说,$\rclim:\CAP\to\RC$ 是满射。
\end{lem}
\begin{proof}
  通过 $\RC$-归纳,我们可以对 $u$ 进行分情况讨论。当然,如果 $u$ 是一个极限 $\rclim(x)$,那么该命题是显然成立的。所以假设 $u$ 是一个有理点 $\rcrat(q)$;我们声称 $u$ 等于 $\rclim(\lam{\epsilon} \rcrat(q))$。通过 $\RC$ 的路径构造器,只需证明 $\rcrat(q) \close\epsilon \rclim(\lam{\epsilon} \rcrat(q))$ 对所有 $\epsilon:\Qp$ 成立。而根据 $\closesym$ 的第二个构造器,为此只需找到 $\delta:\Qp$ 使得 $\rcrat(q)\close{\epsilon-\delta} \rcrat(q)$。但根据 $\closesym$ 的第一个构造器,我们可以取任意 $\delta:\Qp$ 使得 $\delta<\epsilon$。
\end{proof}


%

\begin{lem} \label{RC-lim-factor}
如果 $A$ 是一个集合并且 $f : \CAP \to A$ 满足Cauchy近似的重合性\index{coincidence!of Cauchy approximations},即
%
\begin{equation*}
  \fall{x, y : \CAP} \rclim(x) = \rclim(y) \Rightarrow f(x) = f(y),
\end{equation*}
%
那么 $f$ 唯一地通过 $\rclim : \CAP \to \RC$ 因子化。
\end{lem}
\begin{proof}
  由于 $\rclim$ 是满射,根据 \cref{lem:images_are_coequalizers},$\RC$ 是 $\CAP$ 通过 $\rclim$ 的核对偶\index{kernel!pair} 的商集。
  这正是引理的表述。
\end{proof}

对于归纳原理的第二个特例,假设我们将 $A$ 取为常量 $\unit$。
在这种情况下,$\bsim$ 只是对 $\epsilon$ 近似对实数对的 $\epsilon$ 索引的关系族,因此我们可以写 $u\bsim_\epsilon v$ 代替 $(u,\ttt)\bsim_\epsilon (v,\ttt)$。
那么所需的数据简化为如下内容,其中 $q, r$ 表示有理数,$\epsilon, \delta, \eta$ 是正有理数,而 $x, y$ 是 Cauchy 近似:
\begin{itemize}
  \item 如果 $-\epsilon < q - r < \epsilon$,那么
  $\rcrat(q) \bsim_\epsilon \rcrat(r)$,
  \item 如果 $\rcrat(q) \close{\epsilon - \delta} y_\delta$ 并且
  $\rcrat(q)\bsim_{\epsilon-\delta} y_\delta$,
  那么 $\rcrat(q) \bsim_\epsilon \rclim(y)$,
  \item 如果 $x_\delta \close{\epsilon - \delta} \rcrat(r)$ 并且
  $x_\delta \bsim_{\epsilon-\delta} \rcrat(r)$,
  那么 $\rclim(y) \bsim_\epsilon \rcrat(q)$,
  \item 如果 $x_\delta \close{\epsilon - \delta - \eta} y_\eta$ 并且
  $x_\delta\bsim_{\epsilon - \delta - \eta} y_\eta$,
  那么 $\rclim(x) \bsim_\epsilon \rclim(y)$。
\end{itemize}
由此得出的结论是 $\fall{u,v:\RC}{\epsilon:\Qp} (u\close\epsilon v) \to (u \bsim_\epsilon v)$。
我们称这个原理为\define{$\closesym$-归纳};它本质上表明,如果我们将 $\RC$ 视为已知,那么 $\close\epsilon$ 由\emph{它的}构造函数归纳生成(作为类型族)。
例如,我们可以使用它来证明 $\closesym$ 是对称的。

\begin{lem}\label{thm:RCsim-symmetric}
对于任意的 $u,v:\RC$ 和 $\epsilon:\Qp$,我们有 $(u\close\epsilon v) = (v\close\epsilon u)$。
\end{lem}
\begin{proof}
  由于两者都是单纯命题,通过对称性我们只需证明一个方向的蕴涵。
  因此,令 $(u\bsim_\epsilon v) \defeq (v\close\epsilon u)$。
  通过 $\closesym$-归纳,我们可以将问题归结为 $u\close\epsilon v$ 由 $\closesym$ 的四个有趣构造函数之一导出。
  在第一种情况下,当 $u$ 和 $v$ 都是有理数时,结果是显然的(我们可以再次应用第一个构造函数)。
  在其他三种情况下,归纳假设(以及 $\Q$ 中加法的交换性)正好得出 $\closesym$ 的另一个构造函数的输入(第二个和第三个构造函数互换,而第四个保持不变)。
\end{proof}

一般归纳原理,我们可以称之为\define{$(\RC,\closesym)$-归纳},因此是一种联合的 $\RC$-归纳和 $\closesym$-归纳。
例如,考虑它的非依赖版本,我们称之为\define{$(\RC,\closesym)$-递归},这是我们最常用的。
\index{recursion principle!for Cauchy reals}%
普通的 $\RC$-递归告诉我们,要定义一个函数 $f : \RC \to A$,只需:
\begin{enumerate}
  \item 对每个 $q : \Q$ 构造 $f(\rcrat(q)) : A$,
  \item 对每个 Cauchy 近似 $x : \Qp \to \RC$,构造 $f(x) : A$,
  假设已为所有 $\epsilon : \Qp$ 定义了 $f(x_\epsilon)$,
  \item 证明对于所有满足 $\fall{\epsilon:\Qp} u\close\epsilon v$ 的 $u, v : \RC$,有 $f(u) = f(v)$。\label{item:rcrec3}
\end{enumerate}
然而,在不了解 $f$ 如何作用于 $\epsilon$ 近似的 Cauchy 实数的情况下,通常很难证明~\ref{item:rcrec3}。
$(\RC,\closesym)$-递归的增强原理弥补了这一缺陷,使我们能够指定一个 \emph{任意的} ``$f$ 作用于 $\epsilon$ 近似的 Cauchy 实数的方式'',然后我们可以通过与 $f$ 的定义同时归纳来证明这一点。
这就是关系族 $\bsim$。
由于 $A$ 独立于 $\RC$,为简单起见,我们可以假设 $\bsim$ 仅依赖于 $A$ 和 $\Qp$,因此在写 $a\bsim_\epsilon b$ 时没有歧义,而不是 $(u,a) \bsim_\epsilon (v,b)$。
在这种情况下,通过 $(\RC,\closesym)$-递归定义函数 $f:\RC\to A$ 需要以下情况(我们现在使用模式匹配约定来写)。
\begin{itemize}
  \item 对于每个 $q : \Q$,构造 $f(\rcrat(q)) : A$。
  \item 对于每个 Cauchy 近似 $x : \Qp \to \RC$,构造 $f(\rclim(x)) : A$,假设归纳地已经为所有 $\epsilon : \Qp$ 定义了 $f(x_\epsilon)$ 并形成一个 ``相对于 $\bsim$ 的 Cauchy 近似'',即 $\fall{\epsilon,\delta:\Qp} (f(x_\epsilon) \bsim_{\epsilon+\delta} f(x_\delta))$。
  \item 证明这些关系 $\bsim$ 是\emph{分离的},即对于任意 $a,b:A$,
  \indexdef{relation!separated family of}%
  \indexdef{separated family of relations}%
  \narrowequation{(\fall{\epsilon:\Qp} a\bsim_\epsilon b) \Rightarrow (a=b)。}
  \item 证明如果 $-\epsilon< q-r <\epsilon$ 对于 $q,r:\Q$,那么 $f(\rcrat(q))\bsim_\epsilon f(\rcrat(r))$。
  \item 对于任意 $q:\Q$ 和任意 Cauchy 近似 $y$,证明
  \narrowequation{f(\rcrat(q)) \bsim_\epsilon f(\rclim(y)),} 假设归纳地 $\rcrat(q)\close{\epsilon-\delta} y_\delta$ 并且 $f(\rcrat(q)) \bsim_{\epsilon-\delta} f(y_\delta)$ 对于某个 $\delta:\Qp$,并且 $\eta \mapsto f(x_\eta)$ 是相对于 $\bsim$ 的 Cauchy 近似。
  \item 对于任意 Cauchy 近似 $x$ 和任意 $r:\Q$,证明
  \narrowequation{f(\rclim(x)) \bsim_\epsilon f(\rcrat(r)),}
  假设归纳地 $x_\delta \close{\epsilon-\delta} \rcrat(r)$ 并且 $f(x_\delta) \bsim_{\epsilon-\delta} f(\rcrat(r))$ 对于某个 $\delta:\Qp$,并且 $\eta\mapsto f(x_\eta)$ 是相对于 $\bsim$ 的 Cauchy 近似。
  \item 对于任意 Cauchy 近似 $x,y$,证明
  \narrowequation{f(\rclim(x)) \bsim_\epsilon f(\rclim(y)),}
  假设归纳地 $x_\delta \close{\epsilon-\delta-\eta} y_\eta$ 并且 $f(x_\delta) \bsim_{\epsilon-\delta-\eta} f(y_\eta)$ 对于某个 $\delta,\eta:\Qp$,并且 $\theta\mapsto f(x_\theta)$ 和 $\theta\mapsto f(y_\theta)$ 是相对于 $\bsim$ 的 Cauchy 近似。
\end{itemize}
注意,在最后四个证明中,我们可以自由使用在前两个数据中给出的 $f(\rcrat(q))$ 和 $f(\rclim(x))$ 的具体定义。
然而,分离性的证明必须适用于 $A$ 的\emph{任意} 两个元素,而与 $f$ 无关:它是一种对关系族 $\bsim$ 的``可接纳性''条件。
因此,我们通常首先验证它,在定义 $\bsim$ 之后立即进行,然后再继续定义 $f(\rcrat(q))$ 和 $f(\rclim(x))$。

在上述假设下,$(\RC,\closesym)$-递归产生一个函数 $f:\RC\to A$,使得 $f(\rcrat(q))$ 和 $f(\rclim(x))$ 在判断上等于在前两个子句中给出的定义。
此外,我们还可以得出
\begin{equation}
  \fall{u,v:\RC}{\epsilon:\Qp} (u\close\epsilon v) \to (f(u) \bsim_\epsilon f(v)).\label{eq:RC-sim-recursion-extra}
\end{equation}

作为一个典型的例子,$(\RC,\closesym)$-递归允许我们将定义在 $\Q$ 上的函数扩展到所有 $\RC$ 上,只要它们足够连续。
\index{function!continuous}%

\begin{defn}\label{defn:lipschitz}
函数 $f:\Q\to\RC$ 是 \define{Lipschitz}
\indexdef{function!Lipschitz}%
\indexdef{Lipschitz!function}%
\indexdef{Lipschitz!constant}%
\indexdef{constant!Lipschitz}%
的,如果存在 $L:\Qp$(\define{Lipschitz常数}),使得
\[ |q - r|<\epsilon \Rightarrow (f(q) \close{L\epsilon} f(r)) \]
对于所有 $\epsilon:\Qp$ 和 $q,r:\Q$。
%
类似地,$g:\RC\to\RC$ 是 \define{Lipschitz} 的,如果存在 $L:\Qp$ 使得
\[ (u\close\epsilon v) \Rightarrow (g(u) \close{L\epsilon} g(v)) \]
对于所有 $\epsilon:\Qp$ 和 $u,v:\RC$。
\end{defn}

特别地,注意到通过 $\closesym$ 的第一个构造函数,如果 $f:\Q\to\Q$ 在显然的意义上是 Lipschitz 的,那么复合函数 $\Q\xrightarrow{f} \Q \to \RC$ 也是如此。

\begin{lem}\label{RC-extend-Q-Lipschitz}
假设 $f : \Q \to \RC$ 是常数为 $L : \Qp$ 的 Lipschitz 函数。
那么存在一个 Lipschitz 映射 $\bar{f} : \RC \to \RC$,其常数也是 $L$,并且对于所有 $q:\Q$,有 $\bar{f}(\rcrat(q)) \jdeq f(q)$。
\end{lem}

\begin{proof}
    % 唯一性直接由 \cref{RC-continuous-eq} 推出。
  我们通过 $(\RC,\closesym)$-递归定义 $\bar{f}$,其余象为 $A\defeq \RC$。
  我们定义关系 $\mathord{\bsim}: \RC \to \RC \to \Qp \to \prop$ 为
  \begin{align*}
  (u \bsim_\epsilon v) &\defeq (u \close{L\epsilon} v)。
  \end{align*}
  对于 $q : \Q$,我们定义
  %
  \begin{equation*}
    \bar{f}(\rcrat(q)) \defeq \rcrat(f(q))。
  \end{equation*}
  %
  对于 Cauchy 近似 $x : \Qp \to \RC$,我们定义
  %
  \begin{equation*}
    \bar{f}(\rclim(x)) \defeq \rclim(\lamu{\epsilon : \Qp} \bar{f}(x_{\epsilon/L}))。
  \end{equation*}
  %
  为了使这有意义,我们必须验证 $y \defeq \lamu{\epsilon : \Qp} \bar{f}(x_{\epsilon/L})$ 是一个 Cauchy 近似。
  然而,这一步的归纳假设是,对于任何 $\delta,\epsilon:\Qp$,我们有 $\bar{f}(x_\delta) \bsim_{\delta+\epsilon} \bar{f}(x_\epsilon)$,即 $\bar{f}(x_\delta) \close{L\delta+L\epsilon} \bar{f}(x_\epsilon)$。
  因此我们有
  \[y_\delta \jdeq f(x_{\delta/L}) \close{\delta + \epsilon} f(x_{\epsilon/L})   \jdeq y_\epsilon。 \]

  为了证明分离性,我们简单地观察到 $\fall{\epsilon:\Qp} a\bsim_\epsilon b$ 意味着 $\fall{\epsilon:\Qp} a\close{L\epsilon} b$,这意味着 $\fall{\epsilon:\Qp}a\close\epsilon b$,从而 $a=b$。

  为了完成 $(\RC,\closesym)$-递归,还需要验证 $\bsim$ 上的四个条件。
  这基本上相当于证明 $\bar f$ 对 $\closesym$ 的所有四个构造函数都是 Lipschitz 的。
  \begin{enumerate}
    \item 当 $u$ 是 $\rcrat(q)$ 并且 $v$ 是 $\rcrat(r)$ 时,如果 $-\epsilon < |q-r| <\epsilon$,假设 $f$ 是 Lipschitz 的得出 $f(q) \close{L\epsilon} f(r)$,因此 $\bar{f}(\rcrat(q)) \bsim_\epsilon \bar{f}(\rcrat(r))$ 通过定义。
    \item 当 $u$ 是 $\rclim(x)$ 并且 $v$ 是 $\rcrat(q)$,如果 $x_\eta \close{\epsilon - \eta} \rcrat(q)$,那么归纳假设是 $\bar{f}(x_\eta) \close{L \epsilon - L \eta} \rcrat(f(q))$,这证明了
    \narrowequation{\bar{f}(\rclim(x)) \close{L \epsilon} \bar{f}(\rcrat(q))。}
    \item 当 $u$ 是有理数而 $v$ 是极限时,对称情况基本相同。
    \item 当 $u$ 是 $\rclim(x)$ 并且 $v$ 是 $\rclim(y)$,其中 $\delta, \eta : \Qp$ 使得 $x_\delta \close{\epsilon - \delta - \eta} y_\eta$,
    归纳假设是 $\bar{f}(x_\delta) \close{L \epsilon - L \delta - L \eta} \bar{f}(y_\eta)$,这证明了 $\bar{f}(\rclim(x)) \close{L
    \epsilon} \bar{f}(\rclim(y))$ 通过 $\closesym$ 的第四个构造函数。
  \end{enumerate}
  这完成了 $(\RC,\closesym)$-递归,从而完成了 $\bar f$ 的构造。
  所需的等式 $\bar f(\rcrat(q))\jdeq f(q)$ 正是 $(\RC,\closesym)$-递归的第一个计算规则,附加条件~\eqref{eq:RC-sim-recursion-extra} 正是表明 $\bar f$ 是常数为 $L$ 的 Lipschitz 的。
\end{proof}

在这一点上,如果没有对 $\closesym$ 的更好表征,我们已经做到了尽可能多的事情。
我们在 $\closesym$ 的构造函数中指定了我们希望何种形式的 Cauchy 实数在 $\epsilon$-近似时是 $\epsilon$-close。
然而,我们如何知道在所生成的归纳-归纳类型族中,这些是 \emph{唯一的} 证明此事实的见证者?
我们已经看到,归纳类型族(如恒等类型,参见 \cref{sec:identity-systems})和更高的归纳类型有包含 ``超出它们的内容'' 的倾向,因此这不是一个空洞的问题。

为了更精确地表征 $\closesym$,我们将定义一个关系族 $\approx_\epsilon$ 在 $\RC$ 上\emph{递归}地计算,使其能够在构造函数上计算,并证明这个关系族等价于 $\close\epsilon$。

\begin{thm}\label{defn:RC-approx}
存在一个关系族 $\mathord\approx:\RC\to\RC\to\Qp\to\prop$,使得
\begin{align}
(\rcrat(q) \approx_\epsilon \rcrat(r))  &\defeq
(-\epsilon < q - r < \epsilon)\label{eq:RCappx1}\\
(\rcrat(q) \approx_\epsilon \rclim(y)) &\defeq
\exis{\delta : \Qp} \rcrat(q) \approx_{\epsilon - \delta} y_\delta\label{eq:RCappx2}\\
(\rclim(x) \approx_\epsilon \rcrat(r)) &\defeq
\exis{\delta : \Qp} x_\delta \approx_{\epsilon - \delta} \rcrat(r)\label{eq:RCappx3}\\
(\rclim(x) \approx_\epsilon \rclim(y)) &\defeq
\exis{\delta, \eta : \Qp} x_\delta \approx_{\epsilon - \delta - \eta} y_\eta。\label{eq:RCappx4}
\end{align}
此外,我们有
\begin{gather}
(u \approx_\epsilon v) \Leftrightarrow \exis{\theta:\Qp} (u \approx_{\epsilon-\theta} v) \label{RC-sim-rounded}\\
(u \approx_\epsilon v) \to (v\close\delta w) \to (u\approx_{\epsilon+\delta} w)\label{eq:RC-sim-rtri}\\
(u \close\epsilon v) \to (v\approx_\delta w) \to (u\approx_{\epsilon+\delta} w)\label{eq:RC-sim-ltri}。
\end{gather}
\end{thm}

附加条件~\eqref{RC-sim-rounded}--\eqref{eq:RC-sim-ltri} 证明了使得递归定义能够进行。
条件~\eqref{RC-sim-rounded} 被称为\define{圆滑的}。
\indexsee{relation!rounded}{rounded relation}%
\indexdef{rounded!relation}%
从右向左阅读得出 $\approx$ 的 \define{单调性},
\index{monotonicity}%
\index{relation!monotonic}%
%
\begin{equation*}
(\delta < \epsilon) \land (u \approx_\delta v) \Rightarrow (u \approx_\epsilon v)
\end{equation*}
%
而从左向右阅读则得出 $\approx$ 的 \define{开放性},
\index{open!relation}%
\index{relation!open}%
%
\begin{equation*}
(u \approx_\epsilon v) \Rightarrow \exis{\delta : \Qp} (\delta < \epsilon) \land (u \approx_\delta v)。
\end{equation*}
%
条件~\eqref{eq:RC-sim-rtri} 和~\eqref{eq:RC-sim-ltri} 是三角不等式的形式,表明 $\approx$ 在两侧是 $\closesym$ 的``模''。

\begin{proof}
  我们将通过双 $(\RC,\closesym)$-递归定义 $\mathord\approx:\RC\to\RC\to\Qp\to\prop$。
  首先我们将应用 $(\RC,\closesym)$-递归,其余象是 $\RC\to\Qp\to\prop$ 的子集,包含那些圆滑并满足一种适当形式的三角不等式的谓词族。
  将这些谓词视为二元关系的一半,我们将它们写为 $(u,\epsilon) \mapsto (\hapx_\epsilon u)$,其中符号 $\hapname$ 指整个关系。
  现在我们可以精确地写出 $A$
  \begin{multline*}
    A \defeq\; \Bigg\{ \hapname :\RC\to\Qp\to\prop \;\bigg|\; \\
    \Big(\fall{u:\RC}{\epsilon:\Qp}
    \big((\hapx_\epsilon u) \Leftrightarrow \exis{\theta:\Qp} (\hapx_{\epsilon-\theta} u)\big)\Big)  \\
    \land \Big(\fall{u,v:\RC}{\eta,\epsilon:\Qp} (u\close\epsilon v) \to\\
    \big((\hapx_\eta u) \to (\hapx_{\eta+\epsilon} v) \big) \land \big((\hapx_\eta v) \to (\hapx_{\eta+\epsilon} u) \big)\Big)\Bigg\}
  \end{multline*}
  与通常的子集一样,我们将使用相同的符号来表示 $A$ 的一个元素及其第一个分量 $\hapname$。
  作为 $(\RC,\closesym)$-递归所需的关系族,我们考虑以下内容,这将确保三角不等式的另一种形式:
  \begin{narrowmultline*}
  (\hapname \bsim_\epsilon \hapbname ) \defeq \narrowbreak
  \fall{u:\RC}{\eta:\Qp} ((\hapx_\eta u) \to (\hapxb_{\epsilon+\eta} u))
  \land \narrowbreak
  ((\hapxb_\eta u) \to (\hapx_{\epsilon+\eta} u))。
  \end{narrowmultline*}
  我们观察到这些关系是分离的。
  因为假设
  \narrowequation{\fall{\epsilon:\Qp} (\hapname \bsim_\epsilon \hapbname),}
  要证明 $\hapname = \hapbname$,只需证明对于所有 $u:\RC$,$(\hapx_\epsilon u) \Leftrightarrow (\hapxb_\epsilon u)$。
  但 $\hapx_\epsilon u$ 意味着对于某个 $\theta$,$\hapx_{\epsilon-\theta} u$,根据圆滑性,并且 $\hapname \bsim_\epsilon \hapbname$ 意味着 $\hapxb_\epsilon u$;反之亦然。

  现在,递归原则所需的前两个数据如下。
  \begin{itemize}
    \item 对于任何 $q:\Q$,我们必须给出 $A$ 的一个元素,我们将其记作 $(\rcrat(q)\approx_{(\blank)} \blank)$。
    \item 对于任何 Cauchy 近似 $x$,如果我们假设定义了函数 $\Qp \to A$,我们将其记作 $\epsilon \mapsto (x_\epsilon \approx_{(\blank)} \blank)$,其具有以下属性:
    % \[ \fall{u,v:\RC}{\delta,\epsilon,\eta:\Qp} (x_\delta \approx_\eta u) \to (u\close{\delta+\epsilon} v) \to (x_\epsilon \approx_{\eta+\delta+\epsilon} v) \]
    \begin{equation}
      \fall{u:\RC}{\delta,\epsilon,\eta:\Qp} (x_\delta \approx_\eta u) \to (x_\epsilon \approx_{\eta+\delta+\epsilon} u),\label{eq:appxrec2}
    \end{equation}
    我们必须给出 $A$ 的一个元素,我们将其写作 $(\rclim(x)\approx_{(\blank)} \blank)$。
  \end{itemize}
  在这两种情况下,我们通过使用嵌套 $(\RC,\closesym)$-递归给出所需定义,其余象为由圆滑的命题构成的 $\Qp\to\prop$ 的子集。
  将这些命题视为二元关系的零一半,我们将它们写作 $\epsilon \mapsto (\tap{\epsilon})$,符号 $\tapname$ 指整个关系族。
  现在我们可以精确地写出这些内递归的余象:
  \begin{narrowmultline*}
    C \defeq
    \bigg\{ \tapname :\Qp\to\prop \;\;\Big|\;\; \narrowbreak
    \fall{\epsilon:\Qp} \Big((\tap\epsilon) \Leftrightarrow \exis{\theta:\Qp} (\tap{\epsilon-\theta})\Big)\bigg\}
  \end{narrowmultline*}
  我们将所需的关系族定义为三角不等式的剩余部分:
  \begin{narrowmultline*}
  (\tapname \bbsim_\epsilon \tapbname) \defeq
  \fall{\eta:\Qp} ((\tap\eta) \to (\tapb{\epsilon+\eta})) \land
  \narrowbreak
  ((\tapb\eta) \to (\tap{\epsilon+\eta}))。
  \end{narrowmultline*}
  通过与 $\bsim$ 类似的论证,这些关系是分离的,使用 $C$ 的所有元素的圆滑性。

  注意,如果这样的内递归成功,它将生成一个谓词族 $\hapname : \RC\to\Qp\to \prop$,这些谓词族是圆滑的
  (因为它们在 $\Qp\to\prop$ 中的象属于 $C$)并且满足
  \[ \fall{u,v:\RC}{\epsilon:\Qp} (u\close\epsilon v) \to \big((\hapx_{(\blank)} u) \bbsim_\epsilon (\hapx_{(\blank)} u)\big)。 \]
  展开 $\bbsim$ 的定义,这正是使得 $\hapname$ 属于 $A$ 的第三个条件;因此,这正是我们需要的。

  此时我们可以给出定义~\eqref{eq:RCappx1}--\eqref{eq:RCappx4},作为两个内递归的前两个子句,分别对应于有理点和极限。
  在每种情况下,我们必须验证关系是圆滑的,因此属于 $C$。
  在有理-有理的情况下~\eqref{eq:RCappx1},这是显然的,而在其他情况下则来自归纳假设。
  (在~\eqref{eq:RCappx2} 中,相关的归纳假设是 $(\rcrat(q) \approx_{(\blank)} y_\delta) : C$,而在~\eqref{eq:RCappx3} 和~\eqref{eq:RCappx4} 中,它是 $(x_\delta \approx_{(\blank)} \blank) : A$。)

  余下的子递归数据包括证明~\eqref{eq:RCappx1}--\eqref{eq:RCappx4} 满足相对于 $\closesym$ 构造函数的右侧三角不等式。
  有八种情况 —— 每个子递归中的四个 —— 对应于 $u$、$v$ 和 $w$ 在~\eqref{eq:RC-sim-rtri} 中可以选择为有理点或极限的八种方式。
  首先考虑 $u$ 是 $\rcrat(q)$ 的情况。
  \begin{enumerate}
    \item 假设 $\rcrat(q)\approx_\phi \rcrat(r)$ 并且 $-\epsilon<|r-s|<\epsilon$,我们必须证明 $\rcrat(q)\approx_{\phi+\epsilon} \rcrat(s)$。
    但根据 $\approx$ 的定义,这简化为有理数的三角不等式。
    \item 我们假设 $\phi,\epsilon,\delta:\Qp$ 使得 $\rcrat(q)\approx_\phi \rcrat(r)$ 并且 $\rcrat(r) \close{\epsilon-\delta} y_\delta$,并且归纳地
    \begin{equation}
      \fall{\psi:\Qp}(\rcrat(q) \approx_{\psi} \rcrat(r)) \to (\rcrat(q) \approx_{\psi+\epsilon-\delta} y_\delta)。\label{eq:RCappx-rtri-rrl1}
    \end{equation}
    我们还假设 $\psi,\delta\mapsto (\rcrat(q) \approx_{\psi} y_\delta)$ 是相对于 $\bbsim$ 的 Cauchy 近似,即
    \begin{equation}
      \fall{\psi,\xi、zeta:\Qp} (\rcrat(q) \approx_{\psi} y_\xi) \to (\rcrat(q) \approx_{\psi+\xi+\zeta} y_\zeta),\label{eq:RCappx-rtri-rrl2}
    \end{equation}
    尽管在这种情况下我们不需要这种假设。
    事实上,\eqref{eq:RCappx-rtri-rrl1} 直接给出了 $\rcrat(q) \approx_{\phi+\epsilon-\delta} y_\delta$,从而通过 $\approx$ 的定义 $\rcrat(q) \approx_{\phi+\epsilon} \rclim(y)$。
    \item 我们假设 $\phi,\epsilon,\delta:\Qp$ 使得 $\rcrat(q)\approx_\phi \rclim(y)$ 并且 $y_\delta \close{\epsilon-\delta} \rcrat(r)$,并且归纳地
    \begin{gather}
      \fall{\psi:\Qp}(\rcrat(q) \approx_{\psi} y_\delta) \to (\rcrat(q) \approx_{\psi+\epsilon-\delta} \rcrat(r))。\label{eq:RCappx-rtri-rlr1}\\
      \fall{\psi,\xi,\zeta:\Qp} (\rcrat(q) \approx_{\psi} y_\xi) \to (\rcrat(q) \approx_{\psi+\xi+\zeta} y_\zeta)。\label{eq:RCappx-rtri-rlr2}
    \end{gather}
    根据定义,$\rcrat(q)\approx_\phi \rclim(y)$ 意味着我们有 $\theta:\Qp$ 使得 $\rcrat(q) \approx_{\phi-\theta} y_\theta$。
    根据假设~\eqref{eq:RCappx-rtri-rlr2},因此我们还得出 $\rcrat(q) \approx_{\phi+\delta} y_\delta$,然后根据~\eqref{eq:RCappx-rtri-rlr1} 得出 $\rcrat(q) \approx_{\phi+\epsilon} \rcrat(r)$,如所需。
    \item 我们假设 $\phi,\epsilon,\delta、eta:\Qp$ 使得 $\rcrat(q)\approx_\phi \rclim(y)$ 并且 $y_\delta \close{\epsilon-\delta-\eta} z_\eta$,并且归纳地
    \begin{gather}
      \fall{\psi:\Qp}(\rcrat(q) \approx_{\psi} y_\delta) \to (\rcrat(q) \approx_{\psi+\epsilon-\delta-\eta} z_\eta), \label{eq:RCappx-rtri-rll1}\\
      \fall{\psi,\xi,\zeta:\Qp} (\rcrat(q) \approx_{\psi} y_\xi) \to (\rcrat(q) \approx_{\psi+\xi+\zeta} y_\zeta), \label{eq:RCappx-rtri-rll2}\\
      \fall{\psi,\xi,\zeta:\Qp} (\rcrat(q) \approx_{\psi} z_\xi) \to (\rcrat(q) \approx_{\psi+\xi+\zeta} z_\zeta)。 \label{eq:RCappx-rtri-rll3}
    \end{gather}
    同样地,$\rcrat(q)\approx_\phi \rclim(y)$ 意味着我们有 $\xi:\Qp$ 使得 $\rcrat(q) \approx_{\phi-\xi} y_\xi$,而~\eqref{eq:RCappx-rtri-rll2} 然后意味着 $\rcrat(q) \approx_{\phi+\delta} y_\delta$ 并且~\eqref{eq:RCappx-rtri-rll1} 意味着 $\rcrat(q) \approx_{\phi+\epsilon-\eta} z_\eta$。
    但根据 $\approx$ 的定义,这意味着 $\rcrat(q) \approx_{\phi+\epsilon} \rclim(z)$ 如所需。
  \end{enumerate}
  现在我们继续讨论 $u$ 是 $\rclim(x)$ 的情况,其中 $x$ 是 Cauchy 近似。
  在这种情况下,$(\rclim(x) \approx_{(\blank)} {\blank}) : A$ 的定义的背景归纳假设是我们有 $(x_\delta \approx_{(\blank)} {\blank}) : A$,因此除了是圆滑的外,它们还满足右侧三角不等式。
  \begin{enumerate}\setcounter{enumi}{4}
  \item 假设 $\rclim(x)\approx_\phi \rcrat(r)$ 并且 $-\epsilon<|r-s|<\epsilon$,我们必须证明 $\rclim(x)\approx_{\phi+\epsilon} \rcrat(s)$。
  根据 $\approx$ 的定义,前者意味着 $x_\delta \approx_{\phi-\delta} \rcrat(r)$,因此上述三角不等式得出 $x_\delta \approx_{\epsilon+\phi-\delta} \rcrat(s)$,从而 $\rclim(x)\approx_{\phi+\epsilon} \rcrat(s)$ 如所需。
  \item 我们假设 $\phi,\epsilon、delta:\Qp$ 使得 $\rclim(x)\approx_\phi \rcrat(r)$ 并且 $\rcrat(r) \close{\epsilon-\delta} y_\delta$,并且有两个不需要的归纳假设。
  %
  根据定义,我们有 $\eta:\Qp$ 使得 $x_\eta \approx_{\phi-\eta} \rcrat(r)$,因此归纳三角不等式得出 $x_\eta \approx_{\phi+\epsilon-\eta-\delta} y_\delta$。
  然后 $\approx$ 的定义立即得出 $\rclim(x) \approx_{\phi+\epsilon} \rclim(y)$。
  \item 我们假设 $\phi,\epsilon、delta:\Qp$ 使得 $\rclim(x)\approx_\phi \rclim(y)$ 并且 $y_\delta \close{\epsilon-\delta} \rcrat(r)$,并且有两个不需要的归纳假设。
  根据定义,我们有 $\xi、theta:\Qp$ 使得 $x_\xi \approx_{\phi-\xi-\theta} y_\theta$。
  由于 $y$ 是 Cauchy 近似,我们有 $y_\theta \close{\theta+\delta} y_\delta$,因此归纳三角不等式得出 $x_\xi \approx_{\phi+\delta-\xi} y_\delta$ 然后 $x_\xi \close{\phi+\epsilon-\xi} \rcrat(r)$。
  然后 $\approx$ 的定义得出 $\rclim(x) \approx_{\phi+\epsilon}\rcrat(r)$,如所需。
  \item 最后,我们假设 $\phi,\epsilon、delta、eta:\Qp$ 使得 $\rclim(x)\approx_\phi \rclim(y)$ 并且 $y_\delta \close{\epsilon-\delta-\eta} z_\eta$。
  然后如前所述,我们有 $\xi、theta:\Qp$ 使得 $x_\xi \approx_{\phi-\xi-\theta} y_\theta$,并且如前所述的两个三角不等式足够了。
  \end{enumerate}

  这完成了两个内递归,从而完成了 $(\rclim(x) \approx_{(\blank)} {\blank}) : A$ 的定义。
  由于所有都是 $A$ 的元素,它们是圆滑的,并且满足相对于 $\closesym$ 的右侧三角不等式。
% , 并且满足~\eqref{eq:appxrec2}。
  余下的任务是验证与 $\bsim$ 相关的条件,即这些关系满足相对于 $\closesym$ 构造函数的\emph{左侧}三角不等式。
  四种情况对应于在~\eqref{eq:RC-sim-ltri} 中为 $u$ 和 $v$ 选择有理点或极限的四种选择,并且由于它们都是纯命题,我们可以应用 $\RC$ 归纳,并假设 $w$ 也是有理点或极限。
  这产生了另外八种情况,其证明基本与前述证明相同;因此我们不再让读者经历它们。
\end{proof}

我们现在可以证明:

\begin{thm}\label{thm:RC-sim-characterization}
对于任意的 $u,v:\RC$ 和 $\epsilon:\Qp$,我们有 $(u\close\epsilon v) = (u\approx_\epsilon v)$。
\end{thm}
\begin{proof}
  由于两者都是纯命题,足以证明双向蕴涵。
  对于左到右方向,我们使用 $\closesym$-归纳,应用于 $C(u,v,\epsilon)\defeq (u\approx_\epsilon v)$。
  因此,足以考虑 $\closesym$ 的四个构造函数。
  在每种情况下,$u$ 和 $v$ 专门化为有理点或极限,因此 $\approx$ 的定义计算并且归纳假设总是适用。

  对于右到左方向,我们使用 $\RC$-归纳来假设 $u$ 和 $v$ 是有理点或极限,允许 $\approx$ 计算。
  但现在 $\approx$ 的定义以及归纳假设提供了 $\closesym$ 构造函数所需的数据。
\end{proof}

\index{encode-decode method}%
略微夸张地说,可以将 $\approx$ 称为 $\closesym$ 的``编码'',上述证明的两个方向分别是 \encode 和 \decode。
根据 $\approx$ 的定义,从 \cref{thm:RC-sim-characterization} 我们得出等价关系
\begin{align*}
(\rcrat(q) \close\epsilon \rcrat(r))  &=
(-\epsilon < q - r < \epsilon)\\
(\rcrat(q) \close\epsilon \rclim(y)) &=
\exis{\delta : \Qp} \rcrat(q) \close{\epsilon - \delta} y_\delta\\
(\rclim(x) \close\epsilon \rcrat(r)) &=
\exis{\delta : \Qp} x_\delta \close{\epsilon - \delta} \rcrat(r)\\
(\rclim(x) \close\epsilon \rclim(y)) &=
\exis{\delta, \eta : \Qp} x_\delta \close{\epsilon - \delta - \eta} y_\eta。
\end{align*}
我们的证明还提供了以下附加信息。

\begin{cor}
  \index{triangle!inequality for R@inequality for $\RC$}%
  \indexsee{inequality!triangle}{triangle inequality}%
  $\closesym$ 是圆滑的\index{rounded!relation}并且满足三角不等式:
  \begin{gather}
    \eqvspaced{
      (u \close\epsilon v)
    }{
      \exis{\theta : \Qp} u \close{\epsilon - \theta} v
    }\\
    (u\close\epsilon v) \to (v\close\delta w) \to (u\close{\epsilon+\delta} w)。 \label{item:RC-sim-triangle}
  \end{gather}
\end{cor}
% \begin{proof}
%   $\approx$ 的构造同时证明了它是圆滑的,并且满足``三角不等式'',如
%   \[ (u\approx_\epsilon v) \to (v\close\delta w) \to (u\approx_{\epsilon+\delta} w)。 \]
%   因此,这两个性质由 \cref{thm:RC-sim-characterization} 得出。
% \end{proof}

有了三角不等式,我们可以证明 Cauchy 近似的``极限''实际上表现得像极限。

\begin{lem}\label{thm:RC-sim-lim}
对于任意的 $u:\RC$,Cauchy 近似 $y$,和 $\epsilon,\delta:\Qp$,如果 $u\close\epsilon y_\delta$ 那么 $u\close{\epsilon+\delta} \rclim(y)$。
\end{lem}
\begin{proof}
  我们对 $u$ 使用 $\RC$-归纳。
  如果 $u$ 是 $\rcrat(q)$,那么这正是 $\closesym$ 的第二个构造函数。
  现在假设 $u$ 是 $\rclim(x)$,并且每个 $x_\eta$ 具有以下性质:对于任意 $y,\epsilon,\delta$,如果 $x_\eta\close\epsilon y_\delta$ 那么 $x_\eta \close{\epsilon+\delta} \rclim(y)$。
  特别地,对于 $y\defeq x$ 和 $\delta\defeq\eta$ 在此假设下,我们得出对于任意 $\eta,\theta$,$x_\eta \close{\eta+\theta} \rclim(x)$。

  现在令 $y,\epsilon,\delta$ 任意,并假设 $\rclim(x) \close\epsilon y_\delta$。
  通过圆滑性,存在一个 $\theta$ 使得 $\rclim(x) \close{\epsilon-\theta} y_\delta$。
  然后,根据上述观察,对于任何 $\eta$,我们有 $x_\eta \close{\eta+\theta/2} \rclim(x)$,因此通过三角不等式得出 $x_\eta \close{\epsilon+\eta-\theta/2} y_\delta$。
  因此,$\closesym$ 的第四个构造函数得出 $\rclim(x) \close{\epsilon+2\eta+\delta-\theta/2} \rclim(y)$。
  因此,如果我们选择 $\eta \defeq \theta/4$,则结果成立。
\end{proof}

\begin{lem}\label{thm:RC-sim-lim-term}
对于任意的 Cauchy 近似 $y$ 和任意的 $\delta,\eta:\Qp$ 我们有 $y_\delta \close{\delta+\eta} \rclim(y)$。
\end{lem}
\begin{proof}
  在前一个引理中取 $u\defeq y_\delta$ 和 $\epsilon\defeq \eta$。
\end{proof}

\begin{rmk}
  我们可能期望有 $y_\delta \close{\delta} \rclim(y)$,但在某些例子中这会失败。
  例如,考虑定义为 $x_\epsilon \defeq \epsilon$ 的 $x$。
  它的极限显然是 $0$,但我们没有 $|\epsilon - 0 |<\epsilon$,只有 $\le$。
\end{rmk}

作为一个应用,\cref{thm:RC-sim-lim-term} 使我们能够证明 \cref{RC-extend-Q-Lipschitz} 中 Lipschitz 函数的扩展是唯一的。

\begin{lem}\label{RC-continuous-eq}
\index{function!continuous}%
令 $f,g:\RC\to\RC$ 是连续的,意思是
\[ \fall{u:\RC}{\epsilon:\Qp}\exis{\delta:\Qp}\fall{v:\RC} (u\close\delta v) \to (f(u) \close\epsilon f(v)) \]
并且类似地对于 $g$。
如果对于所有 $q:\Q$ 有 $f(\rcrat(q))=g(\rcrat(q))$,那么 $f=g$。
\end{lem}
\begin{proof}
  我们通过 $\RC$-归纳证明对于所有 $u$ 有 $f(u)=g(u)$。
  有理数情况只是假设。
  因此,假设对于所有 $\delta$ 有 $f(x_\delta)=g(x_\delta)$。
  我们将证明对于所有 $\epsilon$,$f(\rclim(x))\close\epsilon g(\rclim(x))$,因此 $\RC$ 的路径构造函数适用。

  由于 $f$ 和 $g$ 是连续的,存在 $\theta,\eta$ 使得对于所有 $v$,我们有
  \begin{align*}
  (\rclim(x)\close\theta v) &\to (f(\rclim(x)) \close{\epsilon/2} f(v))\\
  (\rclim(x)\close\eta v) &\to (g(\rclim(x)) \close{\epsilon/2} g(v))。
  \end{align*}
  选择 $\delta < \min(\theta,\eta)$,通过 \cref{thm:RC-sim-lim-term} 我们有 $\rclim(x)\close\theta y_\delta$ 和 $\rclim(x)\close\eta y_\delta$。
  因此
  \[ f(\rclim(x)) \close{\epsilon/2} f(y_\delta) = g(y_\delta) \close{\epsilon/2} g(\rclim(x))\]
  因此通过三角不等式 $f(\rclim(x))\close\epsilon g(\rclim(x))$。
\end{proof}

\subsection{Cauchy 实数的代数结构}
\label{sec:algebr-struct-cauchy}

我们首先定义加法结构 $(\RC, 0, {+}, {-})$。显然,加法单位元素
$0$ 就是 $\rcrat(0)$,而加法逆元 ${-} : \RC \to \RC$ 是通过加法逆元 ${-} : \Q \to \Q$ 的扩展得到的,使用 \cref{RC-extend-Q-Lipschitz}
Lipschitz 常数为 $1$。对于加法我们必须做更多工作。

\begin{lem} \label{RC-binary-nonexpanding-extension}
假设 $f : \Q \times \Q \to \Q$ 满足,对于所有 $q, r, s : \Q$,
%
\begin{equation*}
  |f(q, s) - f(r, s)| \leq |q - r|
  \qquad\text{和}\qquad
  |f(q, r) - f(q, s)| \leq |r - s|。
\end{equation*}
%
那么存在一个函数 $\bar{f} : \RC \times \RC \to \RC$,使得
对于所有 $q, r : \Q$,有 $\bar{f}(\rcrat(q), \rcrat(r)) = f(q,r)$。此外,
对于所有 $u, v, w : \RC$ 和 $q : \Qp$,
%
\begin{equation*}
  u \close\epsilon v \Rightarrow \bar{f}(u,w) \close\epsilon \bar{f}(v,w)
  \quad\text{和}\quad
  v \close\epsilon w \Rightarrow \bar{f}(u,v) \close\epsilon \bar{f}(u,w)。
\end{equation*}
\end{lem}


\begin{proof}
  我们使用 $(\RC, {\closesym})$-递归来构造 $\bar{f}$ 的柯里化形式,作为映射
  $\RC \to A$,其中 $A$ 是非扩展\index{function!non-expanding}\index{non-expanding function}实值函数的空间:
  %
  \begin{equation*}
    A \defeq
    \setof{ h : \RC \to \RC |
    \fall{\epsilon : \Qp} \fall{u, v : \RC}
    u \close\epsilon v \Rightarrow h(u) \close\epsilon h(v)
    }.
  \end{equation*}
  %
  我们还需要一个合适的 $\bsim_\epsilon$ 在 $A$ 上,我们定义为
  %
  \begin{equation*}
  (h \bsim_\epsilon k) \defeq \fall{u : \RC} h(u) \close\epsilon k(u)。
  \end{equation*}
  %
  显然,如果 $\fall{\epsilon : \Qp} h \bsim_\epsilon k$ 那么对所有 $u : \RC$ 有 $h(u) = k(u)$,因此 $\bsim$ 是分离的。

  对于基准情况,我们定义 $\bar{f}(\rcrat(q)) : A$,其中 $q : \Q$,作为 Lipschitz 映射 $\lam{r} f(q,r)$ 从 $\Q \to \Q$ 到 $\RC \to \RC$ 的扩展,使用 Lipschitz 常数为~$1$ 的 \cref{RC-extend-Q-Lipschitz} 构造。接下来,对于一个 Cauchy 近似 $x$,我们定义 $\bar{f}(\rclim(x)) : \RC \to \RC$ 为
  %
  \begin{equation*}
    \bar{f}(\rclim(x))(v) \defeq \rclim (\lam{\epsilon} \bar{f}(x_\epsilon)(v))。
  \end{equation*}
  %
  为了使这个定义有效,$\lam{\epsilon} \bar{f}(x_\epsilon)(v)$ 应该是一个 Cauchy 近似,因此考虑任意 $\delta, \epsilon : \Q$。然后根据假设 $\bar{f}(x_\delta) \bsim_{\delta + \epsilon} \bar{f}(x_\epsilon)$,因此 $\bar{f}(x_\delta)(v) \close{\delta + \epsilon} \bar{f}(x_\epsilon)(v)$。此外,$\bar{f}(\rclim(x))$ 是非扩展的,因为根据归纳假设 $\bar{f}(x_\epsilon)$ 是这样的。事实上,如果 $u \close\epsilon v$,那么对于所有 $\epsilon : \Q$,
  %
  \begin{equation*}
    \bar{f}(x_{\epsilon/3})(u) \close{\epsilon/3} \bar{f}(x_{\epsilon/3})(v),
  \end{equation*}
  %
  因此通过 $\closesym$ 的第四个构造函数 $\bar{f}(\rclim(x))(u) \close\epsilon \bar{f}(\rclim(x))(v)$。

  我们还需要检查四个条件,我们来说明其中一个。假设 $\epsilon : \Qp$ 并且对于某个 $\delta : \Qp$ 有 $\rcrat(q) \close{\epsilon - \delta} y_\delta$ 和 $\bar{f}(\rcrat(q)) \bsim_{\epsilon - \delta} \bar{f}(y_\delta)$。为了证明 $\bar{f}(\rcrat(q)) \bsim_\epsilon \bar{f}(\rclim(y))$,考虑任意 $v : \RC$ 并观察到
  %
  \begin{equation*}
    \bar{f}(\rcrat(q))(v) \close{\epsilon - \delta} \bar{f}(y_\delta)(v)。
  \end{equation*}
  %
  因此,通过 $\closesym$ 的第二个构造函数,我们有
  \narrowequation{\bar{f}(\rcrat(q))(v) \close\epsilon \bar{f}(\rclim(y))(v)}
  如所需。
\end{proof}

我们可以将 \cref{RC-binary-nonexpanding-extension} 应用于任何在每个变量中分别是非扩展的二元有理函数。加法就是这样的函数,因此我们得到了 ${+} : \RC \times \RC \to \RC$。
\indexdef{addition!of Cauchy reals}%
此外,只要我们要求它在每个变量中是非扩展的,这个扩展就是唯一的,并且与一元情况下相同,有理数上的恒等式扩展为实数上的恒等式。由于非扩展映射的复合再次是非扩展的,我们可以得出加法满足通常的性质,例如交换律和结合律。
\index{associativity!of addition!of Cauchy reals}%
因此,$(\RC, 0, {+}, {-})$ 是一个交换群。

我们还可以将 \cref{RC-binary-nonexpanding-extension} 应用于函数 $\min : \Q \times \Q \to \Q$ 和 $\max : \Q \times \Q \to \Q$,这将 $\RC$ 变成一个格。$\RC$ 上的偏序 $\leq$ 以 $\max$ 的形式定义为
%
\symlabel{leq-RC}
\index{order!non-strict}%
\index{non-strict order}%
\begin{equation*}
(u \leq v) \defeq (\max(u, v) = v)。
\end{equation*}
%
关系 $\leq$ 是偏序,因为它在 $\Q$ 上是这样的,并且偏序的公理可以用 $\min$ 和 $\max$ 表示为方程,因此它们转移到 $\RC$。

\index{absolute value}%
另一个通过相同方法扩展到 $\RC$ 的函数是绝对值 $|{\blank}|$。
同样,它具有预期的性质,因为它们从 $\Q$ 转移到 $\RC$。

\symlabel{lt-RC}
从 $\leq$ 我们得到严格顺序 $<$,其定义为
\index{strict!order}%
\index{order!strict}%
%
\begin{equation*}
(u < v) \defeq \exis{q, r : \Q} (u \leq \rcrat(q)) \land (q < r) \land (\rcrat(r) \leq v)。
\end{equation*}
%
即,当仅存在一对有理数 $q < r$ 使得 $x \leq \rcrat(q)$ 且 $\rcrat(r) \leq v$ 时,$u < v$ 成立。不难检查 $<$ 是反身和传递的,并且具有有序域预期的其他性质。
阿基米德原则直接来自于 $<$ 的定义。

\index{ordered field!archimedean}%
\begin{thm}[RC 的阿基米德原理] \label{RC-archimedean}
%
对于每个 $u, v : \RC$ 使得 $u < v$,仅存在 $q : \Q$ 使得 $u < q < v$。
\end{thm}

\begin{proof}
  从 $u < v$ 我们仅得到 $r, s : \Q$ 使得 $u \leq r < s \leq v$,我们可以取 $q \defeq (r + s) / 2$。
\end{proof}

我们现在有足够的结构来用标准概念表达 $u \close\epsilon v$。

\begin{lem}\label{thm:RC-le-grow}
如果 $q:\Q$ 和 $u:\RC$ 满足 $u\le \rcrat(q)$,那么对于任何 $v:\RC$ 和 $\epsilon:\Qp$,如果 $u\close\epsilon v$ 那么 $v\le \rcrat(q+\epsilon)$。
\end{lem}
\begin{proof}
  注意函数 $\max(\rcrat(q),\blank):\RC\to\RC$ 是 Lipschitz 的,常数为 $1$。
  首先考虑 $u=\rcrat(r)$ 为有理数的情况。
  对于此,我们使用归纳法 $v$。
  如果 $v$ 是有理数,那么该陈述是显然的。
  如果 $v$ 是 $\rclim(y)$,我们假设归纳地,对于任何 $\epsilon,\delta$,如果 $\rcrat(r)\close\epsilon y_\delta$ 那么 $y_\delta \le \rcrat(q+\epsilon)$,即 $\max(\rcrat(q+\epsilon),y_\delta)=\rcrat(q+\epsilon)$。

  现在假设 $\epsilon$ 并且 $\rcrat(r)\close\epsilon \rclim(y)$,我们有 $\theta$ 使得 $\rcrat(r)\close{\epsilon-\theta} \rclim(y)$,因此 $\rcrat(r)\close\epsilon y_\delta$ 只要 $\delta<\theta$。
  因此,归纳假设给出了 $\max(\rcrat(q+\epsilon),y_\delta)=\rcrat(q+\epsilon)$ 对于这样的 $\delta$。
  但根据定义,
  \[\max(\rcrat(q+\epsilon),\rclim(y)) \jdeq \rclim(\lam{\delta} \max(\rcrat(q+\epsilon),y_\delta))。\]
  由于最终常数 Cauchy 近似的极限是那个常数,我们有
  \[\max(\rcrat(q+\epsilon),\rclim(y)) = \rcrat(q+\epsilon),\] 因此 $\rclim(y)\le \rcrat(q+\epsilon)$。

  现在考虑一个一般的 $u:\RC$。
  由于 $u\le \rcrat(q)$ 意味着 $\max(\rcrat(q),u)=\rcrat(q)$,假设 $u\close\epsilon v$ 和 $\max(\rcrat(q),-)$ 的 Lipschitz 性质意味着 $\max(\rcrat(q),v) \close\epsilon \rcrat(q)$。
  因此,$\rcrat(q)\le \rcrat(q)$ 的第一个情况意味着 $\max(\rcrat(q),v) \le \rcrat(q+\epsilon)$,因此通过 $\le$ 的传递性 $v\le \rcrat(q+\epsilon)$。
\end{proof}

\begin{lem}\label{thm:RC-lt-open}
假设 $q:\Q$ 和 $u:\RC$ 满足 $u<\rcrat(q)$。 那么:
\begin{enumerate}
  \item 对于任何 $v:\RC$ 和 $\epsilon:\Qp$,如果 $u\close\epsilon v$ 那么 $v< \rcrat(q+\epsilon)$。\label{item:RCltopen1}
  \item 存在 $\epsilon:\Qp$ 使得对于任何 $v:\RC$,如果 $u\close\epsilon v$ 我们有 $v<\rcrat(q)$。\label{item:RCltopen2}
\end{enumerate}
\end{lem}
\begin{proof}
  根据定义,$u<\rcrat(q)$ 意味着存在 $r:\Q$ 使得 $r<q$ 并且 $u\le \rcrat(r)$。
  然后根据 \cref{thm:RC-le-grow},对于任何 $\epsilon$,如果 $u\close\epsilon v$ 那么 $v\le \rcrat(r+\epsilon)$。
  结论~\ref{item:RCltopen1} 立即得出,因为 $r+\epsilon<q+\epsilon$,而对于~\ref{item:RCltopen2} 我们可以取任何 $\epsilon <q-r$。
\end{proof}

我们现在可以证明辅助关系 $\closesym$ 确实如我们所想的那样。

\begin{thm} \label{RC-sim-eqv-le}
\index{distance}%
对于所有 $u, v : \RC$ 和 $\epsilon : \Qp$,$\eqv{(u \close\epsilon v)}{(|u - v| < \rcrat(\epsilon))}$。
\end{thm}
\begin{proof}
  减法和绝对值的 Lipschitz 性质意味着如果 $u\close\epsilon v$,那么 $|u-v| \close\epsilon |u-u| = 0$。
  因此,对于从左到右的方向,足以证明如果 $u\close\epsilon 0$,那么 $|u|<\rcrat(\epsilon)$。
  我们通过 $\RC$-归纳法证明 $u$。

  如果 $u$ 是有理数,那么该陈述立即得出,因为绝对值和顺序扩展了 $\Qp$ 上的标准。
  如果 $u$ 是 $\rclim(x)$,那么通过圆滑性我们有 $\theta:\Qp$,使得 $\rclim(x)\close{\epsilon-\theta} 0$。
  因此,通过三角不等式,我们有 $x_{\theta/3} \close{\epsilon-2\theta/3} 0$,因此归纳假设得出 $|x_{\theta/3}|<\rcrat(\epsilon-2\theta/3)$。
  但是 $x_{\theta/3} \close{2\theta/3} \rclim(x)$,因此根据 Lipschitz 性质 $|x_{\theta/3}| \close{2\theta/3} |\rclim(x)|$,因此 \cref{thm:RC-lt-open}\ref{item:RCltopen1} 意味着 $|\rclim(x)|<\rcrat(\epsilon)$。

  另一个方向上,我们使用 $\RC$-归纳法证明 $u$ 和 $v$。
  如果两者都是有理数,这是 $\closesym$ 的第一个构造函数。

  如果 $u$ 是 $\rcrat(q)$ 而 $v$ 是 $\rclim(y)$,我们归纳假设,对于任何 $\epsilon,\delta$,如果 $|\rcrat(q)-y_\delta|<\rcrat(\epsilon)$ 那么 $\rcrat(q) \close{\epsilon} y_\delta$。
  固定一个 $\epsilon$ 使得 $|\rcrat(q) - \rclim(y)|<\rcrat(\epsilon)$。
  由于 $\Q$ 在 $\RC$ 中是序致密的,存在 $\theta<\epsilon$ 使得 $|\rcrat(q) - \rclim(y)|<\rcrat(\theta)$。
  现在对于任何 $\delta,\eta$,我们有 $\rclim(y)\close{2\delta} y_\delta$,因此根据 Lipschitz 性质
  \[ |\rcrat(q) - \rclim(y)| \close{\delta+\eta} |\rcrat(q) - y_\delta|。 \]
  因此,根据 \cref{thm:RC-lt-open}\ref{item:RCltopen1},我们有 $|\rcrat(q) - y_\delta| < \rcrat(\theta+2\delta)$。
  因此,根据归纳假设,$\rcrat(q) \close{\theta+2\delta} y_\delta$,因此通过三角不等式 $\rcrat(q)\close{\theta+4\delta} \rclim(y)$。
  因此,足以选择 $\delta \defeq (\epsilon-\theta)/4$。

  剩下的两个情况完全类似。
\end{proof}

\indexdef{multiplication!of Cauchy reals}%
接下来,我们想为 $\RC$ 配备乘法结构。对于每个 $q : \Q$,映射 $r \mapsto q \cdot r$ 是 Lipschitz 的,常数\footnote{我们将 Lipschitz 常数定义为 \emph{正}有理数。} 为 $|q| + 1$,因此我们可以将其扩展到实数上的乘法。因此 $\RC$ 是一个 $\Q$ 上的向量空间\index{vector!space}。
通常,我们可以将实数乘法定义为
%
\begin{equation}
  u \cdot v \defeq
  {\textstyle \frac{1}{2}} \cdot ((u + v)^2 - u^2 - v^2),\label{mult-from-square}
\end{equation}
%
因此我们只需要平方\index{squaring function} $u \mapsto u^2$ 作为一个映射 $\RC \to \RC$。平方不是一个 Lipschitz 映射,但它在每个有界域上是 Lipschitz 的,这使得我们可以将其拼接在一起。定义开区间和闭区间
%
\indexdef{interval!open and closed}%
\indexdef{open!interval}%
\indexdef{closed!interval}%
\begin{equation*}
[u,v] \defeq \setof{ x : \RC | u \leq x \leq v }
\qquad\text{和}\qquad
(u,v) \defeq \setof{ x : \RC | u < x < v }。
\end{equation*}
%
尽管技术上 $[u,v]$ 或 $(u,v)$ 的元素是一个 Cauchy 实数,连同一个证明,因为后者居住在一个无关紧要的命题中,所以它是无趣的。
因此,与子集类型常见的情况一样,我们通常仅当 $x:\RC$ 满足 $u\leq x \leq v$ 时写作 $x:[u,v]$,同样。

\begin{thm} \label{RC-squaring}
%
存在一个唯一的函数 ${(\blank)}^2 : \RC \to \RC$ 扩展了有理数的平方 $q \mapsto q^2$ 并满足
%
\begin{equation*}
  \fall{n : \N}
  \fall{u, v : [-n, n]}
  |u^2 - v^2| \leq 2 \cdot n \cdot |u - v|。
\end{equation*}
\end{thm}

\begin{proof}
  我们首先观察到,对于每个 $u : \RC$ 仅存在 $n : \N$ 使得 $-n \leq u \leq n$,参见 \cref{ex:traditional-archimedean},因此映射
  %
  \begin{equation*}
    e : \Parens{\sm{n : \N} [-n, n]} \to \RC
    \qquad定义为\qquad
    e(n, x) \defeq x
  \end{equation*}
  %
  是满射。接下来,对于每个 $n : \N$,平方映射
  %
  \begin{equation*}
    s_n : \setof{ q : \Q | -n \leq q \leq n } \to \Q
    \qquad定义为\qquad
    s_n(q) \defeq q^2
  \end{equation*}
  %
  是 Lipschitz 的,常数为 $2 n$,因此我们可以使用 \cref{RC-extend-Q-Lipschitz} 将其扩展到 Lipschitz 常数为 $2 n$ 的映射 $\bar{s}_n : [-n, n] \to \RC$,参见 \cref{RC-Lipschitz-on-interval} 了解详细信息。这些映射 $\bar{s}_n$ 是兼容的:如果 $m < n$ 对于某些 $m, n : \N$,那么 $s_n$ 限制到 $[-m, m]$ 必须与 $s_m$ 一致,因为两者都是 Lipschitz 的,因此在 \cref{RC-continuous-eq} 的意义上是连续的。因此,根据 \cref{lem:images_are_coequalizers},映射
  %
  \begin{equation*}
    \Parens{\sm{n : \N} [-n, n]} \to \RC,
    \qquad给定为\qquad
    (n, x) \mapsto s_n(x)
  \end{equation*}
  %
  唯一地分解为 $\RC$ 给我们所需的函数。
\end{proof}

此时我们得到了实数的环结构和阿基米德顺序。要将 $\RC$ 作为一个阿基米德有序域,我们仍然需要逆元。

\begin{thm}
  \index{apartness}%
  Cauchy 实数是可逆的,当且仅当它远离零。
\end{thm}

\begin{proof}
  首先,假设 $u : \RC$ 有一个逆元 $v : \RC$ 根据阿基米德原则存在 $q :
  \Q$ 使得 $|v| < q$。那么 $1 = |u v| < |u| \cdot v < |u| \cdot q$,因此 $|u| >
  1/q$,也就是说 $u \apart 0$。

  相反,我们通过拼接函数来构造逆映射
  %
  \begin{equation*}
  ({\blank})^{-1} : \setof{ u : \RC | u \apart 0 } \to \RC。
  \end{equation*}
  %
  只给出主要步骤。对于每个 $q : \Q$,定义
  %
  \begin{equation*}
  [q, \infty) \defeq \setof{u : \RC | q \leq u}
  \qquad和\qquad
  (-\infty, q] \defeq \setof{u : \RC | u \leq -q}。
  \end{equation*}
  %
  然后,当 $q$ 取遍 $\Qp$ 时,类型 $(-\infty, q]$ 和 $[q, \infty)$ 共同覆盖
  $\setof{u : \RC | u \apart 0}$。在每个 $[q, \infty)$ 和 $(-\infty, q]$ 上,逆函数通过应用 \cref{RC-extend-Q-Lipschitz} 与 Lipschitz 常数 $1/q^2$ 得到。最后,\cref{lem:images_are_coequalizers} 保证逆函数唯一地分解为 $\setof{ u : \RC | u apart 0 }$。
\end{proof}

我们用定理总结 $\RC$ 的代数结构。

\begin{thm} \label{RC-archimedean-ordered-field}
Cauchy 实数组成一个阿基米德有序域。
\end{thm}

\subsection{Cauchy 实数是 Cauchy 完备的}
\label{sec:cauchy-reals-cauchy-complete}

我们通过将 $\Q$ 在 Cauchy 近似的极限下闭合来构造 $\RC$,所以 $\RC$ 应该是 Cauchy 完备的。根据 \cref{RC-sim-eqv-le},Cauchy 近似 $x : \Qp \to \RC$ 的定义与 $\RC$ 构造中的定义一致,并且与 \cref{defn:cauchy-approximation}(适用于 $\RC$)中的 Cauchy 近似一致。

因此,给定一个 Cauchy 近似 $x : \Qp \to \RC$,我们很自然地期望 $\rclim(x)$ 是它的极限,极限的概念定义如 \cref{defn:cauchy-approximation}。但事实如此,通过 \cref{RC-sim-eqv-le} 和 \cref{thm:RC-sim-lim-term}。我们已经证明了:

\begin{thm}
  每个 $\RC$ 中的 Cauchy 近似都有一个极限。
\end{thm}

一个阿基米德有序域中每个 Cauchy 近似都有一个极限的域称为\define{Cauchy 完备}。
\indexdef{Cauchy!completeness}%
\indexdef{complete!ordered field, Cauchy}%
\index{ordered field}%
Cauchy 实数是最小的此类域。

\begin{thm} \label{RC-initial-Cauchy-complete}
Cauchy 实数嵌入到每个 Cauchy 完备的阿基米德有序域中。
\end{thm}

\begin{proof}
  \index{limit!of a Cauchy approximation}%
  假设 $F$ 是一个 Cauchy 完备的阿基米德有序域。因为极限是唯一的,所以存在一个运算符 $\lim$,它将 $F$ 中的 Cauchy 近似映射到它们的极限。我们定义嵌入 $e : \RC \to F$ 通过 $(\RC, {\closesym})$-递归作为
  %
  \begin{equation*}
    e(\rcrat(q)) \defeq q
    \qquad和\qquad
    e(\rclim(x)) \defeq \lim (e \circ x)。
  \end{equation*}
  %
  在 $F$ 上合适的 $\bsim$ 是
  %
  \begin{equation*}
  (a \bsim_\epsilon b) \defeq |a - b| < \epsilon。
  \end{equation*}
  %
  这是一个分离的关系,因为 $F$ 是阿基米德的。对于 $(\RC, {\closesym})$-递归的其余条款很容易验证。还需要检查 $e$ 是一个固定有理数的有序域的嵌入。
\end{proof}

\index{real numbers!Cauchy|)}%

\section{柯西实数与戴德金实数的比较 (Comparison of Cauchy and Dedekind reals)}
\label{sec:comp-cauchy-dedek}

\index{real numbers!Dedekind|(实数!戴德金|(}%
\index{real numbers!Cauchy|(实数!柯西|(}%
\index{depression|(抑郁|(}

我们来讨论一下柯西实数 ($\RC$) 和戴德金实数 ($\RD$) 之间的关系。根据\cref{RC-archimedean-ordered-field},$\RC$ 是阿基米德有序域 (archimedean ordered field)。它对于 $\Omega$ 也是可容许的(admissible),这一点很容易验证。(如果 $\Omega$ 是初始 $\sigma$-框架 (initial $\sigma$-frame),则只需简单的归纳法即可验证;在其他情况下,这一点是直接显然的。)因此,根据\cref{RD-final-field},有一个有序域的嵌入 (embedding of ordered fields):

\begin{equation*}
  \RC \to \RD
\end{equation*}

这个嵌入固定了有理数 (rational numbers)。
(我们也可以从\cref{RC-initial-Cauchy-complete,RD-cauchy-complete} 中得出这一结论。)
通常情况下,如果没有进一步的假设,我们并不期望 $\RC$ 和 $\RD$ 会相等。

\begin{lem} \label{lem:untruncated-linearity-reals-coincide}
如果对于每个 $x : \RD$ ,仅存在一个满足以下条件的 $c$:
\begin{equation}
  \label{eq:untruncated-linearity}
  c : \prd{q, r : \Q} (q < r) \to (q < x) + (x < r)
\end{equation}
那么柯西实数 (Cauchy reals) 和戴德金实数 (Dedekind reals) 是一致的。
\end{lem}

\begin{proof}
  请注意,\eqref{eq:untruncated-linearity} 中的类型是~\eqref{eq:RD-linear-order} 的非截断版本,后者表明~$<$ 是一个弱线性序 (weak linear order)。
  我们已经知道 $\RC$ 嵌入了 $\RD$,因此只需证明每个戴德金实数 (Dedekind real) 都仅是一个有理数柯西序列 (Cauchy sequence) 的极限即可。

  考虑任意 $x : \RD$。根据假设,存在一个如引理陈述中所描述的 $c$,并且根据切分 (cuts) 的居留性 (inhabitation),存在有理数 $a, b : \Q$ 满足 $a < x < b$。
  我们通过递归构造一个序列 $f : \N \to \setof{ \pairr{q, r} \in \Q \times \Q | q < r }$:
  \begin{enumerate}
    \item 设 $f(0) \defeq \pairr{a, b}$。
    \item 假设 $f(n)$ 已经定义为 $\pairr{q_n, r_n}$,且 $q_n < r_n$。定义 $s \defeq (2 q_n + r_n)/3$ 和 $t \defeq (q_n + 2 r_n)/3$。然后 $c(s,t)$ 决定了 $s < x$ 还是 $x < t$。如果决定 $s < x$,那么我们设置 $f(n+1) \defeq \pairr{s, r_n}$,否则设置 $f(n+1) \defeq \pairr{q_n, t}$。
  \end{enumerate}

  让我们用 $\pairr{q_n, r_n}$ 表示序列 $f$ 的第 $n$ 项。那么很容易看出,对于所有 $n : \N$,都有 $q_n < x < r_n$ 并且 $|q_n - r_n| \leq (2/3)^n \cdot |q_0 - r_0|$。因此,$q_0, q_1, \ldots$ 和 $r_0, r_1, \ldots$ 都是收敛于戴德金实数 $x$ 的柯西序列 (Cauchy sequences)。
  我们已经证明,对于每个 $x : \RD$,仅存在一个收敛于 $x$ 的柯西序列。
\end{proof}

该引理表明,无论是可数选择公理 (countable choice) 还是排中律 (excluded middle) 都足以保证 $\RC$ 和 $\RD$ 的一致性。

\begin{cor} \label{when-reals-coincide}
如果排中律 (excluded middle) 或者可数选择公理 (countable choice) 成立,那么 $\RC$ 和 $\RD$ 是等价的。
\end{cor}

\begin{proof}
  如果排中律成立,那么 $(x < y) \to (x < z) + (z < y)$ 可以被证明:要么 $x < z$,要么 $\lnot (x < z)$。在前一种情况下,我们完成证明;而在后一种情况下,我们得出 $z < y$,因为 $z \leq x < y$。因此,我们得到~\eqref{eq:untruncated-linearity},可以应用\cref{lem:untruncated-linearity-reals-coincide}。

  假设可数选择公理成立。集合 $S = \setof{ \pairr{q, r} \in \Q \times \Q | q < r }$ 与 $\N$ 是等价的,因此我们可以将可数选择公理应用于 $x$ 的定位 (located),即:
  \begin{equation*}
    \fall{\pairr{q, r} : S} (q < x) \lor (x < r)。
  \end{equation*}
  请注意,$(q < x) \lor (x < r)$ 可以表示为一个存在量词语句 (existential statement) $\exis{b : \bool} (b = \bfalse \to q < x) \land (b = \btrue \to x < r)$。选择函数的柯里形式 (curried form) 就是~\eqref{eq:untruncated-linearity},因此\cref{lem:untruncated-linearity-reals-coincide} 再次适用。
\end{proof}

\index{real numbers!Dedekind|)实数!戴德金|)}%
\index{real numbers!Cauchy|)实数!柯西|)}%
\index{real numbers!agree实数!一致性}%

\index{depression|)抑郁|)}

\section{区间的紧致性 (Compactness of the interval)}
\label{sec:compactness-interval}

\index{mathematics!classical|(数学!经典|(}%
\index{mathematics!constructive|(数学!构造性|(}%

我们已经指出,我们对于实数的构造与经典逻辑 (classical logic) 完全兼容。因此,假设排中律 (law of excluded middle) \eqref{eq:lem} 和选择公理 (axiom of choice) \eqref{eq:ac} 成立,我们可以发展经典分析 (classical analysis),\index{classical!analysis经典!分析}\index{analysis!classical分析!经典}这实际上相当于复制任何标准的分析书籍。

\index{analysis!constructive分析!构造性}%
\index{constructive!analysis构造性!分析}%
然而,对于任何对计算感兴趣的人,例如数值分析学家,应该对在一个计算上有意义的环境中发展分析感到好奇。构造性环境中的分析是可能的,这一点已经由 \cite{Bishop1967} 证明了。作为经典分析和构造性分析之间差异和相似性的一个例子,我们将简要讨论一个主题——闭区间 $[0,1]$ 的紧致性 (compactness) 以及围绕这一概念的一些定理。

在构造性数学中,经典上等价的概念常常会分裂,紧致性也不例外。最常用的三种紧致性概念是:
%
\indexdef{compactness紧致性}%
\begin{enumerate}
  \item \define{度量紧致性 (metrically compact):} “柯西完备 (Cauchy complete) 和全有界 (totally bounded)”,
  \indexdef{metrically compact度量紧致性}%
  \indexdef{compactness!metric紧致性!度量}%
  \item \define{波尔查诺-魏尔斯特拉斯紧致性 (Bolzano--Weierstra\ss{} compact):} “每个序列都有一个收敛的子序列”,
  \index{compactness!Bolzano--Weierstrass@Bolzano--Weierstra\ss{}紧致性!波尔查诺-魏尔斯特拉斯}%
  \indexsee{Bolzano--Weierstrass@Bolzano--Weierstra\ss{}紧致性}{compactness紧致性}%
  \index{sequence序列}%
  \item \define{海涅-博雷尔紧致性 (Heine--Borel compact):} “每个开覆盖 (open cover) 都有一个有限子覆盖 (finite subcover)”。
  \index{compactness!Heine--Borel紧致性!海涅-博雷尔}%
  \indexsee{Heine--Borel海涅-博雷尔}{compactness紧致性}%
\end{enumerate}
%
这些在经典数学中都是等价的。
让我们看看它们在同伦类型论 (homotopy type theory) 中的表现。我们可以使用戴德金实数 (Dedekind reals) 或柯西实数 (Cauchy reals),因此我们将实数记作 $\R$。首先我们回顾一些基本定义。

\indexsee{space!metric空间!度量}{metric space度量空间}
\index{metric space|(度量空间|(}%

\begin{defn} \label{defn:metric-space}
\define{度量空间 (metric space)}
\indexdef{metric space度量空间}%
$(M, d)$ 是一个集合 $M$,其上有一个映射 $d : M \times M \to \R$,
满足对所有 $x, y, z : M$,
%
\begin{align*}
  d(x,y) &\geq 0, &
  d(x,y) &= d(y,x), \\
  d(x,y) &= 0 \Leftrightarrow x = y, &
  d(x,z) &\leq d(x,y) + d(y,z).
\end{align*}
  %
\end{defn}

\begin{defn} \label{defn:complete-metric-space}
\define{柯西逼近 (Cauchy approximation)}
\index{Cauchy!approximation柯西!逼近}%
在 $M$ 中是一个序列 $x : \Qp \to M$,满足
%
\begin{equation*}
  \fall{\delta, \epsilon} d(x_\delta, x_\epsilon) < \delta + \epsilon.
\end{equation*}
%
\index{limit!of a Cauchy approximation柯西逼近的极限}%
一个柯西逼近 $x : \Qp \to M$ 的 \define{极限 (limit)} 是一个点 $\ell : M$,满足
%
\begin{equation*}
  \fall{\epsilon, \theta : \Qp} d(x_\epsilon, \ell) < \epsilon + \theta.
\end{equation*}
%
\indexdef{metric space!complete度量空间!完备}%
\indexdef{complete!metric space完备!度量空间}%
一个 \define{完备度量空间 (complete metric space)} 是其中每个柯西逼近都有极限的空间。
\end{defn}

\begin{defn} \label{defn:total-bounded-metric-space}
对于一个正有理数 $\epsilon$,一个 \define{$\epsilon$-网 ($\epsilon$-net)}
\indexdef{epsilon-net@$\epsilon$-net}%
在度量空间 $(M, d)$ 中是一个元素
%
\begin{equation*}
  \sm{n : \N}{x_1, \ldots, x_n : M}
  \fall{y : M} \exis{k \leq n} d(x_k, y) < \epsilon.
\end{equation*}
%
换句话说,这是一个有限的点序列 $x_1, \ldots, x_n$,使得 $M$ 中的每个点仅在某个 $x_k$ 的 $\epsilon$ 内。

一个度量空间 $(M, d)$ 是 \define{全有界 (totally bounded)} 的,
\indexdef{totally bounded metric space全有界度量空间}%
\indexdef{metric space!totally bounded度量空间!全有界}%
当它有所有大小的 $\epsilon$-网时:
%
\begin{equation*}
  \prd{\epsilon : \Qp}
  \sm{n : \N}{x_1, \ldots, x_n : M}
  \fall{y : M} \exis{k \leq n} d(x_k, y) < \epsilon.
\end{equation*}
\end{defn}

\begin{rmk}
  在全有界性的定义中,我们使用了不严格的记法 $\sm{n : \N}{x_1, \ldots, x_n : M}$。正式来说,我们应该写成 $\sm{x : \lst{M}}$,其中 $\lst{M}$ 是来自 \cref{sec:bool-nat} 的有限列表的归纳类型 (inductive type of finite lists)\index{type!of lists类型!列表}。
  然而,这会使得表达其余部分的陈述更加麻烦。
\end{rmk}

注意,在全有界性的定义中,我们要求纯粹存在 (pure existence) 的 $\epsilon$-网,而不是仅仅存在 (mere existence)。这样我们得到一个函数,该函数为每个 $\epsilon : \Qp$ 分配一个特定的 $\epsilon$-网。这样的函数可以称为“全有界性的模数 (modulus of total boundedness)”。通常,在将经典的度量概念移植到同伦类型论时,我们应该谨慎地使用命题截断 (propositional truncation),通常这样做是为了避免要求从 $\R$ 到 $\Q$ 或 $\N$ 的非常数映射。例如,以下是均匀连续性的“正确”定义。

\begin{defn} \label{defn:uniformly-continuous}
在度量空间上的映射 $f : M \to \R$ 是 \define{均匀连续 (uniformly continuous)} 的,
\indexdef{function!uniformly continuous函数!均匀连续}%
\indexdef{uniformly continuous function均匀连续函数}%
当
%
\begin{equation*}
  \prd{\epsilon : \Qp}
  \sm{\delta : \Qp}
  \fall{x, y : M}
  d(x,y) < \delta \Rightarrow |f(x) - f(y)| < \epsilon.
\end{equation*}
%
特别地,一个均匀连续映射有一个均匀连续性的模数 (modulus of uniform continuity)\indexdef{modulus!of uniform continuity模数!均匀连续性},
这是一个为每个 $\epsilon$ 分配相应 $\delta$ 的函数。
\end{defn}

让我们证明 $[0,1]$ 在第一种意义上是紧致的。

\begin{thm} \label{analysis-interval-ctb}
\index{compactness!metric紧致性!度量}%
\index{interval!open and closed区间!开和闭}%
闭区间 $[0,1]$ 是完备的且全有界的。
\end{thm}

\begin{proof}
  给定 $\epsilon : \Qp$,存在 $k : \N$ 使得 $2/k < \epsilon$,所以我们可以取 $\epsilon$-网 $x_i = i/k$,其中 $i = 0, \ldots, k$。这是一个 $\epsilon$-网,因为对于每个 $y : [0,1]$,仅存在某个 $i$,使得 $0 \leq i \leq k$ 且 $(i - 1)/k < y < (i+1)/k$,因此 $|y - x_i| < 2/k < \epsilon$。

  对于 $[0,1]$ 的完备性,考虑一个柯西逼近 $x : \Qp \to [0,1]$ 并让 $\ell$ 是它在 $\R$ 中的极限。由于 $\max$ 和 $\min$ 是 Lipschitz 映射,由 $r(x) \defeq \max(0, \min(1, x))$ 定义的从 $\R$ 到 $[0,1]$ 的缩回 (retraction) $r$ 与柯西逼近的极限交换,因此
  %
  \begin{equation*}
    r(\ell) =
    r (\lim x) =
    \lim (r \circ x) =
    \lim x =
    \ell,
  \end{equation*}
  %
  这意味着 $0 \leq \ell \leq 1$,这是我们所要求的。
\end{proof}

因此,我们在同伦类型论中至少有一种好的紧致性概念。不幸的是,它仅限于度量空间,因为全有界性是一个度量概念。我们很快将考虑其他两个概念,但首先我们证明在全有界空间上的均匀连续映射具有 \define{上确界 (supremum)}\indexsee{least upper bound上确界}{supremum上确界},即一个小于或等于所有其他上界的上界。

\begin{thm} \label{ctb-uniformly-continuous-sup}
\indexdef{supremum!of uniformly continuous function均匀连续函数的上确界}%
在一个全有界度量空间 $(M, d)$ 上的均匀连续映射 $f : M \to \R$ 有一个上确界 $m : \R$。对于每个 $\epsilon : \Qp$,存在 $u : M$,使得 $|m - f(u)| < \epsilon$。
\end{thm}

\begin{proof}
  设 $h : \Qp \to \Qp$ 是 $f$ 的均匀连续性模数 (modulus of uniform continuity)。
  我们如下定义一个逼近 $x : \Qp \to \R$:对于任意 $\epsilon : \Q$,$M$ 的全有界性给出了一个 $h(\epsilon)$-网 $y_0, \ldots, y_n$。定义
  %
  \begin{equation*}
    x_\epsilon \defeq \max (f(y_0), \ldots, f(y_n))。
  \end{equation*}
  %
  我们声称 $x$ 是一个柯西逼近。考虑任意 $\epsilon, \eta : \Q$,使得
  %
  \begin{equation*}
    x_\epsilon \jdeq \max (f(y_0), \ldots, f(y_n))
    \quad\text{和}\quad
    x_\eta \jdeq \max (f(z_0), \ldots, f(z_m))
  \end{equation*}
  %
  对于某个 $h(\epsilon)$-网 $y_0, \ldots, y_n$ 和 $h(\eta)$-网 $z_0, \ldots, z_m$。
  每个 $z_i$ 仅与某个 $y_j$ 的 $h(\epsilon)$-近,因此 $|f(z_i) - f(y_j)| < \epsilon$,我们可以得出
  %
  \begin{equation*}
    f(z_i) < \epsilon + f(y_j) \leq \epsilon + x_\epsilon,
  \end{equation*}
  %
  因此 $x_\eta < \epsilon + x_\epsilon$。对称地,我们得到 $x_\eta < \eta + x_\eta$,因此 $|x_\eta - x_\epsilon| < \eta + \epsilon$。

  我们声称 $m \defeq \lim x$ 是 $f$ 的上确界。为了证明 $f(x) \leq m$ 对所有 $x : M$ 成立,只需证明 $\lnot (m < f(x))$。假设相反的情况,即 $m < f(x)$。存在 $\epsilon : \Qp$ 使得 $m + \epsilon < f(x)$。但是现在仅对定义 $x_\epsilon$ 的某个 $y_i$,我们得到 $|f(x) - f(y_i)| < \epsilon$,因此 $m < f(x) - \epsilon < f(y_i) \leq m$,这与假设矛盾。

  最后,我们通过证明 $m$ 满足定理的第二部分来结束证明,因为它自动是一个最小的上界。给定任意 $\epsilon : \Qp$,一方面 $|m - f(x_{\epsilon/2})| < 3 \epsilon/4$,另一方面 $|f(x_{\epsilon/2}) - f(y_i)| < \epsilon/4$ 仅对定义 $x_{\epsilon/2}$ 的某个 $y_i$ 成立,因此通过取 $u \defeq y_i$,我们通过三角不等式 (triangle inequality) 得到 $|m - f(u)| < \epsilon$。
\end{proof}

现在,如果在 \cref{ctb-uniformly-continuous-sup} 中我们也知道 $M$ 是完备的,我们可以希望将均匀连续性的假设减弱为连续性 (continuity),并将结论加强为存在一个点在该点上达到上确界。通常的证明这些改进依赖于以下事实:在一个完备的全有界空间中
%
\begin{enumerate}
  \item 连续性意味着均匀连续性,
  \item 每个序列都有一个收敛的子序列。
\end{enumerate}
%
第一个陈述很容易从海涅-博雷尔紧致性推导出来,第二个则只是波尔查诺-魏尔斯特拉斯紧致性。
\index{compactness!Bolzano--Weierstrass@Bolzano--Weierstra\ss{}紧致性!波尔查诺-魏尔斯特拉斯}%
不幸的是,这两者都有些问题。首先我们证明波尔查诺-魏尔斯特拉斯紧致性蕴含了一个排中律的实例,称为\define{全知原理的有限形式 (limited principle of omniscience)}:
\indexsee{axiom!limited principle of omniscience公理!全知原理的有限形式}{limited principle of omniscience全知原理的有限形式}%
\indexdef{limited principle of omniscience全知原理的有限形式}%
对于每个 $\alpha : \N \to \bool$,
%
\begin{equation} \label{eq:lpo}
\Parens{\sm{n : \N} \alpha(n) = \btrue} +
\Parens{\prd{n : \N} \alpha(n) = \bfalse}.
\end{equation}
%
从计算的角度来看,我们不会期望这个原理成立,因为它要求我们决定一个函数是否有无穷多个值为 $\bfalse$。

\begin{thm} \label{analysis-bw-lpo}
波尔查诺-魏尔斯特拉斯紧致性 (Bolzano--Weierstra\ss{} compactness) 对于 $[0,1]$ 蕴含了全知原理的有限形式。
\index{compactness!Bolzano--Weierstrass@Bolzano--Weierstra\ss{}紧致性!波尔查诺-魏尔斯特拉斯}%
\end{thm}

\begin{proof}
  给定任意 $\alpha : \N \to \bool$,定义序列 (sequence) $x : \N \to [0,1]$ 为
  %
  \begin{equation*}
    x_n \defeq
    \begin{cases}
      0 & \text{如果对所有 $k < n$,$\alpha(k) = \bfalse$,}\\
      1 & \text{如果对某个 $k < n$,$\alpha(k) = \btrue$}.
    \end{cases}
  \end{equation*}
  %
  如果波尔查诺-魏尔斯特拉斯性质成立,存在一个严格递增的 $f : \N \to \N$,使得 $x \circ f$ 是一个柯西序列 (Cauchy sequence)。对于一个足够大的 $n : \N$,第 $n$ 项 $x_{f(n)}$ 距离其极限小于 $1/6$。要么 $x_{f(n)} < 2/3$,要么 $x_{f(n)} > 1/3$。如果 $x_{f(n)} < 2/3$,则 $x_n$ 收敛到 $0$,因此 $\prd{n : \N} \alpha(n) = \bfalse$。如果 $x_{f(n)} > 1/3$,则 $x_{f(n)} = 1$,因此 $\sm{n : \N} \alpha(n) = \btrue$。
\end{proof}

虽然我们可能不会太在意波尔查诺-魏尔斯特拉斯紧致性,但没有海涅-博雷尔紧致性似乎更难以接受,正如经典数学和布劳威尔直觉主义 (Brouwer's Intuitionism) 都接受了它一样。由于我们不想深入一般拓扑学,我们将使用基本开集 (basic open sets) 来工作。在 $\R$ 的情况下,这些是具有有理数端点的开区间。一个由类型 $I$ 索引的此类区间的族将是一个映射
%
\begin{equation*}
  \mathcal{F} : I \to \setof{(q, r) : \Q \times \Q | q < r},
\end{equation*}
%
其想法是,一个有理数对 $(q, r)$ 与 $q < r$ 确定类型 $\setof{ x : \R | q < x < r}$。允许退化区间稍微更方便一些,因此我们取一个 \define{基本区间族 (family of basic intervals)} \indexdef{family!of basic intervals族!基本区间的}%
\indexdef{interval!family of basic区间!基本的族}%
为一个映射
%
\begin{equation*}
  \mathcal{F} : I \to \Q \times \Q。
\end{equation*}
%
要非常精确,一个族是一个依赖对 $(I, \mathcal{F})$,而不仅仅是 $\mathcal{F}$。一个 \define{有限基本区间族 (finite family of basic intervals)} 是由 $\setof{ m : \N | m < n}$ 对某个 $n : \N$ 索引的族。我们通常用有限列表 $[(q_0, r_0), \ldots, (q_{n-1}, r_{n-1})]$ 来表示它。最后,一个 $(I, \mathcal{F})$ 的 \define{有限子族 (finite subfamily)} \indexdef{subfamily, finite, of intervals子族,有限的,区间的} 是由索引列表 $[i_1, \ldots, i_n]$ 给出的,这些索引决定了有限族 $[\mathcal{F}(i_1), \ldots, \mathcal{F}(i_n)]$。

只要我们知道对 $(q, r)$ 和对应的区间 $\setof{ x : \R | q < x < r}$ 之间的区别,我们可以安全地对两者使用相同的符号 $(q, r)$。区间的交集 (intersections)\indexdef{intersection!of intervals交集!区间的} 和包含 (inclusions)\indexdef{inclusion!of intervals包含!区间的}\indexdef{containment!of intervals包含!区间的} 可以用它们的端点表示:
%
\symlabel{interval-intersection}
\symlabel{interval-subset}
\begin{align*}
(q, r) \cap (s, t) &\ \defeq\  (\max(q, s), \min(r, t)),\\
(q, r) \subseteq (s, t) &\ \defeq\ (q < r \Rightarrow s \leq q < r \leq t)。
\end{align*}
%
我们说 $\intfam{i}{I}{(q_i, r_i)}$ \define{(逐点)覆盖 $[a,b]$ ((pointwise) covers $[a,b]$)},\indexdef{interval!pointwise cover区间!逐点覆盖的}%
\indexdef{cover!pointwise覆盖!逐点的}%
\indexdef{pointwise!cover逐点!覆盖的}%
当
%
\begin{equation} \label{eq:cover-pointwise-truncated}
\fall{x : [a,b]} \exis{i : I} q_i < x < r_i。
\end{equation}
%
$[0,1]$ 的 \define{海涅-博雷尔紧致性 (Heine--Borel compactness)}\indexdef{compactness!Heine--Borel紧致性!海涅-博雷尔}%
表示每个覆盖 $[0,1]$ 的区间族都有一个有限子族仍然覆盖 $[0,1]$。

\index{depression}
\begin{thm} \label{classical-Heine-Borel}
\index{excluded middle排中律}%
如果排中律成立,那么 $[0,1]$ 是海涅-博雷尔紧致的。
\end{thm}

\begin{proof}
  假设为了达到矛盾,一个族 $\intfam{i}{I}{(a_i, b_i)}$ 覆盖 $[0,1]$,但没有有限子族覆盖它。我们构造一系列闭区间 $[q_n, r_n]$,这些区间是嵌套的,它们的大小收缩到 $0$,且其中没有一个被 $\intfam{i}{I}{(a_i, b_i)}$ 的有限子族覆盖。

  我们设 $[q_0, r_0] \defeq [0,1]$。假设 $[q_n, r_n]$ 已经构造,设 $s \defeq (2 q_n + r_n)/3$ 和 $t \defeq (q_n + 2 r_n)/3$。$[q_n, t]$ 和 $[s, r_n]$ 都被 $\intfam{i}{I}{(a_i, b_i)}$ 覆盖,但它们不能同时有一个有限子覆盖,否则 $[q_n, r_n]$ 也会有一个有限子覆盖。要么 $[q_n, t]$ 有一个有限子覆盖,要么它没有。如果有,我们设 $[q_{n+1}, r_{n+1}] \defeq [s, r_n]$,否则我们设 $[q_{n+1}, r_{n+1}] \defeq [q_n, t]$。

  序列 $q_0, q_1, \ldots$ 和 $r_0, r_1, \ldots$ 都是柯西的,它们收敛到 $[0,1]$ 中的一个点 $x$,该点包含在每个 $[q_n, r_n]$ 中。
  仅存在 $i : I$ 使得 $a_i < x < b_i$。由于区间 $[q_n, r_n]$ 的大小收缩到零,存在 $n : \N$ 使得 $a_i < q_n \leq x \leq r_n < b_i$,但这意味着 $[q_n, r_n]$ 被单个区间 $(a_i, b_i)$ 覆盖,而同时它没有有限子覆盖。这是矛盾。
\end{proof}

没有排中律,或布劳威尔直觉主义 (Brouwerian Intuitionism) 的一点点帮助,我们似乎陷入了困境。然而,在构造性环境中,$[0,1]$ 的海涅-博雷尔紧致性 \emph{可以} 得到恢复,并且仍然与经典数学兼容!为此,我们需要重新审视覆盖的概念。\eqref{eq:cover-pointwise-truncated} 的问题在于截断的存在使得一个空间可以以任何随意的方式被覆盖,从计算的角度来看,我们几乎没有希望仅提取一个有限子覆盖。通过移除截断,我们得到
%
\begin{equation} \label{eq:cover-pointwise}
\prd{x : [0,1]} \sm{i : I} q_i < x < r_i,
\end{equation}
%
这可能有所帮助,但它对覆盖的要求太高了。使用这个定义,我们甚至无法证明 $(0,3)$ 和 $(2,5)$ 覆盖 $[1,4]$,因为这相当于展示一个从 $[1,4]$ 到 $\bool$ 的非常数映射,参见 \cref{ex:reals-non-constant-into-Z}。在这里,我们可以从“无点拓扑 (pointfree topology)”\index{pointfree topology无点拓扑}%
\index{topology!pointfree拓扑!无点的}%
(即位置理论 (locale theory))\index{locale位置}%
中吸取教训:覆盖的概念应该用开集来表达,而不涉及点。这样一种对空间的“整体”观点将允许我们分析覆盖的概念,我们将能够恢复海涅-博雷尔紧致性。位置理论使用幂集 (power sets),\index{power set幂集}%
我们可以通过假设命题调整 (propositional resizing) 获得;\index{propositional!resizing命题!调整}%
但我们可以从位置理论的预测性 (predicative) 近亲中偷取想法,这被称为“形式拓扑 (formal topology)”\index{formal!topology形式!拓扑}%
\index{mathematics!predicative数学!预测性}。

\index{acceptance|(接受|(}

假设我们有一个族 $\pairr{I, \mathcal{F}}$ 和一个区间 $(a, b)$。我们如何表达 $(a,b)$ 被该族覆盖的事实,而不涉及点呢?这里有一种方法:如果 $(a, b)$ 等于某个 $\mathcal{F}(i)$,那么它就被该族覆盖了。还有一种方法:如果 $(a,b)$ 被另一个族 $(J, \mathcal{G})$ 覆盖,而每个 $\mathcal{G}(j)$ 都被 $\pairr{I, \mathcal{F}}$ 覆盖,那么 $(a,b)$ 就被 $\pairr{I, \mathcal{F}}$ 覆盖。注意,我们正在列出可以用来推导出 $\pairr{I, \mathcal{F}}$ 覆盖 $(a,b)$ 的\emph{规则}。我们应该找到足够好的规则并将它们转化为一个归纳定义。

\begin{defn} \label{defn:inductive-cover}
\define{归纳覆盖 $\cover$ (inductive cover $\cover$)}\indexdef{inductive!cover归纳!覆盖}%
\indexdef{cover!inductive覆盖!归纳的}%
是一个单纯关系 (mere relation)
%
\begin{equation*}
{\cover} : (\Q \times \Q) \to \Parens{\sm{I : \type} (I \to \Q \times \Q)} \to \prop
\end{equation*}
%
通过以下规则进行归纳定义,其中 $q, r, s, t$ 是有理数,$\pairr{I, \mathcal{F}}$ 和 $\pairr{J, \mathcal{G}}$ 是基本区间族:
%
\begin{enumerate}

  \item \emph{自反性 (reflexivity):}\index{reflexivity!of inductive cover自反性!归纳覆盖的}%
  对于所有 $i : I$,$\mathcal{F}(i) \cover \pairr{I, \mathcal{F}}$,

  \item \emph{传递性 (transitivity):}\index{transitivity!of inductive cover传递性!归纳覆盖的}%
  如果 $(q, r) \cover \pairr{J, \mathcal{G}}$ 并且 $\fall{j : J} \mathcal{G}(j) \cover \pairr{I,\mathcal{F}}$,那么 $(q, r) \cover \pairr{I, \mathcal{F}}$,

  \item \emph{单调性 (monotonicity):}\index{monotonicity!of inductive cover单调性!归纳覆盖的}%
  如果 $(q, r) \subseteq (s, t)$ 并且 $(s,t) \cover \pairr{I, \mathcal{F}}$,那么 $(q, r) \cover \pairr{I, \mathcal{F}}$,

  \item \emph{局部化 (localization):}\index{localization of inductive cover局部化归纳覆盖的}%
  如果 $(q, r) \cover (I, \mathcal{F})$,那么 $(q, r) \cap (s, t) \cover \intfam{i}{I}{(\mathcal{F}(i) \cap (s, t))}$。

  \item \label{defn:inductive-cover-interval-1}
  如果 $q < s < t < r$,那么 $(q, r) \cover [(q, t), (r, s)]$,

  \item \label{defn:inductive-cover-interval-2}
  $(q, r) \cover \intfam{u}{\setof{ (s,t) : \Q \times \Q | q < s < t < r}}{u}$。
\end{enumerate}
\end{defn}

该定义应被视为一个高阶归纳类型 (higher-inductive type),其中列出的规则是点构造器 (point constructors),并且该类型是 $(-1)$ 截断的。前四个条款是一般性的,应该是直观的。最后两条是特定于实数的:一条说,如果它们重叠,则一个区间可以被两个区间覆盖,而另一条说,一个区间可以从内部覆盖。如果 $r \leq q$,则根据最后一条规则,$(q, r)$ 被空族覆盖。

归纳覆盖享有海涅-博雷尔性质,其证明需要一个引理。

\begin{lem} \label{reals-formal-topology-locally-compact}
假设 $q < s < t < r$ 并且 $(q, r) \cover \pairr{I, \mathcal{F}}$。那么仅存在 $\pairr{I, \mathcal{F}}$ 的一个有限子族,它归纳覆盖 $(s, t)$。
\end{lem}

\begin{proof}
  我们通过归纳证明 $(q, r) \cover \pairr{I, \mathcal{F}}$。有六种情况:
  %
  \begin{enumerate}

    \item 自反性:如果 $(q, r) = \mathcal{F}(i)$,那么根据单调性 $(s, t)$ 被有限子族 $[\mathcal{F}(i)]$ 覆盖。

    \item 传递性:
    假设 $(q, r) \cover \pairr{J, \mathcal{G}}$ 并且 $\fall{j : J} \mathcal{G}(j) \cover \pairr{I, \mathcal{F}}$。根据归纳假设,仅存在 $[\mathcal{G}(j_1), \ldots, \mathcal{G}(j_n)]$ 覆盖 $(s, t)$。
    再次根据归纳假设,它们中的每一个都被 $\pairr{I, \mathcal{F}}$ 的一个有限子族覆盖,并且我们可以将这些子族收集成一个有限子族覆盖 $(s, t)$。

    \item 单调性:
    如果 $(q, r) \subseteq (u, v)$ 并且 $(u, v) \cover \pairr{I, \mathcal{F}}$,那么我们可以应用归纳假设到 $(u, v) \cover \pairr{I, \mathcal{F}}$,因为 $u < s < t < v$。

    \item 局部化:
    假设 $(q', r') \cover \pairr{I, \mathcal{F}}$ 并且 $(q, r) = (q', r') \cap (a, b)$。
    因为 $q' < s < t < r'$,根据归纳假设,存在一个有限子覆盖 $[\mathcal{F}(i_1), \ldots, \mathcal{F}(i_n)]$ 覆盖 $(s, t)$。我们还知道 $a < s < t < b$,因此 $(s, t) = (s, t) \cap (a, b)$ 被 $[\mathcal{F}(i_1) \cap (a,b), \ldots, \mathcal{F}(i_n) \cap (a,b)]$ 覆盖,这是 $\intfam{i}{I}{(\mathcal{F}(i) \cap (a, b))}$ 的一个有限子族。

    \item 如果 $(q, r) \cover [(q, v), (u, r)]$ 对于某个 $q < u < v < r$,那么根据单调性 $(s, t) \cover [(q, v), (u, r)]$。

    \item 最后,通过自反性 $(s, t) \cover \intfam{z}{\setof{ (u,v):\Q \times \Q | q < u < v < r}}{z}$。 \qedhere
  \end{enumerate}
\end{proof}

说 $\pairr{I, \mathcal{F}}$ 归纳覆盖 $[a, b]$ 当仅存在 $\epsilon : \Qp$,使得 $(a - \epsilon, b + \epsilon) \cover \pairr{I, \mathcal{F}}$。

\begin{cor} \label{interval-Heine-Borel}
\index{compactness!Heine-Borel紧致性!海涅-博雷尔}%
\index{interval!open and closed区间!开和闭}%
闭区间对归纳覆盖是海涅-博雷尔紧致的。
\end{cor}

\begin{proof}
  假设 $[a, b]$ 被 $\pairr{I, \mathcal{F}}$ 归纳覆盖,因此仅存在 $\epsilon : \Qp$,使得 $(a - \epsilon, b + \epsilon) \cover \pairr{I, \mathcal{F}}$。根据 \cref{reals-formal-topology-locally-compact} 存在一个有限子覆盖 $(a - \epsilon/2, b + \epsilon/2)$,因此是 $[a, b]$ 的有限子覆盖。
\end{proof}

形式拓扑的经验 (Experience from formal topology)\index{topology!formal拓扑!形式} 表明,归纳覆盖的规则对于构造性的无点拓扑 (pointfree topology) 发展是足够的。但我们也可以提供自己的证据,证明它们是一个合理的概念。

\begin{thm} \label{inductive-cover-classical}
\mbox{}
%
\begin{enumerate}
  \item 一个归纳覆盖也是一个逐点覆盖。
  \item 假设排中律,一个逐点覆盖也是一个归纳覆盖。
\end{enumerenumerate}
\end{thm}

\begin{proof}
  \mbox{}
  %
  \begin{enumerate}

    \item
    考虑一个基本区间族 $\pairr{I, \mathcal{F}}$,其中我们写 $(q_i, r_i) \defeq \mathcal{F}(i)$,一个由 $\pairr{I, \mathcal{F}}$ 逐点覆盖的区间 $(a,b)$ 和 $x$,使得 $a < x < b$。
    我们通过 $(a,b) \cover \pairr{I, \mathcal{F}}$ 的归纳证明,仅存在 $i : I$,使得 $q_i < x < r_i$。大多数情况都很明显,所以我们只展示两个。如果 $(a,b) \cover \pairr{I, \mathcal{F}}$ 是通过自反性覆盖的,那么仅存在某个 $i : I$,使得 $(a,b) = (q_i, r_i)$,因此 $q_i < x < r_i$。如果 $(a,b) \cover \pairr{I, \mathcal{F}}$ 是通过 $\intfam{j}{J}{(s_j, t_j)}$ 的传递性覆盖的,那么根据归纳假设,仅存在 $j : J$,使得 $s_j < x < t_j$,然后因为 $(s_j, t_j) \cover \pairr{I, \mathcal{F}}$,再次根据归纳假设,仅存在 $i : I$,使得 $q_i < x < r_i$。其他情况同样激动人心。

    \item 假设 $\intfam{i}{I}{(q_i, r_i)}$ 逐点覆盖 $(a, b)$。根据 \cref{defn:inductive-cover-interval-2} 和 \cref{defn:inductive-cover} 的定义,只需证明 $\intfam{i}{I}{(q_i, r_i)}$ 归纳覆盖 $(c, d)$,只要 $a < c < d < b$,所以考虑这样的 $c$ 和 $d$。根据 \cref{classical-Heine-Borel},存在一个有限子族 $[i_1, \ldots, i_n]$,它已经逐点覆盖 $[c, d]$,因此 $(c,d)$。设 $\epsilon : \Qp$ 为 $[c, d]$ 的 $[(q_{i_1}, r_{i_1}), \ldots, (q_{i_n}, r_{i_n})]$ 的 Lebesgue 数 (Lebesgue number)\index{Lebesgue number}%
    ,如 \cref{ex:finite-cover-lebesgue-number} 中所示。存在一个正数 $k : \N$,使得 $2 (d - c)/k < \min(1, \epsilon)$。对于 $0 \leq i \leq k$,设
    %
    \begin{equation*}
      c_k \defeq ((k - i) c + i d) / k。
    \end{equation*}
    %
    区间 $(c_0, c_2)$、$(c_1, c_3)$、……、$(c_{k-2}, c_k)$ 通过反复使用传递性和 \cref{defn:inductive-cover-interval-1} 归纳覆盖 $(c,d)$。由于它们的宽度低于 $\epsilon$,因此每个都包含在某个 $(q_i, r_i)$ 中,并且我们可以使用传递性和单调性得出 $\intfam{i}{I}{(q_i, r_i)}$ 归纳覆盖 $(c, d)$。 \qedhere
  \end{enumerenumerate}
\end{proof}

前述定理的结果是,就经典数学而言,逐点覆盖和归纳覆盖之间没有区别。特别是,由于在同伦类型论中假设排中律是自洽的,我们不能展示一个归纳覆盖无法逐点覆盖。或者换句话说,逐点覆盖和归纳覆盖之间的区别不在于它们覆盖什么,而在于它们\emph{证明}它们覆盖的内容。

我们可以写另一本书来继续讨论这些内容,但让我们在这里停止,希望我们已经提供了充分的理由来证明分析可以在同伦类型论中进行发展。好奇的读者应参阅 \cref{ex:mean-value-theorem},以了解中值定理 (intermediate value theorem) 的构造性版本。

\index{acceptance|接受|)}

\index{mathematics!classical|数学!经典|}%
\index{mathematics!constructive|数学!构造性|}%

\section{The surreal numbers}
\label{sec:surreals}

\index{surreal numbers|(}%

In this section we consider another example of a higher inductive-in\-duc\-tive type, which draws together many of our threads: Conway's field \NO of \emph{surreal numbers}~\cite{conway:onag}.
The surreal numbers are the natural common generalization of the (Dedekind) real numbers (\cref{sec:dedekind-reals}) and the ordinal numbers (\cref{sec:ordinals}).
Conway, working in classical\index{mathematics!classical} mathematics with excluded middle and Choice, defines a surreal number to be a pair of \emph{sets} of surreal numbers, written $\surr L R$, such that every element of $L$ is strictly less than every element of $R$.
This obviously looks like an inductive definition, but there are three issues with regarding it as such.

Firstly, the definition requires the relation of (strict) inequality between surreals, so that relation must be defined simultaneously with the type \NO of surreals.
(Conway avoids this issue by first defining \emph{games}\index{game!Conway}, which are like surreals but omit the compatibility condition on $L$ and $R$.)
As with the relation $\closesym$ for the Cauchy reals, this simultaneous definition could \emph{a priori} be either inductive-inductive or inductive-recursive.
We will choose to make it inductive-inductive, for the same reasons we made that choice for $\closesym$.

Moreover, we will define strict inequality $<$ and non-strict inequality $\le$ for surreals separately (and mutually inductively).
Conway defines $<$ in terms of $\le$, in a way which is sensible classically but not constructively.
\index{mathematics!constructive}%
Furthermore, a negative definition of $<$ would make it unacceptable as a hypothesis of the constructor of a higher inductive type (see \cref{sec:strictly-positive}).

Secondly, Conway says that $L$ and $R$ in $\surr L R$ should be ``sets of surreal numbers'', but the naive meaning of this as a predicate $\NO\to\prop$ is not positive, hence cannot be used as input to an inductive constructor.
However, this would not be a good type-theoretic translation of what Conway means anyway, because in set theory the surreal numbers form a proper class, whereas the sets $L$ and $R$ are true (small) sets, not arbitrary subclasses of \NO.
In type theory, this means that \NO will be defined relative to a universe \UU, but will itself belong to the next higher universe $\UU'$, like the sets \ord and \card of ordinals and cardinals, the cumulative hierarchy $V$, or even the Dedekind reals in the absence of propositional resizing.
\index{propositional!resizing}%
We will then require the ``sets'' $L$ and $R$ of surreals to be \UU-small, and so it is natural to represent them by \emph{families} of surreals indexed by some \UU-small type.
(This is all exactly the same as what we did with the cumulative hierarchy in \cref{sec:cumulative-hierarchy}.)
That is, the constructor of surreals will have type
\[ \prd{\LL,\RR:\UU} (\LL\to\NO) \to (\RR\to \NO) \to (\text{some condition}) \to \NO \]
which is indeed strictly positive.\index{strict!positivity}

Finally, after giving the mutual definitions of \NO and its ordering, Conway declares two surreal numbers $x$ and $y$ to be \emph{equal} if $x\le y$ and $y\le x$.
This is naturally read as passing to a quotient of the set of ``pre-surreals'' by an equivalence relation.
%(In set-theoretic foundations, one has to us an additional trick to deal with large equivalence classes.)
However, in the absence of the axiom of choice, such a quotient presents the same problem as the quotient in the usual construction of Cauchy reals: it will no longer be the case that a pair of families \emph{of surreals} yield a new surreal $\surr L R$, since we cannot necessarily ``lift'' $L$ and $R$ to families of pre-surreals.
Of course, we can solve this problem in the same way we did for Cauchy reals, by using a \emph{higher} inductive-inductive definition.

\begin{defn}\label{defn:surreals}
  The type \NO of \define{surreal numbers},
  \indexdef{surreal numbers}%
  \indexsee{number!surreal}{surreal numbers}%
  along with the relations $\mathord<:\NO\to\NO\to\type$ and $\mathord\le:\NO\to\NO\to\type$, are defined higher inductive-inductively as follows.
  The type \NO has the following constructors.
  \begin{itemize}
  \item For any $\LL,\RR:\UU$ and functions $\LL\to \NO$ and $\RR\to \NO$, whose values we write as $x^L$ and $x^R$ for $L:\LL$ and $R:\RR$ respectively, if $\fall{L:\LL}{R:\RR} x^L<x^R$, then there is a surreal number $x$.
  \item For any $x,y:\NO$ such that $x\le y$ and $y\le x$, we have $\noeq(x,y):x=y$.
  \end{itemize}
  We will refer to the inputs of the first constructor as a \define{cut}.
  \indexdef{cut!of surreal numbers}%
  If $x$ is the surreal number constructed from a cut, then the notation $x^L$ will implicitly assume $L:\LL$, and similarly $x^R$ will assume $R:\RR$.
  In this way we can usually avoid naming the indexing types $\LL$ and $\RR$, which is convenient when there are many different cuts under discussion.
  Following Conway, we call $x^L$ a \emph{left option}\indexdef{option of a surreal number} of $x$ and $x^R$ a \emph{right option}.

  The path constructor implies that different cuts can define the same surreal number.
  Thus, it does not make sense to speak of the left or right options of an arbitrary surreal number $x$, unless we also know that $x$ is defined by a particular cut.
  Thus in what follows we will say, for instance, ``given a cut defining a surreal number $x$'' in contrast to ``given a surreal number $x$''.

  The relation $\le$ has the following constructors.
  \index{non-strict order}%
  \index{order!non-strict}%
  \begin{itemize}
  \item Given cuts defining two surreal numbers $x$ and $y$, if $x^L<y$ for all $L$, and $x<y^R$ for all $R$, then $x\le y$.
  \item Propositional truncation:
    for any $x,y:\NO$, if $p,q:x\le y$, then $p=q$.
  \end{itemize}
  And the relation $<$ has the following constructors.
  \index{strict!order}%
  \index{order!strict}%
  \begin{itemize}
    % Don't technically need x in the first one and y in the second one to be defined by cuts?
  \item Given cuts defining two surreal numbers $x$ and $y$, if there is an $L$ such that $x\le y^L$, then $x<y$.
  \item Given cuts defining two surreal numbers $x$ and $y$, if there is an $R$ such that $x^R\le y$, then $x<y$.
  \item Propositional truncation: for any $x,y:\NO$, if $p,q:x<y$, then $p=q$.
  \end{itemize}
\end{defn}

\noindent
We compare this with Conway's definitions:
\begin{itemize}\footnotesize
\item[-] If $L,R$ are any two sets of numbers, and no member of $L$ is $\ge$ any member of $R$, then there is a number $\surr L R$.
  All numbers are constructed in this way.
\item[-] $x\ge y$ iff (no $x^R\le y$ and $x\le$ no $y^L$).
\item[-] $x=y$ iff ($x \ge y$ and $y\ge x$).
\item[-] $x>y$ iff ($x\ge y$ and $y\not\ge x$).
\end{itemize}
The inclusion of $x\ge y$ in the definition of $x>y$ is unnecessary if all objects are [surreal] numbers rather than ``games''\index{game!Conway}.
Thus, Conway's $<$ is just the negation of his $\ge$, so that his condition for $\surr L R$ to be a surreal is the same as ours.
Negating Conway's $\le$ and canceling double negations, we arrive at our definition of $<$, and we can then reformulate his $\le$ in terms of $<$ without negations.

We can immediately populate $\NO$ with many surreal numbers.
Like Conway, we write
\symlabel{surreal-cut}
\[\surr{x,y,z,\dots}{u,v,w,\dots}\]
for the surreal number defined by a cut where $\LL\to\NO$ and $\RR\to\NO$ are families described by $x,y,z,\dots$ and $u,v,w,\dots$.
Of course, if $\LL$ or $\RR$ are $\emptyt$, we leave the corresponding part of the notation empty.
There is an unfortunate clash with the standard notation $\setof{x:A | P(x)}$ for subsets, but we will not use the latter in this section.
\begin{itemize}
\item We define $\iota_{\nat}:\nat\to\NO$ recursively by
  \begin{align*}
    \iota_{\nat}(0) &\defeq \surr{}{},\\
    \iota_\nat(\suc(n)) &\defeq \surr{\iota_\nat(n)}{}.
  \end{align*}
  That is, $\iota_\nat(0)$ is defined by the cut consisting of $\emptyt\to\NO$ and $\emptyt\to\NO$.
  Similarly, $\iota_\nat(\suc(n))$ is defined by $\unit\to\NO$ (picking out $\iota_\nat(n)$) and $\emptyt\to\NO$.
\item Similarly, we define $\iota_{\Z}:\Z\to\NO$ using the sign-case recursion principle (\cref{thm:sign-induction}):
  \begin{align*}
    \iota_{\Z}(0) &\defeq \surr{}{},\\
    \iota_\Z(n+1) &\defeq \surr{\iota_\Z(n)}{} & &\text{$n\ge 0$,}\\
    \iota_\Z(n-1) &\defeq \surr{}{\iota_\Z(n)} & &\text{$n\le 0$.}
  \end{align*}
\item By a \define{dyadic rational}
  \indexdef{rational numbers!dyadic}%
  \indexsee{dyadic rational}{rational numbers, dyadic}%
  we mean a pair $(a,n)$ where $a:\Z$ and $n:\nat$, and such that if $n>0$ then $a$ is odd.
  We will write it as $a/2^n$, and identify it with the corresponding rational number.
  If $\Q_D$ denotes the set of dyadic rationals, we define $\iota_{\Q_D}:\Q_D\to\NO$ by induction on $n$:
  \begin{align*}
    \iota_{\Q_D}(a/2^0) &\defeq \iota_\Z(a),\\
    \iota_{\Q_D}(a/2^n) &\defeq \surr{\iota_{\Q_D}(a/2^n - 1/2^n)}{\iota_{\Q_D}(a/2^n + 1/2^n)},
    \quad \text{for $n>0$.}
  \end{align*}
  Here we use the fact that if $n>0$ and $a$ is odd, then $a/2^n \pm 1/2^n$ is a dyadic rational with a smaller denominator than $a/2^n$.
\item We define $\iota_{\RD}:\RD\to\NO$,\label{reals-into-surreals} where $\RD$ is (any version of) the Dedekind reals from \cref{sec:dedekind-reals}, by
  \begin{align*}
    \iota_{\RD}(x) &\defeq
    \surr{q\in\Q_D \text{ such that } q<x}{q\in\Q_D \text{ such that } x<q}.
  \end{align*}
  Unlike in the previous cases, it is not obvious that this extends $\iota_{\Q_D}$ when we regard dyadic rationals as Dedekind reals.
  This follows from the simplicity theorem (\cref{thm:NO-simplicity}).
\item Recall the type \ord of \emph{ordinals}\index{ordinal} from \cref{sec:ordinals}, which is well-ordered by the relation $<$, where $A<B$ means that $A = \ordsl B b$ for some $b:B$.
  We define $\iota_{\ord}:\ord\to\NO$\label{ord-into-surreals} by well-founded recursion (\cref{thm:wfrec}) on $\ord$:
  \begin{equation*}
    \iota_{\ord}(A) \defeq
    \surr{\iota_\ord(\ordsl A a) \text{ for all } a:A}{}.
  \end{equation*}
  It will also follow from the simplicity theorem that $\iota_\ord$ restricted to finite ordinals agrees with $\iota_\nat$.
  (We caution the reader, however, that unlike the above examples, $\iota_\ord$ is not constructively injective unless we restrict it to a smaller class of ordinals; see \cref{ex:ord-into-surreals,ex:hiit-plump}.)
\item A few more interesting examples taken from Conway:
  \begin{align*}
    \omega &\defeq \surr{0,1,2,3,\dots}{} \qquad\text{(also an ordinal)}\\
    -\omega &\defeq \surr{}{\dots,-3,-2,-1,0}\\
    1/\omega &\defeq \textstyle\surr{0}{1,\frac12,\frac14,\frac18,\dots}\\
    \omega-1 &\defeq \surr{0,1,2,3,\dots}{\omega}\\
    \omega/2 &\defeq \surr{0,1,2,3,\dots}{\dots,\omega-2,\omega-1,\omega}.
  \end{align*}
\end{itemize}

In identifying surreal numbers presented by different cuts, the following simple observation is useful.

\begin{thm}[Conway's simplicity theorem]\label{thm:NO-simplicity}
  \index{simplicity theorem}%
  \index{theorem!Conway's simplicity}%
  Suppose $x$ and $z$ are surreal numbers defined by cuts, and that the following hold.
  \begin{itemize}
  \item $x^L < z < x^R$ for all $L$ and $R$.
  \item For every left option $z^L$ of $z$, there exists a left option $x^{L'}$ with $z^L\le x^{L'}$.
  \item For every right option $z^R$ of $z$, there exists a right option $x^{R'}$ with $x^{R'}\le z^R$.
  \end{itemize}
  Then $x=z$.
\end{thm}
\begin{proof}
  Applying the path constructor of $\NO$, we must show $x\le z$ and $z\le x$.
  The first entails showing $x^L<z$ for all $L$, which we assumed, and $x<z^R$ for all $R$.
  But by assumption, for any $z^R$ there is an $x^{R'}$ with $x^{R'}\le z^R$ hence $x<z^R$ as desired.
  Thus $x\le z$; the proof of $z\le x$ is symmetric.
\end{proof}

\index{induction principle!for surreal numbers}
In order to say much more about surreal numbers, however, we need their induction principle.
The mutual induction principle for $(\NO,\le,<)$ applies to three families of types:
\begin{align*}
  A &: \NO\to\type\\
  B &: \prd{x,y:\NO}{a:A(x)}{b:A(y)} (x\le y) \to \type\\
  C &: \prd{x,y:\NO}{a:A(x)}{b:A(y)} (x<y) \to \type.
\end{align*}
As with the induction principle for Cauchy reals, it is helpful to think of $B$ and $C$ as families of relations between the types $A(x)$ and $A(y)$.
\symlabel{NO-recursion}
Thus we write $B(x,y,a,b,\xi)$ as $(x,a) \ble^\xi (y,b)$ and $C(x,y,a,b,\xi)$ as $(x,a) \blt^\xi (y,b)$.
Similarly, we usually omit the $\xi$ since it inhabits a mere proposition and so is uninteresting, and we may often omit $x$ and $y$ as well, writing simply $a\ble b$ or $a\blt b$.
With these notations, the hypotheses of the induction principle are the following.
\begin{itemize}
\item For any cut defining a surreal number $x$, together with
  \begin{enumerate}
  \item for each $L$, an element $a^L:A(x^L)$, and
  \item for each $R$, an element $a^R:A(x^R)$, such that
  \item for all $L$ and $R$ we have $(x^L,a^L) \blt (x^R,a^R)$
  \end{enumerate}
  there is a specified element $f_a:A(x)$.
  We call such data a \define{dependent cut}
  \indexdef{cut!of surreal numbers!dependent}%
  \indexdef{dependent!cut}%
  over the cut defining~$x$.
\item For any $x,y:\NO$ with $a:A(x)$ and $b:A(y)$, if $x\le y$ and $y\le x$ and also $(x,a) \ble (y,b)$
  and $(y,b) \ble (x,a)$,
  then $\dpath{A}{\noeq}{a}{b}$.
\item Given cuts defining two surreal numbers $x$ and $y$, and dependent cuts $a$ over $x$ and $b$ over $y$, such that for all $L$ we have $x^L<y$ and $(x^L,a^L)\blt (y,f_b)$,
  and for all $R$ we have $x<y^R$ and $(x,f_a) \blt (y^R,b^R)$,
  then $(x,f_a) \ble (y,f_b)$.
\item $\ble$ takes values in mere propositions.
\item Given cuts defining two surreal numbers $x$ and $y$, dependent cuts $a$ over $x$ and $b$ over $y$, and an $L_0$ such that $x\le y^{L_0}$ and $(x,f_a) \ble (y^{L_0},b^{L_0})$,
  we have $(x,f_a) \blt (y,f_b)$.
\item Given cuts defining two surreal numbers $x$ and $y$, dependent cuts $a$ over $x$ and $b$ over $y$, and an ${R_0}$ such that $x^{R_0}\le y$ together with $(x^{R_0},a^{R_0}),\ble (y,f_b)$,
  we have $(x,f_a) \blt (y,f_b)$.
\item $\blt$ takes values in mere propositions.
\end{itemize}
Under these hypotheses we deduce a function $f:\prd{x:\NO} A(x)$ such that
\begin{align}
  f(x) &\;\jdeq\; f_{f[x]} \label{eq:noind1}\\
  (x\le y) &\;\Rightarrow\; (x,f(x)) \ble (y,f(y)) \notag\\
  (x< y) &\;\Rightarrow\; (x,f(x)) \blt (y,f(y)). \notag
\end{align}
In the computation rule~\eqref{eq:noind1} for the point constructor, $x$ is a surreal number defined by a cut, and $f[x]$ denotes the dependent cut over $x$ defined by applying $f$ (and using the fact that $f$ takes $<$ to $\blt$).
As usual, we will generally use pattern-matching notation, where the definition of $f$ on a cut $\surr{x^L}{x^R}$ may use the symbols $f(x^L)$ and $f(x^R)$ and the assumption that they form a dependent cut.

As with the Cauchy reals, we have special cases resulting from trivializing some of $A$, $\ble$, and~$\blt$.
Taking $\ble$ and $\blt$ to be constant at \unit, we have \define{\NO-induction}, which for simplicity we state only for mere properties:
\begin{itemize}
\item Given $P:\NO\to\prop$, if $P(x)$ holds whenever $x$ is a surreal number defined by a cut such that $P(x^L)$ and $P(x^R)$ hold for all
$L$ and $R$, then $P(x)$ holds for all $x:\NO$.
\end{itemize}
This should be compared with Conway's remark:
\begin{quote}\footnotesize
  In general when we wish to establish a proposition $P(x)$ for all numbers $x$, we will prove it inductively by deducing $P(x)$ from the truth of all the propositions $P(x^L)$ and $P(x^R)$.
  We regard the phrase ``all numbers are constructed in this way'' as justifying the legitimacy of this procedure.
\end{quote}
With $\NO$-induction, we can prove

\begin{thm}[Conway's Theorem 0]\label{thm:NO-refl-opt}\
  \index{theorem!Conway's 0}%
  \begin{enumerate}
  \item For any $x:\NO$, we have $x\le x$.\label{item:NO-le-refl}
  \item For any $x:\NO$ defined by a cut, we have $x^L <x$ and $x<x^R$ for all $L$ and $R$.\label{item:NO-lt-opt}
  \end{enumerate}
\end{thm}
\begin{proof}
  Note first that if $x\le x$, then whenever $x$ occurs as a left option of some cut $y$, we have $x<y$ by the first constructor of $<$, and similarly whenever $x$ occurs as a right option of a cut $y$, we have $y<x$ by the second constructor of $<$.
  In particular,~\ref{item:NO-le-refl}$\Rightarrow$\ref{item:NO-lt-opt}.

  We prove~\ref{item:NO-le-refl} by $\NO$-induction on $x$.
  Thus, assume $x$ is defined by a cut such that $x^L\le x^L$ and $x^R \le x^R$ for all $L$ and $R$.
  But by our observation above, these assumptions imply $x^L<x$ and $x<x^R$ for all $L$ and $R$, yielding $x\le x$ by the constructor of $\le$.
\end{proof}

\begin{cor}\label{thm:NO-set}
  \NO is a 0-type.
%  (As with $V$, it might be confusing to say that it is a ``set''.)
\end{cor}
\begin{proof}
  The mere relation $R(x,y)\defeq (x\le y) \land (y\le x)$ implies identity by the path constructor of $\NO$, and contains the diagonal by \cref{thm:NO-refl-opt}\ref{item:NO-le-refl}.
  Thus, \cref{thm:h-set-refrel-in-paths-sets} applies.
\end{proof}

By contrast, Conway's Theorem 1 (transitivity of $\le$) is somewhat harder to establish with our definition; see \cref{thm:NO-unstrict-transitive}.

% Of course, we also have:

% \begin{lem}
%   Every surreal number is merely defined by a cut.
% \end{lem}
% \begin{proof}
%   Obvious by $\NO$-induction.
% \end{proof}

We will also need the joint recursion principle, \define{$(\NO,\le,<)$-recursion}.
It is convenient to state this as follows.
Suppose $A$ is a type equipped with relations $\mathord\ble:A\to A\to\prop$ and $\mathord\blt:A\to A\to\prop$.
Then we can define $f:\NO\to A$ by doing the following.
\begin{enumerate}
\item For any $x$ defined by a cut, assuming $f(x^L)$ and $f(x^R)$ to be defined such that $f(x^L)\blt f(x^R)$ for all $L$ and $R$, we must define $f(x)$.  (We call this the \emph{primary clause} of the recursion.)\label{item:NO-rec-primary}
\item Prove that $\ble$ is \emph{antisymmetric}\index{relation!antisymmetric}: if $a\ble b$ and $b\ble a$, then $a=b$.
\item For $x,y$ defined by cuts such that $x^L<y$ for all $L$ and $x<y^R$ for all $R$, and assuming inductively that $f(x^L)\blt f(y)$ for all $L$, $f(x)\blt f(y^R)$ for all $R$, and also that $f(x^L)\blt f(x^R)$ and $f(y^L)\blt f(y^R)$ for all $L$ and $R$, we must prove $f(x)\ble f(y)$.
\item For $x,y$ defined by cuts and an $L_0$ such that $x\le y^{L_0}$, and assuming inductively that $f(x)\ble f(y^{L_0})$, and also that $f(x^L)\blt f(x^R)$ and $f(y^L)\blt f(y^R)$ for all $L$ and $R$, we must prove $f(x)\blt f(y)$.
\item For $x,y$ defined by cuts and an $R_0$ such that $x^{R_0}\le y$, and assuming inductively that $f(x^{R_0})\ble f(y)$, and also that $f(x^L)\blt f(x^R)$ and $f(y^L)\blt f(y^R)$ for all $L$ and $R$, we must prove $f(x)\blt f(y)$.\label{item:NO-rec-last}
\end{enumerate}
The last three clauses can be more concisely described by saying we must prove that $f$ (as defined in the first clause) takes $\le$ to $\ble$ and $<$ to $\blt$.
We will refer to these properties by saying that \emph{$f$ preserves inequalities}.
Moreover, in proving that $f$ preserves inequalities, we may assume the particular instance of $\le$ or $<$ to be obtained from one of its constructors, and we may also use inductive hypotheses that $f$ preserves all inequalities appearing in the input to that constructor.

If we succeed at~\ref{item:NO-rec-primary}--\ref{item:NO-rec-last} above, then we obtain $f:\NO\to A$, which computes on cuts as specified by~\ref{item:NO-rec-primary}, and which preserves all inequalities:
%
\begin{narrowmultline*}
  \fall{x,y:\NO}\Big((x\le y) \to (f(x)\ble f(y))\Big) \land
  \narrowbreak
  \Big((x< y) \to (f(x)\blt f(y))\Big).
\end{narrowmultline*}
%
Like $(\RC,\closesym)$-recursion for the Cauchy reals, this recursion principle is essential for defining functions on $\NO$, since we cannot first define a function on ``pre-surreals'' and only later prove that it respects the notion of equality.

\begin{eg}
  Let us define the \emph{negation} function $\NO\to\NO$.
  We apply the joint recursion principle with $A\defeq\NO$, with $(x\ble y)\defeq (y\le x)$, and $(x\blt y)\defeq (y< x)$.
  Clearly this $\ble$ is antisymmetric.

  For the main clause in the definition, we assume $x$ defined by a cut, with $-x^L$ and $-x^R$ defined such that $-x^L \blt -x^R$ for all $L$ and $R$.
  By definition, this means $-x^R< -x^L$ for all $L$ and $R$, so we can define $-x$ by the cut $\surr{-x^R}{-x^L}$.
  This notation, which follows Conway, refers to the cut whose left options are indexed by the type $\RR$ indexing the right options of $x$, and whose right options are indexed by the type $\LL$ indexing the left options of $x$, with the corresponding families $\RR\to\NO$ and $\LL\to\NO$ defined by composing those for $x$ with negation.

  We now have to verify that $f$ preserves inequalities.
  \begin{itemize}
  \item For $x\le y$, we may assume $x^L<y$ for all $L$ and $x < y^R$ for all $R$, and show $-y\le -x$.
    But inductively, we may assume $-y <-x^L$ and $-y^R<-x$, which gives the desired result, by definition of $-y$, $-x$, and the constructor of $\le$.
  \item For $x<y$, in the first case when it arises from some $x\le y^{L_0}$, we may inductively assume $-y^{L_0} \le -x$, in which case $-y<-x$ follows by the constructor of $<$.
  \item Similarly, if $x<y$ arises from $x^{R_0}\le y$, the inductive hypothesis is $-y \le -x^R$, yielding $-y<-x$ again.
  \end{itemize}
\end{eg}

To do much more than this, however, we will need to characterize the relations $\le$ and $<$ more explicitly, as we did for the Cauchy reals in \cref{thm:RC-sim-characterization}.
Also as there, we will have to simultaneously prove a couple of essential properties of these relations, in order for the induction to go through.

\begin{thm}\label{defn:No-codes}
  There are relations $\mathord\preceq:\NO\to\NO\to\prop$ and $\mathord\prec:\NO\to\NO\to\prop$ such that if $x$ and $y$ are surreals defined by cuts, then
  \begin{align*}
    (x\preceq y) &\defeq
    \big(\fall{L} x^L\prec y\big) \land \big(\fall{R} x\prec y^R\big)\\
    (x\prec y) &\defeq
    \big(\exis{L} x\preceq y^L\big) \lor \big(\exis{R} x^R \preceq y\big).
  \end{align*}
  Moreover, we have
  \begin{equation}\label{eq:NO-codes-unstrict}
    (x\prec y) \to (x\preceq y)
  \end{equation}
  and all the reasonable transitivity properties making $\prec$ and $\preceq$ into a ``bimodule''\index{bimodule} over $\le$ and $<$:
  \begin{equation}\label{eq:NO-codes-transitivity}
    \begin{array}{c@{\hspace{1cm}}c}
      (x \le y) \to (y\preceq z) \to (x\preceq z) &
      (x \preceq y) \to (y\le z) \to (x\preceq z) \\
      (x \le y) \to (y\prec z) \to (x\prec z) &
      (x \preceq y) \to (y< z) \to (x\prec z) \\
      (x < y) \to (y\preceq z) \to (x\prec z) &
      (x \prec y) \to (y\le z) \to (x\prec z).
  \end{array}
  \end{equation}
\end{thm}

\begin{proof}
  We define $\preceq$ and $\prec$ by double $(\NO,\le,<)$-induction on $x,y$.
  The first induction is a simple recursion, whose codomain is the subset $A$ of $(\NO\to\prop)\times (\NO\to\prop)$ consisting of pairs of predicates of which one implies the other and which satisfy ``transitivity on the right'', i.e.~\eqref{eq:NO-codes-unstrict} and the right column of~\eqref{eq:NO-codes-transitivity} with $(x\preceq \blank)$ and $(x\prec \blank)$ replaced by the two given predicates.
  As in the proof of \cref{defn:RC-approx}, we regard these predicates as half of binary relations, writing them as $y\mapsto (\hle y)$ and $y\mapsto (\hlt y)$, with $\hlname$ denoting the pair of relations.
  % The precise definition of $A$ is
  % \begin{align*}
  %   A\defeq \bigg\{ \hlname : (\NO\to\prop)\times (\NO\to\prop) \;\bigg|\;\\
  %   \begin{split}
  %     \fall{y,z:\NO}
  %     &\Big( (\hle y) \to (y\le z) \to (\hle z) \Big)\\
  %     \land\; &\Big( (\hle y) \to (y< z) \to (\hlt z) \Big)\\
  %     \land\; &\Big( (\hlt y) \to (y\le z) \to (\hlt z) \Big)\\
  %     \land\; &\Big( (\hlt y) \to (y< z) \to (\hlt z) \Big) \bigg\}
  %   \end{split}
  % \end{align*}
  We equip $A$ with the following two relations:
  \begin{align*}
    (\hlname \ble \hlbname) &\defeq
    \fall{y:\NO} \Big( (\hleb y) \to (\hle y) \Big) \land
    \Big( (\hltb y) \to (\hlt y) \Big),\\
    (\hlname \blt \hlbname) &\defeq
    \fall{y:\NO} \Big( (\hleb y) \to (\hlt y) \Big).
    %\land \Big( (\hltb y) \to (\hlt y) \Big)
  \end{align*}
  Note that $\ble$ is antisymmetric, since if $\hlname \ble \hlbname$ and $\hlbname \ble \hlname$, then $(\hleb y) \Leftrightarrow (\hle y)$ and $(\hltb y) \Leftrightarrow (\hlt y)$ for all $y$, hence $\hlname=\hlbname$ by univalence for mere propositions and function extensionality.
  Moreover, to say that a function $\NO\to A$ preserves inequalities is exactly to say that, when regarded as a pair of binary relations on $\NO$, it satisfies ``transitivity on the left'' (the left column of~\eqref{eq:NO-codes-transitivity}).

  Now for the primary clause of the recursion, we assume given $x$ defined by a cut, and relations $(x^L \prec \blank)$, $(x^R \prec \blank)$, $(x^L \preceq \blank)$, and $(x^R \preceq \blank)$ for all $L$ and $R$, of which the strict ones imply the non-strict ones, which satisfy transitivity on the right, and such that
  \begin{equation}\label{eq:NO-prec-outer-IH}
    \fall{L,R}{y:\NO}\Big( (x^R\preceq y) \to (x^L \prec y) \Big).
    % \land\Big( (x^R \prec y) \to (x^L \prec y) \Big)
  \end{equation}
  We now have to define $(x\prec y)$ and $(x\preceq y)$ for all $y$.
  Here in contrast to \cref{defn:RC-approx}, rather than a nested recursion, we use a nested induction, in order to be able to inductively use transitivity on the left with respect to the inequalities $x^L<x$ and $x<x^R$.
  Define $A':\NO\to\type$ by taking $A'(y)$ to be the subset $A'$ of $\prop\times\prop$ consisting of two mere propositions, denoted $\tle y$ and $\tlt y$ (with $\tlname:A'(y)$), such that
  \begin{gather}
    (\tlt y) \to (\tle y)\\
    \fall{L} (\tle y)\to (x^L\prec y) \label{eq:NO-prec-IHL}\\
    \fall{R} (x^R \preceq y) \to (\tlt y) \label{eq:NO-prec-IHR}.
  \end{gather}
  Using notation analogous to $\ble$ and $\blt$, we equip $A'$ with the two relations defined for $\tlname:A'(y)$ and $\tlbname:A'(z)$ by
  \begin{align*}
    (\tlname \bble \tlbname) &\defeq
    \Big((\tle y) \to (\tleb z)\Big) \land \Big((\tlt y) \to (\tltb z)\Big)\\
    (\tlname \bblt \tlbname) &\defeq
    \Big((\tle y) \to (\tltb z)\Big). % \land \Big(\tlt \to \tltb\Big).
  \end{align*}
  % (These are the type families $B$ and $C$ in the general induction principle.)
  Again, $\bble$ is evidently antisymmetric in the appropriate sense.
  Moreover, a function $\prd{y:\NO} A'(y)$ which preserves inequalities is precisely a pair of predicates of which one implies the other, which satisfy transitivity on the right, and transitivity on the left with respect to the inequalities $x^L<x$ and $x<x^R$.
  Thus, this inner induction will provide what we need to complete the primary clause of the outer recursion.

  For the primary clause of the inner induction, we assume also given $y$ defined by a cut, and properties $(x\prec y^L)$, $(x\prec y^R)$, $(x\preceq y^L)$, and $(x\preceq y^R)$ for all $L$ and $R$, with the strict ones implying the non-strict ones, transitivity on the left with respect to $x^L<x$ and $x<x^R$, and on the right with respect to $y^L<y^R$.
  % \begin{equation}
  %   \fall{L,R}\Big((x \preceq y^L) \to (x \prec y^R)\Big) % \land \Big((x \prec y^L) \to (x\prec y^R)\Big).
  %   \label{eq:NO-prec-inner-IH}
  % \end{equation}
  We can now give the definitions specified in the theorem statement:
  \begin{align}
    (x\preceq y) &\defeq
    (\fall{L} x^L\prec y) \land (\fall{R} x\prec y^R), \label{eq:NO-preceq-def}\\
    (x\prec y) &\defeq
    (\exis{L} x\preceq y^L) \lor (\exis{R} x^R \preceq y).\label{eq:NO-prec-def}
  \end{align}
  For this to define an element of $A'(y)$, we must show first that $(x\prec y) \to (x\preceq y)$.
  The assumption $x\prec y$ has two cases.
  On one hand, if there is $L_0$ with $x\preceq y^{L_0}$, then by transitivity on the right with respect to $y^{L_0}<y^R$, we have $x\prec y^R$ for all $R$.
  Moreover, by transitivity on the left with respect to $x^L<x$, we have $x^L \prec y^{L_0}$ for any $L$, hence $x^L\prec y$ by transitivity on the right.
  Thus, $x\preceq y$.

  On the other hand, if there is $R_0$ with $x^{R_0}\preceq y$, then by~\eqref{eq:NO-prec-outer-IH}, we have $x^L \prec y$ for all $L$.
  And by transitivity on the left and right with respect to $x<x^{R_0}$ and $y<y^R$, we have $x\prec y^R$ for any $R$.
  Thus, $x\preceq y$.

  We also need to show that these definitions are transitive on the left with respect to $x^L<x$ and $x<x^R$.
  But if $x\preceq y$, then $x^L\prec y$ for all $L$ by definition; while if $x^R\preceq y$, then $x\prec y$ also by definition.

  Thus,~\eqref{eq:NO-preceq-def} and~\eqref{eq:NO-prec-def} do define an element of $A'(y)$.
  We now have to verify that this definition preserves inequalities, as a dependent function into $A'$, i.e.\ that these relations are transitive on the right.
  Remember that in each case, we may assume inductively that they are transitive on the right with respect to all inequalities arising in the inequality constructor.
  \begin{itemize}
  \item Suppose $x\preceq y$ and $y\le z$, the latter arising from $y^L<z$ and $y<z^R$ for all $L$ and $R$.
    Then the inductive hypothesis (of the inner recursion) applied to $y<z^R$ yields $x\prec z^R$ for any $R$.
    Moreover, by definition $x\preceq y$ implies that $x^L \prec y$ for any $L$, so by the inductive hypothesis of the outer recursion we have $x^L \prec z$.
    Thus, $x\preceq z$.
  \item Suppose $x\preceq y$ and $y<z$.
    First, suppose $y<z$ arises from $y\le z^{L_0}$.
    Then the inner inductive hypothesis applied to $y\le z^{L_0}$ yields $x \preceq z^{L_0}$, hence $x\prec z$.

    Second, suppose $y<z$ arises from $y^{R_0}\le z$.
    Then by definition, $x\preceq y$ implies $x\prec y^{R_0}$, and then the inner inductive hypothesis for $y^{R_0}\le z$ yields $x\prec z$.
  \item Suppose $x\prec y$ and $y\le z$, the latter arising from $y^L<z$ and $y<z^R$ for all $L$ and $R$.
    By definition, $x\prec y$ implies there merely exists $R_0$ with $x^{R_0}\preceq y$ or $L_0$ with $x\preceq y^{L_0}$.
    If $x^{R_0}\preceq y$, then the outer inductive hypothesis yields $x^{R_0}\preceq z$, hence $x\prec z$.
    If $x\preceq y^{L_0}$, then the inner inductive hypothesis for $y^{L_0}<z$ (which holds by the constructor of $y\le z$) yields $x\prec z$.
  % \item Suppose $x\prec y$ and $y<z$.
  %   First, suppose $y<z$ arises from $y\le z^{L_0}$.
  %   Then the inner inductive hypothesis for $y\le z^{L_0}$ yields $x\prec z^{L_0}$, hence $x\preceq z^{L_0}$; thus $x\prec z$.

  %   Second, suppose $y<z$ arises from $y^{R_0}\le z$.
  %   Then by definition, $x\prec y$ implies there merely exists $R_1$ with $x^{R_1}\preceq y$ or $L_1$ with $x\preceq y^{L_1}$.
  %   If $x^{R_1}\preceq y$, then the outer inductive hypothesis implies $x^{R_1}\prec z$, hence $x^{R_1}\preceq z$, and thus $x\prec z$.
  %   And if $x\preceq y^{L_1}$, then the inner inductive hypothesis applied to $y^{L_1}<y^{R_0}$ (which comes from $y$ being defined as a cut) and $y^{R_0}\le z$ yields $x\prec z$.
  \end{itemize}
  This completes the inner induction.
  Thus, for any $x$ defined by a cut, we have $(x\prec \blank)$ and $(x\preceq \blank)$ defined by~\eqref{eq:NO-preceq-def} and~\eqref{eq:NO-prec-def}, and transitive on the right.

  To complete the outer recursion, we need to verify these definitions are transitive on the left.
  After a $\NO$-induction on $z$, we end up with three cases that are essentially identical to those just described above for transitivity on the right.
  Hence, we omit them.
\end{proof}

\begin{thm}\label{thm:NO-encode-decode}
  For any $x,y:\NO$ we have $(x<y)=(x\prec y)$ and $(x\le y)=(x\preceq y)$.
\end{thm}
\begin{proof}
  From left to right, we use $(\NO,\le,<)$-induction where $A(x)\defeq\unit$, with $\preceq$ and $\prec$ supplying the relations $\ble$ and $\blt$.
  In all the constructor cases, $x$ and $y$ are defined by cuts, so the definitions of $\preceq$ and $\prec$ evaluate, and the inductive hypotheses apply.

  From right to left, we use $\NO$-induction to assume that $x$ and $y$ are defined by cuts.
  But now the definitions of $\preceq$ and $\prec$, and the inductive hypotheses, supply exactly the data required for the relevant constructors of $\le$ and $<$.
  % From right to left, we first prove by $\NO$-induction on $x$ that for any $y:\NO$ we have $(x\prec y) \to (x<y)$ and $(x\preceq y) \to (x\le y)$.
  % Thus, we assume this to be true for all $x^L$ and $x^R$ in a cut, and show it for the resulting $x:\NO$.
  % Next, we prove by $\NO$-induction on $y$ that $(x\prec y) \to (x<y)$ and $(x\preceq y) \to (x\le y)$, hence we assume it to be true for all $y^L$ and $y^R$ in a cut, and show it for the resulting $y:\NO$.
  % Now since $x$ and $y$ are both defined by cuts, $x\preceq y$ means that $x^L\prec y$ and $x\prec y^R$ for all $L$ and $R$.
  % By the inductive hypotheses, this gives $x^L<y$ and $x<y^R$, hence $x\le y$ by the constructor of $\le$.
  % Similarly, $x\prec y$ yields merely an $R_0$ with $x^{R_0}\preceq y$ or an $L_0$ with $x\preceq y^{L_0}$.
  % Hence merely $x^{R_0}\le y$ or $x\le y^{L_0}$ by the inductive hypothesis, so $x<y$ by a constructor.
\end{proof}

\begin{cor}\label{thm:NO-unstrict-transitive}
  The relations $\le$ and $<$ on $\NO$ satisfy
  \[ \fall{x,y:\NO} (x<y) \to (x\le y) \]
  and are transitive:
  \index{transitivity!of . for surreals@of $<$ for surreals}
  \index{transitivity!of . for surreals@of $\leq$ for surreals}
  \begin{gather*}
    (x\le y) \to (y\le z) \to (x\le z)\\
    (x\le y) \to (y< z) \to (x< z)\\
    (x< y) \to (y\le z) \to (x< z).
  \end{gather*}
\end{cor}

As with the Cauchy reals, the joint $(\NO,\le,<)$-recursion principle remains essential when defining all operations on $\NO$.

\begin{eg}\label{eg:surreal-addition}
\index{addition!of surreal numbers}%
We define $\mathord+:\NO\to\NO\to\NO$ by recursion on the first argument, followed by induction on the second argument.
For the outer recursion, we take the codomain to be the subset of $\NO\to\NO$ consisting of functions $g$ such that $(x<y) \to (g(x)<g(y))$ and $(x\le y) \to (g(x)\le g(y))$ for all $x,y$.
For such $g,h$ we define $(g\ble h)\defeq \fall{x:\NO} g(x)\le h(x)$ and $(g\blt h)\defeq \fall{x:\NO} g(x)< h(x)$.
Clearly $\ble$ is antisymmetric.

For the primary clause of the recursion, we suppose $x$ defined by a cut, that the functions $(x^L+\blank)$ and $(x^R+\blank)$ are defined, preserve inequalities, and satisfy $x^L+y<x^R+y$, and we define $(x+\blank)$.
As in \cref{defn:No-codes}, rather than an inner recursion, we use an inner induction into the family $A:\NO\to\type$, where $A(y)$ is the subset of those $z:\NO$ such that each $x^L + y < z$ and each $x^R + y > z$.
We equip $A$ with the relations $\le$ and $<$ induced from $\NO$, so that antisymmetry is obvious.
For the primary clause of the inner recursion, we suppose also $y$ defined by a cut, with each $x+y^L$ and $x+y^R$ defined and satisfying $x^L+y^L < x+y^L$, $x^L+y^R < x+y^R$, $x+y^L < x^R + y^L$, and $x+y^R < x^R+y^R$ (these come from the additional conditions imposed on elements of $A(y)$), and also $x+y^L < x+y^R$ (since the elements $x+y^L$ and $x+y^R$ of $A(y)$ form a dependent cut).
Now we give Conway's definition:
\[ x+y \defeq \surr{x^L+y, x+y^L}{x^R+y,x+y^R}. \]
In other words, the left options of $x+y$ are all numbers of the form $x^L+y$ for some left option $x^L$, or $x+y^L$ for some left option $y^L$.
We must show that each of these left options is less than each of these right options:
\begin{itemize}
\item $x^L+y < x^R+y$ by the outer inductive hypothesis.
\item $x^L+y < x^L + y^R < x + y^R$, the first since $(x^L+\blank)$ preserves inequalities, and the second since $x+y^R : A(y^R)$.
\item $x+y^L < x^R+ y^L < x^R + y$, the first since $x+y^L : A(y^L)$ and the second since $(x^R+\blank)$ preserves inequalities.
\item $x+y^L < x+y^R$ by the inner inductive hypothesis (specifically, the fact that we have a dependent cut).
\end{itemize}
We also have to show that $x+y$ thusly defined lies in $A(y)$, i.e.\ that $x^L + y < x+y$ and $x+y < x^R + y$; but this is true by \cref{thm:NO-refl-opt}\ref{item:NO-lt-opt}.

Next we have to verify that the definition of $(x+\blank)$ preserves inequality:
\begin{itemize}
\item If $y\le z$ arises from knowing that $y^L<z$ and $y<z^R$ for all $L$ and $R$, then the inner inductive hypothesis gives $x+y^L<x+z$ and $x+y < x+z^R$, while the outer inductive hypotheses give $x^L+y \le x^L+z$ and $x^R+ y \le x^R+z$.
  Moreover, since $x^R+y$ is by definition a right option of $x+y$, we have $x+y < x^R+y$.
  Similarly, we find that $x^L+z$ is a left option of $x+z$, so that $x^L+z < x+z$.
  Thus, using transitivity, we have $x^L+y < x+z$ and $x+y < x^R+z$; so we may conclude $x+y \le x+z$ by the constructor of $\le$.
\item If $y<z$ arises from an $L_0$ with $y\le z^{L_0}$, then inductively $x+y \le x+z^{L_0}$, hence $x+y<x+z$ since $x+z^{L_0}$ is a right option of $x+z$.
\item Similarly, if $y<z$ arises from $y^{R_0}\le z$, then $x+y<x+z$ since $x+y^{R_0}\le x+z$.
\end{itemize}
This completes the inner induction.
For the outer recursion, we have to verify that $+$ preserves inequality on the left as well.
After an $\NO$-induction, this proceeds in exactly the same way.
\end{eg}

\index{acceptance|(}%
\index{mathematics!formalized}%
In the Appendix to Part Zero of~\cite{conway:onag}, Conway discusses how the surreal numbers may be formalized in ZFC set theory: by iterating along the ordinals and passing to sets of representatives of lowest rank for each equivalence class, or by representing numbers with ``sign-expansions''.
He then remarks that
\begin{quote}\footnotesize
  The curiously complicated nature of these constructions tells us more about the nature of formalizations within ZF than about our system of numbers\dots
\end{quote}
and goes on to advocate for a general theory of ``permissible kinds of construction'' which should include
\begin{enumerate}\footnotesize
\item Objects may be created from earlier objects in any reasonably constructive fashion.\label{item:conway1}
\item Equality among the created objects can be any desired equivalence relation.\label{item:conway2}
\end{enumerate}
\noindent
Condition~\ref{item:conway1} can be naturally read as justifying general principles of \emph{inductive definition}, such as those presented in \cref{sec:strictly-positive,sec:generalizations}.
In particular, the condition of strict positivity for constructors can be regarded as a formalization of what it means to be ``reasonably constructive''.
Condition~\ref{item:conway2} then suggests we should extend this to \emph{higher} inductive definitions of all sorts, in which we can impose path constructors making objects equal in any reasonable way.
For instance, in the next paragraph Conway says:
\begin{quote}\footnotesize
  \dots we could also, for instance, freely create a new object $(x,y)$ and call it the ordered pair of $x$ and $y$.
  We could also create an ordered pair $[x,y]$ different from $(x,y)$ but co-existing with it\dots
  If instead we wanted to make $(x,y)$ into an unordered pair, we could define equality by means of the equivalence relation $(x,y)=(z,t)$ if and only if $x=z,y=t$ \emph{or} $x=t,y=z$.
\end{quote}
The freedom to introduce new objects with new names, generated by certain forms of constructors, is precisely what we have in the theory of inductive definitions.
Just as with our two copies of the natural numbers $\nat$ and $\nat'$ in \cref{sec:appetizer-univalence}, if we wrote down an identical definition to the cartesian product type $A\times B$, we would obtain a distinct product type $A\times' B$ whose canonical elements we could freely write as $[x,y]$.
And we could make one of these a type of unordered pairs by adding a suitable path constructor. % (and perhaps 0-truncating).

To be sure, Conway's point was not to complain about ZF in particular, but to argue against all foundational theories at once:
\begin{quote}\footnotesize
  \dots this proposal is not of any particular theory as an alternative to ZF\dots{}
  What is proposed is instead that we give ourselves the freedom to create arbitrary mathematical theories of these kinds, but prove a metatheorem which ensures once and for all that any such theory could be formalized in terms of any of the standard foundational theories.
\end{quote}
One might respond that, in fact, univalent foundations is not one of the ``standard foundational theories'' which Conway had in mind, but rather the \emph{metatheory} in which we may express our ability to create new theories, and about which we may prove Conway's metatheorem.
For instance, the surreal numbers are one of the ``mathematical theories'' Conway has in mind, and we have seen that they can be constructed and justified inside univalent foundations.
Similarly, Conway remarked earlier that
\begin{quote}\footnotesize
  \dots set theory would be such a theory, sets being constructed from earlier ones by processes corresponding to the usual axioms, and the equality relation being that of having the same members.
\end{quote}
This description closely matches the higher-inductive construction of the cumulative hierarchy of set theory in \cref{sec:cumulative-hierarchy}.
Conway's metatheorem would then correspond to the fact we have referred to several times that we can construct a model of univalent foundations inside ZFC (which is outside the scope of this book).

However, univalent foundations is so rich and powerful in its own right that it would be foolish to relegate it to only a metatheory in which to construct set-like theories.
We have seen that even at the level of sets (0-types), the higher inductive types in univalent foundations yield direct constructions of objects by their universal properties (\cref{sec:free-algebras}), such as a constructive theory of Cauchy completion (\cref{sec:cauchy-reals}).
But most importantly, the potential to model homotopy theory and category theory directly in the foundational system (\cref{cha:homotopy,cha:category-theory}) gives univalent foundations an advantage which no set-theoretic foundation can match.
\index{acceptance|)}%

\index{surreal numbers|)}%

\sectionNotes

Defining algebraic operations on Dedekind reals, especially multiplication, is both somewhat tricky and tedious.
There are several ways to get arithmetic going: each has its own advantages, but they all seem to require some technical work.
For instance, Richman~\cite{Richman:reals} defines multiplication on the Dedekind reals first on the positive cuts and then extends it algebraically to all Dedekind cuts, while Conway~\cite{conway:onag} has observed that the definition of multiplication for surreal numbers works well for Dedekind reals.

Our treatment of the Dedekind reals borrows many ideas from~\cite{BauerTaylor09} where the Dedekind reals are constructed in the context of Abstract Stone Duality.
\index{Abstract Stone Duality}%
This is a (restricted) form of simply typed $\lambda$-calculus with a distinguished object $\Sigma$ which classifies open sets, and by duality also the closed ones. In~\cite{BauerTaylor09} you can also find detailed proofs of the basic properties of arithmetical operations.

The fact that $\RC$ is the least Cauchy complete archimedean ordered field, as was proved in \cref{RC-initial-Cauchy-complete}, indicates that our Cauchy reals probably coincide with the Escard{\'o}-Simpson reals~\cite{EscardoSimpson:01}.
\index{real numbers!Escardo-Simpson@Escard\'o-Simpson}%
It would be interesting to check\index{open!problem} whether this is really the case. The notion of Escard{\'o}-Simpson reals, or more precisely the corresponding closed interval, is interesting because it can be stated in any category with finite products.

In constructive set theory augmented by the ``regular extension axiom'', one may also try to define Cauchy completion by closing under limits of Cauchy sequences with a transfinite iteration.
It would also be interesting to check whether this construction agrees with ours.

It is constructive folklore that coincidence of Cauchy and Dedekind reals requires dependent choice but it is less well known that countable choice suffices. Recall that \define{dependent choice}
\indexdef{axiom!of choice!dependent}%
\index{axiom!of choice!countable}%
\index{total!relation}%
states that for a total relation $R$ on $A$, by which we mean $\fall{x : A} \exis{y : A} R(x,y)$, and for any $a : A$ there merely exists $f : \N \to A$ such that $f(0) = a$ and $R(f(n), f(n+1))$ for all $n : \N$. Our \cref{when-reals-coincide} uses the typical trick for converting an application of dependent choice to one using countable choice. Namely, we use countable choice once to make in advance all the choices that could come up, and then use the choice function to avoid the dependent choices.

The intricate relationship between various notions of compactness in a constructive
setting is discussed in \cite{bridges2002compactness}. Palmgren~\cite{Palmgren:FT} has a
good comparison between pointwise analysis and
pointfree topology.

The surreal numbers were defined by~\cite{conway:onag}, using a sort of inductive definition but without justifying it explicitly in terms of any foundational system.
For this reason, some later authors have tended to use sign-expansions or other more explicit presentations which can be coded more obviously into set theory.
The idea of representing them in type theory was first considered by Hancock, while
Setzer and Forsberg~\cite{forsbergfinite} noted that the surreals and their inequality relations $<$ and $\le$ naturally form an inductive-inductive definition.
The \emph{higher} inductive-inductive version presented here, which builds in the correct notion of equality for surreals, is new.


\sectionExercises

\begin{ex}\label{ex:alt-dedekind-reals}
 Give an alternative definition of the Dedekind reals by first defining the square and then use \cref{mult-from-square}.
 Check that one obtains a commutative ring.
\end{ex}

\begin{ex} \label{ex:RD-extended-reals}
  %
  Suppose we remove the boundedness condition
  \ref{defn:dedekind-reals-inhabited} in \cref{defn:dedekind-reals}.
  Then we obtain the \define{extended reals}
  \indexdef{real numbers!extended}%
  \indexdef{extended real numbers}%
  which contain $-\infty \defeq
  (\emptyt, \Q)$ and $\infty \defeq (\Q, \emptyt)$. Which definitions of arithmetical
  operations on cuts still make sense for extended reals? What algebraic structure do we
  get?
\end{ex}

\begin{ex} \label{ex:RD-lower-cuts}
  %
  By considering one-sided cuts we obtain \define{lower} and \define{upper} Dedekind reals,
  \indexdef{real numbers!Dedekind!upper}%
  \indexdef{real numbers!Dedekind!lower}%
  \indexdef{lower Dedekind reals}%
  \indexdef{upper Dedekind reals}%
  \index{cut!Dedekind}%
  respectively. For example, a lower real is given by a predicate $L : \Q \to \Omega$
  which is
  %
  \begin{enumerate}
  \item \emph{inhabited:} $\exis{q : \Q} L(q)$ and
  \item \emph{rounded:} $L(q) = \exis{r : \Q} q < r \land L(r)$.
    \index{rounded!Dedekind cut}
  \end{enumerate}
  %
  (We could also require $\exis{r : \Q} \lnot L(r)$ to exclude the cut $\infty \defeq
  \Q$.) Which arithmetical operations can you define on the lower reals? In particular,
  what happens with the additive inverse?
\end{ex}

\begin{ex} \label{ex:RD-interval-arithmetic}
  %
  \index{interval!arithmetic}%
  Suppose we remove the locatedness condition in \cref{defn:dedekind-reals}.
  Then we obtain the \define{interval domain}
  \indexdef{interval!domain}%
  $\mathbb{I}$ because cuts are allowed
  to have ``gaps'', which are just intervals. Define the partial order $\sqsubseteq$ on
  $\mathbb{I}$ by
  %
  \begin{narrowmultline*}
    ((L, U) \sqsubseteq (L', U'))
    \defeq \narrowbreak
    (\fall{q : \Q} L(q) \Rightarrow L'(q)) \land
    (\fall{q : \Q} U(q) \Rightarrow U'(q)).
  \end{narrowmultline*}
  %
  What are the maximal elements of $\mathbb{I}$ with respect to $\mathbb{I}$? Define the
  ``endpoint'' operations which assign to an element of the interval domain its lower and
  upper endpoints. Are the endpoints reals, lower reals, or upper reals (see
  \cref{ex:RD-lower-cuts})? Which definitions of arithmetical operations on cuts still
  make sense for the interval domain?
\end{ex}

\begin{ex} \label{ex:RD-lt-vs-le}
  Show that, for all $x, y : \RD$,
  %
  \begin{equation*}
    \lnot (x < y) \Rightarrow y \leq x
  \end{equation*}
  %
  and
  %
  \begin{equation*}
    \eqv{(x \leq y)}{\Parens{\prd{\epsilon : \Qp} x < y + \epsilon}}.
  \end{equation*}
  %
  Does $\lnot (x \leq y)$ imply $y < x$?
\end{ex}

\begin{ex} \label{ex:reals-non-constant-into-Z}
  \mbox{}
  %
  \begin{enumerate}
  \item
    Assuming excluded middle, construct a non-constant map $\RD \to \Z$.
  \item
    Suppose $f : \RD \to \Z$ is a map such that $f(0) = 0$ and $f(x) \neq 0$ for all $x >
    0$. Derive from this the limited principle of omniscience~\eqref{eq:lpo}.
\index{limited principle of omniscience}%
  \end{enumerate}
\end{ex}

\begin{ex} \label{ex:traditional-archimedean}
  \index{ordered field!archimedean}%
  Show that in an ordered field $F$, density of $\Q$ and the traditional archimedean axiom
  are equivalent:
  %
  \begin{equation*}
    (\fall{x, y : F} x < y \Rightarrow \exis{q : \Q} x < q < y)
    \Leftrightarrow
    (\fall{x : F} \exis{k : \Z} x < k).
  \end{equation*}
\end{ex}

\begin{ex} \label{RC-Lipschitz-on-interval} Suppose $a, b : \Q$ and $f : \setof{ q : \Q |
    a \leq q \leq b } \to \RC$ is Lipschitz with constant~$L$. Show that there exists a unique
  extension $\bar{f} : [a,b] \to \RC$ of $f$ which is Lipschitz with
  constant~$L$. Hint: rather than redoing \cref{RC-extend-Q-Lipschitz} for closed
  intervals, observe that there is a retraction $r : \RC \to [-n,n]$ and apply
  \cref{RC-extend-Q-Lipschitz} to $f \circ r$.
\end{ex}

\begin{ex} \label{ex:metric-completion}
  \index{completion!of a metric space}%
  Generalize the construction of $\RC$ to construct the Cauchy completion of any metric space. First, think about which notion of real numbers is most natural as the codomain for the distance\index{distance} function of a metric space. Does it matter? Next, work out the details of two constructions:
  %
  \begin{enumerate}
  \item Follow the construction of Cauchy reals to define the completion of a metric space as an inductive-inductive type closed under limits of Cauchy sequences.\index{Cauchy!sequence}
  \item Use the following construction due to Lawvere~\cite{lawvere:metric-spaces}\index{Lawvere} and Richman~\cite{Richman00thefundamental}, where the completion of a metric space $(M, d)$ is given as the type of \define{locations}.
    \indexdef{location}%
    A location is a function $f : M \to \R$ such that
    %
    \begin{enumerate}
    \item $f(x) \geq |f(y) - d(x,y)|$ for all $x, y : M$, and
    \item $\inf_{x \in M} f(x) = 0$, by which we mean $\fall{\epsilon : \Qp} \exis{x : M} |f(x)| < \epsilon$ and $\fall{x : M} f(x) \geq 0$.
    \end{enumerate}
    %
    The idea is that $f$ looks like it is measuring the distance from a point.
  \end{enumerate}
  %
  \index{universal!property!of metric completion}%
  Finally, prove the following universal property of metric completions: a locally uniformly continuous map from a metric space to a Cauchy complete metric space extends uniquely to a locally uniformly continuous map on the completion. (We say that a map is \define{locally uniformly continuous}
  \indexdef{function!locally uniformly continuous}%
  \indexdef{locally uniformly continuous map}%
  if it is uniformly continuous on open balls.)
\end{ex}

\index{metric space|)}%

\begin{ex} \label{ex:reals-apart-neq-MP}
  \define{Markov's principle}
  \indexdef{axiom!Markov's principle}%
  \indexdef{Markov's principle}%
  says that for all $f : \nat \to \bool$,
  %
  \begin{equation*}
    (\lnot \lnot \exis{n : \nat} f(n) = \btrue)
    \Rightarrow
    \exis{n : \nat} f(n) = \btrue.
  \end{equation*}
  %
  This is a particular instance of the law of double negation~\eqref{eq:ldn}. Show that
  $\fall{x, y: \RD} x \neq y \Rightarrow x \apart y$ implies Markov's principle. Does the
  converse hold as well?
\end{ex}

\begin{ex} \label{ex:reals-apart-zero-divisors}
  \index{apartness}%
  Verify that the following ``no zero divisors'' property holds for the real numbers:
  $x y \apart 0 \Leftrightarrow x \apart 0 \land y \apart 0$.
\end{ex}

\begin{ex} \label{ex:finite-cover-lebesgue-number}
  %
  Suppose $(q_1, r_1), \ldots, (q_n, r_n)$ pointwise cover $(a, b)$. Then there is
  $\epsilon : \Qp$ such that whenever $a < x < y < b$ and $|x - y| < \epsilon$
  then there merely exists $i$ such that $q_i < x < r_i$ and $q_i < y < r_i$. Such an
  $\epsilon$ is called a \define{Lebesgue number}
  \indexdef{Lebesgue number}%
  for the given cover.
\end{ex}

\begin{ex} \label{ex:mean-value-theorem}
  %
  Prove the following approximate version of the intermediate value theorem:
  %
  \begin{quote}
    \emph{
      If $f : [0,1] \to \R$ is uniformly continuous and $f(0) < 0 < f(1)$ then
      for every $\epsilon : \Qp$ there merely exists $x : [0,1]$ such that $|f(x)| <
      \epsilon$.
    }
  \end{quote}
  %
  Hint: do not try to use the bisection method because it leads to the axiom of choice.
  Instead, approximate $f$ with a piecewise linear map. How do you construct a piecewise
  linear map?
\end{ex}

\begin{ex}\label{ex:knuth-surreal-check}
  Check whether everything in~\cite{knuth74:_surreal_number} can be done using the higher
  inductive-inductive surreals of \cref{sec:surreals}.
\end{ex}

\begin{ex}\label{ex:reals-into-surreals}
  Recall the function $\iota_{\RD}:\RD\to\NO$ defined on page~\pageref{reals-into-surreals}.
  \begin{enumerate}
  \item Show that $\iota_{\RD}$ is injective.
  \item There are obvious extensions of $\iota_{\RD}$ to the extended reals (\cref{ex:RD-extended-reals}) and the interval domain (\cref{ex:RD-interval-arithmetic}).
    Are they injective?
  \end{enumerate}
\end{ex}

\begin{ex}\label{ex:ord-into-surreals}
  Show that the function $\iota_{\ord}:\ord\to\NO$ defined on page~\pageref{ord-into-surreals} is injective if and only if \LEM{} holds.
\end{ex}

\begin{ex}\label{ex:hiit-plump}
  Define a type $\mathsf{POrd}$ equipped with binary relations $\le$ and $<$ by mimicking the definition of \NO but using only left options.
  \begin{enumerate}
  \item Construct a map $j:\mathsf{POrd} \to \NO$ and show that it is an embedding.
  \item Show that $\mathsf{POrd}$ is an ordinal (in the next higher universe, like \ord) under the relation $<$.
  \item Assuming propositional resizing, show that $\mathsf{POrd}$ is equivalent to the subset
    \[\setof{A:\ord | \mathsf{isPlump}(A)}\]
    of \ord from \cref{ex:plump-ordinals}.
    Conclude that $\iota_{\ord}:\ord\to\NO$ is injective when restricted to plump ordinals.
  \end{enumerate}
  In the absence of propositional resizing, we may still refer to elements of $\mathsf{POrd}$ (or their images in \NO) as \define{plump ordinals}.\index{ordinal!plump}\index{plump!ordinal}
\end{ex}

\begin{ex}\label{ex:pseudo-ordinals}
  Define a surreal number to be a \define{pseudo-ordinal}\index{pseudo-ordinal}\index{ordinal!pseudo-} if it is equal to a cut $\surr{x^L}{}$ with no right options (but its left options may themselves have right options).
  Show that the statement ``every pseudo-ordinal is a plump ordinal'' is equivalent to \LEM{}.
\end{ex}

\begin{ex}\label{ex:double-No-recursion}
  Note that \cref{defn:No-codes} and \cref{eg:surreal-addition} both use a similar pattern to define a function $\NO \to \NO \to B$: an outer \NO-recursion whose codomain is the set of order-preserving functions $\NO\to B$, followed by an inner \NO-induction into a family $A:\NO\to\type$ where $A(y)$ is a subset of $B$ ensuring that the inequalities $x^L<x$ and $x<x^R$ are also preserved.
  Formulate and prove a general principle of ``double \NO-recursion'' that generalizes these proofs.
\end{ex}

\index{real numbers|)}%

%%% Local Variables:
%%% mode: latex
%%% TeX-master: "hott-online"
%%% End:
