\chapter{集合论 (Set theory)}
\label{cha:set-math}

\index{集合|(set|(}%

我们将集合理解为具有特别简单的同伦特征的类型,参见\cref{sec:basics-sets}。这种理解与Zermelo--Fraenkel\index{集合论!Zermelo--Fraenkel(set theory!Zermelo--Fraenkel)}集合论中的集合完全不同,后者形成了一个包含复杂嵌套隶属关系的累积层次结构。对于许多数学目的,同伦理论中的集合与Zermelo--Fraenkel集合一样好用,但它们之间存在重要差异。

我们在本章开始于\cref{sec:piw-pretopos},展示了范畴$\uset$具有通常集合范畴的(大部分)性质。
\index{数学!构造性(mathematics!constructive)}%
\index{数学!预测性(mathematics!predicative)}%
在构造性、预测性、单值性基础上,它是一个``$\Pi\mathsf{W}$-前拓扑(pretopos)'';而如果我们假设命题的缩放
\index{命题!缩放(propositional!resizing)}%
(\cref{subsec:prop-subsets}),它是一个初等拓扑(elementary topos)\index{拓扑(topos)},如果我们假设\LEM{}和\choice{},那么它是Lawvere的\emph{集合范畴的初等理论}的模型\index{Lawvere}。
\index{集合范畴的初等理论(Elementary Theory of the Category of Sets)}%
这足以确保在同伦类型论中的集合行为类似于大多数数学家在集合论之外使用的集合。

在本章的其余部分,我们研究了一些传统上属于“集合论”的主题。在\cref{sec:cardinals,sec:ordinals,sec:wellorderings}中,我们研究了基数和序数。这些通常在集合论中使用全局隶属关系来定义,但我们将看到,单值性公理使得一种同样方便、更具“结构性”的方法成为可能。

最后,在\cref{sec:cumulative-hierarchy}中,我们考虑在同伦类型论内部构建一个具有二元隶属关系的类似于Zermelo--Fraenkel集合论的累积层次结构的可能性。这结合了高阶归纳类型与代数集合论领域的思想。
\index{代数集合论(algebraic set theory)}%
\index{集合论!代数(algebraic)}%

在本章中,我们将经常使用\cref{subsec:prop-trunc}中描述的传统逻辑符号。除了\cref{cha:basics,cha:logic}的基本理论外,我们还使用了\cref{sec:colimits,sec:set-quotients}中关于余极限和商集的高阶归纳类型,以及\cref{cha:hlevels}中关于截断的某些理论,特别是在\cref{sec:image-factorization}中提到的因子分解系统在$n=-1$的情况。在\cref{sec:ordinals}中,我们使用了一个归纳族(\cref{sec:generalizations})来描述良序性,在\cref{sec:cumulative-hierarchy}中,我们使用了一个更复杂的高阶归纳类型来呈现累积层次结构。


\section{集合范畴 (The category of sets)}
\label{sec:piw-pretopos}

回顾在\cref{cha:category-theory}中,我们定义了范畴\uset由所有$0$-类型(在某个宇宙\UU中)及其之间的映射组成,并且观察到它是一个范畴(不仅仅是一个预范畴)。我们将依次考虑\uset所具有的结构层次。

\subsection{极限和余极限 (Limits and colimits)}
\label{subsec:limits-sets}

\index{极限!集合的(limit!of sets)}%
\index{余极限!集合的(colimit!of sets)}%

由于集合在积下是封闭的,\cref{thm:prod-ump}中的积的泛性质立即表明\uset具有有限积。实际上,无穷积也同样容易从等价式中得出:
\[ \Parens{X\to \prd{a:A} B(a)} \eqvsym \Parens{\prd{a:A} (X\to B(a))}.\]
我们在\cref{ex:pullback}中看到,$f:A\to C$和$g:B\to C$的拉回可以定义为$\sm{a:A}{b:B} f(a)=g(b)$;如果$A,B,C$都是集合,则它是一个集合并继承了正确的泛性质。因此,\uset在显然的意义上是一个\emph{完备}范畴。
\index{范畴!完备的(category!complete)}%
\index{完备!范畴(complete!category)}%

由于集合在$+$下是封闭的并包含\emptyt,\uset具有有限余积。同样,由于$\sm{a:A}B(a)$是一个集合,当且仅当$A$和每个$B(a)$是集合时,它在\uset中产生了族$B$的余积。最后,我们在\cref{sec:pushouts}中证明了$n$-类型中的推出存在,这特别包括了\uset。因此,\uset也是\emph{余完备的}。
\index{范畴!余完备的(category!cocomplete)}%
\index{余完备范畴(cocomplete category)}%

\subsection{像 (Images)}
\label{sec:image}

接下来,我们展示\uset是一个\define{正则范畴 (regular category)},即:
\indexdef{范畴!正则的(category!regular)}%
\indexdef{正则!范畴(regular!category)}%
%
\begin{enumerate}
  \item \uset是有限完备的。\label{item:reg1}
  \item 任何函数$f : A \to B$的核对偶$\proj1,\proj2: (\sm{x,y:A} f(x)= f(y)) \to A$具有余等化子。\label{item:reg2}
  \indexdef{核!对偶(kernel!pair)}
  \item 正则满态射的拉回再次是正则满态射。\label{item:reg3}
\end{enumerate}
%
回想一个\define{正则满态射 (regular epimorphism)}
\indexdef{满态射!正则的(epimorphism!regular)}%
\indexdef{正则!满态射(regular!epimorphism)}%
是某对映射的余等化子。因此在\ref{item:reg3}中,余等化子的拉回需要再次是余等化子,但不一定是被拉回对偶的。

\index{集合余等化子(set-coequalizer)}%
\index{像(image)}%
$f:A\to B$的核对偶的余等化子的明显候选者是$f$的\emph{像},如\cref{sec:image-factorization}中定义的那样。回想我们定义了$\im(f)\defeq \sm{b:B} \brck{\hfib f b}$,并且定义了$\tilde{f}:A\to\im(f)$和$i_f:\im(f)\to B$,如下所示:
\begin{align*}
  \tilde{f} & \defeq \lam{a} \Pairr{f(a),\,\bproj{\pairr{a,\refl{f(a)}}}}\\
  i_f & \defeq \proj1
\end{align*}
它们构成了一个图:
\begin{equation*}
  \xymatrix{
    **[l]{\sm{x,y:A} f(x)= f(y)}
    \ar@<0.25em>[r]^{\proj1}
    \ar@<-0.25em>[r]_{\proj2}
    &
      {A}
    \ar[r]^(0.4){\tilde{f}}
    \ar[rd]_{f}
    &
      {\im(f)}
    \ar@{..>}[d]^{i_f}
    \\ & &
    B
  }
\end{equation*}

回想一个函数$f:A\to B$称为\emph{满射 (surjective)},如果
\index{函数!满射(function!surjective)}%
\narrowequation{\fall{b:B}\brck{\hfib f b},}
或者等价地$\fall{b:B} \exis{a:A} f(a)=b$。我们还说过,两个集合之间的函数$f:A\to B$称为\emph{单射 (injective)},如果
\index{函数!单射(function!injective)}%
$\fall{a,a':A} (f(a) = f(a')) \Rightarrow (a=a')$,或者等价地,如果它的每个纤维是一个简单命题。由于这些是在\cref{cha:hlevels}意义上的$(-1)$-连通和$(-1)$-截断映射,一般理论表明,上述$\tilde f$是满射而$i_f$是单射,并且这种因子分解在拉回下是稳定的。

我们现在将单射性和满射性与适当的范畴理论概念进行比较。首先我们观察到范畴中的单态射和满态射有一个略微更强的等价公式。

\begin{lem}\label{thm:mono}
对于范畴$A$中的一个态射$f:\hom_A(a,b)$,以下条件是等价的。
\begin{enumerate}
  \item $f$是一个\define{单态射 (monomorphism)}:
  \indexdef{单态射(monomorphism)}%
  对于所有$x:A$和${g,h:\hom_A(x,a)}$,如果$f\circ g = f\circ h$,则$g=h$。\label{item:mono1}
  \item (如果$A$有拉回)对角线映射$a\to a\times_b a$是一个同构。\label{item:mono4}
  \item 对于所有$x:A$和$k:\hom_A(x,b)$,类型$\sm{h:\hom_A(x,a)} (k = f\circ h)$是一个简单命题。\label{item:mono2}
  \item 对于所有$x:A$和${g:\hom_A(x,a)}$,类型$\sm{h:\hom_A(x,a)} (f\circ g = f\circ h)$是一个收缩的。\label{item:mono3}
\end{enumerate}
\end{lem}
\begin{proof}
  条件~\ref{item:mono1}和~\ref{item:mono4}的等价性是标准范畴论。现在考虑$\hom_A(x,a)$和$\hom_A(x,b)$之间的函数$(f\circ \blank )$。条件~\ref{item:mono1}表示它是单射,而~\ref{item:mono2}表示它的纤维是简单命题;因此它们是等价的。~\ref{item:mono2}通过取$k\defeq f\circ g$并记住被占用的简单命题是收缩的来隐含~\ref{item:mono3}。最后,~\ref{item:mono3}隐含~\ref{item:mono1},因为如果$p:f\circ g= f\circ h$,那么$(g,\refl{})$和$(h,p)$都包含在~\ref{item:mono3}中的类型中,因此是相等的,所以$g=h$。
\end{proof}

\begin{lem}\label{thm:inj-mono}
集合之间的一个函数$f:A\to B$是单射的当且仅当它是\uset中的单态射\index{单态射(monomorphism)}。
\end{lem}
\begin{proof}
  留给读者。
\end{proof}

当然,\define{满态射 (epimorphism)}
\indexdef{满态射(epimorphism)}%
\indexsee{满态射(epi)}{满态射(epimorphism)}%
是在对偶范畴中的单态射。我们现在展示,在\uset中,满态射正是满射,同时也正是余等化子(正则满态射)。

两个集合$A$和$B$之间的$f,g:A\to B$的余等化子在$\uset$中定义为一般(同伦)余等化子的$0$-截断。为了清楚起见,我们可以将其称为\define{集合余等化子 (set-coequalizer)}。
\indexdef{集合余等化子(set-coequalizer)}%
\indexsee{集合余等化子(coequalizer!of sets)}{set-coequalizer}%
它的泛性质方便地表达如下。

\begin{lem}
  \index{集合余等化子的泛性质(universal!property!of set-coequalizer)}%
  设$f,g:A\to B$为两个集合$A$和$B$之间的函数。{集合余}等化子$c_{f,g}:B\to Q$具有如下性质:对于任意集合$C$和任意$h:B\to C$,满足$h\circ f = h\circ g$,类型
  \begin{equation*}
    \sm{k:Q\to C} (k\circ c_{f,g} = h)
  \end{equation*}
  是收缩的。
\end{lem}

\begin{lem}\label{epis-surj}
对于集合之间的任意函数$f:A\to B$,以下条件是等价的:
\begin{enumerate}
  \item $f$是一个满态射。
  \item 考虑推出图
  \begin{equation*}
    \xymatrix{
        {A}
      \ar[r]^{f}
      \ar[d]
      &
        {B}
      \ar[d]^{\iota}
      \\
      {\unit}
      \ar[r]_{t}
      &
        {C_f}
    }
  \end{equation*}
  在$\uset$中定义了映射锥\index{映射锥(cone!of a function)}。那么类型$C_f$是收缩的。
  \item $f$是满射。
\end{enumerate}
\end{lem}

\begin{proof}
  设$f:A\to B$为一个集合之间的函数,假设它是一个满态射;我们证明$C_f$是收缩的。$C_f$的构造器$\unit\to C_f$给了我们一个元素$t:C_f$。我们需要证明
  \begin{equation*}
    \prd{x:C_f} x= t.
  \end{equation*}
  请注意$x= t$是一个简单命题,因此我们可以对$C_f$进行归纳。当然,当$x$为$t$时,我们有$\refl{t}:t=t$,所以足以找到
  \begin{align*}
    I_0 & : \prd{b:B} \iota(b)= t\\
    I_1 & : \prd{a:A} \opp{\alpha_1(a)} \ct I_0(f(a))=\refl{t}
  \end{align*}
  其中$\iota:B\to C_f$和$\alpha_1:\prd{a:A} \iota(f(a))= t$是$C_f$的其他构造器。请注意$\alpha_1$是$\iota\circ f$到$\mathsf{const}_t\circ f$的一个同伦,因此我们可以找到元素
  \begin{equation*}
    \pairr{\iota,\refl{\iota\circ f}},\pairr{\mathsf{const}_t,\alpha_1}:
    \sm{h:B\to C_f} \iota\circ f \htpy h\circ f.
  \end{equation*}
  通过\cref{thm:mono}\ref{item:mono3}的对偶(以及函数扩展性),我们有一条路径
  \begin{equation*}
    \gamma:\pairr{\iota,\refl{\iota\circ f}}=\pairr{\mathsf{const}_t,\alpha_1}.
  \end{equation*}
  因此,我们可以定义$I_0(b)\defeq \happly(\projpath1(\gamma),b):\iota(b)=t$。
  我们还有
  \[\projpath2(\gamma) : \trans{\projpath1(\gamma)}{\refl{\iota\circ f}} = \alpha_1。 \]
  此传输涉及$f$的前置,它与$\happly$一起工作。因此,从路径类型中的传输我们得到$I_0(f(a)) = \alpha_1(a)$,对于任何$a:A$,这给了我们$I_1$。

  现在假设$C_f$是收缩的;我们证明$f$是满射。我们首先通过$C_f$上的递归构造一个类型族$P:C_f\to\prop$,这是有效的,因为\prop是一个集合。在点构造器上,我们定义
  \begin{align*}
    P(t) & \defeq \unit\\
    P(\iota(b)) & \defeq \brck{\hfiber{f}b}.
  \end{align*}
  为了完成$P$的构造,我们还需要为所有$a:A$给出一条路径
  \narrowequation{\brck{\hfiber{f}{f(a)}} =_\prop \unit。}
  然而,$\brck{\hfiber{f}{f(a)}}$是由$(f(a),\refl{f(a)})$居住的。由于它是一个简单命题,这意味着它是收缩的——因此是等价的,因此与\unit相等。这完成了$P$的定义。现在,由于$C_f$被假设为收缩的,因此$P(x)$对于任何$x:C_f$都是等价于$P(t)$的。特别是,$P(\iota(b))\jdeq \brck{\hfiber{f}b}$等价于$P(t)\jdeq \unit$,对于每个$b:B$,因此是收缩的。因此,$f$是满射。

  最后,假设$f:A\to B$是满射,并考虑一个集合$C$和两个函数$g,h:B\to C$,它们具有$g\circ f = h\circ f$的性质。由于$f$被假设为满射,因此对于所有$b:B$,类型$\brck{\hfib f b}$是收缩的。因此我们有以下等价:
  \begin{align*}
    \prd{b:B} (g(b)= h(b))
    & \eqvsym \prd{b:B} \Parens{\brck{\hfib f b} \to (g(b)= h(b))}\\
    & \eqvsym \prd{b:B} \Parens{\hfib f b \to (g(b)= h(b))}\\
    & \eqvsym \prd{b:B}{a:A}{p:f(a)= b} g(b)= h(b)\\
    & \eqvsym \prd{a:A} g(f(a))= h(f(a))。
  \end{align*}
  使用在第二行中的事实,即$g(b)=h(b)$是一个简单命题,因为$C$是一个集合。但根据假设,有该类型的一个元素。
\end{proof}

\begin{thm}\label{thm:set_regular}\label{lem:images_are_coequalizers}
范畴$\uset$是正则的。此外,集合之间的满射是正则满态射。
\end{thm}

\begin{proof}
  这是范畴论中的一个标准引理,即范畴是正则的,只要它承认有限极限和稳定于拉回的正交因子分解系统\index{正交因子分解系统(orthogonal factorization system)} $(\mathcal{E},\mathcal{M})$,其中$\mathcal{M}$是单态射,在这种情况下,$\mathcal{E}$自动由正则满态射组成。
  (参见例如\cite[A1.3.4]{elephant})。
  因子分解系统的存在性在\cref{thm:orth-fact}中得到证明。
\end{proof}

\begin{lem}\label{lem:pb_of_coeq_is_coeq}
在\uset中,正则满态射的拉回是正则满态射。
\end{lem}
\begin{proof}
  我们在\cref{thm:stable-images}中展示了,$n$-连通函数的拉回是$n$-连通的。通过\cref{lem:images_are_coequalizers},当$n=-1$时应用这一结论就足够了。
\end{proof}

\indexdef{子集的像(image!of a subset)}
\uset作为正则范畴的一个后果是,我们有了“像”运算作用于子集。也就是说,给定$f:A\to B$,任何子集$P:\power A$(即谓词$P:A\to \prop$)都有一个\define{像 (image)},它是$B$的一个子集。这可以直接定义为$\setof{ y:B | \exis{x:A} f(x)=y \land P(x)}$,或间接地定义为复合函数
\[ \setof{ x:A | P(x) } \to A \xrightarrow{f} B的像。]
\symlabel{subset-image}
我们有时也会使用常见的记号$\setof{f(x) | P(x)}$来表示$P$的像。


\subsection{商 (Quotients)}\label{subsec:quotients}

\index{集合商(set-quotient|(}%
现在我们知道$\uset$是正则的,要表明$\uset$是精确的,我们需要证明每个等价关系都是有效的。
\index{有效!等价关系(effective!equivalence relation|(}%
\index{关系!有效等价(effective equivalence|(}%
换句话说,给定等价关系$R:A\to A\to\prop$,存在一个对偶的余等化子$c_R$,并且$\proj1$和$\proj2$形成$c_R$的核对偶。

我们已经在\cref{sec:set-quotients}中看到了两个构造集合按等价关系$R:A\to A\to\prop$的商的方法。第一个可以描述为
\[ \proj1,\proj2:\Parens{\sm{x,y:A} R(x,y)} \to A的集合余}等化子。]
其商的一个重要性质如下。

\begin{defn}
一个关系$R:A\to A\to\prop$被称为\define{有效的 (effective)},
\indexdef{有效!关系(effective!relation)}
\indexdef{有效!等价关系(effective!equivalence relation)}%
\indexdef{关系!有效等价(effective equivalence)}%
如果方框
\begin{equation*}
\xymatrix{
{\sm{x,y:A} R (x,y)}
\ar[r]^(0.7){\proj1}
\ar[d]_{\proj2}
&
{A}
\ar[d]^{c_R}
\\
{A}
\ar[r]_{c_R}
&
{A/R}
}
\end{equation*}
是一个拉回。
\end{defn}

由于$c_R$和它本身的标准拉回是$\sm{x,y:A} (c_R(x)=c_R(y))$,通过\cref{thm:total-fiber-equiv},这相当于要求$c_R(x)=c_R(y)$的典型转换是一个纤维等价。

\begin{lem}\label{lem:sets_exact}
假设$\pairr{A,R}$是一个等价关系。那么对于任何$x,y:A$,有一个等价关系
\begin{equation*}
(c_R(x)= c_R(y))\eqvsym R(x,y)。
\end{equation*}
换句话说,等价关系是有效的。
\end{lem}

\begin{proof}
我们首先通过对$A/R$进行双重归纳将$R$扩展为一个关系$\widetilde{R}:A/R\to A/R\to\prop$,然后我们将证明它与$A/R$上的恒等类型等价。我们定义$\widetilde{R}(c_R(x),c_R(y)) \defeq R(x,y)$。对于$r:R(x,x')$和$s:R(y,y')$,$R$的传递性和对称性给出了$R(x,y)$到$R(x',y')$的一个等价关系。这完成了$\widetilde{R}$的定义。

现在要证明对于每个$w,w':A/R$,$\widetilde{R}(w,w')\eqvsym (w= w')$。
方向$(w=w')\to \widetilde{R}(w,w')$通过传输一次我们证明了$\widetilde{R}$是反射的,这是一种简单的归纳。
另一个方向$\widetilden{R}(w,w')\to (w= w')$是一个简单命题,因此由于$c_R:A\to A/R$是满射,只需要假设$w$和$w'$是$c_R(x)$和$c_R(y)$形式的。但在这种情况下,我们有典型映射$\widetilden{R}(c_R(x),c_R(y)) \defeq R(x,y) \to (c_R(x)=c_R(y))$。(再次注意到编码解码方法的出现。\index{编码解码方法(encode-decode method)})
\end{proof}

第二个商的构造是作为$R$的等价类的集合(它的幂集的子集):
\[ A\sslash R \defeq \setof{ P:A\to\prop | P \text{ is an equivalence class of } R}。]
这需要命题缩放\index{命题缩放(propositional resizing)}\index{非预测性商(impredicative!quotient)}\index{缩放(resizing)}来保持在与$A$和$R$相同的宇宙中。

注意,如果我们将$R$视为$A$到$A\to \prop$的函数,那么$A\sslash R$等价于\cref{sec:image}中构造的$\im(R)$。现在在\cref{lem:images_are_coequalizers}中我们已经证明了图像是
余等化子。特别是,我们立即得到余等化子图
\begin{equation*}
\xymatrix{
**[l]{\sm{x,y:A} R (x)= R (y)}
\ar@<0.25em>[r]^{\proj1}
\ar@<-0.25em>[r]_{\proj2}
&
{A}
\ar[r]
&
{A \sslash R。}
}
\end{equation*}
我们可以用这个来给出另一个证明,即任何等价关系是有效的,并且两个商的定义是一致的。

\begin{thm}\label{prop:kernels_are_effective}
对于任意两个集合之间的函数$f:A\to B$,
关系$\ker(f):A\to A\to\prop$由
$\ker(f,x,y)\defeq (f(x)= f(y))$给出是有效的。
\end{thm}
\begin{proof}
我们将使用$\proj1,\proj2: (\sm{x,y:A} f(x)= f(y))\to A$的余等化子$\im(f)$。
注意,函数
\[c_f\defeq\lam{a} \Parens{f(a),\brck{\pairr{a,\refl{f(a)}}}}
: A \to \im(f)
\]
的核对偶由两个投影
\begin{equation*}
\proj1,\proj2:\Parens{\sm{x,y:A} c_f(x)= c_f(y)}\to A。
\end{equation*}
对于任何$x,y:A$,我们有等价
\begin{align*}
(c_f(x)= c_f(y))
& \eqvsym \Parens{\sm{p:f(x)= f(y)} \trans{p}{\brck{\pairr{x,\refl{f(x)}}}} =\brck{\pairr{y,\refl{f(y)}}}}\\
& \eqvsym (f(x)= f(y)),
\end{align*}
其中最后一个等价关系成立,因为
$\brck{\hfiber{f}b}$对于任何$b:B$来说是一个简单命题。
因此,我们得到
\begin{equation*}
\Parens{\sm{x,y:A} c_f(x)= c_f(y)}\eqvsym \Parens{\sm{x,y:A} f(x)= f(y)}
\end{equation*}
并且我们可以得出结论,对于任何函数$f$,$\ker f$是一个有效的关系。
\end{proof}

\begin{thm}
等价关系是有效的,并且$A/R \eqvsym A\sslash  R$。
\end{thm}

\begin{proof}
我们需要分析余等化子图
\begin{equation*}
\xymatrix{
**[l]{\sm{x,y:A} R (x)= R (y)}
\ar@<0.25em>[r]^{\proj1}
\ar@<-0.25em>[r]_{\proj2}
&
{A}
\ar[r]
&
{A \sslash R}
}
\end{equation*}
通过单值化公理,类型$R(x) = R(y)$等价于从$R(x)$到$R(y)$的同伦类型,并且进一步等价于
\narrowequation{\prd{z:A} R (x,z)\eqvsym R (y,z)}。
由于$R$是一个等价关系,后者空间等价于$R(x,y)$。总之,我们得到$(R(x) = R(y)) \eqvsym R(x,y)$,因此$R$是有效的,因为它等价于一个有效的关系。此外,图
\begin{equation*}
\xymatrix{
**[l]{\sm{x,y:A} R(x, y)}
\ar@<0.25em>[r]^{\proj1}
\ar@<-0.25em>[r]_{\proj2}
&
{A}
\ar[r]
&
{A \sslash R。}
}
\end{equation*}
是一个余等化子图。由于余等化子是等价的,因此可以得出$A/R \eqvsym A\sslash  R$。
\end{proof}

我们通过提到商的第三种可能的构造来结束本节。考虑一个以$A$为对象的预范畴,其同态集为$R$;此预范畴的Rezk完成\index{完成!Rezk}(参见\cref{sec:rezk})的对象类型将是该等价关系的商。读者可以检查详细信息。

\index{有效!等价关系|)}%
\index{关系!有效等价|)}%
\index{集合商|)}%

\subsection{\texorpdfstring{$\uset$}{Set}是一个\texorpdfstring{$\Pi\mathsf{W}$}{ΠW}-前拓扑范畴}
\label{subsec:piw}

\index{结构化!集合论|(}%

所谓的\emph{$\Pi\mathsf{W}$-前拓扑范畴} \index{PiW-pretopos@$\Pi\mathsf{W}$-pretopos}%
\indexsee{前拓扑范畴}{$\Pi\mathsf{W}$-pretopos}是一种局部笛卡尔闭范畴
\index{局部笛卡尔闭范畴(locally cartesian closed category)}%
\index{范畴!局部笛卡尔闭(locally cartesian closed)}%
具有不相交的有限上积,有效等价关系,以及多项式自函子的初始代数的范畴——这被认为是一种“预测性”的拓扑概念,即“预测性集合”的范畴,可用于构造数学
\index{数学!构造性}%
就像通常的集合范畴对于经典数学
\index{数学!经典}%
的用途一样。

通常,在构造性类型论中,求助于“集合体”——一种精确补全——的外部构造来获得具有此类闭包性质的范畴。
\index{集合体}\index{补全!精确}%
特别是,良好行为的商在数学中通常涉及(非构造性)幂集的许多构造中是必需的。值得注意的是,统一基础通过更高归纳类型(higher inductive types)提供了这些内部构造,而无需此类外部构造。这代表了我们方法的强大优势,我们将在后续示例中看到。

\begin{thm}
\index{PiW-pretopos@$\Pi\mathsf{W}$-pretopos}
范畴$\uset$是一个$\Pi\mathsf{W}$-前拓扑范畴。
\end{thm}
\begin{proof}
我们有一个初始对象
\index{初始!集合}%
$\emptyt$和有限、不相交的上积$A+B$。这些在拉回时保持稳定,原因很简单,因为拉回有一个右伴随\index{伴随!函子}。事实上,$\uset$是局部笛卡尔闭的,因为对于集合之间的任何映射$f:A\to B$,使用“纤维化替换”\index{纤维化替换(fibrant replacement)}$\sm{a:A}f(a)=b$等价于$A$(在$B$上),并且我们有该替换的依赖函数类型。
我们刚刚展示了$\uset$是正则的(\cref{thm:set_regular}),并且商是有效的(\cref{lem:sets_exact})。因此,我们有一个局部笛卡尔闭前拓扑范畴。最后,由于$n$-类型在\cref{ex:ntypes-closed-under-wtypes}中通过多项式自函子的初始代数(\cref{thm:w-hinit})构成,我们看到$\uset$是一个$\Pi\mathsf{W}$-前拓扑范畴。
\end{proof}

\index{拓扑|(}
人们自然会想知道,有什么(如果有的话)阻止$\uset$成为一个(基本)拓扑?
除了已经提到的结构,拓扑还具有一个\emph{子对象分类器}:
\indexdef{子对象分类器(subobject classifier)}%
\index{分类器!子对象(classifier!subobject)}%
\index{幂集(power set)}%
这是一个指示对象,用于分类(等价类的)单态射(monomorphisms)。实际上,在具有子对象分类器的情况下,事情变得稍微简单一些:仅需要笛卡尔闭包即可获得余积。
在同伦类型论中,单值化公理表明,类型$\prop \defeq \sm{X:\UU}\isprop(X)$确实分类单态射(通过类似于\cref{sec:object-classification}的论证),但通常它与周围的宇宙$\UU$一样大。因此,它在某种意义上是一个“集合”,因为它是一个$0$-类型,但它不是“小的”,因为它不是$\UU$的对象,因此不是范畴$\uset$的对象。然而,如果我们假设一种适当形式的命题缩放(见\cref{subsec:prop-subsets}),那么我们可以找到$\prop$的一个小版本,使得$\uset$成为一个基本的拓扑范畴。

\begin{thm}\label{thm:settopos}
\index{命题!缩放(propositional resizing)}%
如果存在一个类型$\Omega:\UU$,包含所有简单命题,那么范畴$\uset_\UU$是一个基本拓扑范畴。
\end{thm}
\index{拓扑|)}

一个足够的条件是排中律,在“简单命题”形式中,我们称之为 \LEM{};因为在这种情况下,我们有 $\prop = \bool$,这是“小的”,并且可以分类所有的简单命题。此外,拓扑理论中一个众所周知的充分条件是选择公理,这是经典\index{数学!经典} 集合论中经常假设的公理。在下一节中,我们将简要探讨这些条件在我们环境下的关系。

\index{结构化!集合论|)}%

%%%%%%%%%%%%%%%%%%%%%%%%%%%%%%%%%%%%%%%%%%%%%%%%
\subsection{选择公理蕴含排中律}
\label{subsec:emacinsets}

我们从以下引理开始。

\begin{lem}\label{prop:trunc_of_prop_is_set}
如果 $A$ 是一个简单命题,那么其悬挂 $\susp(A)$ 是一个集合,并且 $A$ 等价于 $\id[\susp(A)]{\north}{\south}$。
\end{lem}

\begin{proof}
为了证明 $\susp(A)$ 是一个集合,我们定义一个族 $P:\susp(A)\to\susp(A)\to\type$,使得对于每个 $x,y:\susp(A)$,$P(x,y)$ 是一个简单命题,并且它等价于 $\susp(A)$ 上的同一类型 $\idtypevar{\susp(A)}$。
%
我们做以下定义:
\begin{align*}
P(\north,\north) & \defeq \unit &
P(\south,\north) & \defeq A\\
P(\north,\south) & \defeq A &
P(\south,\south) & \defeq \unit。
\end{align*}
我们需要检查定义是否保持路径。对于任何 $a : A$,有一个子午线 $\merid(a) : \north = \south$,因此我们应该有
%
\begin{equation*}
P(\north, \north) = P(\north, \south) = P(\south, \north) = P(\south, \south)。
\end{equation*}
%
但由于 $A$ 由 $a$ 占据,它等价于 $\unit$,因此我们有
%
\begin{equation*}
P(\north, \north) \eqvsym P(\north, \south) \eqvsym P(\south, \north) \eqvsym P(\south, \south)。
\end{equation*}
%
单值化公理将这些转换为所需的相等性。此外,$P(x,y)$ 对于所有 $x, y : \susp(A)$ 是一个简单命题,这通过对 $x$ 和 $y$ 的归纳以及简单命题是一个简单命题这一事实来证明。

注意,$P$ 是一个自反关系。因此,我们可以应用 \cref{thm:h-set-refrel-in-paths-sets},因此只需构造 $\tau : \prd{x,y:\susp(A)}P(x,y)\to(x=y)$。我们通过双重归纳来实现。当 $x$ 是 $\north$ 时,我们定义 $\tau(\north)$ 为
%
\begin{equation*}
\tau(\north,\north,u) \defeq \refl{\north}
\qquad\text{以及}\qquad
\tau(\north,\south,a) \defeq \merid(a)。
\end{equation*}
%
如果 $A$ 由 $a$ 占据,那么 $\merid(a) : \north = \south$,因此我们还需要
\narrowequation{
\trans{\merid(a)}{\tau(\north, \north)} = \tau(\north, \south)。
}
通过函数外延性,我们使用以下事实来完成这一步:
%
\begin{multline*}
\trans{\merid(a)}{\tau(\north,\north,x)} =
\tau(\north,\north,x) \ct \opp{\merid(a)} \jdeq \\
\refl{\north} \ct \merid(a) =
\merid(a) =
\merid(x) \jdeq
\tau(\north, \south, x)。
\end{multline*}
以对称的方式,我们可以通过以下方式定义 $\tau(\south)$:
%
\begin{equation*}
\tau(\south,\north, a) \defeq \opp{\merid(a)}
\qquad\text{以及}\qquad
\tau(\south,\south, u) \defeq \refl{\south}。
\end{equation*}
%
为了完成 $\tau$ 的构造,我们需要检查 $\trans{\merid(a)}{\tau(\north)} = \tau(\south)$,对于任何 $a : A$。验证过程与上述类似,通过对 $\tau$ 的第二个参数的归纳来进行。

因此,通过 \cref{thm:h-set-refrel-in-paths-sets} 我们得出 $\susp(A)$ 是一个集合,并且对于所有 $x,y:\susp(A)$ 有 $P(x,y) \eqvsym (\id{x}{y})$。取 $x\defeq \north$ 和 $y\defeq \south$ 即可得到所需的 $A \eqvsym (\id[\susp(A)]\north\south)$。
\end{proof}

\begin{thm}[Diaconescu 定理]\label{thm:1surj_to_surj_to_pem}
\index{选择公理(axiom of choice)}%
\index{排中律(excluded middle)}%
\index{Diaconescu's theorem(迪亚孔涅斯库定理)}\index{theorem!Diaconescu's(迪亚孔涅斯库定理)}%
选择公理蕴含排中律。
\end{thm}

\begin{proof}
我们使用在 \cref{thm:ac-epis-split} 中给出的选择公理的等价形式。考虑一个简单命题 $A$。定义一个函数 $f:\bool\to\susp(A)$,其定义为 $f(\bfalse) \defeq \north$ 和 $f(\btrue) \defeq \south$。这个函数是满射的。实际上,我们有 $\pairr{\bfalse,\refl{\north}} : \hfiber{f}{\north}$ 和 $\pairr{\btrue,\refl{\south}} :\hfiber{f}{\south}$。由于 $\bbrck{\hfiber{f}{x}}$ 是一个简单命题,通过归纳法可以得出所需的满射性。

根据 \cref{prop:trunc_of_prop_is_set},悬挂 $\susp(A)$ 是一个集合,因此根据选择公理,存在一个从 $\susp(A)$ 到 $\bool$ 的截面 $g: \susp(A) \to \bool$。由于 $\bool$ 上的相等性是可判定的,我们得到
\begin{equation*}
(g(f(\bfalse))= g(f(\btrue))) +
\lnot (g(f(\bfalse))= g(f(\btrue))),
\end{equation*}
并且,由于 $g$ 是 $f$ 的一个截面,因此是单射,
\begin{equation*}
(f(\bfalse) = f(\btrue)) +
\lnot (f(\bfalse) = f(\btrue))。
\end{equation*}
最后,由于 $(f(\bfalse)=f(\btrue)) = (\north=\south) = A$ 根据 \cref{prop:trunc_of_prop_is_set},我们有 $A+\neg A$。
\end{proof}

% This conclusion needs only \LEM{}, see \cref{ex:lemnm}.

% \begin{cor}\label{cor:ACtoLEM0}
%   If the axiom of choice \choice{} holds then $\brck{A + \neg A}$ for every set $A$.
% \end{cor}

% \begin{proof}
%   There is a surjection
%   \[
%   A + \neg A \epi \brck{A} + \brck{\neg A} \epi
%   \brck{(\brck{A} + \brck{\neg A})} = \brck{A} \vee \brck{\neg A} = \brck{A} \vee \neg \brck{A} = \unit,
%   \]
%   %
%   where in the last step excluded middle is available as a consequence of the axiom of choice.
%   Again by the axiom of choice there merely exists a section of the surjection, but this
%   is none other than an inhabitant of $A + \neg A$. Therefore $\brck{A+\neg A}$.
% \end{proof}

\index{denial(否定)}
\begin{thm}\label{thm:ETCS}
\index{Elementary Theory of the Category of Sets(集合范畴的初等理论)}%
\index{category!well-pointed(范畴!良好定点)}%
如果选择公理成立,则类别 $\uset$ 是一个带选择的良好定点布尔初等拓扑。
\end{thm}

\begin{proof}
由于 \choice{} 蕴含 \LEM{},通过 \cref{thm:settopos} 以及后面的注释,我们可以得到一个带选择的布尔初等拓扑。我们将良好定点性的证明留给读者作为练习 (\cref{ex:well-pointed})。
\end{proof}

\begin{rmk}
定理中提到的关于范畴的条件被称为 Lawvere 用于“集合范畴的初等理论”的公理~\cite{lawvere:etcs-long}。
\end{rmk}

\section{基数 (Cardinal numbers)}
\label{sec:cardinals}

\begin{defn}\label{defn:card}
\define{基数类型 (type of cardinal numbers)}
\indexdef{type!of cardinal numbers}%
\indexdef{cardinal number}%
\indexsee{number!cardinal}{cardinal number}%
是集合类型 (\set) 的 0-截断:
\[ \card \defeq \pizero{\set} \]
因此,一个 \define{基数 (cardinal number)},或称 \define{基数 (cardinal)},是 $\card\jdeq \pizero\set$ 的一个元素。
技术上,当然每个宇宙 \type 都有一个单独的基数类型 $\card_\UU$ 。
\end{defn}

%\begin{rmk}

% , but with these conventions we can state theorems beginning with ``for all cardinal numbers\dots''\ and give them exactly the same sort of meaning as those beginning ``for all types\dots''.
%\end{rmk}

和通常的截断一样,如果 $A$ 是一个集合,那么 $\cd{A}$ 表示其在从集合类型到 0-截断 $\trunc0\set \jdeq \card$ 的标准映射下的像;我们称 $\cd{A}$ 为 $A$ 的 \define{基数 (cardinality)}\indexdef{cardinality}。
根据定义,\card 是一个集合。
它还继承了来自 \set 的半环结构。

\begin{defn}
\define{基数加法 (cardinal addition)}
\indexdef{addition!of cardinal numbers}%
\index{cardinal number!addition of}%
的运算定义为截断上的归纳:
\[ (\blank+\blank) : \card \to \card \to \card \]
定义为:
\[ \cd{A} + \cd{B} \defeq \cd{A+B}.\]
\end{defn}
\begin{proof}
由于 $\card\to\card$ 是一个集合,要为所有 $\alpha:\card$ 定义 $(\alpha+\blank):\card\to\card$,通过归纳,只需要假设 $\alpha$ 是某个 $A:\set$ 的 $\cd{A}$。
现在我们想要定义 $(\cd{A}+\blank) :\card\to\card$,即我们想要为所有 $\beta:\card$ 定义 $\cd{A}+\beta :\card$。
然而,由于 \card 是一个集合,通过归纳,只需要假设 $\beta$ 是某个 $B:\set$ 的 $\cd{B}$。
但是现在我们可以定义 $\cd{A}+\cd{B}$ 为 $\cd{A+B}$。
\end{proof}

\begin{defn}
类似地,\define{基数乘法 (cardinal multiplication)}
\indexdef{multiplication!of cardinal numbers}%
\index{cardinal number!multiplication of}%
的运算定义为截断上的归纳:
\[ (\blank\cdot\blank) : \card \to \card \to \card \]
定义为:
\[ \cd{A} \cdot \cd{B} \defeq \cd{A\times B} \]
\end{defn}

\begin{lem}\label{card:semiring}
\card 是一个交换半环 (commutative semiring)\index{semiring},即对于 $\alpha,\beta,\gamma:\card$ 我们有以下性质。
\begin{align*}
(\alpha+\beta)+\gamma &= \alpha+(\beta+\gamma)\\
\alpha+0 &= \alpha\\
\alpha + \beta &= \beta + \alpha\\
(\alpha \cdot \beta) \cdot \gamma &= \alpha \cdot (\beta\cdot\gamma)\\
\alpha \cdot 1 &= \alpha\\
\alpha\cdot\beta &= \beta\cdot\alpha\\
\alpha\cdot(\beta+\gamma) &= \alpha\cdot\beta + \alpha\cdot\gamma
\end{align*}
其中 $0 \defeq \cd{\emptyt}$ 且 $1\defeq\cd{\unit}$。
\end{lem}
\begin{proof}
我们证明乘法的交换性,即 $\alpha\cdot\beta = \beta\cdot\alpha$;其他性质的证明完全类似。
由于 \card 是一个集合,类型 $\alpha\cdot\beta = \beta\cdot\alpha$ 是一个纯命题,并且特别地是一个集合。
因此,通过截断上的归纳,只需要假设 $\alpha$ 和 $\beta$ 分别是某些 $A,B:\set$ 的 $\cd{A}$ 和 $\cd{B}$。
现在 $\cd{A}\cdot \cd{B} \jdeq \cd{A\times B}$ 和 $\cd{B}\cdot\cd{A} \jdeq \cd{B\times A}$,所以只需要证明 $A\times B = B\times A$。
最后,通过同一性原理 (univalence),只需要给出一个等价 $A\times B \eqvsym B\times A$。
这很简单:取 $(a,b) \mapsto (b,a)$ 及其显然的逆映射即可。
\end{proof}

\begin{defn}
\define{基数指数 (cardinal exponentiation)} 的运算同样定义为截断上的归纳:
\indexdef{exponentiation, of cardinal numbers}%
\index{cardinal number!exponentiation of}%
\[ \cd{A}^{\cd{B}} \defeq \cd{B\to A}. \]
\end{defn}

\begin{lem}\label{card:exp}
对于 $\alpha,\beta,\gamma:\card$ 我们有以下性质:
\begin{align*}
\alpha^0 &= 1\\
1^\alpha &= 1\\
\alpha^1 &= \alpha\\
\alpha^{\beta+\gamma} &= \alpha^\beta \cdot \alpha^\gamma\\
\alpha^{\beta\cdot \gamma} &= (\alpha^{\beta})^\gamma\\
(\alpha\cdot\beta)^\gamma &= \alpha^\gamma \cdot \beta^\gamma
\end{align*}
\end{lem}
\begin{proof}
证明与 \cref{card:semiring} 类似。
\end{proof}

\begin{defn}
\define{基数不等式 (cardinal inequality)}
\index{order!non-strict}%
\index{cardinal number!inequality of}%
的关系定义为截断上的归纳:
\symlabel{inj}
\[ \cd{A} \le \cd{B} \defeq \brck{\inj(A,B)} \]
其中 $\inj(A,B)$ 是从 $A$ 到 $B$ 的单射 (injections) 的类型。
\index{function!injective}%
换句话说,$\cd{A} \le \cd{B}$ 意味着从 $A$ 到 $B$ 仅仅存在一个单射。
\end{defn}

\begin{lem}
基数不等式是一个预序 (preorder),即对于 $\alpha,\beta:\card$ 我们有:
\index{preorder!of cardinal numbers}%
\begin{gather*}
\alpha \le \alpha\\
(\alpha \le \beta) \to (\beta\le\gamma) \to (\alpha\le\gamma)
\end{gather*}
\end{lem}
\begin{proof}
同前,通过截断上的归纳。
例如,类型 $(\alpha \le \beta) \to (\beta\le\gamma) \to (\alpha\le\gamma)$ 是一个纯命题,因此通过 0-截断上的归纳,我们可以假设 $\alpha$、$\beta$ 和 $\gamma$ 分别是 $\cd{A}$、$\cd{B}$ 和 $\cd{C}$。
现在,由于 $\cd{A} \le \cd{C}$ 是一个纯命题,通过 $(-1)$-截断上的归纳,我们可以假设给定了从 $A$ 到 $B$ 的单射 $f$ 和从 $B$ 到 $C$ 的单射 $g$。
但是,$g\circ f$ 是从 $A$ 到 $C$ 的一个单射,因此 $\cd{A} \le \cd{C}$ 成立。
反身性 (reflexivity) 更容易证明。
\end{proof}

我们还可以证明基数不等式与半环运算的兼容性。

\begin{lem}\label{thm:injsurj}
\index{function!injective}%
\index{function!surjective}%
考虑以下命题:
\begin{enumerate}
\item 存在从 $A$ 到 $B$ 的一个单射。\label{item:cle-inj}
\item 存在从 $B$ 到 $A$ 的一个满射。\label{item:cle-surj}
\end{enumerate}
那么,在假设排中律 (excluded middle) 的情况下:
\index{excluded middle}%
\index{axiom!of choice}%
\begin{itemize}
\item 给定 $a_0:A$,我们有 \ref{item:cle-inj}$\to$\ref{item:cle-surj}。
\item 因此,如果 $A$ 是居留 (merely inhabited) 的,我们有 \ref{item:cle-inj} $\to$ 仅仅存在 \ref{item:cle-surj}。
\item 假设选择公理 (axiom of choice),我们有 \ref{item:cle-surj} $\to$ 仅仅存在 \ref{item:cle-inj}。
\end{itemize}
\end{lem}
\begin{proof}
如果 $f:A\to B$ 是一个单射,则定义 $g:B\to A$ 如下。
由于 $f$ 是单射,因此 $f$ 在 $b$ 处的纤维是一个纯命题。
因此,根据排中律,要么存在 $a:A$ 满足 $f(a)=b$,要么不存在。
在第一种情况下,定义 $g(b)\defeq a$;否则设定 $g(b)\defeq a_0$。
那么对于任何 $a:A$,我们有 $a = g(f(a))$,所以 $g$ 是满射。

第二条是通过截断上的归纳得到的。
对于第三条,如果 $g:B\to A$ 是满射,那么根据选择公理,仅仅存在一个函数 $f:A\to B$ 满足对所有的 $a$ 有 $g(f(a)) = a$。
但此时 $f$ 必然是单射。
\end{proof}

\begin{thm}[Schroeder--Bernstein (施罗德–贝尔斯坦定理)]
\index{theorem!Schroeder--Bernstein}%
\index{Schroeder--Bernstein theorem}%
在假设排中律的情况下,对于集合 $A$ 和 $B$ 我们有:
\[ \inj(A,B) \to \inj(B,A) \to (A\cong B) \]
\end{thm}
\begin{proof}
常见的“来回反复 (back-and-forth)”论证在此仍然适用。
注意,这实际上构造了一个同构 $A\cong B$(假设排中律,以便我们可以决定给定元素是属于循环、无限链、从 $A$ 开始的链,还是从 $B$ 开始的链)。
\end{proof}

\begin{cor}
在假设排中律的情况下,基数不等式是一个偏序 (partial order),即对于 $\alpha,\beta:\card$ 我们有:
\[ (\alpha\le\beta) \to (\beta\le\alpha) \to (\alpha=\beta). \]
\end{cor}
\begin{proof}
由于 $\alpha=\beta$ 是一个纯命题,通过截断上的归纳,我们可以假设 $\alpha$ 和 $\beta$ 分别是 $\cd{A}$ 和 $\cd{B}$,并且我们有从 $A$ 到 $B$ 的单射 $f$ 和从 $B$ 到 $A$ 的单射 $g$。
但根据施罗德–贝尔斯坦定理,我们得到了一个同构 $A\cong B$,因此得到了一个等式 $\cd{A}=\cd{B}$。
\end{proof}

最后,我们可以重现康托尔 (Cantor) 定理,证明对于每个基数,都存在一个更大的基数。

\begin{thm}[康托尔定理 (Cantor's theorem)]
\index{Cantor's theorem}%
\index{theorem!Cantor's}%
对于 $A:\set$,不存在满射 $A \to (A\to \bool)$。
\end{thm}
\begin{proof}
假设 $f:A \to (A\to \bool)$ 是任意一个函数,定义 $g:A\to \bool$ 为 $g(a) \defeq \neg f(a)(a)$。
如果 $g = f(a_0)$,那么 $g(a_0) = f(a_0)(a_0)$ 但 $g(a_0) = \neg f(a_0)(a_0)$,这就矛盾了。
因此,$f$ 不是满射。
\end{proof}

\begin{cor}
在假设排中律的情况下,对于任意 $\alpha:\card$,存在一个基数 $\beta$ 使得 $\alpha\le\beta$ 且 $\alpha\neq\beta$。
\end{cor}
\begin{proof}
令 $\beta = 2^\alpha$。
现在我们想要证明一个纯命题,所以通过归纳我们可以假设 $\alpha$ 是 $\cd{A}$,因此 $\beta\jdeq \cd{A\to \bool}$。
在排中律的帮助下,我们定义了一个函数 $f:A\to (A\to \bool)$,定义为:
\[f(a)(a') \defeq
\begin{cases}
\btrue &\quad a=a'\\
\bfalse &\quad a\neq a'.
\end{cases}
\]
如果 $f(a)=f(a')$,那么 $f(a')(a) = f(a)(a) = \btrue$,所以 $a=a'$;因此 $f$ 是单射。
因此,$\alpha \jdeq \cd{A} \le \cd{A\to \bool} \jdeq 2^\alpha$。

另一方面,如果 $2^\alpha \le \alpha$,那么我们将有一个单射 $(A\to\bool)\to A$。
根据 \cref{thm:injsurj},由于我们有 $(\lam{x} \bfalse):A\to \bool$ 和排中律,那么就会有一个从 $A$ 到 $(A\to \bool)$ 的满射,这与康托尔定理相矛盾。
\end{proof}

\section{Ordinal numbers}
\label{sec:ordinals}

\index{ordinal|(}%

\begin{defn}\label{defn:accessibility}
  Let $A$ be a set and
  \[(\blank<\blank):A\to A\to \prop\]
  a binary relation on $A$.
  We define by induction what it means for an element $a:A$ to be \define{accessible}
  \indexdef{accessibility}%
  \indexsee{accessible}{accessibility}%
  by $<$:
  \begin{itemize}
  \item If $b$ is accessible for every $b<a$, then $a$ is accessible.
  \end{itemize}
  We write $\acc(a)$ to mean that $a$ is accessible.
\end{defn}

It may seem that such an inductive definition can never get off the ground, but of course if $a$ has the property that there are \emph{no} $b$ such that $b<a$, then $a$ is vacuously accessible.

Note that this is an inductive definition of a family of types, like the type of vectors considered in \cref{sec:generalizations}.
More precisely, it has one constructor, say $\acc_<$, with type
\[ \acc_< : \prd{a:A} \Parens{\prd{b:A} (b<a) \to \acc(b)} \to \acc(a). \]
\index{induction principle!for accessibility}%
The induction principle for $\acc$ says that for any $P:\prd{a:A} \acc(a) \to \type$, if we have
\[f:\prd{a:A}{h:\prd{b:A} (b<a) \to \acc(b)}
\Parens{\prd{b:A}{l:b<a} P(b,h(b,l))} \to
P(a,\acc_<(a,h)),
\]
then we have $g:\prd{a:A}{c:\acc(a)} P(a,c)$ defined by induction, with
\[g(a,\acc_<(a,h)) \jdeq f(a,\,h,\,\lam{b}{l} g(b,h(b,l))).\]
This is a mouthful, but generally we apply it only in the simpler case where $P:A\to\type$ depends only on $A$.
In this case the second and third arguments of $f$ may be combined, so that what we have to prove is
\[f:\prd{a:A} \Parens{\prd{b:A} (b<a) \to \acc(b) \times P(b)}
\to P(a).
\]
That is, we assume every $b<a$ is accessible and $g(b):P(b)$ is defined, and from these define $g(a):P(a)$.

The omission of the second argument of $P$ is justified by the following lemma, whose proof is the only place where we use the more general form of the induction principle.

\begin{lem}
  Accessibility\index{accessibility} is a mere property.
\end{lem}
\begin{proof}
  We must show that for any $a:A$ and $s_1,s_2:\acc(a)$ we have $s_1=s_2$.
  We prove this by induction on $s_1$, with
  \[P_1(a,s_1) \defeq \prd{s_2:\acc(a)} (s_1=s_2). \]
  Thus, we must show that for any $a:A$ and ${h_1:\prd{b:A} (b<a) \to \acc(b)}$ and
  \[ k_1:{\prd{b:A}{l:b<a}{t:\acc(b)} h_1(b,l) = t},\]
  we have $\acc_<(a,h) = s_2$ for any $s_2:\acc(a)$.
  We regard this statement as $\prd{a:A}{s_2:\acc(a)} P_2(a,s_2)$, where
  \[P_2(a,s_2) \defeq
  \prd{h_1 : \cdots } %{h_1:\prd{b:A} (b<a) \to \acc(b)}
  {k_1 : \cdots} % \Parens{\prd{b:A}{l:b<a}{t:\acc(b)} h_1(b,l) = t} \to
  (\acc_<(a,h_1) = s_2);
  \]
  thus we may prove it by induction on $s_2$.
  Therefore, we assume $h_2 : \prd{b:A} (b<a) \to \acc(b)$, and $k_2$ with a monstrous but irrelevant type,
  % \begin{narrowmultline*}
  %   k_2:\prd{b:A}{l:b<a}
  %   \prd{h_1:\prd{b':A} (b'<b) \to \acc(b')}
  %   \narrowbreak
  %   \Parens{\prd{b':A}{l':b'<b}{t':\acc(b')} h_1(b',l') = t'} \to
  %   (\acc_<(b,h_1) = h_2(b,l)).
  % \end{narrowmultline*}
  and must show that for any $h_1$ and $k_1$ with types as above,
  we have $\acc_<(a,h_1) = \acc_<(a,h_2)$.
  By function extensionality, it suffices to show $h_1(b,l) = h_2(b,l)$ for all $b:A$ and $l:b<a$.
  This follows from $k_1$.
\end{proof}

\begin{defn}
  A binary relation $<$ on a set $A$ is \define{well-founded}
  \indexdef{relation!well-founded}%
  \indexdef{well-founded!relation}%
  if every element of $A$ is accessible.
\end{defn}

The point of well-foundedness is that for $P:A\to \type$, we can use the induction principle of $\acc$ to conclude $\prd{a:A} \acc(a) \to P(a)$, and then apply well-foundedness to conclude $\prd{a:A} P(a)$.
In other words, if from $\fall{b:A} (b<a) \to P(b)$ we can prove $P(a)$, then $\fall{a:A} P(a)$.
This is called \define{well-founded induction}\indexdef{well-founded!induction}.

\begin{lem}
  Well-foundedness is a mere property.
\end{lem}
\begin{proof}
  Well-foundedness of $<$ is the type $\prd{a:A} \acc(a)$, which is a mere proposition since each $\acc(a)$ is.
\end{proof}

\begin{eg}\label{thm:nat-wf}
  Perhaps the most familiar well-founded relation is the usual strict ordering on \nat.
  To show that this is well-founded, we must show that $n$ is accessible for each $n:\nat$.
  \index{strong!induction}%
  This is just the usual proof of ``strong induction'' from ordinary induction on \nat.

  Specifically, we prove by induction on $n:\nat$ that $k$ is accessible for all $k\le n$.
  The base case is just that $0$ is accessible, which is vacuously true since nothing is strictly less than $0$.
  For the inductive step, we assume that $k$ is accessible for all $k\le n$, which is to say for all $k<n+1$; hence by definition $n+1$ is also accessible.

  A different relation on \nat which is also well-founded is obtained by setting only $n < \suc(n)$ for all $n:\nat$.
  Well-foundedness of this relation is almost exactly the ordinary induction principle of \nat.
\end{eg}

\begin{eg}\label{thm:wtype-wf}
  Let $A:\set$ and $B : A \to \set$ be a family of sets.
  Recall from \cref{sec:w-types} that the $W$-type $\wtype{a:A} B(a)$ is inductively generated by the single constructor
  \begin{itemize}
  \item $\supp : \prd{a:A} (B(a) \to \wtype{x:A} B(x)) \to \wtype{x:A} B(x)$
  \end{itemize}
  We define the relation $<$ on $\wtype{x:A} B(x)$ by recursion on its second argument:
  \begin{itemize}
  \item For any $a:A$ and $f:B(a) \to \wtype{x:A} B(x)$, we define $w<\supp(a,f)$ to mean that there merely exists a $b:B(a)$ such that $w = f(b)$.
  \end{itemize}
  Now we prove that every $w:\wtype{x:A} B(x)$ is accessible for this relation, using the usual induction principle for $\wtype{x:A}B(x)$.
  This means we assume given $a:A$ and $f:B(a) \to \wtype{x:A} B(x)$, and also a lifting $f' : \prd{b:B(a)} \acc(f(b))$.
  But then by definition of $<$, we have $\acc(w)$ for all $w<\supp(a,f)$; hence $\supp(a,f)$ is accessible.
\end{eg}

Well-foundedness allows us to define functions by recursion and prove statements by induction, such as for instance the following.
Recall from \cref{subsec:prop-subsets} that $\power B$ denotes the \emph{power set}\index{power set} $\power B \defeq (B\to\prop)$.

\begin{lem}\label{thm:wfrec}
  Suppose $B$ is a set and we have a function
  \[ g : \power B \to B \]
  Then if $<$ is a well-founded relation on $A$, there is a function $f:A\to B$ such that for all $a:A$ we have
  \begin{equation*}
    f(a) = g\Big(\setof{ f(a') | a'<a }\Big).
  \end{equation*}
\end{lem}
\noindent
(We are using the notation for images of subsets from \cref{sec:image}.)
\begin{proof}
  We first define, for every $a:A$ and $s:\acc(a)$, an element $\bar f(a,s):B$.
  By induction, it suffices to assume that $s$ is a function assigning to each $a'<a$ a witness $s(a'):\acc(a')$, and that moreover for each such $a'$ we have an element $\bar f(a',s(a')):B$.
  In this case, we define
  \begin{equation*}
    \bar f(a,s) \defeq g\Big(\setof{ \bar f(a',s(a')) | a'<a }\Big).
  \end{equation*}

  Now since $<$ is well-founded, we have a function $w:\prd{a:A} \acc(a)$.
  Thus, we can define $f(a)\defeq \bar f (a,w(a))$.
\end{proof}

In classical\index{mathematics!classical} logic, well-foundedness has a more well-known reformulation.
In the following, we say that a subset $B: \power A$ is \define{nonempty}
\indexdef{nonempty subset}
if it is unequal to the empty subset $(\lam{x}\bot) : \power X$.
We leave it to the reader to verify that assuming excluded middle, this is equivalent to mere inhabitation, i.e.\ to the condition $\exis{x:A} x\in B$.

\begin{lem}\label{thm:wfmin}
  \index{excluded middle}%
  Assuming excluded middle, $<$ is well-founded if and only if every nonempty subset $B: \power A$ merely has a minimal element.
\end{lem}
\begin{proof}
  Suppose first $<$ is well-founded, and suppose $B\subseteq A$ is a subset with no minimal element.
  That is, for any $a:A$ with $a\in B$, there merely exists a $b:A$ with $b<a$ and $b\in B$.

  We claim that for any $a:A$ and $s:\acc(a)$, we have $a\notin B$.
  By induction, we may assume $s$ is a function assigning to each $a'<a$ a proof $s(a'):\acc(a')$, and that moreover for each such $a'$ we have $a'\notin B$.
  If $a\in B$, then by assumption, there would merely exist a $b<a$ with $b\in B$, which contradicts this assumption.
  Thus, $a\notin B$; this completes the induction.
  Since $<$ is well-founded, we have $a\notin B$ for all $a:A$, i.e. $B$ is empty.

  Now suppose each nonempty subset merely has a minimal element.
  Let $B = \setof{ a:A | \neg \acc(a) }$.
  Then if $B$ is nonempty, it merely has a minimal element.
  Thus there merely exists an $a:A$ with $a\in B$ such that for all $b<a$, we have $\acc(b)$.
  But then by definition (and induction on truncation), $a$ is merely accessible, and hence accessible, contradicting $a\in B$.
  Thus, $B$ is empty, so $<$ is well-founded.
\end{proof}

\begin{defn}
  A well-founded relation $<$ on a set $A$ is \define{extensional}
  \indexdef{relation!extensional}%
  \indexdef{extensional!relation}%
  if for any $a,b:A$, we have
  \[ \Parens{\fall{c:A} (c<a) \Leftrightarrow (c<b)} \to (a=b). \]
\end{defn}

Note that since $A$ is a set, extensionality is a mere proposition.
This notion of ``extensionality'' is unrelated to function extensionality, and also unrelated to the extensionality of identity types.
\index{axiom!of extensionality}%
Rather, it is a ``local'' counterpart of the axiom of extensionality in classical set theory.

\begin{thm}
  The type of extensional well-founded relations is a set.
\end{thm}
\begin{proof}
  By the univalence axiom, it suffices to show that if $(A,<)$ is extensional and well-founded and $f:(A,<) \cong (A,<)$, then $f=\idfunc[A]$.
  \index{automorphism!of extensional well-founded relations}%
  We prove by induction on $<$ that $f(a)=a$ for all $a:A$.
  The inductive hypothesis is that for all $a'<a$, we have $f(a')=a'$.

  Now since $A$ is extensional, to conclude $f(a)=a$ it is sufficient to show
  \[\fall{c:A}(c<f(a)) \Leftrightarrow (c<a).\]
  However, since $f$ is an automorphism, we have $(c<a) \Leftrightarrow (f(c)<f(a))$.
  But $c<a$ implies $f(c)=c$ by the inductive hypothesis, so $(c<a) \to (c<f(a))$.
  On the other hand, if $c<f(a)$, then $f^{-1}(c)<a$, and so $c = f(f^{-1}(c)) = f^{-1}(c)$ by the inductive hypothesis again; thus $c<a$.
  Therefore, we have $(c<a) \Leftrightarrow (c<f(a))$ for any $c:A$, so $f(a)=a$.
\end{proof}

\begin{defn}\label{def:simulation}
  If $(A,<)$ and $(B,<)$ are extensional and well-founded, a \define{simulation}
  \indexdef{simulation}%
  \indexsee{function!simulation}{simulation}%
  is a function $f:A\to B$ such that
  \begin{enumerate}
  \item if $a<a'$, then $f(a)<f(a')$, and\label{item:sim1}
  \item for all $a:A$ and $b:B$, if $b<f(a)$, then there merely exists an $a'<a$ with $f(a')=b$.\label{item:sim2}
  \end{enumerate}
\end{defn}

\begin{lem}
  Any simulation is injective.
\end{lem}
\begin{proof}
  We prove by double well-founded induction that for any $a,b:A$, if $f(a)=f(b)$ then $a=b$.
  The inductive hypothesis for $a:A$ says that for any $a'<a$, and any $b:B$, if $f(a')=f(b)$ then $a=b$.
  The inner inductive hypothesis for $b:A$ says that for any $b'<b$, if $f(a)=f(b')$ then $a=b'$.

  Suppose $f(a)=f(b)$; we must show $a=b$.
  By extensionality, it suffices to show that for any $c:A$ we have $(c<a)\Leftrightarrow (c<b)$.
  If $c<a$, then $f(c)<f(a)$ by \cref{def:simulation}\ref{item:sim1}.
  Hence $f(c)<f(b)$, so by \cref{def:simulation}\ref{item:sim2} there merely exists $c':A$ with $c'<b$ and $f(c)=f(c')$.
  By the inductive hypothesis for $a$, we have $c=c'$, hence $c<b$.
  The dual argument is symmetrical.
\end{proof}

In particular, this implies that in \cref{def:simulation}\ref{item:sim2} the word ``merely'' could be omitted without change of sense.

\begin{cor}
  If $f:A\to B$ is a simulation, then for all $a:A$ and $b:B$, if $b<f(a)$, there \emph{purely} exists an $a'<a$ with $f(a')=b$.
\end{cor}
\begin{proof}
  Since $f$ is injective, $\sm{a:A} (f(a)=b)$ is a mere proposition.
\end{proof}

We say that a subset $C :\power B$ is an \define{initial segment}
\indexdef{initial!segment}%
\indexsee{segment, initial}{initial segment}%
if $c\in C$ and $b<c$ imply $b\in C$.
The image of a simulation must be an initial segment, while the inclusion of any initial segment is a simulation.
Thus, by univalence, every simulation $A\to B$ is \emph{equal} to the inclusion of some initial segment of $B$.

\begin{thm}
  For a set $A$, let $P(A)$ be the type of extensional well-founded relations on $A$.
  If $\mathord{<_A} : P(A)$ and $\mathord{<_B} : P(B)$ and $f:A\to B$, let $H_{\mathord{<_A}\mathord{<_B}}(f)$ be the mere proposition that $f$ is a simulation.
  Then $(P,H)$ is a standard notion of structure over \uset in the sense of \cref{sec:sip}.
\end{thm}
\begin{proof}
  We leave it to the reader to verify that identities are simulations, and that composites of simulations are simulations.
  Thus, we have a notion of structure.
  For standardness, we must show that if $<$ and $\prec$ are two extensional well-founded relations on $A$, and $\idfunc[A]$ is a simulation in both directions, then $<$ and $\prec$ are equal.
  Since extensionality and well-foundedness are mere propositions, for this it suffices to have $\fall{a,b:A} (a<b) \Leftrightarrow (a\prec b)$.
  But this follows from \cref{def:simulation}\ref{item:sim1} for $\idfunc[A]$.
\end{proof}

\begin{cor}\label{thm:wfcat}
  There is a category whose objects are sets equipped with extensional well-founded relations, and whose morphisms are simulations.
\end{cor}

In fact, this category is a poset.

\begin{lem}
  For extensional and well-founded $(A,<)$ and $(B,<)$, there is at most one simulation $f:A\to B$.
\end{lem}
\begin{proof}
  Suppose $f,g:A\to B$ are simulations.
  Since being a simulation is a mere property, it suffices to show $\fall{a:A}(f(a)=g(a))$.
  By induction on $<$, we may suppose $f(a')=g(a')$ for all $a'<a$.
  And by extensionality of $B$, to have $f(a)=g(a)$ it suffices to have $\fall{b:B}(b<f(a)) \Leftrightarrow (b<g(a))$.

  But since $f$ is a simulation, if $b<f(a)$, then we have $a'<a$ with $f(a')=b$.
  By the inductive hypothesis, we have also $g(a')=b$, hence $b<g(a)$.
  The dual argument is symmetrical.
\end{proof}

Thus, if $A$ and $B$ are equipped with extensional and well-founded relations, we may write $A\le B$ to mean there exists a simulation $f:A\to B$.
\cref{thm:wfcat} implies that if $A\le B$ and $B\le A$, then $A=B$.

\begin{defn}
  An \define{ordinal}
  \indexdef{ordinal}%
  \indexsee{number!ordinal}{ordinal}%
  is a set $A$ with an extensional well-founded relation which is \emph{transitive}, i.e.\ satisfies $\fall{a,b,c:A}(a<b)\to (b<c) \to (a<c)$.
\end{defn}

\begin{eg}
  Of course, the usual strict order on \nat is transitive.
  It is easily seen to be extensional as well; thus it is an ordinal.
  As usual, we denote this ordinal by $\omega$.
\end{eg}

\symlabel{ord}
Let \ord denote the type of ordinals.
By the previous results, \ord is a set and has a natural partial order.
We now show that \ord also admits a well-founded relation.

\symlabel{initial-segment}
If $A$ is an ordinal and $a:A$, let $\ordsl A a \defeq \setof{ b:A | b<a}$ denote the initial segment.
\index{initial!segment}%
Note that if $\ordsl A a = \ordsl A b$ as ordinals, then that isomorphism must respect their inclusions into $A$ (since simulations form a poset), and hence they are equal as subsets of $A$.
Therefore, since $A$ is extensional, $a=b$.
Thus the function $a\mapsto \ordsl A a$ is an injection $A\to \ord$.

\begin{defn}
  For ordinals $A$ and $B$, a simulation $f:A\to B$ is said to be \define{bounded}
  \indexdef{simulation!bounded}%
  \indexdef{bounded!simulation}%
  if there exists $b:B$ such that $A = \ordsl B b$.
\end{defn}

The remarks above imply that such a $b$ is unique when it exists, so that boundedness is a mere property.

We write $A<B$ if there exists a bounded simulation from $A$ to $B$.
Since simulations are unique, $A<B$ is also a mere proposition.

\begin{thm}\label{thm:ordord}
  $(\ord,<)$ is an ordinal.
\end{thm}

\noindent
More precisely, this theorem says that the type $\ord_{\UU_i}$ of ordinals in one universe\index{universe level} is itself an ordinal in the next higher universe, i.e.\ $(\ord_{\UU_i},<):\ord_{\UU_{i+1}}$.

\begin{proof}
  Let $A$ be an ordinal; we first show that $\ordsl A a$ is accessible (in \ord) for all $a:A$.
  By well-founded induction on $A$, suppose $\ordsl A b$ is accessible for all $b<a$.
  By definition of accessibility, we must show that $B$ is accessible in \ord for all $B<\ordsl A a$.
  However, if $B<\ordsl A a$ then there is some $b<a$ such that $B = \ordsl{(\ordsl A a)}{b} = \ordsl A b$, which is accessible by the inductive hypothesis.
  Thus, $\ordsl A a$ is accessible for all $a:A$.

  Now to show that $A$ is accessible in \ord, by definition we must show $B$ is accessible for all $B<A$.
  But as before, $B<A$ means $B=\ordsl A a$ for some $a:A$, which is accessible as we just proved.
  Thus, \ord is well-founded.

  For extensionality, suppose $A$ and $B$ are ordinals such that
  \narrowequation{\prd{C:\ord} (C<A) \Leftrightarrow (C<B).}
  Then for every $a:A$, since $\ordsl A a<A$, we have $\ordsl A a<B$, hence there is $b:B$ with $\ordsl A a = \ordsl B b$.
  Define $f:A\to B$ to take each $a$ to the corresponding $b$; it is straightforward to verify that $f$ is an isomorphism.
  Thus $A\cong B$, hence $A=B$ by univalence.

  Finally, it is easy to see that $<$ is transitive.
\end{proof}

Treating \ord as an ordinal is often very convenient, but it has its pitfalls as well.
For instance, consider the following lemma, where we pay attention to how universes are used.

\begin{lem}\label{thm:ordsucc}
  Let \bbU be a universe.
  For any $A:\ord_\bbU$, there is a $B:\ord_\bbU$ such that $A<B$.
\end{lem}
\begin{proof}
  Let $B=A+\unit$, with the element $\ttt:\unit$ being greater than all elements of $A$.
  Then $B$ is an ordinal and it is easy to see that $A\cong \ordsl B \ttt$.
\end{proof}

The ordinal $B$ constructed in the proof of \cref{thm:ordsucc} is called the \define{successor}\indexdef{successor!of an ordinal} of $A$.

This lemma illustrates a potential pitfall of the ``typically ambiguous''\index{typical ambiguity} style of using \UU to denote an arbitrary, unspecified universe.
Consider the following alternative proof of it.

\begin{proof}[Another putative proof of \cref{thm:ordsucc}]
  Note that $C<A$ if and only if $C=\ordsl A a$ for some $a:A$.
  This gives an isomorphism $A \cong \ordsl \ord A$, so that $A<\ord$.
  Thus we may take $B\defeq\ord$.
\end{proof}

The second proof would be valid if we had stated \cref{thm:ordsucc} in a typically ambiguous style.
But the resulting lemma would be less useful, because the second proof would constrain the second ``\ord'' in the lemma statement to refer to a higher universe level than the first one.
The first proof allows both universes to be the same.

Similar remarks apply to the next lemma, which could be proved in a less useful way by observing that $A\le \ord$ for any $A:\ord$.

\begin{lem}\label{thm:ordunion}
  Let \bbU be a universe.
  For any $X:\type$ and $F:X\to \ord_\bbU$, there exists $B:\ord_\bbU$ such that $Fx\le B$ for all $x:X$.
\end{lem}
\begin{proof}
  Let $B$ be the set-quotient (see \cref{rmk:quotient-of-non-set}) of the equivalence relation $\eqr$ on $\sm{x:X} Fx$ defined as follows:
  \[ (x,y) \eqr (x',y')
  \;\defeq\;
  \Big(\ordsl{(Fx)}{y} \cong \ordsl{(Fx')}{y'}\Big).
  \]
  Define $(x,y)<(x',y')$ if $\ordsl{(Fx)}{y} < \ordsl{(Fx')}{y'}$.
  This clearly descends to the quotient, and can be seen to make $B$ into an ordinal.
  Moreover, for each $x:X$ the induced map $Fx\to B$ is a simulation.
\end{proof}



\section{Classical well-orderings}
\label{sec:wellorderings}

\index{denial|(}%
We now show the equivalence of our ordinals with the more familiar classical\index{mathematics!classical} well-orderings.

\begin{lem}
  \index{excluded middle}%
  Assuming excluded middle, every ordinal is trichotomous:
  \index{trichotomy of ordinals}%
  \index{ordinal!trichotomy of}%
  \[ \fall{a,b:A} (a<b) \vee (a=b) \vee (b<a). \]
\end{lem}
\begin{proof}
  By induction on $a$, we may assume that for every $a'<a$ and every $b':A$, we have $(a'<b') \vee (a'=b') \vee (b'<a')$.
  Now by induction on $b$, we may assume that for every $b'<b$, we have $(a<b') \vee (a=b') \vee (b'<a)$.

  By excluded middle, either there merely exists a $b'<b$ such that $a<b'$, or there merely exists a $b'<b$ such that $a=b'$, or for every $b'<b$ we have $b'<a$.
  In the first case, merely $a<b$ by transitivity, hence $a<b$ as it is a mere proposition.
  Similarly, in the second case, $a<b$ by transport.
  Thus, suppose $\fall{b':A}(b'<b)\to (b'<a)$.

  Now analogously, either there merely exists $a'<a$ such that $b<a'$, or there merely exists $a'<a$ such that $a'=b$, or for every $a'<a$ we have $a'<b$.
  In the first and second cases, $b<a$, so we may suppose $\fall{a':A}(a'<a)\to (a'<b)$.
  However, by extensionality, our two suppositions now imply $a=b$.
\end{proof}

\begin{lem}
  A well-founded relation contains no cycles, i.e.\
  \[ \fall{n:\mathbb{N}}{a:\mathbb{N}_n\to A} \neg\Big((a_0<a_1) \wedge \dots \wedge (a_{n-1}<a_n)\wedge (a_n<a_0)\Big). \]
\end{lem}
\begin{proof}
  We prove by induction on $a:A$ that there is no cycle containing $a$.
  Thus, suppose by induction that for all $a'<a$, there is no cycle containing $a'$.
  But in any cycle containing $a$, there is some element less than $a$ and contained in the same cycle.
\end{proof}

\indexdef{relation!irreflexive}%
\index{irreflexivity!of well-founded relation}%
In particular, a well-founded relation must be \define{irreflexive}, i.e.\ $\neg(a<a)$ for all $a$.

\begin{thm}\label{thm:wellorder}
  Assuming excluded middle, $(A,<)$ is an ordinal if and only if every nonempty subset $B\subseteq A$ has a least element.
\end{thm}
\begin{proof}
  If $A$ is an ordinal, then by \cref{thm:wfmin} every nonempty subset merely has a minimal element.
  But trichotomy implies that any minimal element is a least element.
  Moreover, least elements are unique when they exist, so merely having one is as good as having one.

  Conversely, if every nonempty subset has a least element, then by \cref{thm:wfmin}, $A$ is well-founded.
  We also have trichotomy, since for any $a,b$ the subset
  $ \setof{a,b} \defeq \setof{x:A | x=a \lor x=b} $
  merely has a least element, which must be either $a$ or $b$.
  This implies transitivity, since if $a<b$ and $b<c$, then either $a=c$ or $c<a$ would produce a cycle.
  Similarly, it implies extensionality, for if $\fall{c:A}(c<a)\Leftrightarrow (c<b)$, then $a<b$ implies (letting $c$ be $a$) that $a<a$, which is a cycle, and similarly if $b<a$; hence $a=b$.
\end{proof}

In classical\index{mathematics!classical} mathematics, the characterization of \cref{thm:wellorder} is taken as the definition of a \emph{well-ordering}, with the \emph{ordinals} being a canonical set of representatives of isomorphism classes for well-orderings.
In our context, the structure identity principle means that there is no need to look for such representatives: any well-ordering is as good as any other.

We now move on to consider consequences of the axiom of choice.
For any set $X$, let $\powerp X$ denote the type of merely inhabited subsets of $X$:
\symlabel{inhabited-powerset}
\[ \powerp X \defeq \setof{ Y : \power X | \exis{x:X} x\in Y}. \]
Assuming excluded middle, this is equivalently the type of \emph{nonempty}\index{nonempty subset} subsets of $X$, and we have $\power X \eqvsym (\powerp X) + \unit$.

\begin{thm}\label{thm:wop}
  \index{axiom!of choice}%
  \index{excluded middle}%
  Assuming excluded middle, the following are equivalent.
  \begin{enumerate}
  \item For every set $X$, there merely exists a function
    $ f: \powerp X \to X $
    such that $f(Y)\in Y$ for all $Y:\powerp X$.\label{item:wop1}
  \item Every set merely admits the structure of an ordinal.\label{item:wop2}
  \end{enumerate}
\end{thm}

\noindent
Of course,~\ref{item:wop1} is a standard classical\index{mathematics!classical} version of the axiom of choice; see \cref{ex:choice-function}.

\begin{proof}
  One direction is easy: suppose~\ref{item:wop2}.
  Since we aim to prove the mere proposition~\ref{item:wop1}, we may assume $A$ is an ordinal.
  But then we can define $f(B)$ to be the least element of $B$.

  Now suppose~\ref{item:wop1}.
  As before, since~\ref{item:wop2} is a mere proposition, we may assume given such an $f$.
  We extend $f$ to a function
  \[ \bar f:\power X \eqvsym (\powerp X) + \unit \longrightarrow X+\unit
  \]
  in the obvious way.
  Now for any ordinal $A$, we can define $g_A:A\to X+\unit$ by well-founded recursion:
  \[ g_A(a) \defeq
    \bar f\Big(X \setminus \setof{ g_A(b) | \strut (b<a) \wedge (g_A(b) \in X) }\Big)
  \]
  (regarding $X$ as a subset of $X+\unit$ in the obvious way).

  Let $A'\defeq \setof{a:A | g_A(a) \in X}$ be the preimage of $X\subseteq X+\unit$; then we claim the restriction $g_A':A' \to X$ is injective.
  For if $a,a':A$ with $a\neq a'$, then by trichotomy and without loss of generality, we may assume $a'<a$.
  Thus $g_A(a') \in \setof{ g_A(b) | b<a }$, so since $f(Y)\in Y$ for all $Y$ we have $g_A(a) \neq g_A(a')$.

  Moreover, $A'$ is an initial segment of $A$.
  For $g_A(a)$ lies in \unit if and only if $\setof{g_A(b)|b<a} = X$, and if this holds then it also holds for any $a'>a$.
  Thus, $A'$ is itself an ordinal.

  Finally, since \ord is an ordinal, we can take $A\defeq\ord$.
  Let $X'$ be the image of $g_\ord':\ord' \to X$; then the inverse of $g_\ord'$ yields an injection $H:X'\to \ord$.
  By \cref{thm:ordunion}, there is an ordinal $C$ such that $Hx\le C$ for all $x:X'$.
  Then by \cref{thm:ordsucc}, there is a further ordinal $D$ such that $C<D$, hence $Hx<D$ for all $x:X'$.
  Now we have
  \begin{align*}
    g_{\ord}(D) &= \bar f\Big( X \setminus \setof{ g_\ord(B) | \rule{0pt}{1em} B<D \wedge (g_\ord(B) \in X)} \Big)\\
    &=\bar f\Big( X \setminus \setof{ g_\ord(B) | \rule{0pt}{1em} g_\ord(B) \in X} \Big)
  \end{align*}
  since if $B:\ord$ and $(g_\ord(B) \in X)$, then $B = Hx$ for some $x:X'$, hence $B<D$.
  Now if
  \[\setof{ g_\ord(B) | \rule{0pt}{1em} g_\ord(B) \in X}\]
  is not all of $X$, then $g_\ord(D)$ would lie in $X$ but not in this subset, which would be a contradiction since $D$ is itself a potential value for $B$.
  So this set must be all of $X$, and hence $g_\ord'$ is surjective as well as injective.
  Thus, we can transport the ordinal structure on $\ord'$ to $X$.
\end{proof}

\begin{rmk}
  If we had given the wrong proof of \cref{thm:ordsucc} or \cref{thm:ordunion}, then the resulting proof of \cref{thm:wop} would be invalid: there would be no way to consistently assign universe levels\index{universe level}.
  As it is, we require propositional resizing (which follows from \LEM{}) to ensure that $X'$ lives in the same universe as $X$ (up to equivalence).
\end{rmk}

\begin{cor}
  Assuming the axiom of choice, the function $\ord\to\set$ (which forgets the order structure) is a surjection.
\end{cor}

Note that \ord is a set, while \set is a 1-type.
In general, there is no reason for a 1-type to admit any surjective function from a set.
Even the axiom of choice does not appear to imply that \emph{every} 1-type does so (although see \cref{ex:acnm-surjset}), but it readily implies that this is so for 1-types constructed out of \set, such as the types of objects of categories of structures as in \cref{sec:sip}.
The following corollary also applies to such categories.

\begin{cor}
  \index{weak equivalence!of precategories}%
  Assuming \choice{}, \uset admits a weak equivalence functor from a strict category.
\end{cor}
\begin{proof}
  Let $X_0\defeq \ord$, and for $A,B:X_0$ let $\hom_X(A,B) \defeq (A\to B)$.
  Then $X$ is a strict category, since \ord is a set, and the above surjection $X_0 \to \set$ extends to a weak equivalence functor $X\to \uset$.
\end{proof}

Now recall from \cref{sec:cardinals} that we have a further surjection $\cd{\blank}:\set\to\card$, and hence a composite surjection $\ord\to\card$ which sends each ordinal to its cardinality.

\begin{thm}
  Assuming \choice{}, the surjection $\ord\to\card$ has a section.
\end{thm}
\begin{proof}
  There is an easy and wrong proof of this: since \ord and \card are both sets, \choice{} implies that any surjection between them \emph{merely} has a section.
  However, we actually have a canonical \emph{specified} section: because \ord is an ordinal, every nonempty subset of it has a uniquely specified least element.
  Thus, we can map each cardinal to the least element in the corresponding fiber.
\end{proof}

It is traditional in set theory to identify cardinals with their image in \ord: the least ordinal having that cardinality.

It follows that \card also canonically admits the structure of an ordinal: in fact, one isomorphic to \ord.
Specifically, we define by well-founded recursion a function $\aleph:\ord\to\ord$, such that $\aleph(A)$ is the least ordinal having cardinality greater than $\aleph({\ordsl A a})$ for all $a:A$.
Then (assuming \choice{}) the image of $\aleph$ is exactly the image of \card.

\index{denial|)}%

\index{ordinal|)}%

\section{The cumulative hierarchy}
\label{sec:cumulative-hierarchy}

\index{bargaining|(}%
We can define a cumulative hierarchy $V$ of all sets in a given universe $\UU$ as a higher inductive type, in such a way that $V$ is again a set (in a larger universe $\UU'$), equipped with a binary ``membership'' relation $x\in y$ which satisfies the usual laws of set theory.

\begin{defn}\label{defn:V}
  The \define{cumulative hierarchy}
  \indexdef{cumulative!hierarchy, set-theoretic}%
  \indexdef{hierarchy!cumulative, set-theoretic}%
  $V$ relative to a type universe $\UU$ is the
  higher inductive type generated by the following constructors.
  %
  \begin{enumerate}
  \item For every $A : \UU$ and $f : A \to V$, there is an element $\vset(A, f) : V$.
  \item For all $A, B : \UU$, $f : A \to V$ and $g : B \to V$ such that
    %
    \begin{narrowmultline} \label{eq:V-path}
      \big(\fall{a:A} \exis{b:B} \id[V]{f(a)}{g(b)}\big) \land \narrowbreak
      \big(\fall{b:B} \exis{a:A} \id[V]{f(a)}{g(b)}\big)
    \end{narrowmultline}
    %
    there is a path $\id[V]{\vset(A,f)}{\vset(B,g)}$.
  \item The 0-truncation constructor: for all $x,y:V$ and $p,q:x=y$, we have $p=q$.
  \end{enumerate}
\end{defn}

In set-theoretic language, $\vset(A,f)$ can be understood as the set (in the sense of classical set theory) that is the image of $A$ under $f$, i.e.\ $\setof{ f(a) | a \in A }$.
However, we will avoid this notation, since it would clash with our notation for subtypes (but see~\eqref{eq:class-notation} and \cref{def:TypeOfElements} below).

The hierarchy $V$ is
bootstrapped from the empty map $\rec\emptyt(V) : \emptyt \to V$, which gives the empty set as $\emptyset = \vset(\emptyt,\rec\emptyt(V))$.
Then the singleton $\{\emptyset\}$ enters $V$ through $\unit \to V$, defined as $\ttt \mapsto \emptyset$, and so
on.
(The definition can also be adapted to include an arbitrary set of ``atoms'' or ``urelements'', by adding an additional point constructor.)
The type $V$ lives in the same universe as the base universe $\UU$.

The second constructor of $V$ has a form unlike any we have seen before: it involves not only paths in $V$ (which in \cref{sec:hittruncations} we claimed were slightly fishy) but truncations of sums of them.
It certainly does not fit the general scheme described in \cref{sec:naturality}, and thus it may not be obvious what its induction principle should be.
Fortunately, like our first definition of the 0-truncation in \cref{sec:hittruncations}, it can be re-expressed using auxiliary higher inductive types.
We leave it to the reader to work out the details (see \cref{ex:cumhierhit}).

\index{induction principle!for cumulative hierarchy}%
At the end of the day, the induction principle for $V$ (written in pattern matching language) says that given $P:V\to \set$, in order to construct $h:\prd{x:V} P(x)$, it suffices to give the following.
\begin{enumerate}
\item For any $f:A\to V$, construct $h(\vset(A,f))$, assuming as given $h(f(a))$ for all $a:A$.
\item Verify that if $f : A \to V$ and $g : B \to V$ satisfy~\eqref{eq:V-path}, then $\dpath{P}{q}{h(\vset(A,f))}{h(\vset(B,g))}$, where $q$ is the path arising from the second constructor of $V$ and~\eqref{eq:V-path}, assuming inductively that $h(f(a))$ and $h(g(b))$ are defined for all $a:A$ and $b:B$, and that the following condition holds:
\begin{eqnarray*}
    &       & \big(\fall{a:A} \exis{b:B} \exis{p:f(a)=g(b)} \dpath{P}{p}{h(f(a))}{h(g(b))}\big) \\
    & \land & \big(\fall{b:B} \exis{a:A} \exis{p:f(a)=g(b)} \dpath{P}{p}{h(f(a))}{h(g(b))}\big)
\end{eqnarray*}
\end{enumerate}
The second clause checks that the map being defined must respect the paths introduced in \eqref{eq:V-path}.
As usual when we state higher induction principles using pattern matching, it may seem tautologous, but is not.
The point is that ``$h(f(a))$'' is essentially a formal symbol which we cannot peek inside of, which $h(\vset(A,f))$ must be defined in terms of. Thus, in the second clause, we assume equality of these formal symbols when appropriate, and verify that the elements resulting from the construction of the first clause are also equal.
Of course, if $P$ is a family of mere propositions, then the second clause is automatic.

Observe that, by induction, for each $v:V$ there merely exist $A:\UU$ and $f:A\to V$ such that $v=\vset(A,f)$.
Thus, it is reasonable to try to define the \define{membership relation}
\indexdef{membership, for cumulative hierarchy}%
$x\in v$ on $V$ by setting:
%
% Note: "membership" rather than "elementhood", because "element" is taken.
%
\symlabel{V-membership}
\begin{equation*}
  (x \in \vset(A,f)) \defeq (\exis{a : A} x = f(a)).
\end{equation*}
%
To see that the definition is valid, we must use the recursion principle of $V$.  Thus, suppose we have a path $\vset(A, f) = \vset(B, g)$
constructed through~\eqref{eq:V-path}. If $x \in \vset(A,f)$ then there merely is $a : A$ such
that $x = f(a)$, but by~\eqref{eq:V-path} there merely is $b : B$ such that $f(a) = g(b)$, hence
$x = g(b)$ and $x \in \vset(B,g)$. The converse is symmetric.

The \define{subset relation}
\indexdef{subset!relation on the cumulative hierarchy}%
$x\subseteq y$ is defined on $V$ as usual by
%
\begin{equation*}
  (x \subseteq y) \defeq \fall{z : V} z \in x \Rightarrow z \in y.
\end{equation*}

A \define{class}
\indexdef{class}%
may be taken to be a mere predicate on~$V$. We can say that a class $C : V \to \prop$ is a
\define{$V$-set}
\indexdef{set!in the cumulative hierarchy}%
if there merely exists $v:V$ such that
%
\begin{equation*}
  \fall{x : V} C(x) \Leftrightarrow x \in v.
\end{equation*}
We may also use the conventional notation for classes, which matches our standard notation for subtypes:
\begin{equation}
  \setof{ x | C(x) } \defeq \lam{x}C(x).\label{eq:class-notation}
\end{equation}
%
A class $C: V\to \prop$ will be called \define{$\UU$-small}
\indexdef{class!small}%
\indexdef{small!class}%
if all of its values $C(x)$ lie in $\UU$, specifically $C: V\to \prop_{\UU}$.
Since $V$ lives in the same universe $\UU'$ as does the base universe $\UU$ from which it is built, the same is true for the identity types $v=_V w$ for any $v,w:V$. To obtain a well-behaved theory in the absence of propositional resizing,
\index{propositional!resizing}%
\index{resizing}%
therefore, it will be convenient to have a $\UU$-small ``resizing'' of the identity relation, which we can define by induction as follows.

\begin{defn}\label{def:bisimulation}
  Define the \define{bisimulation}
  \indexdef{bisimulation}%
  relation
  %
  \begin{equation*}
    \mathord\bisim : V \times V \longrightarrow \prop_{\UU}
  \end{equation*}
  %
  by double induction over $V$, where for $\vset(A,f)$ and $\vset(B,g)$ we let:
  \begin{narrowmultline*}
    \vset(A,f)  \bisim \vset(B,g) \defeq \narrowbreak
    \big(\fall{a:A}\exis{b:B} f(a)  \bisim g(b)\big) \land
    \big(\fall{b:B}\exis{a:A} f(a) \bisim g(b)\big).
  \end{narrowmultline*}
\end{defn}
%
To verify that the definition is correct, we just need to check that it respects paths $\vset(A, f) = \vset(B, g)$ constructed through~\eqref{eq:V-path}, but this is obvious, and that $\prop_{\UU}$ is a set, which it is.  Note that $u \bisim v$ is in $\propU$ by construction.

\begin{lem}\label{lem:BisimEqualsId}
For any $u,v:V$ we have $(u=_V v) = (u \bisim v)$.
\end{lem}

\begin{proof}
An easy induction shows that $\bisim$ is reflexive, so by transport we have $(u=_V v)\to (u \bisim v)$.
Thus, it remains to show that $(u \bisim v)\to (u=_V v)$.
By induction on $u$ and $v$, we may assume they are $\vset(A,f)$ and $\vset(B,g)$ respectively.
(We can ignore the path constructors of $V$, since $(u \bisim v)\to (u=_V v)$ is a mere proposition.)
Then by definition, $\vset(A,f)\bisim\vset(B,g)$ implies $(\fall{a:A}\exis{b:B}f(a)  \bisim g(b))$ and conversely.
But the inductive hypothesis then tells us that $(\fall{a:A}\exis{b:B}f(a) = g(b))$ and conversely.
So by the path con\-struc\-tor for $V$ we have $\vset(A,f) =_V \vset(B,g)$.
\end{proof}

One might think that we could omit the 0-truncation constructor of $V$ and \emph{prove} that $V$ is 0-truncated by applying \cref{thm:h-set-refrel-in-paths-sets} to the bisimulation.
However, in the proof of \cref{lem:BisimEqualsId} we used the fact that $V$ is 0-truncated, to conclude that $(u \bisim v)\to (u=_V v)$ is a mere proposition so that in the induction it suffices to assume $u$ and $v$ are $\vset(A,f)$ and $\vset(B,g)$.

Now we can use the resized identity relation to get the following useful principle.

\begin{lem}\label{lem:MonicSetPresent}
For every $u:V$ there is a given $A_u:\UU$ and monic $m_u: A_u \mono V$ such that $u = \vset(A_u, m_u)$.
\end{lem}

\begin{proof}
  Take any presentation $u = \vset(A,f)$ and factor $f:A\to V$ as a surjection followed by an injection:
  %
  \begin{equation*}
    f = m_u\circ e_u : A \epi A_u \mono V.
  \end{equation*}
  %
  Clearly $u = \vset(A_u, m_u)$ if only $A_u$ is still in $\UU$, which holds if the kernel of $e_u : A \epi A_u$ is in $\UU$.  But the kernel of $e_u : A \epi A_u$ is the pullback along $f : A\to V$ of the identity on $V$, which we just showed to be $\UU$-small, up to equivalence.  Now, this construction of the pair $(A_u, m_u)$ with $m_u :A_u \mono V$ and $u = \vset(A_u, m_u)$ from $u:V$ is unique up to equivalence over $V$, and hence up to identity by univalence.  Thus by the principle of unique choice \eqref{cor:UC} there is a map $c : V\to\sm{A:\UU}(A\to V)$ such that $c(u) = (A_u, m_u)$, with $m_u :A_u \mono V$ and $u = \vset(c(u))$, as claimed.
\end{proof}

\begin{defn}\label{def:TypeOfElements}
  For $u:V$, the just constructed monic presentation $m_u: A_u \mono V$ such that $u = \vset(A_u, m_u)$ may be called the \define{type of members}
  \indexdef{type!of members}%
  of $u$ and denoted $m_u : [u] \mono V$, or even $[u] \mono V$.  We can think of $[u]$ as the ``subclass of $V$ consisting of members of $u$''.
\end{defn}

\begin{thm}\label{thm:VisCST}
  \index{axiom!of set theory, for the cumulative hierarchy}%
  The following hold for $(V, {\in})$:
  %
  \begin{enumerate}
  \item \emph{extensionality:}
    %
    \begin{equation*}
      \fall{x, y : V} x \subseteq y \land y \subseteq x \Leftrightarrow x = y.
    \end{equation*}
    %
     \item \emph{empty set:} for all $x:V$, we have $\neg (x\in \emptyset)$.
    %
    \item \emph{pairing:} for all $u, v:V$, the class $\{u, v\} \defeq \setof{ x | x = u \vee x = v}$ is a $V$-set.
      %
    \item \emph{infinity:}\index{axiom!of infinity}  there is a $v:V$ with $\emptyset\in v$ and $x\in v$ implies $x\cup \{x\}\in v$.
    %
  \item \emph{union:} for all $v:V$, the class $\cup v\defeq \setof{ x | \exis{u:V} x \in u \in v}$ is a $V$-set.
    %
    \item \emph{function set:} for all $u, v:V$, the class $v^u \defeq \setof{ x | x : u\to v}$ is a $V$-set.%
      \footnote{Here $x:u\to v$ means that $x$ is an appropriate set of ordered pairs, according to the usual way of encoding functions in set theory.}
    %
   \item \emph{$\in$-induction:} if $C : V \to \prop$ is a class such that $C(a)$ holds whenever $C(x)$ for all $x\in a$, then $C(v)$ for all $v:V$.
   %
     \item \emph{replacement:}\index{axiom!of replacement} given any $r : V \to V$ and $x : V$, the class
       %
       \begin{equation*}
         \setof{ y | \exis{z : V} z \in x \land y = r(z)}
       \end{equation*}
       %
       is a $V$-set.
  %
   \item \emph{separation:}\index{axiom!of separation}  given any $a : V$ and $\UU$-small $C : V \to \propU$, the class
     %
     \begin{equation*}
       \setof{ x | x \in a \land C(x)}
     \end{equation*}
     %
     is a $V$-set.
  \end{enumerate}
\end{thm}


\begin{proof}[Sketch of proof]
  \mbox{}
  %
  \begin{enumerate}
  \item Extensionality: if $\vset(A,f) \subseteq \vset(B, g)$ then $f(a) \in \vset(B, g)$
    for every $a : A$, therefore for every $a : A$ there merely exists $b : B$ such that
    $f(a) = g(b)$. The assumption $\vset(B, g) \subseteq \vset(A, f)$ gives the other half
    of~\eqref{eq:V-path}, therefore $\vset(A,f) = \vset(B,g)$.

  \item Empty set: suppose $x\in \emptyset = \vset(\emptyt,\rec\emptyt(V))$.  Then $\exis{a:\emptyt}x=\, \rec\emptyt(V,a)$, which is absurd.

  \item Pairing: given $u$ and $v$, let $w=\vset(\bool,\rec\bool(V,u,v))$.
    \index{pair!unordered}

  \item Infinity: take $w = \vset(\nat,I)$, where $I: \nat \to V$ is given by the recursion $I(0) \defeq \emptyset$ and $I(n+1) \defeq I(n)\cup \{I(n)\}$.

  \item Union: Take any $v:V$ and any presentation $f :A\to V$ with $v=\vset(A,f)$.  Then let $\tilde{A} \defeq \sm{a:A}[fa]$, where $m_{fa} : [fa] \mono V$ is the type of members from \cref{def:TypeOfElements}.  $\tilde{A}$ is plainly $\UU$-small, and we have $\cup v \defeq \vset(\tilde{A}, \lam{x} m_{f(\proj1(x))}(\proj2(x)))$.

  \item Function set: given $u, v:V$, take the types of members $[u] \mono V$ and $[v] \mono V$, and the function type $[u]\to [v]$.  We want to define a map
  \[
 r: ([u]\to [v])\ \longrightarrow\ V
  \]
   with ``$r(f) = \setof{ \pairr{x, f(x)} | x : [u] }$'', but in order for this to make sense we must first define the ordered pair $\pairr{x, y}$, and then we take the map $r': x \mapsto \pairr{x, f(x)}$, and then we can put $r(f)\defeq \vset([u], r')$.  But the ordered pair can be defined in terms of unordered pairing as usual.

  \item $\in$-induction: let $C : V \to \prop$ be a class such that $C(a)$ holds whenever $C(x)$ for all $x\in a$, and take any $v=\vset(B,g)$.  To show that $C(v)$ by induction, assume that $C(g(b))$ for all $b:B$.  For every $x\in v$ there merely exists some $b:B$ with $x = g(b)$, and so $C(x)$.  Thus $C(v)$.

  \item Replacement: let $C$ denote the class in question.
    The statement ``$C$ is a $V$-set'' is a mere proposition, so we may
    proceed by induction as follows. Supposing $x$ is $\vset(A, f)$, we claim that $w
    \defeq \vset(A, r \circ f)$ is the set we are looking for.  If $C(y)$ then there merely exists
    $z : V$ and $a : A$ such that $z = f(a)$ and $y = r(z)$, therefore $y \in w$.
    Conversely, if $y \in w$ then there merely exists $a : A$ such that $y = r(f(a))$, so
    if we take $z \defeq f(a)$ we see that $C(y)$ holds.

  \item Let us say that a class $C: V\to\prop$ is \define{separable}
    \indexdef{class!separable}%
    \indexdef{separable class}%
    if for any $a:V$ the class
  %
  \symlabel{class-intersection}
  \begin{equation*}
    a \cap C \defeq\setof{x | x\in a \wedge C(x)}
  \end{equation*}
  %
  is a $V$-set.
We need to show that any $\UU$-small  $C: V \to \propU$ is separable. Indeed, given $a=\vset(A,f)$, let $A' = \sm{x:A}C(fx)$, and take $f' = f\circ i$, where $i : A' \to A$ is the obvious inclusion.  Then we can take $a' = \vset(A',f')$ and we have $x\in a\wedge C(x) \Leftrightarrow x\in a'$ as claimed.  We needed the assumption that $C$ lands in $\UU$ in order for $A' = \sm{x:A}C(fx)$ to be in $\UU$.\qedhere
\end{enumerate}
\end{proof}

It is also convenient to have a strictly syntactic criterion of separability, so that one can read off from the expression for a class that it produces a $V$-set.  One such familiar condition is being ``$\Delta_0$'', which means that the expression is built up from equality $x=_V y$ and membership $x\in y$, using only mere-propositional connectives $\neg$, $\land$, $\lor$, $\Rightarrow$ and quantifiers $\forall$, $\exists$ over particular sets, i.e.\ of the form $\exists(x\in a)$ and $\forall(y\in b)$ (these are called \define{bounded} quantifiers\index{bounded!quantifier}\index{quantifier!bounded}).\indexdef{separation!.Delta0@$\Delta_0$}%

\begin{cor}\label{cor:Delta0sep}
If the class $C: V \to \prop$ is $\Delta_0$ in the above sense, then it is separable.
\end{cor}
\index{axiom!of $\Delta_0$-separation}%

\begin{proof}
Recall that we have a $\UU$-small resizing $x \bisim y$ of identity $x = y$. Since $x\in y$ is defined in terms of $x=y$, we also have a $\UU$-small resizing of membership
%
\symlabel{resized-membership}
\begin{equation*}
  x\bin\vset(A,f) \defeq \exis{a:A} x \bisim f(a).
\end{equation*}
%
Now, let $\Phi$ be a $\Delta_0$ expression for $C$, so that as classes $\Phi = C$ (strictly speaking, we should distinguish expressions from their meanings, but we will blur the difference). Let $\widetilde{\Phi}$ be the result of replacing all occurrences of $=$ and $\in$ by their resized equivalents $\bisim$ and $\bin$.  Clearly then $\widetilde{\Phi}$ also expresses $C$, in the sense that for all $x:V$, $\widetilde{\Phi}(x) \Leftrightarrow C(x)$, and hence $\widetilde{\Phi}=C$ by univalence.  It now suffices to show that $\widetilde{\Phi}$ is $\UU$-small, for then it will be separable by the theorem.

We show that  $\widetilde{\Phi}$ is $\UU$-small by induction on the construction of the expression.  The base cases are $x \bisim y$ and $x\bin y$, which have already been resized into $\UU$.  It is also clear that $\UU$ is closed under the mere-propositional operations (and $(-1)$-truncation), so it just remains to check the bounded quantifiers $\exists(x\in a)$ and $\forall(y\in b)$.  By definition,
\begin{align*}
\exists(x\in a) P(x) &\defeq \Brck {\sm{x:V}(x\bin a \land P(x))},\\
\forall(y\in b) P(x) &\defeq  \prd{x:V}(x\bin a \to P(x)).
\end{align*}
Let us consider $\brck {\sm{x:V}(x\bin a \land P(x))}$.  Although the body $(x\bin a \land P(x))$ is $\UU$-small since $P(x)$ is so by the inductive hypothesis, the quantification over $V$ need not stay inside $\UU$.  However, in the present case we can replace this with a quantification over the type $[a]\mono V$ of members of $a$, and easily show that
\begin{equation*}
  \sm{x:V}(x\bin a \land P(x)) = \sm{x:[a]} P(x).
\end{equation*}
The right-hand side does remain in $\UU$, since both $[a]$ and $P(x)$ are in $\UU$.  The case of $\prd{x:V}(x\bin a \to P(x))$ is analogous, using $\prd{x:V}(x\bin a \to P(x)) = \prd{x:[a]}P(x)$.
\end{proof}

We have shown that in type theory with a universe $\UU$, the cumulative hierarchy $V$ is a model of a ``constructive set theory''
\index{constructive!set theory}%
with many of the standard axioms.
However, as far as we know, it lacks the \emph{strong collection}
\index{axiom!strong collection}%
\index{collection!strong}%
\index{strong!collection}%
and \emph{subset collection}
\index{axiom!subset collection}%
\index{collection!subset}%
\index{subset!collection}%
axioms which are included in \CZF{}~\cite{AczelCZF}.
In the usual interpretation of this set theory into type theory, these two axioms are consequences of the setoid-like definition of equality; while in other constructed models of set theory, strong collection may hold for other reasons.
We do not know whether either of these axioms holds in our model $(V,\in)$, but it seems unlikely.
Since $V$ is a higher inductive type \emph{inside} the system, rather than being an \emph{external} construction, it is not surprising that it differs in some ways from prior interpretations.

Finally, consider the result of adding the axiom of choice for sets to our type theory, in the form  $\choice{}$ from \cref{subsec:emacinsets} above.  This has the consequence that $\LEM{}$ then also holds, by \cref{thm:1surj_to_surj_to_pem}, and so $\set$ is a topos\index{topos} with subobject classifier $\bool$, by \cref{thm:settopos}.  In this case, we have $\prop = \bool:\UU$, and so \emph{all classes are separable}.
Thus we have shown:

\begin{lem}\label{lem:fullsep}
  In type theory with $\choice{}$, the law of \define{(full) separation}
  \indexdef{separation!full}%
  holds for $V$: given \emph{any} class $C : V \to \prop$ and $a : V$, the class $a \cap C$ is a $V$-set.
\end{lem}

\begin{thm}\label{thm:zfc}
In type theory with $\choice{}$ and a universe $\UU$, the cumulative hierarchy $V$ is a model of Zermelo--Fraenkel\index{set theory!Zermelo--Fraenkel} set theory with choice, ZFC.
\end{thm}

\begin{proof}
We have all the axioms listed in \cref{thm:VisCST}, plus full separation, so we just need to show that there are power sets\index{power set} $\power a:V$ for all $a:V$.  But since we have $\LEM{}$ these are simply function types $\power a = (a\to\bool)$.  Thus $V$ is a model of Zermelo--Fraenkel set theory ZF. We leave the verification of the set-theoretic axiom of choice from $\choice{}$ as an easy exercise.
\end{proof}

\index{bargaining|)}%

%%%%%%%%%%%%%%%%%%%%%%%%%%%%%%%%%%%%%%%%%%%%%%%%%%%%%%%%%%%%%%%%%%%%%%
\sectionNotes

The basic properties one expects of the category of sets date back to the early days of elementary topos theory.
The \emph{Elementary theory of the category of sets} referred to in \cref{subsec:emacinsets} was introduced by Lawvere\index{Lawvere} in
\cite{lawvere:etcs-long}, as a category-theoretic axiomatization of set theory.
\index{Elementary Theory of the Category of Sets}%
The notion of $\Pi W$-pretopos, regarded as a predicative version of an elementary topos, was introduced in~\cite{MoerdijkPalmgren2002}; see also~\cite{palmgren:cetcs}.

The treatment of the category of sets in \cref{sec:piw-pretopos} roughly follows that in~\cite{RijkeSpitters}.
The fact that epimorphisms are surjective (\cref{epis-surj}) is well known in classical mathematics, but is not as trivial as it may seem to prove \emph{predicatively}.
\index{mathematics!predicative}%
The proof in~\cite{Mines/R/R:1988} uses the power set operation (which is impredicative), although it can also be seen as a predicative proof of the weaker statement that a map in a universe $\UU_i$ is surjective if it is an epimorphism in the next universe $\UU_{i+1}$.
A predicative proof for setoids was given by Wilander~\cite{Wilander2010}.
Our proof is similar to Wilander's, but avoids setoids by using pushouts and univalence.

The implication in \cref{thm:1surj_to_surj_to_pem} from $\choice{}$ to $\LEM{}$ is an adaptation to homotopy type
theory of a theorem from topos theory due to Diaconescu~\cite{Diaconescu}; it was posed as a problem already by Bishop~\cite[Problem~2]{Bishop1967}.

For the intuitionistic theory of ordinal numbers, see~\cite{taylor:ordinals,Taylor99} and also \cite{JoyalMoerdijk1995}.
Definitions of well-foundedness in type theory by an induction principle, including the inductive predicate of accessibility\index{accessibility}, were studied in~\cite{Huet80,Paulson86,Nordstrom88}, although the idea dates back to Gentzen's proof of the consistency\index{consistency!of arithmetic} of arithmetic~\cite{Gentzen36}.

The idea of algebraic set theory, which informs our development in \cref{sec:cumulative-hierarchy} of the cumulative hierarchy, is due to~\cite{JoyalMoerdijk1995}, but it derives from earlier work by~\cite{AczelCZF}.
\index{algebraic set theory}%
\index{set theory!algebraic}%


%%%%%%%%%%%%%%%%%%%%%%%%%%%%%%%%%%%%%%%%%%%%%%%%%%%%%%%%%%%%%%%%%%%%%%
\sectionExercises

\begin{ex}\label{ex:utype-ct}
  Following the pattern of $\uset$, we would like to make a category $\utype$ of all types and maps between them (in a given universe $\UU$).  In order for this to be a category in the sense of \cref{sec:cats}, however, we must first declare $\hom(X,Y) \defeq \pizero{X\to Y}$, with composition defined by induction on truncation from ordinary composition $(Y\to Z) \to (X\to Y) \to (X\to Z)$.  This was defined as the \emph{homotopy precategory of types} in \cref{ct:hoprecat}.  It is still not a category, however, but only a precategory (its type of objects $\UU$ is not even a $0$-type).  It becomes a category by Rezk completion
  \index{completion!Rezk}%
  (see \cref{ct:hocat}), and its type of objects can be identified with $\trunc1\type$ by \cref{ct:ex:hocat}.  Show that the resulting category $\utype$, unlike $\uset$, is not a pretopos.
\end{ex}

\begin{ex}\label{ex:surjections-have-sections-impl-ac}
  Show that if every surjection has a section in the category $\uset$, then the axiom of choice holds.
\end{ex}

\begin{ex}\label{ex:well-pointed}
  Show that with $\LEM{}$, the category $\uset$ is well-pointed,
  \indexdef{category!well-pointed}%
  in the sense that the following statement holds: for any $f, g : A\to B$, if $f \neq g$ then there is a function $a : 1\to A$ such that $f(a) \neq g(a)$.
  Show that the slice category
  \index{category!slice}%
  $\uset/\bool$ consisting of functions $A\to \bool$ and commutative triangles does not have this property.
  (Hint: the terminal object in $\uset/\bool$ is the identity function $\bool \to \bool$, so in this category, there are objects $X$ that have no elements $1\to X$.)
\end{ex}

\begin{ex}\label{ex:add-ordinals}
  \index{addition!of ordinal numbers}%
  Prove that if $(A,<_A)$ and $(B,<_B)$ are well-founded, extensional, or ordinals, then so is $A+B$, with $<$ defined by
  \begin{align*}
    (a<a') &\defeq (a<_A a') & \text{for }& a,a':A\\
    (b<b') &\defeq (b<_B b') & \text{for }& b,b':B\\
    (a<b) &\defeq \unit      & \text{for }& (a:A),(b:B)\\
    (b<a) &\defeq \emptyt    & \text{for }& (a:A),(b:B).
  \end{align*}
\end{ex}
% \begin{proof}
%   We first prove by induction on $<_A$ that every element of $A$ is accessible in $A+B$.
%   This is easy since the only elements less than $a:A$ in $A+B$ are also in $A$.
%   We then prove by induction on $<_B$ that every element of $B$ is accessible in $A+B$.
%   This is easy since we have already proven that every element of $A$ is accessible.
% \end{proof}

\begin{ex}\label{ex:multiply-ordinals}
  \index{multiplication!of ordinal numbers}%
  Prove that if $(A,<_A)$ and $(B,<_B)$ are well-founded, extensional, or ordinals, then so is $A\times B$, with $<$ defined by
  \[ ((a,b) <(a',b')) \defeq (a<_A a') \vee ((a=a') \wedge (b<_B b')). \]
\end{ex}
% \begin{proof}
%   We prove by induction on $<_A$ that for every $a:A$, every element of the form $(a,b)$ is accessible in $A\times B$.
%   The inductive hypothesis is that for all $a'<_A a$, every pair $(a',b)$ is accessible.
%   Inside this induction, we prove by induction on $<_B$ that for every $b:B$, the element $(a,b)$ is accessible.
%   The nested inductive hypothesis is that for every $b'<_B b$, the element $(a,b')$ is accessible.
%   But now, if $(a',b')< (a,b)$, then either $a<_A a'$ in which case $(a',b')$ is accessible by the first inductive hypothesis, or $a=a'$ and $b'<_B b$, in which case $(a,b')$ is accessible by the second inductive hypothesis.
%   Thus, by definition of accessibility, $(a,b)$ is accessible.
%   This completes both inductions.
% \end{proof}

\begin{ex}\label{ex:algebraic-ordinals}
  Define the usual algebraic operations on ordinals, and prove that they satisfy the usual properties.
\end{ex}

\begin{ex}\label{ex:prop-ord}
  Note that $\bool$ is an ordinal, under the obvious relation $<$ such that $\bfalse<\btrue$ only.
  \begin{enumerate}
  \item Define a relation $<$ on $\prop$ which makes it into an ordinal.
  \item Show that $\id[\ord]\bool\prop$ if and only if \LEM{} holds.
  \end{enumerate}
\end{ex}

\begin{ex}\label{ex:ninf-ord}
  Recall that we denote \nat by $\omega$ when regarding it as an ordinal; thus we have also the ordinal $\omega+1$.
  On the other hand, let us define
  \[ \nat_\infty \defeq \setof{a:\nat\to\bool | \fall{n:\nat} (a_n \le a_{\suc(n)}) } \]
  where $\le$ denotes the obvious partial order on $\bool$, with $\bfalse\le\btrue$.
  \begin{enumerate}
  \item Define a relation $<$ on $\nat_\infty$ which makes it into an ordinal.
  \item Show that $\id[\ord]{\omega+1}{\nat_\infty}$ if and only if the limited principle of omniscience~\eqref{eq:lpo} holds.%
    \index{limited principle of omniscience}%
  \end{enumerate}
\end{ex}

\begin{ex}\label{ex:well-founded-extensional-simulation}
  Show that if $(A,<)$ is well-founded and extensional and $A:\UU$, then there is a simulation $A\to V$, where $(V,\in)$ is the cumulative hierarchy from \cref{sec:cumulative-hierarchy} built from the universe~\UU.
\end{ex}

\begin{ex}\label{ex:choice-function}
  Show that \cref{thm:wop}\ref{item:wop1} is equivalent to the axiom of choice~\eqref{eq:ac}.
\end{ex}

\begin{ex}\label{ex:cumhierhit}
  Given types $A$ and $B$, define a \define{bitotal relation}
  \indexsee{bitotal relation}{relation, bitotal}%
  \indexdef{relation!bitotal}%
  to be $R:A\to B\to \prop$ such that
  \[ \Big(\fall{a:A}\exis{b:B} R(a,b) \Big) \land \Big(\fall{b:B}\exis{a:A} R(a,b) \Big). \]
  For such $A,B,R$, let $A\sqcup^R B$ be the higher inductive type generated by
  \begin{itemize}
  \item $i:A\to A\sqcup^R B$
  \item $j:B\to A\sqcup^R B$
  \item For each $a:A$ and $b:B$ such that $R(a,b)$, a path $i(a)=j(b)$.
  \end{itemize}
  Show that the cumulative hierarchy $V$ can be defined by the following more straightforward list of constructors, and that the resulting induction principle is the one given in \cref{sec:cumulative-hierarchy}.
  \begin{itemize}
  \item For every $A : \UU$ and $f : A \to V$, there is an element $\vset(A, f) : V$.
  \item For any $A,B:\UU$ and bitotal relation
    \index{relation!bitotal}%
    $R:A\to B\to \prop$, and any map $h:A\sqcup^R B \to V$, there is a path $\id{\vset(A,h\circ i)}{\vset(B,h\circ j)}$.
  \item The 0-truncation constructor.
  \end{itemize}
\end{ex}

\begin{ex}\label{ex:strong-collection}
  In \CZF, the \define{axiom of strong collection}
  \indexdef{axiom!strong collection}%
  \indexdef{collection!strong}%
  \indexdef{strong!collection}%
  has the form:
   \begin{multline*}
   \Parens{\fall{x\in v}\exis{y} R(x,y)} \Rightarrow \\
   \exis{w}\big[\big(\fall{x\in v}\exis{y\in w}R(x,y)\big)\land \big(\fall{y\in w}\exis{x\in v}R(x,y) \big)\big]
   \end{multline*}
   Does it hold in the cumulative hierarchy $V$?  (We do not know the answer to this.)
\end{ex}

\begin{ex}\label{ex:choice-cumulative-hierarchy-choice}
Verify that, if we assume $\choice{}$, then the cumulative hierarchy $V$ satisfies the usual set-theoretic axiom of choice, which may be stated in the form:
  \[
   \fall{x:V} \Parens{(\fall{y\in x}\exis{z:V} z\in y) \Rightarrow  \exis{c\in(\cup x)^x}\fall{y\in x} c(y)\in y}
   \]
\end{ex}

\begin{ex}\label{ex:plump-ordinals}
  Assuming propositional resizing, show that there is a mere predicate $\mathsf{isPlump}:\ord\to\prop$ such that for any $A:\ord$ we have
  \begin{multline*}\label{eq:plump}
    \mathsf{isPlump}(A) = \Parens{\fall{B<A} \mathsf{isPlump}(B)} \wedge\narrowbreak
    \Parens{\fall{C,B:\ord} C\le B < A \wedge \mathsf{isPlump}(C) \Rightarrow C < A}.
  \end{multline*}
  Note that $\mathsf{isPlump}$ cannot be defined by a simple well-founded induction over \ord; you must use a different well-founded relation.
  We say that an ordinal $A$ is \define{plump}~\cite{taylor:ordinals,Taylor99} if $\mathsf{isPlump}(A)$.
  \index{plump!ordinal}\index{ordinal!plump}%
\end{ex}

\begin{ex}\label{ex:not-plump}
  Show that \LEM{} is equivalent to the statement ``all ordinals are plump''.
\end{ex}

\begin{ex}\label{ex:plump-successor}
  Define the \define{plump successor}\index{plump!successor}\indexdef{successor!plump} of an ordinal $A$ to be
  \[ t(A) \defeq \setof{ B:\ord | (B\le A) \wedge \mathsf{isPlump}(B) } \]
  \begin{enumerate}
  \item By definition, $t(A)$ belongs to the next higher universe.
    Show that assuming propositional resizing, it is equal to an ordinal in the same universe as $A$.
  \item Again assuming propositional resizing, show that if $A$ is plump (\cref{ex:plump-ordinals}) then so is $t(A)$.
  \end{enumerate}
\end{ex}

\begin{ex}\label{ex:ZF-algebras}
  A \define{ZF-algebra}~\cite{JoyalMoerdijk1995}
  \index{ZF-algebra}%
  relative to a universe $\UU_i$ is a poset (see \cref{ct:orders}) $V:\UU_{i+1}$, which has all suprema indexed by types in $\UU_i$, and is equipped with a ``successor'' function $s:V\to V$ (not necessarily respecting $\le$ in any way).
  \begin{enumerate}
  \item Show that the cumulative hierarchy $(V_{\UU_i},\subseteq,s)$ is the initial ZF-algebra, where $s(x)$ is the singleton $\setof{x}$.
  \item Show that $(\ord_{\UU_i},\le,s)$ is the initial ZF-algebra with the property that $x\le s(x)$ for all $x$, where $s(A)=A+\unit$ is the successor\index{successor!of an ordinal} from \cref{thm:ordsucc}.
  \item Assuming propositional resizing, show that $\Parens{\setof{A:\ord_{\UU_i} | \mathsf{isPlump}(A) },\le,t}$ is the initial ZF-algebra with the property that $(x\le y) \Rightarrow (t(x)\le t(y))$ for all $x,y$, where $t$ is the plump successor from \cref{ex:plump-successor}.
  \end{enumerate}
\end{ex}

\begin{ex}\label{ex:monos-are-split-monos-iff-LEM-holds}
  For a category $A$, a morphism $f: \hom_A(a,b)$ is said to be a \define{split monomorphism} if there exists a morphism $g: \hom_A(b,a)$ such that $\id{g \circ f}{1_a}$. (Such $g$ is called a \define{retraction} of $f$.)
  Prove that the following are logically equivalent.
  \begin{enumerate}
  \item $\LEM{}$.
  \item For every sets $A$ and $B$, if $A$ is inhabited then for every monomorphism $f: A \to B$ in $\uset$, $f$ is also a split monomorphism in $\uset$.
  \end{enumerate}
\end{ex}

\index{set|)}%

% Local Variables:
% TeX-master: "hott-online"
% End:
